\subsection{Schutz vor Buffer Overflows}
\label{subsec:BO_protection}
Es gibt verschiedene Möglichkeiten um sich vor solchen Problemen zu schützen, unabhängig von der Unachtsamkeit des \CCpp
Programmierers. MSVC kennt Optionen wie\footnote{Compilerseitiger Schutz vor Buffer Overflows:
\href{http://go.yurichev.com/17133}{wikipedia.org/wiki/Buffer\_overflow\_protection}}:

\begin{lstlisting}
 /RTCs Stack Frame runtime checking
 /GZ Enable stack checks (/RTCs)
\end{lstlisting}

\myindex{x86!\Instructions!RET}
\myindex{Function prologue}
\myindex{Security cookie}
Eine Methode ist eine Zufallszahl zwischen die lokalen Variablen auf dem Stack am Funktionsprolog zu schreiben und
diesen im Funktionsepilog vor dem Beenden der Funktion zu überprüfen.
Wenn der Wert nicht identisch ist, sollte der letzte \RET Befehl nicht ausgeführt werden, sondern das Programm
angehalten werden. Der Prozess wird anhalten, aber das ist deutlich besser als eine Fernattacke auf Ihren Rechner.
    
\newcommand{\CANARYURL}{\href{http://go.yurichev.com/17134}{wikipedia.org/wiki/Domestic\_canary\#Miner.27s\_canary}}

\myindex{Canary}
Die Zufallszahl wird auch \q{canary} (dt. Kanarienvogel) genannt. Der Begriff stammt von den Kanarienvögeln der
Minenarbeiter\footnote{\CANARYURL}, die früher benutzt wurde, um giftige Gase schnell zu erkennen

Kanarienvögel reagieren sehr sensibel auf Grubengase und werden bei Gefahr sehr nervös oder sterben sogar.

Wenn wir unser einfaches Arraybeispiel in \ac{MSVC} mit Optionen RTC1 und RTCs kompilieren~(\myref{arrays_simple})
finden wir einen Aufruf von \TT{@\_RTC\_CheckStackVars@8}, eine Funktion am Ende der Funktion, die prüft, ob der
\q{canary} korrekt ist.

Schauen wir uns an, wie GCC die Sache handhabt.
Betrachten wir ein Beispiel mit \TT{alloca()}~(\myref{alloca}):

\lstinputlisting[style=customc]{patterns/02_stack/04_alloca/2_1.c}
Ohne zusätzliche Optionen fügt GCC 4.7.3 standardmäßig dem Code einen \q{canary} zum Überprüfen hinzu:

\lstinputlisting[caption=GCC 4.7.3,style=customasmx86]{patterns/13_arrays/3_BO_protection/gcc_canary_DE.asm}

\myindex{x86!\Registers!GS}
Der Zufallswert befindet sich in \TT{gs:20}.
Er wird auf den Stack geschrieben und am Ende der Funktion wird der Wert auf dem Stack mit dem korrekten \q{canary} in
\TT{gs:20} verglichen.
Wenn die Werte ungleich sind, wird die Funktion \TT{\_\_stack\_chk\_fail} aufgerufen und wir erkennen in der Konsole in
etwa das Folgende (Ubuntu 13.04 x86):

\begin{lstlisting}
*** buffer overflow detected ***: ./2_1 terminated
======= Backtrace: =========
/lib/i386-linux-gnu/libc.so.6(__fortify_fail+0x63)[0xb7699bc3]
/lib/i386-linux-gnu/libc.so.6(+0x10593a)[0xb769893a]
/lib/i386-linux-gnu/libc.so.6(+0x105008)[0xb7698008]
/lib/i386-linux-gnu/libc.so.6(_IO_default_xsputn+0x8c)[0xb7606e5c]
/lib/i386-linux-gnu/libc.so.6(_IO_vfprintf+0x165)[0xb75d7a45]
/lib/i386-linux-gnu/libc.so.6(__vsprintf_chk+0xc9)[0xb76980d9]
/lib/i386-linux-gnu/libc.so.6(__sprintf_chk+0x2f)[0xb7697fef]
./2_1[0x8048404]
/lib/i386-linux-gnu/libc.so.6(__libc_start_main+0xf5)[0xb75ac935]
======= Memory map: ========
08048000-08049000 r-xp 00000000 08:01 2097586    /home/dennis/2_1
08049000-0804a000 r--p 00000000 08:01 2097586    /home/dennis/2_1
0804a000-0804b000 rw-p 00001000 08:01 2097586    /home/dennis/2_1
094d1000-094f2000 rw-p 00000000 00:00 0          [heap]
b7560000-b757b000 r-xp 00000000 08:01 1048602    /lib/i386-linux-gnu/libgcc_s.so.1
b757b000-b757c000 r--p 0001a000 08:01 1048602    /lib/i386-linux-gnu/libgcc_s.so.1
b757c000-b757d000 rw-p 0001b000 08:01 1048602    /lib/i386-linux-gnu/libgcc_s.so.1
b7592000-b7593000 rw-p 00000000 00:00 0
b7593000-b7740000 r-xp 00000000 08:01 1050781    /lib/i386-linux-gnu/libc-2.17.so
b7740000-b7742000 r--p 001ad000 08:01 1050781    /lib/i386-linux-gnu/libc-2.17.so
b7742000-b7743000 rw-p 001af000 08:01 1050781    /lib/i386-linux-gnu/libc-2.17.so
b7743000-b7746000 rw-p 00000000 00:00 0
b775a000-b775d000 rw-p 00000000 00:00 0
b775d000-b775e000 r-xp 00000000 00:00 0          [vdso]
b775e000-b777e000 r-xp 00000000 08:01 1050794    /lib/i386-linux-gnu/ld-2.17.so
b777e000-b777f000 r--p 0001f000 08:01 1050794    /lib/i386-linux-gnu/ld-2.17.so
b777f000-b7780000 rw-p 00020000 08:01 1050794    /lib/i386-linux-gnu/ld-2.17.so
bff35000-bff56000 rw-p 00000000 00:00 0          [stack]
Aborted (core dumped)
\end{lstlisting}

\myindex{MS-DOS}
gs ist das sogenannte Segmentregister. Diese Register wurden zu Zeiten von MS-DOS und DOS-Erweiterungen häufig
verwendet. Heute ist sein Zweck ein anderer:
\myindex{TLS}
\myindex{Windows!TIB}
Kurz gesagt, zeigt das \TT{gs} Register in Linux stets auf den \ac{TLS}~(\myref{TLS})--hier werden threadspezifische
Informationen gespeichert. In win32 spielt das \TT{fs} Register übrigens die gleiche Rolle und zeigt stets auf
\ac{TIB}\footnote{\href{http://go.yurichev.com/17104}{wikipedia.org/wiki/Win32\_Thread\_Information\_Block}}.

Mehr Informationen finden sich im Quellcode des Linux Kernels (zumindest in der Version 3.11), in\\
\emph{arch/x86/include/asm/stackprotector.h} wird diese Variable in den Kommentaren beschrieben.

\subsubsection{ARM}

\myparagraph{\OptimizingKeilVI (\ThumbMode)}

\lstinputlisting[style=customasmARM]{patterns/04_scanf/1_simple/ARM_IDA.lst}

\myindex{\CLanguageElements!\Pointers}
Damit \scanf Elemente einlesen kann, benötigt die Funktion einen Paramter--einen Pointer vom Typ \Tint.
\Tint hat die Größe 32 Bit, wir benötigen also 4 Byte, um den Wert im Speicher abzulegen, und passt daher genau in ein 32-Bit-Register.
\myindex{IDA!var\_?}
Auf dem Stack wird Platz für die lokalen Variable \GTT{x} reserviert und IDA bezeichnet diese Variable mit \emph{var\_8}. 
Eigentlich ist aber an dieser Stelle gar nicht notwendig, Platz auf dem Stack zu reservieren, da \ac{SP} (\gls{stack pointer} 
bereits auf die Adresse zeigt und auch direkt verwendet werden kann.

Der Wert von \ac{SP} wird also in das \Reg{1} Register kopiert und zusammen mit dem Formatierungsstring an \scanf übergeben.

% TBT here
%\INS{PUSH/POP} instructions behaves differently in ARM than in x86 (it's the other way around).
%They are synonyms to \INS{STM/STMDB/LDM/LDMIA} instructions.
%And \INS{PUSH} instruction first writes a value into the stack, \emph{and then} subtracts \ac{SP} by 4.
%\INS{POP} first adds 4 to \ac{SP}, \emph{and then} reads a value from the stack.
%Hence, after \INS{PUSH}, \ac{SP} points to an unused space in stack.
%It is used by \scanf, and by \printf after.

%\INS{LDMIA} means \emph{Load Multiple Registers Increment address After each transfer}.
%\INS{STMDB} means \emph{Store Multiple Registers Decrement address Before each transfer}.

\myindex{ARM!\Instructions!LDR}
Später wird mithilfe des \INS{LDR} Befehls dieser Wert vom Stack in das \Reg{1} Register verschoben um an \printf übergeben werden zu können.

\myparagraph{ARM64}

\lstinputlisting[caption=\NonOptimizing GCC 4.9.1 ARM64,numbers=left,style=customasmARM]{patterns/04_scanf/1_simple/ARM64_GCC491_O0_DE.s}

Im Stack Frame werden 32 Byte reserviert, was deutlich mehr als benötigt ist. Vielleicht handelt es sich um eine Frage des Aligning (dt. Angleichens) von Speicheradressen.
Der interessanteste Teil ist, im Stack Frame einen Platz für die Variable $x$ zu finden (Zeile 22).
Warum 28? Irgendwie hat der Compiler entschieden die Variable am Ende des Stack Frames anstatt an dessen Beginn abzulegen.
Die Adresse wird an \scanf übergeben; diese Funktion speichert den Userinput an der genannten Adresse im Speicher.
Es handelt sich hier um einen 32-Bit-Wert vom Typ \Tint. 
Der Wert wird in Zeile 27 abgeholt und dann an \printf übergeben.




