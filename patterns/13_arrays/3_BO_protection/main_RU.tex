\subsection{Защита от переполнения буфера}
\label{subsec:BO_protection}

В наше время пытаются бороться с переполнением буфера невзирая на халатность программистов на \CCpp. 
В MSVC есть опции вроде\footnote{описания защит, которые компилятор может вставлять в код:
\href{http://go.yurichev.com/17133}{wikipedia.org/wiki/Buffer\_overflow\_protection}}:

\begin{lstlisting}
 /RTCs Stack Frame runtime checking
 /GZ Enable stack checks (/RTCs)
\end{lstlisting}

\myindex{x86!\Instructions!RET}
\myindex{Function prologue}
\myindex{Security cookie}
Одним из методов является вставка в прологе функции некоего случайного значения в область локальных переменных 
и проверка этого значения в эпилоге функции перед выходом. 
Если проверка не прошла, то не выполнять инструкцию \RET, а остановиться (или зависнуть). 
Процесс зависнет, но это лучше, чем удаленная атака на ваш компьютер.

\newcommand{\CANARYURL}{\href{http://go.yurichev.com/17135}{miningwiki.ru/wiki/Канарейка\_в\_шахте}}

\myindex{Canary}
Это случайное значение иногда называют \q{канарейкой}
\footnote{\q{canary} в англоязычной литературе}, 
по аналогии с шахтной канарейкой\footnote{\CANARYURL}.
Раньше использовали шахтеры, чтобы определять, есть ли в шахте опасный газ.

Канарейки очень к нему чувствительны и либо проявляли сильное беспокойство, либо гибли от газа.

Если скомпилировать наш простейший пример работы с массивом ~(\myref{arrays_simple}) в \ac{MSVC}
с опцией RTC1 или RTCs,
в конце нашей функции будет вызов функции \\
\TT{@\_RTC\_CheckStackVars@8}, проверяющей корректность \q{канарейки}.

Посмотрим, как дела обстоят в GCC. 
Возьмем пример из секции про \TT{alloca()}~(\myref{alloca}):

\lstinputlisting[style=customc]{patterns/02_stack/04_alloca/2_1.c}

По умолчанию, без дополнительных ключей, GCC 4.7.3 вставит в код проверку \q{канарейки}:

\lstinputlisting[caption=GCC 4.7.3,style=customasmx86]{patterns/13_arrays/3_BO_protection/gcc_canary_RU.asm}

\myindex{x86!\Registers!GS}
Случайное значение находится в \TT{gs:20}. 
Оно записывается в стек, затем, в конце функции, значение в стеке
сравнивается с корректной \q{канарейкой} в \TT{gs:20}. 
Если значения не равны, будет вызвана функция 
\TT{\_\_stack\_chk\_fail} и в консоли мы увидим что-то вроде такого
 (Ubuntu 13.04 x86):

\begin{lstlisting}
*** buffer overflow detected ***: ./2_1 terminated
======= Backtrace: =========
/lib/i386-linux-gnu/libc.so.6(__fortify_fail+0x63)[0xb7699bc3]
/lib/i386-linux-gnu/libc.so.6(+0x10593a)[0xb769893a]
/lib/i386-linux-gnu/libc.so.6(+0x105008)[0xb7698008]
/lib/i386-linux-gnu/libc.so.6(_IO_default_xsputn+0x8c)[0xb7606e5c]
/lib/i386-linux-gnu/libc.so.6(_IO_vfprintf+0x165)[0xb75d7a45]
/lib/i386-linux-gnu/libc.so.6(__vsprintf_chk+0xc9)[0xb76980d9]
/lib/i386-linux-gnu/libc.so.6(__sprintf_chk+0x2f)[0xb7697fef]
./2_1[0x8048404]
/lib/i386-linux-gnu/libc.so.6(__libc_start_main+0xf5)[0xb75ac935]
======= Memory map: ========
08048000-08049000 r-xp 00000000 08:01 2097586    /home/dennis/2_1
08049000-0804a000 r--p 00000000 08:01 2097586    /home/dennis/2_1
0804a000-0804b000 rw-p 00001000 08:01 2097586    /home/dennis/2_1
094d1000-094f2000 rw-p 00000000 00:00 0          [heap]
b7560000-b757b000 r-xp 00000000 08:01 1048602    /lib/i386-linux-gnu/libgcc_s.so.1
b757b000-b757c000 r--p 0001a000 08:01 1048602    /lib/i386-linux-gnu/libgcc_s.so.1
b757c000-b757d000 rw-p 0001b000 08:01 1048602    /lib/i386-linux-gnu/libgcc_s.so.1
b7592000-b7593000 rw-p 00000000 00:00 0
b7593000-b7740000 r-xp 00000000 08:01 1050781    /lib/i386-linux-gnu/libc-2.17.so
b7740000-b7742000 r--p 001ad000 08:01 1050781    /lib/i386-linux-gnu/libc-2.17.so
b7742000-b7743000 rw-p 001af000 08:01 1050781    /lib/i386-linux-gnu/libc-2.17.so
b7743000-b7746000 rw-p 00000000 00:00 0
b775a000-b775d000 rw-p 00000000 00:00 0
b775d000-b775e000 r-xp 00000000 00:00 0          [vdso]
b775e000-b777e000 r-xp 00000000 08:01 1050794    /lib/i386-linux-gnu/ld-2.17.so
b777e000-b777f000 r--p 0001f000 08:01 1050794    /lib/i386-linux-gnu/ld-2.17.so
b777f000-b7780000 rw-p 00020000 08:01 1050794    /lib/i386-linux-gnu/ld-2.17.so
bff35000-bff56000 rw-p 00000000 00:00 0          [stack]
Aborted (core dumped)
\end{lstlisting}

\myindex{MS-DOS}
gs это так называемый сегментный регистр. Эти регистры широко использовались во времена MS-DOS 
и DOS-экстендеров.
Сейчас их функция немного изменилась.
\myindex{TLS}
\myindex{Windows!TIB}
Если говорить кратко, в Linux \TT{gs} всегда указывает на \ac{TLS}~(\myref{TLS})~--- там находится различная 
информация, специфичная для выполняющегося потока.

Кстати, в win32 эту же роль играет сегментный регистр \TT{fs},
он всегда указывает на
\ac{TIB} \footnote{\href{http://go.yurichev.com/17104}{wikipedia.org/wiki/Win32\_Thread\_Information\_Block}}. 

Больше информации можно почерпнуть из исходных кодов Linux (по крайней мере, в версии 3.11):\\
в файле \emph{arch/x86/include/asm/stackprotector.h} в комментариях описывается эта переменная.

\subsubsection{ARM: \OptimizingKeilVI (\ARMMode)}
\myindex{\CLanguageElements!switch}

\lstinputlisting[style=customasmARM]{patterns/08_switch/1_few/few_ARM_ARM_O3.asm}

Мы снова не сможем сказать, глядя на этот код, был ли в оригинальном исходном коде switch() 
либо же несколько операторов if().

\myindex{ARM!\Instructions!ADRcc}
Так или иначе, мы снова видим здесь инструкции с предикатами, например, \ADREQ (\emph{(Equal)}), 
которая будет исполняться только
если $R0=0$, и тогда в \Reg{0} будет загружен адрес строки \emph{<<zero\textbackslash{}n>>}.

\myindex{ARM!\Instructions!BEQ}
Следующая инструкция \ac{BEQ} перенаправит исполнение на \TT{loc\_170}, если $R0=0$.

Кстати, наблюдательный читатель может спросить, сработает ли \ac{BEQ} нормально,
ведь \ADREQ перед ним уже заполнила регистр \Reg{0} чем-то другим?

Сработает, потому что \ac{BEQ} проверяет флаги, установленные инструкцией \CMP, 
а \ADREQ флаги никак не модифицирует.

Далее всё просто и знакомо. 
Вызов \printf один, и в самом конце, мы уже рассматривали подобный трюк~(\myref{ARM_B_to_printf}).
К вызову функции \printf{} в конце ведут три пути.

\myindex{ARM!\Instructions!ADRcc}
\myindex{ARM!\Instructions!CMP}
Последняя инструкция \TT{CMP R0, \#2} здесь нужна, чтобы узнать $a=2$ или нет.

Если это не так, то при помощи \ADRNE (\emph{Not Equal}) в \Reg{0} будет загружен указатель на 
строку \emph{<<something unknown \textbackslash{}n>>}, ведь $a$ уже было проверено на 0 и 1 до этого, 
и здесь $a$ точно не попадает под эти константы.

Ну а если $R0=2$, в \Reg{0} будет загружен указатель на строку \emph{<<two\textbackslash{}n>>} при помощи инструкции \ADREQ.

\subsubsection{ARM: \OptimizingKeilVI (\ThumbMode)}

\lstinputlisting[style=customasmARM]{patterns/08_switch/1_few/few_ARM_thumb_O3.asm}

% FIXME а каким можно? к каким нельзя? \myref{} ->
Как уже было отмечено, в Thumb-режиме нет возможности добавлять условные предикаты к большинству инструкций,
так что Thumb-код вышел похожим на код x86 в стиле \ac{CISC}, вполне понятный.

\subsubsection{ARM64: \NonOptimizing GCC (Linaro) 4.9}

\lstinputlisting[style=customasmARM]{patterns/08_switch/1_few/ARM64_GCC_O0_RU.lst}

Входное значение имеет тип \Tint, поэтому для него используется регистр \RegW{0},
а не целая часть регистра \RegX{0}.

Указатели на строки передаются в \puts при помощи пары инструкций ADRP/ADD, как было показано в примере
\q{\HelloWorldSectionName}:~\myref{pointers_ADRP_and_ADD}.

\subsubsection{ARM64: \Optimizing GCC (Linaro) 4.9}

\lstinputlisting[style=customasmARM]{patterns/08_switch/1_few/ARM64_GCC_O3_RU.lst}

Фрагмент кода более оптимизированный.
Инструкция \TT{CBZ} (\emph{Compare and Branch on Zero}~--- сравнить и перейти если ноль) совершает переход если \RegW{0} ноль.
Здесь также прямой переход на \puts вместо вызова, как уже было описано:~\myref{JMP_instead_of_RET}.


