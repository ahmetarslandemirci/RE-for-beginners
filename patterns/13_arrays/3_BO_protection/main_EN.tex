\subsection{Buffer overflow protection methods}
\label{subsec:BO_protection}

There are several methods to protect against this scourge, regardless of the \CCpp programmers' negligence.
MSVC has options like\footnote{compiler-side buffer overflow protection methods:
\href{http://go.yurichev.com/17133}{wikipedia.org/wiki/Buffer\_overflow\_protection}}:

\begin{lstlisting}
 /RTCs Stack Frame runtime checking
 /GZ Enable stack checks (/RTCs)
\end{lstlisting}

\myindex{x86!\Instructions!RET}
\myindex{Function prologue}
\myindex{Security cookie}

One of the methods is to write a random value between the local variables in stack at function prologue 
and to check it in function epilogue before the function exits.
If value is not the same, do not execute the last instruction \RET, but stop (or hang).
The process will halt, but that is much better than a remote attack to your host.
    
\newcommand{\CANARYURL}{\href{http://go.yurichev.com/17134}{wikipedia.org/wiki/Domestic\_canary\#Miner.27s\_canary}}

\myindex{Canary}

This random value is called a \q{canary} sometimes, it is related to the miners' canary\footnote{\CANARYURL},
they were used by miners in the past days in order to detect poisonous gases quickly.

Canaries are very sensitive to mine gases, they become very agitated in case of danger, or even die.

If we compile our very simple array example~(\myref{arrays_simple}) in \ac{MSVC}
with RTC1 and RTCs option,\\
you can see a call to \TT{@\_RTC\_CheckStackVars@8} a function at the end of the function that checks if the \q{canary} is correct.

Let's see how GCC handles this. 
Let's take an \TT{alloca()}~(\myref{alloca}) example:

\lstinputlisting[style=customc]{patterns/02_stack/04_alloca/2_1.c}

By default, without any additional options, GCC 4.7.3 inserts a \q{canary} check into the code:

\lstinputlisting[caption=GCC 4.7.3,style=customasmx86]{patterns/13_arrays/3_BO_protection/gcc_canary_EN.asm}

\myindex{x86!\Registers!GS}
The random value is located in \TT{gs:20}. 
It gets written on the stack and then at the end of the function
the value in the stack is compared with the correct \q{canary} in \TT{gs:20}. 
If the values are not equal, the 
\TT{\_\_stack\_chk\_fail} 
function is called and we can see in the console something like that (Ubuntu 13.04 x86):

\begin{lstlisting}
*** buffer overflow detected ***: ./2_1 terminated
======= Backtrace: =========
/lib/i386-linux-gnu/libc.so.6(__fortify_fail+0x63)[0xb7699bc3]
/lib/i386-linux-gnu/libc.so.6(+0x10593a)[0xb769893a]
/lib/i386-linux-gnu/libc.so.6(+0x105008)[0xb7698008]
/lib/i386-linux-gnu/libc.so.6(_IO_default_xsputn+0x8c)[0xb7606e5c]
/lib/i386-linux-gnu/libc.so.6(_IO_vfprintf+0x165)[0xb75d7a45]
/lib/i386-linux-gnu/libc.so.6(__vsprintf_chk+0xc9)[0xb76980d9]
/lib/i386-linux-gnu/libc.so.6(__sprintf_chk+0x2f)[0xb7697fef]
./2_1[0x8048404]
/lib/i386-linux-gnu/libc.so.6(__libc_start_main+0xf5)[0xb75ac935]
======= Memory map: ========
08048000-08049000 r-xp 00000000 08:01 2097586    /home/dennis/2_1
08049000-0804a000 r--p 00000000 08:01 2097586    /home/dennis/2_1
0804a000-0804b000 rw-p 00001000 08:01 2097586    /home/dennis/2_1
094d1000-094f2000 rw-p 00000000 00:00 0          [heap]
b7560000-b757b000 r-xp 00000000 08:01 1048602    /lib/i386-linux-gnu/libgcc_s.so.1
b757b000-b757c000 r--p 0001a000 08:01 1048602    /lib/i386-linux-gnu/libgcc_s.so.1
b757c000-b757d000 rw-p 0001b000 08:01 1048602    /lib/i386-linux-gnu/libgcc_s.so.1
b7592000-b7593000 rw-p 00000000 00:00 0
b7593000-b7740000 r-xp 00000000 08:01 1050781    /lib/i386-linux-gnu/libc-2.17.so
b7740000-b7742000 r--p 001ad000 08:01 1050781    /lib/i386-linux-gnu/libc-2.17.so
b7742000-b7743000 rw-p 001af000 08:01 1050781    /lib/i386-linux-gnu/libc-2.17.so
b7743000-b7746000 rw-p 00000000 00:00 0
b775a000-b775d000 rw-p 00000000 00:00 0
b775d000-b775e000 r-xp 00000000 00:00 0          [vdso]
b775e000-b777e000 r-xp 00000000 08:01 1050794    /lib/i386-linux-gnu/ld-2.17.so
b777e000-b777f000 r--p 0001f000 08:01 1050794    /lib/i386-linux-gnu/ld-2.17.so
b777f000-b7780000 rw-p 00020000 08:01 1050794    /lib/i386-linux-gnu/ld-2.17.so
bff35000-bff56000 rw-p 00000000 00:00 0          [stack]
Aborted (core dumped)
\end{lstlisting}

\myindex{MS-DOS}
gs is the so-called segment register. These registers were used widely in MS-DOS and DOS-extenders
times.
Today, its function is different.
\myindex{TLS}
\myindex{Windows!TIB}

To say it briefly, the \TT{gs} register in Linux always points to the
\ac{TLS}~(\myref{TLS})---some information specific to thread is stored there.
By the way, in win32 the \TT{fs} register plays the same role, pointing to
\ac{TIB} \footnote{\href{http://go.yurichev.com/17104}{wikipedia.org/wiki/Win32\_Thread\_Information\_Block}}. 

More information can be found in the Linux kernel source code (at least in 3.11 version),\\
in \emph{arch/x86/include/asm/stackprotector.h} this variable is described in the comments.

\subsection{ARM}

The ARM processor, just like in any other \q{pure} RISC processor lacks an instruction for division.
It also lacks a single instruction for multiplication by a 32-bit constant (recall that a 32-bit
constant cannot fit into a 32-bit opcode).

By taking advantage of this clever trick (or \emph{hack}), it is possible to do division using only three instructions: addition,
subtraction and bit shifts~(\myref{sec:bitfields}).

Here is an example that divides a 32-bit number by 10, from
\InSqBrackets{\ARMCookBook 3.3 Division by a Constant}.
The output consists of the quotient and the remainder.

\begin{lstlisting}[style=customasmARM]
; takes argument in a1
; returns quotient in a1, remainder in a2
; cycles could be saved if only divide or remainder is required
    SUB    a2, a1, #10             ; keep (x-10) for later
    SUB    a1, a1, a1, lsr #2
    ADD    a1, a1, a1, lsr #4
    ADD    a1, a1, a1, lsr #8
    ADD    a1, a1, a1, lsr #16
    MOV    a1, a1, lsr #3
    ADD    a3, a1, a1, asl #2
    SUBS   a2, a2, a3, asl #1      ; calc (x-10) - (x/10)*10
    ADDPL  a1, a1, #1              ; fix-up quotient
    ADDMI  a2, a2, #10             ; fix-up remainder
    MOV    pc, lr
\end{lstlisting}

\subsubsection{\OptimizingXcodeIV (\ARMMode)}

\begin{lstlisting}[style=customasmARM]
__text:00002C58 39 1E 08 E3 E3 18 43 E3  MOV    R1, 0x38E38E39
__text:00002C60 10 F1 50 E7              SMMUL  R0, R0, R1
__text:00002C64 C0 10 A0 E1              MOV    R1, R0,ASR#1
__text:00002C68 A0 0F 81 E0              ADD    R0, R1, R0,LSR#31
__text:00002C6C 1E FF 2F E1              BX     LR
\end{lstlisting}

This code is almost the same as the one generated by the optimizing MSVC and GCC.

Apparently, LLVM uses the same algorithm for generating constants.

\myindex{ARM!\Instructions!MOV}
\myindex{ARM!\Instructions!MOVT}

The observant reader may ask, how does \MOV writes a 32-bit value in a register, when this is not possible in ARM mode.

it is impossible indeed, but, as we see,
there are 8 bytes per instruction instead of the standard 4,
in fact, there are two instructions.

The first instruction loads \TT{0x8E39} into the low 16 bits of register and the second instruction is
\TT{MOVT}, it loads \TT{0x383E} into the high 16 bits of the register.
\IDA is fully aware of such sequences, and for the sake of compactness reduces them to one single \q{pseudo-instruction}.

\myindex{ARM!\Instructions!SMMUL}
The \TT{SMMUL} (\emph{Signed Most Significant Word Multiply}) 
instruction two multiplies numbers, treating them as signed numbers
and leaving the high 32-bit part of result in the \Reg{0} register,
dropping the low 32-bit part of the result.

\myindex{ARM!Optional operators!ASR}
The\TT{\q{MOV R1, R0,ASR\#1}} instruction is an arithmetic shift right by one bit.

\myindex{ARM!\Instructions!ADD}
\myindex{ARM!Data processing instructions}
\myindex{ARM!Optional operators!LSR}
\TT{\q{ADD R0, R1, R0,LSR\#31}} is $R0=R1 + R0>>31$

% FIXME какие именно инструкции? \myref{} ->
\label{shifts_in_ARM_mode}

There is no separate shifting instruction in ARM mode.
Instead, an instructions like 
(\MOV, \ADD, \SUB, \TT{RSB})\footnote{\DataProcessingInstructionsFootNote}
can have a suffix added, that says if the second operand must be shifted, and if yes, by what value and how.
\TT{ASR} stands for \emph{Arithmetic Shift Right}, \TT{LSR}---\emph{Logical Shift Right}.

\subsubsection{\OptimizingXcodeIV (\ThumbTwoMode)}

\begin{lstlisting}[style=customasmARM]
MOV             R1, 0x38E38E39
SMMUL.W         R0, R0, R1
ASRS            R1, R0, #1
ADD.W           R0, R1, R0,LSR#31
BX              LR
\end{lstlisting}

\myindex{ARM!\Instructions!ASRS}

There are separate instructions for shifting in Thumb mode, 
and one of them is used here---\TT{ASRS} (arithmetic shift right).

\subsubsection{\NonOptimizing Xcode 4.6.3 (LLVM) and Keil 6/2013}

\NonOptimizing LLVM
does not generate the code we saw before in this section, but instead inserts a call to the library function 
\emph{\_\_\_divsi3}.

What about Keil: it inserts a call to the library function \emph{\_\_aeabi\_idivmod} in all cases.


