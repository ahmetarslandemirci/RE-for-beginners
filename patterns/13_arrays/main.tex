\mysection{\Arrays}
\label{arrays}

\RU{Массив это просто набор переменных в памяти, 
обязательно лежащих рядом и обязательно одного типа%
\footnote{\ac{AKA} \q{гомогенный контейнер}}.}
\EN{An array is just a set of variables in memory 
that lie next to each other and that have the same type%
\footnote{\ac{AKA} \q{homogeneous container}}.}
\DEph{}
\FR{Un tableau est simplement un ensemble de variables en mémoire
qui sont situées les unes à côté des autres et qui ont le même type%
\footnote{\ac{AKA} \q{container homogène}}.}
\JPN{配列は、互いに隣り合って、
同じ型を持つメモリ内の変数のセットです。%
\footnote{\ac{AKA} \q{homogeneous container}}}

% sections
\subsection{\RU{Простой пример}\EN{Simple example}\FR{Exemple simple}\JPN{単純な例}}

\label{arrays_simple}
\lstinputlisting[style=customc]{patterns/13_arrays/1_simple/simple.c}

\EN{\subsubsection{x86}

\myparagraph{MSVC}

Let's compile:

\lstinputlisting[caption=MSVC 2008,style=customasmx86]{patterns/13_arrays/1_simple/simple_msvc.asm}

\myindex{x86!\Instructions!SHL}

Nothing very special, just two loops: the first is a filling loop and second is a printing loop.
The \TT{shl ecx, 1} instruction is used for value multiplication by 2 in \ECX, more about it: ~\myref{SHR}.

80 bytes are allocated on the stack for the array, 20 elements of 4 bytes.

\clearpage
Let's try this example in \olly.
\myindex{\olly}

We see how the array gets filled: 

each element is 32-bit word of \Tint type and its value is the index multiplied by 2:

\begin{figure}[H]
\centering
\myincludegraphics{patterns/13_arrays/1_simple/olly.png}
\caption{\olly: after array filling}
\label{fig:array_simple_olly}
\end{figure}

Since this array is located in the stack, we can see all its 20 elements there.

\myparagraph{GCC}

Here is what GCC 4.4.1 does:

\lstinputlisting[caption=GCC 4.4.1,style=customasmx86]{patterns/13_arrays/1_simple/simple_gcc.asm}

By the way, variable $a$ is of type  \emph{int*} 
(the pointer to \Tint{})---you can pass a pointer to an array to another function,
but it's more correct to say that a pointer to the first element of the array is passed
(the addresses of rest of the elements are calculated in an obvious way).

If you index this pointer as \emph{a[idx]}, \emph{idx} is just to be added to the pointer 
and the element placed there (to which calculated pointer is pointing) is to be returned.

An interesting example: a string of characters like 
\emph{string} is an array of characters and it has a type of \emph{const char[]}.

An index can also be applied to this pointer.

And that is why it is possible to write things like \TT{\q{string}[i]}---this is a correct \CCpp expression!

}
\RU{\subsubsection{x86}

\myparagraph{MSVC}

Компилируем:

\lstinputlisting[caption=MSVC 2008,style=customasmx86]{patterns/13_arrays/1_simple/simple_msvc.asm}

\myindex{x86!\Instructions!SHL}
Ничего особенного, просто два цикла. Один изменяет массив, второй печатает его содержимое. 
Команда \INS{shl ecx, 1} используется для умножения \ECX на 2, об этом: ~(\myref{SHR}).

Под массив выделено в стеке 80 байт, это 20 элементов по 4 байта.

\clearpage
Попробуем этот пример в \olly.
\myindex{\olly}

Видно, как заполнился массив: каждый элемент это 32-битное слово типа \Tint, с шагом 2:

\begin{figure}[H]
\centering
\myincludegraphics{patterns/13_arrays/1_simple/olly.png}
\caption{\olly: после заполнения массива}
\label{fig:array_simple_olly}
\end{figure}

А так как этот массив находится в стеке, то мы видим все его 20 элементов внутри стека.

\myparagraph{GCC}

Рассмотрим результат работы GCC 4.4.1:

\lstinputlisting[caption=GCC 4.4.1,style=customasmx86]{patterns/13_arrays/1_simple/simple_gcc.asm}

Переменная $a$ в нашем примере имеет тип \emph{int*} (указатель на \Tint{}).
Вы можете попробовать передать в другую функцию указатель на массив,
но точнее было бы сказать, что передается указатель на первый элемент массива
(а адреса остальных элементов массива можно вычислить очевидным образом).

Если индексировать этот указатель как \emph{a[idx]}, \emph{idx} просто прибавляется к указателю 
и возвращается элемент, расположенный там, куда ссылается вычисленный указатель.

Вот любопытный пример. Строка символов вроде \emph{string} это массив из символов. 
Она имеет тип \emph{const char[]}.
К этому указателю также можно применять индекс.

Поэтому можно написать даже так:  \TT{\q{string}[i]}~--- это совершенно легальное выражение в \CCpp!

}
\DE{\subsubsection{x86}

\myparagraph{MSVC}

Kompilieren wir das Beispiel:

\lstinputlisting[caption=MSVC 2008,style=customasmx86]{patterns/13_arrays/1_simple/simple_msvc.asm}

\myindex{x86!\Instructions!SHL}
Soweit nichts Außergewöhnliches, nur zwei Schleifen: die erste füllt mit Werten auf und die zweite gibt Werte aus.
% TBT
Der Befehl \TT{shl ecx, 1} wird für die Multiplikation mit 2 in \ECX verwendet; mehr dazu unten~\myref{SHR}.

Auf dem Stack werden 80 Bytes für das Array reserviert: 20 Elemente von je 4 Byte.

\clearpage
Untersuchen wir dieses Beispiel in \olly.
\myindex{\olly}

Wir erkennen wie das Array befüllt wird:

jedes Element ist ein 32-Bit-Wort vom Typ \Tint und der Wert ist der Index multipliziert mit 2:

\begin{figure}[H]
\centering
\myincludegraphics{patterns/13_arrays/1_simple/olly.png}
\caption{\olly: nach dem Füllen des Arrays}
\label{fig:array_simple_olly}
\end{figure}
Da sich dieses Array auf dem Stack befindet, finden wir dort alle seine 20 Elemente.

\myparagraph{GCC}

Hier ist was GCC 4.4.1 erzeugt:

\lstinputlisting[caption=GCC 4.4.1,style=customasmx86]{patterns/13_arrays/1_simple/simple_gcc.asm}
Die Variable $a$ ist übrigens vom Typ \emph{int*} (Pointer auf \Tint{})--man kann einen Pointer auf ein Array an eine
andere Funktion übergeben, aber es ist richtiger zu sagen, dass der Pointer auf das erste Element des Arrays übergeben
wird. (Die Adressen der übrigen Elemente werden in bekannter Weise berechnet.)

Wenn man diesen Pointer mittels \emph{a[idx]} indiziert, wird \emph{idx} zum Pointer addiert und das dort abgelegte Element
(auf das der berechnete Pointer zeigt) wird zurückgegeben.

Ein interessantes Beispiel: ein String wie \emph{string} ist ein Array von Chars und hat den Typ \emph{const
char[]}.

Auch auf diesen Pointer kann ein Index angewendet werden.

Das ist der Grund warum es es möglich ist, Dinge wie \TT{\q{string}[i]} zu schreiben--es handelt sich dabei um einen
korrekten \CCpp Ausdruck!

}
\FR{\subsubsection{x86}

\myparagraph{MSVC}

Compilons:

\lstinputlisting[caption=MSVC 2008,style=customasmx86]{patterns/13_arrays/1_simple/simple_msvc.asm}

\myindex{x86!\Instructions!SHL}

Rien de très particulier, juste deux boucles: la première est celle de remplissage
et la seconde celle d'affichage.
L'instruction \TT{shl ecx, 1} est utilisée pour la multiplication par 2 de la valeur
dans \ECX, voir à ce sujet ci-après~\myref{SHR}.

80 octets sont alloués sur la pile pour le tableau, 20 éléments de 4 octets.

\clearpage
Essayons cet exemple dans \olly.
\myindex{\olly}

Nous voyons comment le tableau est rempli:

chaque élément est un mot de 32-bit de type \Tint et sa valeur est l'index multiplié
par 2:

\begin{figure}[H]
\centering
\myincludegraphics{patterns/13_arrays/1_simple/olly.png}
\caption{\olly: après remplissage du tableau}
\label{fig:array_simple_olly}
\end{figure}

Puisque le tableau est situé sur la pile, nous pouvons voir ses 20 éléments ici.

\myparagraph{GCC}

Voici ce que GCC 4.4.1 génère:

\lstinputlisting[caption=GCC 4.4.1,style=customasmx86]{patterns/13_arrays/1_simple/simple_gcc.asm}

À propos, la variable $a$ est de type \emph{int*} (un pointeur sur un \Tint{})---vous
pouvez passer un pointeur sur un tableau à une autre fonction, mais c'est plus juste
de dire qu'un pointeur sur le premier élément du tableau est passé (les adresses
du reste des éléments sont calculées de manière évidente).

% TODO: clarifier
Si vous indexez ce pointeur en \emph{a[idx]}, il suffit d'ajouter \emph{idx} au pointeur
et l'élément placé ici (sur lequel pointe le pointeur calculé) est renvoyé.

Un exemple intéressant: une chaîne de caractères comme \emph{string} est un tableau
de caractères et a un type \emph{const char[]}.

Un index peut aussi être appliqué à ce pointeur.

Et c'est pourquoi il est possible d'écrire des choses comme \TT{\q{string}[i]}---c'est
une expression \CCpp correcte!

}
\JPN{\subsubsection{x86}

\myparagraph{MSVC}

コンパイルしてみましょう。

\lstinputlisting[caption=MSVC 2008,style=customasmx86]{patterns/13_arrays/1_simple/simple_msvc.asm}

\myindex{x86!\Instructions!SHL}

特別なことは何もなくて、2つのループだけです。1つめは配列に値を詰めるループで2つめは値を表示するループです。
\TT{shl ecx, 1}命令は \ECX の値を2倍するのに使用されます。詳細はこちら~\myref{SHR}

80バイトは4バイトの20要素分の配列用としてスタック上に確保されます。

\clearpage
\olly でこの例を試してみましょう。
\myindex{\olly}

配列がどのように埋まるのか見ていきます。

各要素は32ビットの \Tint 型で値はインデックスを2倍したものです。

\begin{figure}[H]
\centering
\myincludegraphics{patterns/13_arrays/1_simple/olly.png}
\caption{\olly: 要素を埋めた後}
\label{fig:array_simple_olly}
\end{figure}

この配列はスタックに位置しているので、20要素すべてを見ることができます。

\myparagraph{GCC}

GCC 4.4.1ではこのようになります。

\lstinputlisting[caption=GCC 4.4.1,style=customasmx86]{patterns/13_arrays/1_simple/simple_gcc.asm}

なお、変数 $a$ は\IT{int*}型です
(\Tint{} へのポインタ)---別の関数に配列へのポインタを渡すことができます。
しかし、もっと正確には、配列の最初の要素へのポインタが渡されます。
(要素の残りのアドレスは明確なやり方で計算されます)

もしこのポインタを\IT{a[idx]}としてインデックスするなら、\IT{idx}はポインタに加算されるだけで、
配置されている要素(計算されたポインタが示されている)がリターンされます。

面白い例:\IT{\q{string}}のような文字列は(1)文字の配列で\IT{const char[]}の型を持ちます。

インデックスもこのポインタに適用されます。

そしてこれが\TT{\q{string}[i]}のように書き込みが可能な理由です。これは \CCpp の正しい表現です!
}

\EN{\subsection{ARM}

The ARM processor, just like in any other \q{pure} RISC processor lacks an instruction for division.
It also lacks a single instruction for multiplication by a 32-bit constant (recall that a 32-bit
constant cannot fit into a 32-bit opcode).

By taking advantage of this clever trick (or \emph{hack}), it is possible to do division using only three instructions: addition,
subtraction and bit shifts~(\myref{sec:bitfields}).

Here is an example that divides a 32-bit number by 10, from
\InSqBrackets{\ARMCookBook 3.3 Division by a Constant}.
The output consists of the quotient and the remainder.

\begin{lstlisting}[style=customasmARM]
; takes argument in a1
; returns quotient in a1, remainder in a2
; cycles could be saved if only divide or remainder is required
    SUB    a2, a1, #10             ; keep (x-10) for later
    SUB    a1, a1, a1, lsr #2
    ADD    a1, a1, a1, lsr #4
    ADD    a1, a1, a1, lsr #8
    ADD    a1, a1, a1, lsr #16
    MOV    a1, a1, lsr #3
    ADD    a3, a1, a1, asl #2
    SUBS   a2, a2, a3, asl #1      ; calc (x-10) - (x/10)*10
    ADDPL  a1, a1, #1              ; fix-up quotient
    ADDMI  a2, a2, #10             ; fix-up remainder
    MOV    pc, lr
\end{lstlisting}

\subsubsection{\OptimizingXcodeIV (\ARMMode)}

\begin{lstlisting}[style=customasmARM]
__text:00002C58 39 1E 08 E3 E3 18 43 E3  MOV    R1, 0x38E38E39
__text:00002C60 10 F1 50 E7              SMMUL  R0, R0, R1
__text:00002C64 C0 10 A0 E1              MOV    R1, R0,ASR#1
__text:00002C68 A0 0F 81 E0              ADD    R0, R1, R0,LSR#31
__text:00002C6C 1E FF 2F E1              BX     LR
\end{lstlisting}

This code is almost the same as the one generated by the optimizing MSVC and GCC.

Apparently, LLVM uses the same algorithm for generating constants.

\myindex{ARM!\Instructions!MOV}
\myindex{ARM!\Instructions!MOVT}

The observant reader may ask, how does \MOV writes a 32-bit value in a register, when this is not possible in ARM mode.

it is impossible indeed, but, as we see,
there are 8 bytes per instruction instead of the standard 4,
in fact, there are two instructions.

The first instruction loads \TT{0x8E39} into the low 16 bits of register and the second instruction is
\TT{MOVT}, it loads \TT{0x383E} into the high 16 bits of the register.
\IDA is fully aware of such sequences, and for the sake of compactness reduces them to one single \q{pseudo-instruction}.

\myindex{ARM!\Instructions!SMMUL}
The \TT{SMMUL} (\emph{Signed Most Significant Word Multiply}) 
instruction two multiplies numbers, treating them as signed numbers
and leaving the high 32-bit part of result in the \Reg{0} register,
dropping the low 32-bit part of the result.

\myindex{ARM!Optional operators!ASR}
The\TT{\q{MOV R1, R0,ASR\#1}} instruction is an arithmetic shift right by one bit.

\myindex{ARM!\Instructions!ADD}
\myindex{ARM!Data processing instructions}
\myindex{ARM!Optional operators!LSR}
\TT{\q{ADD R0, R1, R0,LSR\#31}} is $R0=R1 + R0>>31$

% FIXME какие именно инструкции? \myref{} ->
\label{shifts_in_ARM_mode}

There is no separate shifting instruction in ARM mode.
Instead, an instructions like 
(\MOV, \ADD, \SUB, \TT{RSB})\footnote{\DataProcessingInstructionsFootNote}
can have a suffix added, that says if the second operand must be shifted, and if yes, by what value and how.
\TT{ASR} stands for \emph{Arithmetic Shift Right}, \TT{LSR}---\emph{Logical Shift Right}.

\subsubsection{\OptimizingXcodeIV (\ThumbTwoMode)}

\begin{lstlisting}[style=customasmARM]
MOV             R1, 0x38E38E39
SMMUL.W         R0, R0, R1
ASRS            R1, R0, #1
ADD.W           R0, R1, R0,LSR#31
BX              LR
\end{lstlisting}

\myindex{ARM!\Instructions!ASRS}

There are separate instructions for shifting in Thumb mode, 
and one of them is used here---\TT{ASRS} (arithmetic shift right).

\subsubsection{\NonOptimizing Xcode 4.6.3 (LLVM) and Keil 6/2013}

\NonOptimizing LLVM
does not generate the code we saw before in this section, but instead inserts a call to the library function 
\emph{\_\_\_divsi3}.

What about Keil: it inserts a call to the library function \emph{\_\_aeabi\_idivmod} in all cases.
}
\RU{\subsubsection{ARM: \OptimizingKeilVI (\ARMMode)}
\myindex{\CLanguageElements!switch}

\lstinputlisting[style=customasmARM]{patterns/08_switch/1_few/few_ARM_ARM_O3.asm}

Мы снова не сможем сказать, глядя на этот код, был ли в оригинальном исходном коде switch() 
либо же несколько операторов if().

\myindex{ARM!\Instructions!ADRcc}
Так или иначе, мы снова видим здесь инструкции с предикатами, например, \ADREQ (\emph{(Equal)}), 
которая будет исполняться только
если $R0=0$, и тогда в \Reg{0} будет загружен адрес строки \emph{<<zero\textbackslash{}n>>}.

\myindex{ARM!\Instructions!BEQ}
Следующая инструкция \ac{BEQ} перенаправит исполнение на \TT{loc\_170}, если $R0=0$.

Кстати, наблюдательный читатель может спросить, сработает ли \ac{BEQ} нормально,
ведь \ADREQ перед ним уже заполнила регистр \Reg{0} чем-то другим?

Сработает, потому что \ac{BEQ} проверяет флаги, установленные инструкцией \CMP, 
а \ADREQ флаги никак не модифицирует.

Далее всё просто и знакомо. 
Вызов \printf один, и в самом конце, мы уже рассматривали подобный трюк~(\myref{ARM_B_to_printf}).
К вызову функции \printf{} в конце ведут три пути.

\myindex{ARM!\Instructions!ADRcc}
\myindex{ARM!\Instructions!CMP}
Последняя инструкция \TT{CMP R0, \#2} здесь нужна, чтобы узнать $a=2$ или нет.

Если это не так, то при помощи \ADRNE (\emph{Not Equal}) в \Reg{0} будет загружен указатель на 
строку \emph{<<something unknown \textbackslash{}n>>}, ведь $a$ уже было проверено на 0 и 1 до этого, 
и здесь $a$ точно не попадает под эти константы.

Ну а если $R0=2$, в \Reg{0} будет загружен указатель на строку \emph{<<two\textbackslash{}n>>} при помощи инструкции \ADREQ.

\subsubsection{ARM: \OptimizingKeilVI (\ThumbMode)}

\lstinputlisting[style=customasmARM]{patterns/08_switch/1_few/few_ARM_thumb_O3.asm}

% FIXME а каким можно? к каким нельзя? \myref{} ->
Как уже было отмечено, в Thumb-режиме нет возможности добавлять условные предикаты к большинству инструкций,
так что Thumb-код вышел похожим на код x86 в стиле \ac{CISC}, вполне понятный.

\subsubsection{ARM64: \NonOptimizing GCC (Linaro) 4.9}

\lstinputlisting[style=customasmARM]{patterns/08_switch/1_few/ARM64_GCC_O0_RU.lst}

Входное значение имеет тип \Tint, поэтому для него используется регистр \RegW{0},
а не целая часть регистра \RegX{0}.

Указатели на строки передаются в \puts при помощи пары инструкций ADRP/ADD, как было показано в примере
\q{\HelloWorldSectionName}:~\myref{pointers_ADRP_and_ADD}.

\subsubsection{ARM64: \Optimizing GCC (Linaro) 4.9}

\lstinputlisting[style=customasmARM]{patterns/08_switch/1_few/ARM64_GCC_O3_RU.lst}

Фрагмент кода более оптимизированный.
Инструкция \TT{CBZ} (\emph{Compare and Branch on Zero}~--- сравнить и перейти если ноль) совершает переход если \RegW{0} ноль.
Здесь также прямой переход на \puts вместо вызова, как уже было описано:~\myref{JMP_instead_of_RET}.

}
\DE{\subsubsection{ARM}

\myparagraph{\OptimizingKeilVI (\ThumbMode)}

\lstinputlisting[style=customasmARM]{patterns/04_scanf/1_simple/ARM_IDA.lst}

\myindex{\CLanguageElements!\Pointers}
Damit \scanf Elemente einlesen kann, benötigt die Funktion einen Paramter--einen Pointer vom Typ \Tint.
\Tint hat die Größe 32 Bit, wir benötigen also 4 Byte, um den Wert im Speicher abzulegen, und passt daher genau in ein 32-Bit-Register.
\myindex{IDA!var\_?}
Auf dem Stack wird Platz für die lokalen Variable \GTT{x} reserviert und IDA bezeichnet diese Variable mit \emph{var\_8}. 
Eigentlich ist aber an dieser Stelle gar nicht notwendig, Platz auf dem Stack zu reservieren, da \ac{SP} (\gls{stack pointer} 
bereits auf die Adresse zeigt und auch direkt verwendet werden kann.

Der Wert von \ac{SP} wird also in das \Reg{1} Register kopiert und zusammen mit dem Formatierungsstring an \scanf übergeben.

% TBT here
%\INS{PUSH/POP} instructions behaves differently in ARM than in x86 (it's the other way around).
%They are synonyms to \INS{STM/STMDB/LDM/LDMIA} instructions.
%And \INS{PUSH} instruction first writes a value into the stack, \emph{and then} subtracts \ac{SP} by 4.
%\INS{POP} first adds 4 to \ac{SP}, \emph{and then} reads a value from the stack.
%Hence, after \INS{PUSH}, \ac{SP} points to an unused space in stack.
%It is used by \scanf, and by \printf after.

%\INS{LDMIA} means \emph{Load Multiple Registers Increment address After each transfer}.
%\INS{STMDB} means \emph{Store Multiple Registers Decrement address Before each transfer}.

\myindex{ARM!\Instructions!LDR}
Später wird mithilfe des \INS{LDR} Befehls dieser Wert vom Stack in das \Reg{1} Register verschoben um an \printf übergeben werden zu können.

\myparagraph{ARM64}

\lstinputlisting[caption=\NonOptimizing GCC 4.9.1 ARM64,numbers=left,style=customasmARM]{patterns/04_scanf/1_simple/ARM64_GCC491_O0_DE.s}

Im Stack Frame werden 32 Byte reserviert, was deutlich mehr als benötigt ist. Vielleicht handelt es sich um eine Frage des Aligning (dt. Angleichens) von Speicheradressen.
Der interessanteste Teil ist, im Stack Frame einen Platz für die Variable $x$ zu finden (Zeile 22).
Warum 28? Irgendwie hat der Compiler entschieden die Variable am Ende des Stack Frames anstatt an dessen Beginn abzulegen.
Die Adresse wird an \scanf übergeben; diese Funktion speichert den Userinput an der genannten Adresse im Speicher.
Es handelt sich hier um einen 32-Bit-Wert vom Typ \Tint. 
Der Wert wird in Zeile 27 abgeholt und dann an \printf übergeben.


}
\FR{\subsubsection{ARM: \OptimizingKeilVI (\ARMMode)}
\myindex{\CLanguageElements!switch}

\lstinputlisting[style=customasmARM]{patterns/08_switch/1_few/few_ARM_ARM_O3.asm}

A nouveau, en investiguant ce code, nous ne pouvons pas dire si il y avait un switch()
dans le code source d'origine ou juste un ensemble de déclarations if().

\myindex{ARM!\Instructions!ADRcc}

En tout cas, nous voyons ici des instructions conditionnelles (comme \ADREQ (\emph{Equal}))
qui ne sont exécutées que si $R0=0$, et qui chargent ensuite l'adresse de la chaîne
\emph{<<zero\textbackslash{}n>>} dans \Reg{0}.
\myindex{ARM!\Instructions!BEQ}
L'instruction suivante \ac{BEQ} redirige le flux d'exécution en \TT{loc\_170}, si $R0=0$.

Le lecteur attentif peut se demander si \ac{BEQ} s'exécute correctement puisque \ADREQ
a déjà mis une autre valeur dans le registre \Reg{0}.

Oui, elle s'exécutera correctement, car \ac{BEQ} vérifie les flags mis par l'instruction
\CMP et \ADREQ ne modifie aucun flag.

Les instructions restantes nous sont déjà familières.
Il y a seulement un appel à \printf, à la fin, et nous avons déjà examiné cette
astuce ici~(\myref{ARM_B_to_printf}).
A la fin, il y a trois chemins vers \printf{}.

\myindex{ARM!\Instructions!ADRcc}
\myindex{ARM!\Instructions!CMP}
La dernière instruction, \TT{CMP R0, \#2}, est nécessaire pour vérifier si $a=2$.

Si ce n'est pas vrai, alors \ADRNE charge un pointeur sur la chaîne \emph{<<something unknown \textbackslash{}n>>}
dans \Reg{0}, puisque $a$ a déjà été comparée pour savoir s'elle est égale
à 0 ou 1, et nous sommes sûrs que la variable $a$ n'est pas égale à l'un de
ces nombres, à ce point.
Et si $R0=2$, un pointeur sur la chaîne \emph{<<two\textbackslash{}n>>} sera chargé
par \ADREQ dans \Reg{0}.

\subsubsection{ARM: \OptimizingKeilVI (\ThumbMode)}

\lstinputlisting[style=customasmARM]{patterns/08_switch/1_few/few_ARM_thumb_O3.asm}

% FIXME а каким можно? к каким нельзя? \myref{} ->

Comme il y déjà été dit, il n'est pas possible d'ajouter un prédicat conditionnel
à la plupart des instructions en mode Thumb, donc ce dernier est quelque peu similaire
au code \ac{CISC}-style x86, facilement compréhensible.

\subsubsection{ARM64: GCC (Linaro) 4.9 \NonOptimizing}

\lstinputlisting[style=customasmARM]{patterns/08_switch/1_few/ARM64_GCC_O0_FR.lst}

Le type de la valeur d'entrée est \Tint, par conséquent le registre \RegW{0} est
utilisé pour garder la valeur au lieu du registre complet \RegX{0}.

Les pointeurs de chaîne sont passés à \puts en utilisant la paire d'instructions
\INS{ADRP}/\INS{ADD} comme expliqué dans l'exemple \q{\HelloWorldSectionName}:~\myref{pointers_ADRP_and_ADD}.

\subsubsection{ARM64: GCC (Linaro) 4.9 \Optimizing}

\lstinputlisting[style=customasmARM]{patterns/08_switch/1_few/ARM64_GCC_O3_FR.lst}

Ce morceau de code est mieux optimisé.
L'instruction \TT{CBZ} (\emph{Compare and Branch on Zero} comparer et sauter si zéro)
effectue un saut si \RegW{0} vaut zéro.
Il y a alors un saut direct à \puts au lieu de l'appeler, comme cela a été expliqué
avant:~\myref{JMP_instead_of_RET}.
}
\JPN{\subsubsection{ARM}

\myparagraph{\NonOptimizingKeilVI (\ARMMode)}

\lstinputlisting[style=customasmARM]{patterns/13_arrays/1_simple/simple_Keil_ARM_O0_JPN.asm}

\Tint 型は32ビットのストレージを必要とします(または4バイト)。

20個の \Tint 変数を保存するには80バイト(\TT{0x50})が必要です。
だから、\INS{SUB SP, SP, \#0x50}のようになっています。

関数プロローグの命令はスタックにちょうどその分の空間を確保しています。

最初と次のループの両方で、ループイテレータ \var{i} は\Reg{4}レジスタに置かれています。

\myindex{ARM!Optional operators!LSL}

配列に書かれる数は $i*2$ として計算されます。これは1ビット左シフトすることと同じで、
\INS{MOV R0, R4,LSL\#1}命令がこれをしています。

\myindex{ARM!\Instructions!STR}
\INS{STR R0, [SP,R4,LSL\#2]}は\Reg{0}の内容を配列に書き込んでいます。

配列の要素へのポインタがどのように計算されるかを示しています。\ac{SP}は配列の先頭を示しています。\Reg{4}は $i$ です。

$i$ を2ビット左シフトすると、4倍することに等しいです。
(各配列の要素は4バイトです)そして配列の先頭アドレスに追加されます。

\myindex{ARM!\Instructions!LDR}

次のループは\INS{LDR R2, [SP,R4,LSL\#2]}命令の逆です。
配列から必要とする値をロードし、ポインタもまた同様に計算されます。

\myparagraph{\OptimizingKeilVI (\ThumbMode)}

\lstinputlisting[style=customasmARM]{patterns/13_arrays/1_simple/simple_Keil_thumb_O3_JPN.asm}

Thumbコードも大変似ています。
\myindex{ARM!\Instructions!LSLS}

Thumbコードはビットシフト用の特別な命令を持っています(\TT{LSLS}のような)。
これは配列に書き込まれる値を計算し、また配列の各要素のアドレスも同様に計算します。

コンパイラはもう少し余分な空間をローカルスタックに確保します。しかし、最後の4バイトは使用されません。

\myparagraph{\NonOptimizing GCC 4.9.1 (ARM64)}

\lstinputlisting[caption=\NonOptimizing GCC 4.9.1 (ARM64),style=customasmARM]{patterns/13_arrays/1_simple/ARM64_GCC491_O0_JPN.s}

}

\EN{\subsubsection{MIPS}
% FIXME better start at non-optimizing version?

The function uses a lot of S- registers which must be preserved, so that's why its 
values are saved in the function prologue and restored in the epilogue.

\lstinputlisting[caption=\Optimizing GCC 4.4.5 (IDA),style=customasmMIPS]{patterns/13_arrays/1_simple/MIPS_O3_IDA_EN.lst}

Something interesting: there are two loops and the first one doesn't need $i$, it needs only 
$i*2$ (increased by 2 at each iteration) and also the address in memory (increased by 4 at each iteration).

So here we see two variables, one (in \$V0) increasing by 2 each time, and another (in \$V1) --- by 4.

The second loop is where \printf is called and it reports the value of $i$ to the user, 
so there is a variable
which is increased by 1 each time (in \$S0) and also a memory address (in \$S1) increased by 4 each time.

That reminds us of loop optimizations: \myref{loop_iterators}.

Their goal is to get rid of multiplications.

}
\RU{\subsubsection{MIPS}
% FIXME better start at non-optimizing version?
Функция использует много S-регистров, которые должны быть сохранены. Вот почему их значения сохраняются
в прологе функции и восстанавливаются в эпилоге.

\lstinputlisting[caption=\Optimizing GCC 4.4.5 (IDA),style=customasmMIPS]{patterns/13_arrays/1_simple/MIPS_O3_IDA_RU.lst}

Интересная вещь: здесь два цикла и в первом не нужна переменная $i$, а нужна только переменная
$i*2$ (скачущая через 2 на каждой итерации) и ещё адрес в памяти (скачущий через 4 на каждой итерации).

Так что мы видим здесь две переменных: одна (в \$V0) увеличивается на 2 каждый раз, и вторая (в \$V1) --- на 4.

Второй цикл содержит вызов \printf. Он должен показывать значение $i$ пользователю,
поэтому здесь есть переменная, увеличивающаяся на 1 каждый раз (в \$S0), а также адрес в памяти (в \$S1) 
увеличивающийся на 4 каждый раз.

Это напоминает нам оптимизацию циклов: \myref{loop_iterators}.
Цель оптимизации в том, чтобы избавиться от операций умножения.

}
\DE{\subsubsection{MIPS}
% FIXME better start at non-optimizing version?
Die Funktion verwendet eine Menge S-Register, die gesichert werden müssen. Das ist der Grund dafür, dass die Werte im
Funktionsprolog gespeichert und im Funktionsepilog wiederhergestellt werden.

\lstinputlisting[caption=\Optimizing GCC 4.4.5
(IDA),style=customasmMIPS]{patterns/13_arrays/1_simple/MIPS_O3_IDA_DE.lst}
Interessant: es gibt zwei Schleifen und die erste benötigt $i$ nicht; sie benötigt nur $i\cdot 2$ (erhöht um 2 bei
jedem Iterationsschritt) und die Adresse im Speicher (erhöht um 4 bei jedem Iterationsschritt).

Wir sehen hier also zwei Variablen: eine (in \$V0), die jedes Mal um 2 erhöht wird, und eine andere (in\$V1), die um 4
erhöht wird.

Die zweite Schleife ist der Ort, an dem \printf aufgerufen wird und dem Benutzer den Wert von $i$ zurückliefert, es gibt
also eine Variable die in \$S0 inkrementiert wird und eine Speicheradresse in \$S1, die jedes Mal um 4 erhöht wird.

% TBT
Das erinnert uns an die Optimierung von Schleifen, die wir früher betrachtet haben: \myref{loop_iterators}.

Das Ziel der Optimierung ist es, die Multiplikationen loszuwerden.
}
\FR{\subsubsection{MIPS}

\lstinputlisting[caption=\Optimizing GCC 4.4.5 (IDA),style=customasmMIPS]{patterns/10_strings/1_strlen/MIPS_O3_IDA_FR.lst}

\myindex{MIPS!\Instructions!NOR}
\myindex{MIPS!\Pseudoinstructions!NOT}

Il manque en MIPS une instruction \NOT, mais il y a \NOR qui correspond à l'opération
\TT{OR~+~NOT}.

Cette opération est largement utilisée en électronique digitale\footnote{NOR est
appelé \q{porte universelle}}.
Par exemple, l'Apollo Guidance Computer (ordinateur de guidage Apollo) utilisé dans
le programme Apollo, a été construit en utilisant seulement 5600 portes NOR:
[Jens Eickhoff, \emph{Onboard Computers, Onboard Software and Satellite Operations: An Introduction}, (2011)].
Mais l'élément NOT n'est pas très populaire en programmation informatique.

Donc, l'opération NOT est implémentée ici avec \TT{NOR~DST,~\$ZERO,~SRC}.

D'après le chapitre sur les fondamentaux \myref{sec:signednumbers} nous savons qu'une
inversion des bits d'un nombre signé est la même chose que changer son signe et soustraire
1 du résultat.

Donc ce que \NOT fait ici est de prendre la valeur de $str$ et de la transformer
en $-str-1$.
L'opération d'addition qui suit prépare le résultat.
}
\JPN{\subsubsection{MIPS}
% FIXME better start at non-optimizing version?

関数は保存しなくてはならないたくさんの S- レジスタを使用します。よって、
値は関数プロローグで保存され、エピローグでリストアされます。

\lstinputlisting[caption=\Optimizing GCC 4.4.5 (IDA),style=customasmMIPS]{patterns/13_arrays/1_simple/MIPS_O3_IDA_JPN.lst}

面白いこと:2つのループがあり、最初のループは $i$ がいりません。$i*2$が必要なだけです
(各イテレーションで2をインクリメントする)。それとメモリ上のアドレスが必要です(各イテレーションで4を増やす)。

だから、2つの変数を確認します。1つは(\$V0)毎回2を増やし、もう1つは4増やします(\$V1)。

次のループは \printf が呼び出されるところです。 $i$ の値をユーザに報告します。
毎回1増やす変数があり(\$S0)、そしてメモリアドレス(\$S1)も毎回4増えます。

前に検討したループ最適化を私たちに思いださせます:\myref{loop_iterators}

目的は乗算を取り除くことです。
}


\subsection{\RU{Переполнение буфера}\EN{Buffer overflow}\DE{Puffer-Überlauf}\FR{Débordement de tampon}\JA{バッファオーバーフロー}}
\label{subsec:bufferoverflow}
\myindex{\BufferOverflow}

\EN{\subsubsection{Reading outside array bounds}

So, array indexing is just \emph{array\lbrack{}index\rbrack}.
If you study the generated code closely, you'll probably note the missing index bounds checking,
which could check \emph{if it is less than 20}.
What if the index is 20 or greater?
That's the one \CCpp feature it is often blamed for.

Here is a code that successfully compiles and works:

\lstinputlisting[style=customc]{patterns/13_arrays/2_BO/r.c}

Compilation results (MSVC 2008):

\lstinputlisting[caption=\NonOptimizing MSVC 2008,style=customasmx86]{patterns/13_arrays/2_BO/r_msvc.asm}

The code produced this result:

\lstinputlisting[caption=\olly: console output]{patterns/13_arrays/2_BO/console.txt}

It is just \emph{something} that has been lying in the stack near to the array, 80 bytes away from its first element.

\clearpage
\myindex{\olly}
Let's try to find out where did this value come from, using \olly.

Let's load and find the value located right after the last array element:

\begin{figure}[H]
\centering
\myincludegraphics{patterns/13_arrays/2_BO/olly_r1.png}
\caption{\olly: reading of the 20th element and execution of \printf}
\label{fig:array_BO_olly_r1}
\end{figure}

What is this? 
Judging by the stack layout,
this is the saved value of the EBP register.
\clearpage
Let's trace further and see how it gets restored:

\begin{figure}[H]
\centering
\myincludegraphics{patterns/13_arrays/2_BO/olly_r2.png}
\caption{\olly: restoring value of EBP}
\label{fig:array_BO_olly_r2}
\end{figure}

Indeed, how it could be different?
The compiler may generate some additional code to check the index value to be always
in the array's bounds (like in higher-level programming languages\footnote{Java, Python, etc.})
but this makes the code slower.

}
\RU{\subsubsection{Чтение за пределами массива}

Итак, индексация массива --- это просто \emph{массив\lbrack{}индекс\rbrack}.  % TODO1 как-то плохо отображаются []
Если вы присмотритесь к коду, в цикле печати значений массива через \printf вы 
не увидите проверок индекса, \emph{меньше ли он двадцати?} 
А что будет если он будет 20 или больше? 
Эта одна из особенностей \CCpp, за которую их, собственно, и ругают.

Вот код, который и компилируется и работает:

\lstinputlisting[style=customc]{patterns/13_arrays/2_BO/r.c}

Вот результат компиляции в (MSVC 2008):

\lstinputlisting[caption=\NonOptimizing MSVC 2008,style=customasmx86]{patterns/13_arrays/2_BO/r_msvc.asm}

Данный код при запуске выдал вот такой результат:

\lstinputlisting[caption=\olly: вывод в консоль]{patterns/13_arrays/2_BO/console.txt}

Это просто \emph{что-то}, что волею случая лежало в стеке рядом с массивом, 
через 80 байт от его первого элемента.

\clearpage
\myindex{\olly}
Попробуем узнать в \olly, что это за значение.
Загружаем и находим это значение, находящееся точно после последнего элемента массива:

\begin{figure}[H]
\centering
\myincludegraphics{patterns/13_arrays/2_BO/olly_r1.png}
\caption{\olly: чтение 20-го элемента и вызов \printf}
\label{fig:array_BO_olly_r1}
\end{figure}

Что это за значение? 
Судя по разметке стека, это сохраненное значение регистра EBP.
\clearpage
Трассируем далее, и видим, как оно восстанавливается:

\begin{figure}[H]
\centering
\myincludegraphics{patterns/13_arrays/2_BO/olly_r2.png}
\caption{\olly: восстановление EBP}
\label{fig:array_BO_olly_r2}
\end{figure}

Действительно, а как могло бы быть иначе? Компилятор мог бы встроить какой-то код, 
каждый раз проверяющий индекс на соответствие пределам массива, как в языках программирования 
более высокого уровня\footnote{Java, Python, итд.}, что делало бы запускаемый код медленнее.

}
\DE{\subsubsection{Lesezugriff außerhalb von Arraygrenzen}
Der indizierte Zugriff auf ein Array wird durch \emph{array\lbrack{}index\rbrack} realisiert.
Wenn man sich den erzeugten Code genau ansieht, bemerkt man, dass eine Prüfung der Indexgrenzen fehlt, welche die
Bedingung \emph{kleiner als 20} validiert.
Was also passiert, wenn der Index 20 oder größer ist? 
Hier haben wir es mit einem unschönen Feature von \CCpp zu tun

Hier ein Beipsielcode der erfolgreich kompiliert wurde und funktioniert:

\lstinputlisting[style=customc]{patterns/13_arrays/2_BO/r.c}

Ergebnis des Kompiliervorgangs (MSVC 2008):

\lstinputlisting[caption=\NonOptimizing MSVC 2008,style=customasmx86]{patterns/13_arrays/2_BO/r_msvc.asm}

Der Code produziert dieses Ergebnis:

\lstinputlisting[caption=\olly: console output]{patterns/13_arrays/2_BO/console.txt}
Es handelt sich um \emph{irgendetwas}, das auf dem Stack in der Nähe des Arrays gelegen hat, 80 Byte von dessen erstem
Element entfernt.

\clearpage
\myindex{\olly}
Versuchen wir mit \olly herauszufinden, woher dieser Wert kommt.

Laden und finden wir also den Wert, der sich direkt hinter dem letzten Arrayelement befindet:

\begin{figure}[H]
\centering
\myincludegraphics{patterns/13_arrays/2_BO/olly_r1.png}
\caption{\olly: das 20. Element lesen und \printf ausführen}
\label{fig:array_BO_olly_r1}
\end{figure}

Worum handelt es sich? 
Dem Stacklayout nach zu urteilen ist dies der gespeicherte Wert des EBP Registers.
\clearpage
Verfolgen wir das ganze weiter und schauen uns an, wie dieser wiederhergestellt wird:

\begin{figure}[H]
\centering
\myincludegraphics{patterns/13_arrays/2_BO/olly_r2.png}
\caption{\olly: Wert von EBP wiederherstellen}
\label{fig:array_BO_olly_r2}
\end{figure}
Wie könnte es anders gelöst werden?
Der Compiler könnte zusätzlichen Code erzeugen, der sicherstellt, dass der Index sich stets innerhalb der Arraygrenzen
befindet (wie in höheren Programmiersprachen\footnote{Java, Python, etc.}), aber das würde den Code langsamer machen.
}
\FR{\subsubsection{Lire en dehors des bornes du tableau}

Donc, indexer un tableau est juste \emph{array\lbrack{}index\rbrack}.
Si vous étudiez le code généré avec soin, vous remarquerez sans doute l'absence de
test sur les bornes de l'index, qui devrait vérifier \emph{si il est inférieur à 20}.
Que ce passe-t-il si l'index est supérieur à 20?
C'est une des caractéristiques de \CCpp qui est souvent critiquée.

Voici un code qui compile et fonctionne:

\lstinputlisting[style=customc]{patterns/13_arrays/2_BO/r.c}

Résultat de la compilation (MSVC 2008):

\lstinputlisting[caption=MSVC 2008 \NonOptimizing,style=customasmx86]{patterns/13_arrays/2_BO/r_msvc.asm}

Le code produit ce résultat:

\lstinputlisting[caption=\olly: sortie sur la console]{patterns/13_arrays/2_BO/console.txt}

C'est juste \emph{quelque chose} qui se trouvait sur la pile à côté du tableau, 80 octets
après le début de son premier élément.

\clearpage
\myindex{\olly}
Essayons de trouver d'où vient cette valeur, en utilisant \olly.

Chargeons et trouvons la valeur située juste après le dernier élément du tableau:

\begin{figure}[H]
\centering
\myincludegraphics{patterns/13_arrays/2_BO/olly_r1.png}
\caption{\olly: lecture du 20ème élément et exécution de \printf}
\label{fig:array_BO_olly_r1}
\end{figure}

Qu'est-ce que c'est?
D'après le schéma de la pile, c'est la valeur sauvegardée du registre EBP.
\clearpage
Exécutons encore et voyons comment il est restauré:

\begin{figure}[H]
\centering
\myincludegraphics{patterns/13_arrays/2_BO/olly_r2.png}
\caption{\olly: restaurer la valeur de EBP}
\label{fig:array_BO_olly_r2}
\end{figure}

En effet, comment est-ce ça pourrait être différent?
Le compilateur pourrait générer du code supplémentaire pour vérifier que la valeur
de l'index est toujours entre les bornes du tableau (comme dans les langages de
programmation de plus haut-niveau\footnote{Java, Python, etc.}) mais cela rendrait
le code plus lent.

}
\JA{\subsubsection{配列の範囲外の読み込み}

配列のインデックス化は単に\emph{array\lbrack{}index\rbrack}です。
生成されたコードを詳しく研究したなら、\emph{20未満であるか}チェックするような
インデックスの境界チェックがないことに気づくでしょう。
もしインデックスが20以上だったらどうでしょうか。
これは \CCpp が批判される1つの特徴です。

コンパイルされて動作するコードがあります。

\lstinputlisting[style=customc]{patterns/13_arrays/2_BO/r.c}

コンパイル結果(MSVC 2008)

\lstinputlisting[caption=\NonOptimizing MSVC 2008,style=customasmx86]{patterns/13_arrays/2_BO/r_msvc.asm}

コードは次の結果を生成します。

\lstinputlisting[caption=\olly: console output]{patterns/13_arrays/2_BO/console.txt}

これは単に配列のそばのスタックにある \emph{何か} です。配列の最初の要素から80バイト離れています。

\clearpage
\myindex{\olly}
この値がどこから来るのか \olly を使って見つけてみましょう。

最後の配列の要素のすぐあとに配置された値をロードして見つけましょう。

\begin{figure}[H]
\centering
\myincludegraphics{patterns/13_arrays/2_BO/olly_r1.png}
\caption{\olly: 20番目の要素を読み込み、 \printf を実行する}
\label{fig:array_BO_olly_r1}
\end{figure}

これは何でしょうか?
スタックレイアウトで判断すると、
これは保存されたEBPレジスタの値です。
\clearpage
もっとトレースしてどのようにリストアされるか見てみましょう。

\begin{figure}[H]
\centering
\myincludegraphics{patterns/13_arrays/2_BO/olly_r2.png}
\caption{\olly: EBPの値をリストア}
\label{fig:array_BO_olly_r2}
\end{figure}

本当に、異なっていますか?
コンパイラはインデックス値が配列の境界内かを常にチェックする追加のコードを生成するかもしれません。
(高水準プログラミング言語\footnote{Java, Pythonなど}のように)
しかし、これはコードを遅くします。
}

\EN{\subsubsection{Writing beyond array bounds}

OK, we read some values from the stack \emph{illegally}, but what if we could write something to it?

Here is what we have got:

\lstinputlisting[style=customc]{patterns/13_arrays/2_BO/w.c}

\myparagraph{MSVC}

And what we get:

\lstinputlisting[caption=\NonOptimizing MSVC 2008,style=customasmx86]{patterns/13_arrays/2_BO/w_EN.asm}

The compiled program crashes after running. No wonder. Let's see where exactly does it crash.

\clearpage
\myindex{\olly}

Let's load it into \olly, and trace until all 30 elements are written:

\begin{figure}[H]
\centering
\myincludegraphics{patterns/13_arrays/2_BO/olly_w1.png}
\caption{\olly: after restoring the value of EBP}
\label{fig:array_BO_olly_w1}
\end{figure}

\clearpage
Trace until the function end:

\begin{figure}[H]
\centering
\myincludegraphics{patterns/13_arrays/2_BO/olly_w2.png}
\caption{\olly: 
\TT{EIP} has been restored, but \olly can't disassemble at 0x15}
\label{fig:array_BO_olly_w2}
\end{figure}

Now please keep your eyes on the registers.

\EIP is 0x15 now. It is not a legal address for code---at least for win32 code!
We got there somehow against our will.
It is also interesting that the \EBP register contain 0x14,
\ECX and \EDX contain 0x1D.

Let's study stack layout a bit more.

After the control flow has been passed to \TT{\main}, the value in the \EBP register was saved on the stack.
Then, 84 bytes were allocated for the array and the $i$ variable.
That's \TT{(20+1)*sizeof(int)}.
\ESP now points to the \TT{\_i} variable in the local stack and after the execution of 
the next \TT{PUSH something}, \emph{something} is appearing next to \TT{\_i}.

That's the stack layout while the control is in \main:

\begin{center}
\begin{tabular}{ | l | l | }
\hline
  \TT{ESP}    & 4 bytes allocated for $i$ variable \\
\hline
  \TT{ESP+4}  & 80 bytes allocated for \TT{a[20]} array \\
\hline
  \TT{ESP+84} & saved \EBP value \\
\hline
  \TT{ESP+88} & return address \\
\hline
\end{tabular}
\end{center}

\TT{a[19]=something} statement writes the last \Tint in the bounds of the array (in bounds so far!).

\TT{a[20]=something} statement writes \emph{something} to the place where the value of \EBP is saved.

Please take a look at the register state at the moment of the crash. In our case,
20 has been written in the 20th element. 
At the function end, the function epilogue restores the original \EBP value.
(20 in decimal is \TT{0x14} in hexadecimal).
Then \RET gets executed, which is effectively equivalent to \TT{POP EIP} instruction.

The \RET instruction takes the return address from the stack (that is the address in \ac{CRT},
which has called \main),
and 21 is stored there (\TT{0x15} in hexadecimal).
The CPU traps at address \TT{0x15},
but there is no executable code there, so exception gets raised.

\myindex{\BufferOverflow}

Welcome! It is called a \emph{buffer overflow}\footnote{\href{http://go.yurichev.com/17132}{wikipedia}}.

Replace the \Tint array with a string (\Tchar array), create a long string deliberately
and pass it to the program, to the function, which doesn't check the length of the string and copies it in a short buffer,
and you'll able to point the program to an address to which it must jump.
It's not that simple in reality, but that is how it emerged.
Classic article about it: \AlephOne.

\myparagraph{GCC}

Let's try the same code in GCC 4.4.1. We get:

\lstinputlisting[style=customasmx86]{patterns/13_arrays/2_BO/w_gcc.asm}

Running this in Linux will produce: \TT{Segmentation fault}.

\myindex{GDB}

If we run this in the GDB debugger, we get this:

\begin{lstlisting}
(gdb) r
Starting program: /home/dennis/RE/1 

Program received signal SIGSEGV, Segmentation fault.
0x00000016 in ?? ()
(gdb) info registers
eax            0x0	0
ecx            0xd2f96388	-755407992
edx            0x1d	29
ebx            0x26eff4	2551796
esp            0xbffff4b0	0xbffff4b0
ebp            0x15	0x15
esi            0x0	0
edi            0x0	0
eip            0x16	0x16
eflags         0x10202	[ IF RF ]
cs             0x73	115
ss             0x7b	123
ds             0x7b	123
es             0x7b	123
fs             0x0	0
gs             0x33	51
(gdb) 
\end{lstlisting}

The register values are slightly different than in win32 example, 
since the stack layout is slightly different too.

}
\RU{\subsubsection{Запись за пределы массива}

Итак, мы прочитали какое-то число из стека явно \emph{нелегально}, а что если мы запишем?

Вот что мы пишем:

\lstinputlisting[style=customc]{patterns/13_arrays/2_BO/w.c}

\myparagraph{MSVC}

И вот что имеем на ассемблере:

\lstinputlisting[caption=\NonOptimizing MSVC 2008,style=customasmx86]{patterns/13_arrays/2_BO/w_RU.asm}

Запускаете скомпилированную программу, и она падает. Немудрено. Но давайте теперь узнаем, где именно.

\clearpage
\myindex{\olly}

Загружаем в \olly, трассируем пока запишутся все 30 элементов:

\begin{figure}[H]
\centering
\myincludegraphics{patterns/13_arrays/2_BO/olly_w1.png}
\caption{\olly: после восстановления EBP}
\label{fig:array_BO_olly_w1}
\end{figure}

\clearpage
Доходим до конца функции:

\begin{figure}[H]
\centering
\myincludegraphics{patterns/13_arrays/2_BO/olly_w2.png}
\caption{\olly: EIP восстановлен, но \olly не может дизассемблировать по адресу 0x15}
\label{fig:array_BO_olly_w2}
\end{figure}

Итак, следите внимательно за регистрами.

\EIP теперь 0x15. Это явно нелегальный адрес для кода~--- по крайней мере, win32-кода! 
Мы там как-то очутились, причем, сами того не хотели. Интересен также тот факт, что в \EBP хранится 0x14, 
а в \ECX и \EDX хранится 0x1D.

Ещё немного изучим разметку стека.

После того как управление передалось в \main, в стек было сохранено значение \EBP. 
Затем для массива и переменной $i$ было выделено 84 байта. Это \TT{(20+1)*sizeof(int)}. 
\ESP сейчас указывает на переменную \TT{\_i} в локальном стеке и при исполнении следующего \INS{PUSH что-либо}, 
\emph{что-либо} появится рядом с \TT{\_i}.

Вот так выглядит разметка стека пока управление находится внутри \main:

\begin{center}
\begin{tabular}{ | l | l | }
\hline
  \TT{ESP}    & 4 байта выделенных для переменной $i$ \\
\hline
  \TT{ESP+4}  & 80 байт выделенных для массива \TT{a[20]} \\
\hline
  \TT{ESP+84} & сохраненное значение \EBP \\
\hline
  \TT{ESP+88} & адрес возврата \\
\hline
\end{tabular}
\end{center}

Выражение \TT{a[19]=что\_нибудь} записывает последний \Tint в пределах массива (пока что в пределах!).

Выражение \TT{a[20]=что\_нибудь} записывает \emph{что\_нибудь} на место где сохранено значение \EBP.

Обратите внимание на состояние регистров на момент падения процесса. В нашем случае 
в 20-й элемент записалось значение 20. 
И вот всё дело в том, что заканчиваясь, эпилог функции восстанавливал значение \EBP 
(20 в десятичной системе это как раз \TT{0x14} в шестнадцатеричной). 
Далее выполнилась инструкция \RET, которая на самом деле эквивалентна \TT{POP EIP}.

Инструкция \RET вытащила из стека адрес возврата (это адрес где-то внутри \ac{CRT}, которая вызвала \main),
а там было записано 21 в десятичной системе, то есть 0x15 в шестнадцатеричной. 
И вот процессор оказался по адресу 0x15, но исполняемого кода там нет, так что случилось исключение.

\myindex{\BufferOverflow}
Добро пожаловать! Это называется \emph{buffer overflow}\footnote{\href{http://go.yurichev.com/17132}{wikipedia}}.

Замените массив \Tint на строку (массив \Tchar), нарочно создайте слишком длинную строку, 
передайте её в ту программу, 
в ту функцию, которая не проверяя длину строки скопирует её в слишком короткий буфер, 
и вы сможете указать программе, по какому именно адресу перейти. 
Не всё так просто в реальности, конечно, но началось всё с этого.
Классическая статья об этом: \AlephOne.

\myparagraph{GCC}

Попробуем то же самое в GCC 4.4.1. У нас выходит такое:

\lstinputlisting[style=customasmx86]{patterns/13_arrays/2_BO/w_gcc.asm}

Запуск этого в Linux выдаст: \TT{Segmentation fault}.

\myindex{GDB}
Если запустить полученное в отладчике GDB, получим:

\begin{lstlisting}
(gdb) r
Starting program: /home/dennis/RE/1 

Program received signal SIGSEGV, Segmentation fault.
0x00000016 in ?? ()
(gdb) info registers
eax            0x0	0
ecx            0xd2f96388	-755407992
edx            0x1d	29
ebx            0x26eff4	2551796
esp            0xbffff4b0	0xbffff4b0
ebp            0x15	0x15
esi            0x0	0
edi            0x0	0
eip            0x16	0x16
eflags         0x10202	[ IF RF ]
cs             0x73	115
ss             0x7b	123
ds             0x7b	123
es             0x7b	123
fs             0x0	0
gs             0x33	51
(gdb) 
\end{lstlisting}

Значения регистров немного другие, чем в примере win32, потому что разметка стека чуть другая.

}
\DE{\subsubsection{Schreibzugriff außerhalb von Arraygrenzen}
Nehmen wir an, wir hätte ein paar Werte illegalerweise vom Stack gelesen, wie könnten wir etwas hineinschreiben?

Hier ist, was wir haben:

\lstinputlisting[style=customc]{patterns/13_arrays/2_BO/w.c}

\myparagraph{MSVC}

Wir erhalten das Folgende:

\lstinputlisting[caption=\NonOptimizing MSVC 2008,style=customasmx86]{patterns/13_arrays/2_BO/w_DE.asm}
Das kompilierte Programm stürzt nach der Ausführung ab. Das verwundert nicht. Schauen wir, was genau den Absturz
verursacht.

\clearpage
\myindex{\olly}
Laden wir das Programm in \olly und verfolgen den Ablauf, bis alle 30 Elemente geschrieben worden sind:

\begin{figure}[H]
\centering
\myincludegraphics{patterns/13_arrays/2_BO/olly_w1.png}
\caption{\olly: nach Wiederherstellung des Wertes von EBP}
\label{fig:array_BO_olly_w1}
\end{figure}

\clearpage
Nachverfolgen bis zum Ende der Funktion:

\begin{figure}[H]
\centering
\myincludegraphics{patterns/13_arrays/2_BO/olly_w2.png}
\caption{\olly: 
\TT{EIP} wurde wiederhergestellt, aber \olly kann an 0x15 nicht disassemblieren}
\label{fig:array_BO_olly_w2}
\end{figure}
Richten wir unser Augenmerk auf die Register.

\EIP ist jetzt gerade 0x15. Das ist keine gültige Adreses für Code---zumindest nicht für win32 Code!
Interessant ist auch, dass das \EBP Register 0x14 enthält und \ECX sowie \EDX jeweils 0x1D

Schauen wir uns das Stacklayout etwas genauer an.

Nachdem der Control Flow an \TT{\main} übergeben worde ist, wurde der Wert in \EBP auf dem Stack abgelegt.
Danach wurden 84 Byte für das Array und die Variable $i$ reserviert.
Das entspricht \TT{(20+1)*sizeof(int)}.
\ESP zeigt jetzt auf die Variable \TT{\_i} im lokalen Stack und nach der Ausführung von \TT{PUSH something} scheint sich
\TT{something} neben \TT{\_i} zu befinden.

Hier ist das Stacklayout während der Control Flow in der \main ist:

\begin{center}
\begin{tabular}{ | l | l | }
\hline
  \TT{ESP}    & 4 Byte reserviert für Variable $i$ \\
\hline
  \TT{ESP+4}  & 80 Byte reserviert für Array \TT{a[20]} \\
\hline
  \TT{ESP+84} & sichere Wert von \EBP \\
\hline
  \TT{ESP+88} & Rücksprungadresse \\
\hline
\end{tabular}
\end{center}
Der Befehl \TT{a[19]=something} schreibt den letzten \Tint innerhalb der Grenzen des Arrays (bis hierhin ist alles in
Ordnung!).
Der Befehl \TT{a[20]=something} schreibt \emph{something} an die Stelle, an der der \EBP gespeichert ist.

Sehen wir uns den Zustand der Register im Moment des Absturzes an. In unserem Fall wurde 20 in das zwanzigste Element
geschrieben. Am Ende der Funktion stellt der Funktionsepilog den originalen Wert von \EBP wieder her.
(20 dezimal entspricht \TT{0x14} hexadezimal).
Danach wird \RET ausgeführt, was äquivalent zum Befehl \TT{POP EIP} ist.

Der Befehl \RET nimmt die Rücksprungadresse vom Stack (das ist die Adresse in \ac{CRT}, die \main aufgerufen hat) und
speichert hier den Wert 21 (\TT{0x15} hexadezimal).
Die CPU springt an die Adresse \TT{0x15}, aber hier befindet sich kein ausführbarer Code, sodass eine Exception geworfen
wird.

\myindex{\BufferOverflow}
Dies nennt man einen \emph{Buffer Overflow}\footnote{\href{http://go.yurichev.com/17132}{wikipedia}}.

Ersetzt man das \Tint Array durch einen String (\Tchar Array) und erzeugt absichtlich einen langen String und übergibt
ihn im Programm an eine Funktion, die die Länge des Strings nicht prüft und ihn in einen kurzen Buffer kopiert, kann man
das Programm zwingen an eine bestimmte Adresse zu springen.
In der Realität ist dieses Verhalten nicht so einfach zu erzeugen, funktioniert aber von Prinzip her genau wie hier.
Ein klassischer Artikel dazu:\AlephOne.

\myparagraph{GCC}

Kompilieren wir denselben Code mit GCC 4.4.1, erhalten wir:

\lstinputlisting[style=customasmx86]{patterns/13_arrays/2_BO/w_gcc.asm}

Lässt man das Programm unter Linux laufen, lautet das Ergebnis: \TT{Segmentation fault}.

\myindex{GDB}
Wenn wir es mit dem GDB Debugger laufen lassen, erhalten wir das Folgende:


\begin{lstlisting}
(gdb) r
Starting program: /home/dennis/RE/1 

Program received signal SIGSEGV, Segmentation fault.
0x00000016 in ?? ()
(gdb) info registers
eax            0x0	0
ecx            0xd2f96388	-755407992
edx            0x1d	29
ebx            0x26eff4	2551796
esp            0xbffff4b0	0xbffff4b0
ebp            0x15	0x15
esi            0x0	0
edi            0x0	0
eip            0x16	0x16
eflags         0x10202	[ IF RF ]
cs             0x73	115
ss             0x7b	123
ds             0x7b	123
es             0x7b	123
fs             0x0	0
gs             0x33	51
(gdb) 
\end{lstlisting}
Die Registerwerte unterscheiden sich geringfügig vom win32 Beispiel, da auch das Stacklayout ein wenig anders ist.
}
\FR{\subsubsection{Écrire hors des bornes du tableau}

Ok, nous avons lu quelques valeurs de la pile \emph{illégalement}, mais que se passe-t-il
si nous essayons d'écrire quelque chose?

Voici ce que nous avons:

\lstinputlisting[style=customc]{patterns/13_arrays/2_BO/w.c}

\myparagraph{MSVC}

Et ce que nous obtenons:

\lstinputlisting[caption=MSVC 2008 \NonOptimizing,style=customasmx86]{patterns/13_arrays/2_BO/w_FR.asm}

Le programme compilé plante après le lancement. Pas de miracle. Voyons exactement
où il plante.

\clearpage
\myindex{\olly}

Chargeons le dans \olly, et traçons le jusqu'à ce que les 30 éléments du tableau
soient écrits:

\begin{figure}[H]
\centering
\myincludegraphics{patterns/13_arrays/2_BO/olly_w1.png}
\caption{\olly: après avoir restauré la valeur de EBP}
\label{fig:array_BO_olly_w1}
\end{figure}

\clearpage
Exécutons pas à pas jusqu'à la fin de la fonction:

\begin{figure}[H]
\centering
\myincludegraphics{patterns/13_arrays/2_BO/olly_w2.png}
\caption{\olly: 
\TT{EIP} a été restauré, mais \olly ne peut pas désassembler en 0x15}
\label{fig:array_BO_olly_w2}
\end{figure}

Maintenant, gardez vos yeux sur les registres.

\EIP contient maintenant 0x15. Ce n'est pas une adresse légale pour du code---au
moins pour du code win32!
Nous sommes arrivés ici contre notre volonté.
Il est aussi intéressant de voir que le registre \EBP contient 0x14, \ECX et \EDX
contiennent 0x1D.

Étudions un peu plus la structure de la pile.

Après que le contrôle du flux a été passé à \TT{\main}, la valeur du registre \EBP
a été sauvée sur la pile.
Puis, 84 octets ont été alloués pour le tableau et la variable $i$.
C'est \TT{(20+1)*sizeof(int)}.
\ESP pointe maintenant sur la variable \TT{\_i} dans la pile locale et après l'exécution
du \TT{PUSH quelquechose} suivant, \emph{quelquechose} apparaît à côté de \TT{\_i}.

C'est la structure de la pile pendant que le contrôle est dans \main:

\begin{center}
\begin{tabular}{ | l | l | }
\hline
  \TT{ESP}    & 4 octets alloués pour la variable $i$ \\
\hline
  \TT{ESP+4}  & 80 octets alloués pour le tableau \TT{a[20]} \\
\hline
  \TT{ESP+84} & valeur sauvegardée de \EBP \\
\hline
  \TT{ESP+88} & adresse de retour \\
\hline
\end{tabular}
\end{center}

L'expression \TT{a[19]=quelquechose} écrit le dernier \Tint dans des bornes du tableau
(dans les limites jusqu'ici!)

L'expression \TT{a[20]=quelquechose} écrit \emph{quelquechose} à l'endroit où la valeur
sauvegardée de \EBP se trouve.

S'il vous plaît, regardez l'état du registre lors du plantage. Dans notre cas,
20 a été écrit dans le 20ème élément.
À la fin de la fonction, l'épilogue restaure la valeur d'origine de \EBP.
(20 en décimal est \TT{0x14} en hexadécimal).
Ensuite \RET est exécuté, qui est équivalent à l'instruction \TT{POP EIP}.

L'instruction \RET prend la valeur de retour sur la pile (c'est l'adresse dans \ac{CRT}),
qui a appelé \main), et 21 est stocké ici (\TT{0x15} en hexadécimal).
% TODO: clarifier trap
Le CPU trape à l'adresse \TT{0x15}, mais il n'y a pas de code exécutable ici, donc
une exception est levée.

\myindex{\BufferOverflow}

Bienvenu! Ça s'appelle un \emph{buffer overflow (débordement de tampon)}\footnote{\href{http://go.yurichev.com/17132}{Wikipédia}}.

Remplacez la tableau de \Tint avec une chaîne (\Tchar array), créez délibérément
une longue chaîne et passez-là au programme, à la fonction, qui ne teste pas la longueur
de la chaîne et la copie dans un petit buffer et vous serez capable de faire pointer
le programme à une adresse où il devra sauter.
C'est pas aussi simple dans la réalité, mais c'est comme cela que ça a apparu.
L'article classique à propos de ça: \AlephOne.

\myparagraph{GCC}

Essayons le même code avec GCC 4.4.1. Nous obtenons:

\lstinputlisting[style=customasmx86]{patterns/13_arrays/2_BO/w_gcc.asm}

Lancer ce programme sous Linux donnera: \TT{Segmentation fault}.

\myindex{GDB}

Si nous le lançons dans le débogueur GDB, nous obtenons ceci:

\begin{lstlisting}
(gdb) r
Starting program: /home/dennis/RE/1

Program received signal SIGSEGV, Segmentation fault.
0x00000016 in ?? ()
(gdb) info registers
eax            0x0	0
ecx            0xd2f96388	-755407992
edx            0x1d	29
ebx            0x26eff4	2551796
esp            0xbffff4b0	0xbffff4b0
ebp            0x15	0x15
esi            0x0	0
edi            0x0	0
eip            0x16	0x16
eflags         0x10202	[ IF RF ]
cs             0x73	115
ss             0x7b	123
ds             0x7b	123
es             0x7b	123
fs             0x0	0
gs             0x33	51
(gdb)
\end{lstlisting}

Les valeurs des registres sont légèrement différentes de l'exemple win32, puisque
la structure de la pile est également légèrement différente.
}
\JA{\subsubsection{配列境界を越えて書きこむ}

私たちはスタックからいくつかの値を\emph{不正に}読んでいますが、何かを書くことができたらどうなるでしょうか?

こういう風になります。

\lstinputlisting[style=customc]{patterns/13_arrays/2_BO/w.c}

\myparagraph{MSVC}

そしてこうなります。

\lstinputlisting[caption=\NonOptimizing MSVC 2008,style=customasmx86]{patterns/13_arrays/2_BO/w_JA.asm}

コンパイルしたプログラムは起動後にクラッシュします。当然です。どこでクラッシュするか正確にみてみましょう。

\clearpage
\myindex{\olly}

\olly でロードし、30要素が書かれるまでトレースしてみましょう。

\begin{figure}[H]
\centering
\myincludegraphics{patterns/13_arrays/2_BO/olly_w1.png}
\caption{\olly: EBPの値をリストアした後}
\label{fig:array_BO_olly_w1}
\end{figure}

\clearpage
関数が終了するまでトレースします。

\begin{figure}[H]
\centering
\myincludegraphics{patterns/13_arrays/2_BO/olly_w2.png}
\caption{\olly: 
\TT{EIP} がリストアされるが、 \olly は0x15でディスアセンブルできない}
\label{fig:array_BO_olly_w2}
\end{figure}

レジスタをよく見てください。

\EIP は0x15です。コードでは不正なアドレスではありません。少なくともwin32のコードとしては!
我々の意志に反しています。
\EBP レジスタが0x14を、\ECX と \EDX が0x1Dを含んでいるということが面白いです。

スタックレイアウトをもう少し勉強しましょう。

制御フローが \TT{\main} を通ったあと、 \EBP レジスタの値はスタックに保存されます。
それから、84バイトが配列と $i$ 用に確保されます。
それは\TT{(20+1)*sizeof(int)}です。
\ESP は ローカルスタックの \TT{\_i} 変数を指し、次の\TT{PUSH something}の実行の
後で、\emph{何か}が次の\TT{\_i}に現れます。

これが、制御が \main にあるときのスタックレイアウトです。

\begin{center}
\begin{tabular}{ | l | l | }
\hline
  \TT{ESP}    & 4バイトが $i$ 変数に確保される \\
\hline
  \TT{ESP+4}  & 80バイトが \TT{a[20]} 配列に確保される \\
\hline
  \TT{ESP+84} & \EBP の値を保存 \\
\hline
  \TT{ESP+88} & リターンアドレス \\
\hline
\end{tabular}
\end{center}

\TT{a[19]=something} 文は配列の境界である最後の \Tint を書き込みます(今は境界内です!)。

\TT{a[20]=something} 文は \EBP の値が保存された場所に \emph{何か}を書き込みます。

クラッシュ時のレジスタの状態を見てください。我々の場合、
20番目の要素に20が書かれています。
関数の最後で、関数エピローグがオリジナルの \EBP 値をリストアします。
(10進数の20は16進数で\TT{0x14}です)。
そして、 \RET が実行されます。これは\TT{POP EIP}命令と同じ効果です。

\RET 命令はスタックからリターンアドレスを取って(これは\ac{CRT}の中のアドレスで、
\main を呼び出したアドレスです)、
21が保存されます(16進数で\TT{0x15})。
CPUはアドレス\TT{0x15}をトラップしますが、
実行可能なコードがここにないので、例外が発生します。

\myindex{\BufferOverflow}

ようこそ! \emph{バッファオーバーフロー} です。\footnote{\href{http://go.yurichev.com/17132}{wikipedia}}

\Tint 配列を文字列( \Tchar 配列)で置換するには、意図的に長い文字列を作成し、
それをプログラムに渡し、関数に渡し、文字列の長さをチェックせず、より短いバッファにコピーし、
そこにジャンプするアドレスをプログラムに指し示すことで可能になります。
実際にはそんなに簡単ではありませんが、それが現実にどのように現れたかが重要です。
古典的な記事は: \AlephOne

\myparagraph{GCC}

GCC 4.4.1 で同じコードを試してみましょう。次を得ます。

\lstinputlisting[style=customasmx86]{patterns/13_arrays/2_BO/w_gcc.asm}

Linuxで動かすと\TT{Segmentation fault}が発生します。

\myindex{GDB}

GDBデバッガで動かすと、このようになります。

\begin{lstlisting}
(gdb) r
Starting program: /home/dennis/RE/1 

Program received signal SIGSEGV, Segmentation fault.
0x00000016 in ?? ()
(gdb) info registers
eax            0x0	0
ecx            0xd2f96388	-755407992
edx            0x1d	29
ebx            0x26eff4	2551796
esp            0xbffff4b0	0xbffff4b0
ebp            0x15	0x15
esi            0x0	0
edi            0x0	0
eip            0x16	0x16
eflags         0x10202	[ IF RF ]
cs             0x73	115
ss             0x7b	123
ds             0x7b	123
es             0x7b	123
fs             0x0	0
gs             0x33	51
(gdb) 
\end{lstlisting}

レジスタ値はwin32の例とは少し異なりますし、スタックレイアウトも少し違います。
}


\EN{\subsection{Buffer overflow protection methods}
\label{subsec:BO_protection}

There are several methods to protect against this scourge, regardless of the \CCpp programmers' negligence.
MSVC has options like\footnote{compiler-side buffer overflow protection methods:
\href{http://go.yurichev.com/17133}{wikipedia.org/wiki/Buffer\_overflow\_protection}}:

\begin{lstlisting}
 /RTCs Stack Frame runtime checking
 /GZ Enable stack checks (/RTCs)
\end{lstlisting}

\myindex{x86!\Instructions!RET}
\myindex{Function prologue}
\myindex{Security cookie}

One of the methods is to write a random value between the local variables in stack at function prologue 
and to check it in function epilogue before the function exits.
If value is not the same, do not execute the last instruction \RET, but stop (or hang).
The process will halt, but that is much better than a remote attack to your host.
    
\newcommand{\CANARYURL}{\href{http://go.yurichev.com/17134}{wikipedia.org/wiki/Domestic\_canary\#Miner.27s\_canary}}

\myindex{Canary}

This random value is called a \q{canary} sometimes, it is related to the miners' canary\footnote{\CANARYURL},
they were used by miners in the past days in order to detect poisonous gases quickly.

Canaries are very sensitive to mine gases, they become very agitated in case of danger, or even die.

If we compile our very simple array example~(\myref{arrays_simple}) in \ac{MSVC}
with RTC1 and RTCs option,\\
you can see a call to \TT{@\_RTC\_CheckStackVars@8} a function at the end of the function that checks if the \q{canary} is correct.

Let's see how GCC handles this. 
Let's take an \TT{alloca()}~(\myref{alloca}) example:

\lstinputlisting[style=customc]{patterns/02_stack/04_alloca/2_1.c}

By default, without any additional options, GCC 4.7.3 inserts a \q{canary} check into the code:

\lstinputlisting[caption=GCC 4.7.3,style=customasmx86]{patterns/13_arrays/3_BO_protection/gcc_canary_EN.asm}

\myindex{x86!\Registers!GS}
The random value is located in \TT{gs:20}. 
It gets written on the stack and then at the end of the function
the value in the stack is compared with the correct \q{canary} in \TT{gs:20}. 
If the values are not equal, the 
\TT{\_\_stack\_chk\_fail} 
function is called and we can see in the console something like that (Ubuntu 13.04 x86):

\begin{lstlisting}
*** buffer overflow detected ***: ./2_1 terminated
======= Backtrace: =========
/lib/i386-linux-gnu/libc.so.6(__fortify_fail+0x63)[0xb7699bc3]
/lib/i386-linux-gnu/libc.so.6(+0x10593a)[0xb769893a]
/lib/i386-linux-gnu/libc.so.6(+0x105008)[0xb7698008]
/lib/i386-linux-gnu/libc.so.6(_IO_default_xsputn+0x8c)[0xb7606e5c]
/lib/i386-linux-gnu/libc.so.6(_IO_vfprintf+0x165)[0xb75d7a45]
/lib/i386-linux-gnu/libc.so.6(__vsprintf_chk+0xc9)[0xb76980d9]
/lib/i386-linux-gnu/libc.so.6(__sprintf_chk+0x2f)[0xb7697fef]
./2_1[0x8048404]
/lib/i386-linux-gnu/libc.so.6(__libc_start_main+0xf5)[0xb75ac935]
======= Memory map: ========
08048000-08049000 r-xp 00000000 08:01 2097586    /home/dennis/2_1
08049000-0804a000 r--p 00000000 08:01 2097586    /home/dennis/2_1
0804a000-0804b000 rw-p 00001000 08:01 2097586    /home/dennis/2_1
094d1000-094f2000 rw-p 00000000 00:00 0          [heap]
b7560000-b757b000 r-xp 00000000 08:01 1048602    /lib/i386-linux-gnu/libgcc_s.so.1
b757b000-b757c000 r--p 0001a000 08:01 1048602    /lib/i386-linux-gnu/libgcc_s.so.1
b757c000-b757d000 rw-p 0001b000 08:01 1048602    /lib/i386-linux-gnu/libgcc_s.so.1
b7592000-b7593000 rw-p 00000000 00:00 0
b7593000-b7740000 r-xp 00000000 08:01 1050781    /lib/i386-linux-gnu/libc-2.17.so
b7740000-b7742000 r--p 001ad000 08:01 1050781    /lib/i386-linux-gnu/libc-2.17.so
b7742000-b7743000 rw-p 001af000 08:01 1050781    /lib/i386-linux-gnu/libc-2.17.so
b7743000-b7746000 rw-p 00000000 00:00 0
b775a000-b775d000 rw-p 00000000 00:00 0
b775d000-b775e000 r-xp 00000000 00:00 0          [vdso]
b775e000-b777e000 r-xp 00000000 08:01 1050794    /lib/i386-linux-gnu/ld-2.17.so
b777e000-b777f000 r--p 0001f000 08:01 1050794    /lib/i386-linux-gnu/ld-2.17.so
b777f000-b7780000 rw-p 00020000 08:01 1050794    /lib/i386-linux-gnu/ld-2.17.so
bff35000-bff56000 rw-p 00000000 00:00 0          [stack]
Aborted (core dumped)
\end{lstlisting}

\myindex{MS-DOS}
gs is the so-called segment register. These registers were used widely in MS-DOS and DOS-extenders
times.
Today, its function is different.
\myindex{TLS}
\myindex{Windows!TIB}

To say it briefly, the \TT{gs} register in Linux always points to the
\ac{TLS}~(\myref{TLS})---some information specific to thread is stored there.
By the way, in win32 the \TT{fs} register plays the same role, pointing to
\ac{TIB} \footnote{\href{http://go.yurichev.com/17104}{wikipedia.org/wiki/Win32\_Thread\_Information\_Block}}. 

More information can be found in the Linux kernel source code (at least in 3.11 version),\\
in \emph{arch/x86/include/asm/stackprotector.h} this variable is described in the comments.

\subsection{ARM}

The ARM processor, just like in any other \q{pure} RISC processor lacks an instruction for division.
It also lacks a single instruction for multiplication by a 32-bit constant (recall that a 32-bit
constant cannot fit into a 32-bit opcode).

By taking advantage of this clever trick (or \emph{hack}), it is possible to do division using only three instructions: addition,
subtraction and bit shifts~(\myref{sec:bitfields}).

Here is an example that divides a 32-bit number by 10, from
\InSqBrackets{\ARMCookBook 3.3 Division by a Constant}.
The output consists of the quotient and the remainder.

\begin{lstlisting}[style=customasmARM]
; takes argument in a1
; returns quotient in a1, remainder in a2
; cycles could be saved if only divide or remainder is required
    SUB    a2, a1, #10             ; keep (x-10) for later
    SUB    a1, a1, a1, lsr #2
    ADD    a1, a1, a1, lsr #4
    ADD    a1, a1, a1, lsr #8
    ADD    a1, a1, a1, lsr #16
    MOV    a1, a1, lsr #3
    ADD    a3, a1, a1, asl #2
    SUBS   a2, a2, a3, asl #1      ; calc (x-10) - (x/10)*10
    ADDPL  a1, a1, #1              ; fix-up quotient
    ADDMI  a2, a2, #10             ; fix-up remainder
    MOV    pc, lr
\end{lstlisting}

\subsubsection{\OptimizingXcodeIV (\ARMMode)}

\begin{lstlisting}[style=customasmARM]
__text:00002C58 39 1E 08 E3 E3 18 43 E3  MOV    R1, 0x38E38E39
__text:00002C60 10 F1 50 E7              SMMUL  R0, R0, R1
__text:00002C64 C0 10 A0 E1              MOV    R1, R0,ASR#1
__text:00002C68 A0 0F 81 E0              ADD    R0, R1, R0,LSR#31
__text:00002C6C 1E FF 2F E1              BX     LR
\end{lstlisting}

This code is almost the same as the one generated by the optimizing MSVC and GCC.

Apparently, LLVM uses the same algorithm for generating constants.

\myindex{ARM!\Instructions!MOV}
\myindex{ARM!\Instructions!MOVT}

The observant reader may ask, how does \MOV writes a 32-bit value in a register, when this is not possible in ARM mode.

it is impossible indeed, but, as we see,
there are 8 bytes per instruction instead of the standard 4,
in fact, there are two instructions.

The first instruction loads \TT{0x8E39} into the low 16 bits of register and the second instruction is
\TT{MOVT}, it loads \TT{0x383E} into the high 16 bits of the register.
\IDA is fully aware of such sequences, and for the sake of compactness reduces them to one single \q{pseudo-instruction}.

\myindex{ARM!\Instructions!SMMUL}
The \TT{SMMUL} (\emph{Signed Most Significant Word Multiply}) 
instruction two multiplies numbers, treating them as signed numbers
and leaving the high 32-bit part of result in the \Reg{0} register,
dropping the low 32-bit part of the result.

\myindex{ARM!Optional operators!ASR}
The\TT{\q{MOV R1, R0,ASR\#1}} instruction is an arithmetic shift right by one bit.

\myindex{ARM!\Instructions!ADD}
\myindex{ARM!Data processing instructions}
\myindex{ARM!Optional operators!LSR}
\TT{\q{ADD R0, R1, R0,LSR\#31}} is $R0=R1 + R0>>31$

% FIXME какие именно инструкции? \myref{} ->
\label{shifts_in_ARM_mode}

There is no separate shifting instruction in ARM mode.
Instead, an instructions like 
(\MOV, \ADD, \SUB, \TT{RSB})\footnote{\DataProcessingInstructionsFootNote}
can have a suffix added, that says if the second operand must be shifted, and if yes, by what value and how.
\TT{ASR} stands for \emph{Arithmetic Shift Right}, \TT{LSR}---\emph{Logical Shift Right}.

\subsubsection{\OptimizingXcodeIV (\ThumbTwoMode)}

\begin{lstlisting}[style=customasmARM]
MOV             R1, 0x38E38E39
SMMUL.W         R0, R0, R1
ASRS            R1, R0, #1
ADD.W           R0, R1, R0,LSR#31
BX              LR
\end{lstlisting}

\myindex{ARM!\Instructions!ASRS}

There are separate instructions for shifting in Thumb mode, 
and one of them is used here---\TT{ASRS} (arithmetic shift right).

\subsubsection{\NonOptimizing Xcode 4.6.3 (LLVM) and Keil 6/2013}

\NonOptimizing LLVM
does not generate the code we saw before in this section, but instead inserts a call to the library function 
\emph{\_\_\_divsi3}.

What about Keil: it inserts a call to the library function \emph{\_\_aeabi\_idivmod} in all cases.


}
\RU{\subsection{Защита от переполнения буфера}
\label{subsec:BO_protection}

В наше время пытаются бороться с переполнением буфера невзирая на халатность программистов на \CCpp. 
В MSVC есть опции вроде\footnote{описания защит, которые компилятор может вставлять в код:
\href{http://go.yurichev.com/17133}{wikipedia.org/wiki/Buffer\_overflow\_protection}}:

\begin{lstlisting}
 /RTCs Stack Frame runtime checking
 /GZ Enable stack checks (/RTCs)
\end{lstlisting}

\myindex{x86!\Instructions!RET}
\myindex{Function prologue}
\myindex{Security cookie}
Одним из методов является вставка в прологе функции некоего случайного значения в область локальных переменных 
и проверка этого значения в эпилоге функции перед выходом. 
Если проверка не прошла, то не выполнять инструкцию \RET, а остановиться (или зависнуть). 
Процесс зависнет, но это лучше, чем удаленная атака на ваш компьютер.

\newcommand{\CANARYURL}{\href{http://go.yurichev.com/17135}{miningwiki.ru/wiki/Канарейка\_в\_шахте}}

\myindex{Canary}
Это случайное значение иногда называют \q{канарейкой}
\footnote{\q{canary} в англоязычной литературе}, 
по аналогии с шахтной канарейкой\footnote{\CANARYURL}.
Раньше использовали шахтеры, чтобы определять, есть ли в шахте опасный газ.

Канарейки очень к нему чувствительны и либо проявляли сильное беспокойство, либо гибли от газа.

Если скомпилировать наш простейший пример работы с массивом ~(\myref{arrays_simple}) в \ac{MSVC}
с опцией RTC1 или RTCs,
в конце нашей функции будет вызов функции \\
\TT{@\_RTC\_CheckStackVars@8}, проверяющей корректность \q{канарейки}.

Посмотрим, как дела обстоят в GCC. 
Возьмем пример из секции про \TT{alloca()}~(\myref{alloca}):

\lstinputlisting[style=customc]{patterns/02_stack/04_alloca/2_1.c}

По умолчанию, без дополнительных ключей, GCC 4.7.3 вставит в код проверку \q{канарейки}:

\lstinputlisting[caption=GCC 4.7.3,style=customasmx86]{patterns/13_arrays/3_BO_protection/gcc_canary_RU.asm}

\myindex{x86!\Registers!GS}
Случайное значение находится в \TT{gs:20}. 
Оно записывается в стек, затем, в конце функции, значение в стеке
сравнивается с корректной \q{канарейкой} в \TT{gs:20}. 
Если значения не равны, будет вызвана функция 
\TT{\_\_stack\_chk\_fail} и в консоли мы увидим что-то вроде такого
 (Ubuntu 13.04 x86):

\begin{lstlisting}
*** buffer overflow detected ***: ./2_1 terminated
======= Backtrace: =========
/lib/i386-linux-gnu/libc.so.6(__fortify_fail+0x63)[0xb7699bc3]
/lib/i386-linux-gnu/libc.so.6(+0x10593a)[0xb769893a]
/lib/i386-linux-gnu/libc.so.6(+0x105008)[0xb7698008]
/lib/i386-linux-gnu/libc.so.6(_IO_default_xsputn+0x8c)[0xb7606e5c]
/lib/i386-linux-gnu/libc.so.6(_IO_vfprintf+0x165)[0xb75d7a45]
/lib/i386-linux-gnu/libc.so.6(__vsprintf_chk+0xc9)[0xb76980d9]
/lib/i386-linux-gnu/libc.so.6(__sprintf_chk+0x2f)[0xb7697fef]
./2_1[0x8048404]
/lib/i386-linux-gnu/libc.so.6(__libc_start_main+0xf5)[0xb75ac935]
======= Memory map: ========
08048000-08049000 r-xp 00000000 08:01 2097586    /home/dennis/2_1
08049000-0804a000 r--p 00000000 08:01 2097586    /home/dennis/2_1
0804a000-0804b000 rw-p 00001000 08:01 2097586    /home/dennis/2_1
094d1000-094f2000 rw-p 00000000 00:00 0          [heap]
b7560000-b757b000 r-xp 00000000 08:01 1048602    /lib/i386-linux-gnu/libgcc_s.so.1
b757b000-b757c000 r--p 0001a000 08:01 1048602    /lib/i386-linux-gnu/libgcc_s.so.1
b757c000-b757d000 rw-p 0001b000 08:01 1048602    /lib/i386-linux-gnu/libgcc_s.so.1
b7592000-b7593000 rw-p 00000000 00:00 0
b7593000-b7740000 r-xp 00000000 08:01 1050781    /lib/i386-linux-gnu/libc-2.17.so
b7740000-b7742000 r--p 001ad000 08:01 1050781    /lib/i386-linux-gnu/libc-2.17.so
b7742000-b7743000 rw-p 001af000 08:01 1050781    /lib/i386-linux-gnu/libc-2.17.so
b7743000-b7746000 rw-p 00000000 00:00 0
b775a000-b775d000 rw-p 00000000 00:00 0
b775d000-b775e000 r-xp 00000000 00:00 0          [vdso]
b775e000-b777e000 r-xp 00000000 08:01 1050794    /lib/i386-linux-gnu/ld-2.17.so
b777e000-b777f000 r--p 0001f000 08:01 1050794    /lib/i386-linux-gnu/ld-2.17.so
b777f000-b7780000 rw-p 00020000 08:01 1050794    /lib/i386-linux-gnu/ld-2.17.so
bff35000-bff56000 rw-p 00000000 00:00 0          [stack]
Aborted (core dumped)
\end{lstlisting}

\myindex{MS-DOS}
gs это так называемый сегментный регистр. Эти регистры широко использовались во времена MS-DOS 
и DOS-экстендеров.
Сейчас их функция немного изменилась.
\myindex{TLS}
\myindex{Windows!TIB}
Если говорить кратко, в Linux \TT{gs} всегда указывает на \ac{TLS}~(\myref{TLS})~--- там находится различная 
информация, специфичная для выполняющегося потока.

Кстати, в win32 эту же роль играет сегментный регистр \TT{fs},
он всегда указывает на
\ac{TIB} \footnote{\href{http://go.yurichev.com/17104}{wikipedia.org/wiki/Win32\_Thread\_Information\_Block}}. 

Больше информации можно почерпнуть из исходных кодов Linux (по крайней мере, в версии 3.11):\\
в файле \emph{arch/x86/include/asm/stackprotector.h} в комментариях описывается эта переменная.

\subsubsection{ARM: \OptimizingKeilVI (\ARMMode)}
\myindex{\CLanguageElements!switch}

\lstinputlisting[style=customasmARM]{patterns/08_switch/1_few/few_ARM_ARM_O3.asm}

Мы снова не сможем сказать, глядя на этот код, был ли в оригинальном исходном коде switch() 
либо же несколько операторов if().

\myindex{ARM!\Instructions!ADRcc}
Так или иначе, мы снова видим здесь инструкции с предикатами, например, \ADREQ (\emph{(Equal)}), 
которая будет исполняться только
если $R0=0$, и тогда в \Reg{0} будет загружен адрес строки \emph{<<zero\textbackslash{}n>>}.

\myindex{ARM!\Instructions!BEQ}
Следующая инструкция \ac{BEQ} перенаправит исполнение на \TT{loc\_170}, если $R0=0$.

Кстати, наблюдательный читатель может спросить, сработает ли \ac{BEQ} нормально,
ведь \ADREQ перед ним уже заполнила регистр \Reg{0} чем-то другим?

Сработает, потому что \ac{BEQ} проверяет флаги, установленные инструкцией \CMP, 
а \ADREQ флаги никак не модифицирует.

Далее всё просто и знакомо. 
Вызов \printf один, и в самом конце, мы уже рассматривали подобный трюк~(\myref{ARM_B_to_printf}).
К вызову функции \printf{} в конце ведут три пути.

\myindex{ARM!\Instructions!ADRcc}
\myindex{ARM!\Instructions!CMP}
Последняя инструкция \TT{CMP R0, \#2} здесь нужна, чтобы узнать $a=2$ или нет.

Если это не так, то при помощи \ADRNE (\emph{Not Equal}) в \Reg{0} будет загружен указатель на 
строку \emph{<<something unknown \textbackslash{}n>>}, ведь $a$ уже было проверено на 0 и 1 до этого, 
и здесь $a$ точно не попадает под эти константы.

Ну а если $R0=2$, в \Reg{0} будет загружен указатель на строку \emph{<<two\textbackslash{}n>>} при помощи инструкции \ADREQ.

\subsubsection{ARM: \OptimizingKeilVI (\ThumbMode)}

\lstinputlisting[style=customasmARM]{patterns/08_switch/1_few/few_ARM_thumb_O3.asm}

% FIXME а каким можно? к каким нельзя? \myref{} ->
Как уже было отмечено, в Thumb-режиме нет возможности добавлять условные предикаты к большинству инструкций,
так что Thumb-код вышел похожим на код x86 в стиле \ac{CISC}, вполне понятный.

\subsubsection{ARM64: \NonOptimizing GCC (Linaro) 4.9}

\lstinputlisting[style=customasmARM]{patterns/08_switch/1_few/ARM64_GCC_O0_RU.lst}

Входное значение имеет тип \Tint, поэтому для него используется регистр \RegW{0},
а не целая часть регистра \RegX{0}.

Указатели на строки передаются в \puts при помощи пары инструкций ADRP/ADD, как было показано в примере
\q{\HelloWorldSectionName}:~\myref{pointers_ADRP_and_ADD}.

\subsubsection{ARM64: \Optimizing GCC (Linaro) 4.9}

\lstinputlisting[style=customasmARM]{patterns/08_switch/1_few/ARM64_GCC_O3_RU.lst}

Фрагмент кода более оптимизированный.
Инструкция \TT{CBZ} (\emph{Compare and Branch on Zero}~--- сравнить и перейти если ноль) совершает переход если \RegW{0} ноль.
Здесь также прямой переход на \puts вместо вызова, как уже было описано:~\myref{JMP_instead_of_RET}.


}
\DE{\subsection{Schutz vor Buffer Overflows}
\label{subsec:BO_protection}
Es gibt verschiedene Möglichkeiten um sich vor solchen Problemen zu schützen, unabhängig von der Unachtsamkeit des \CCpp
Programmierers. MSVC kennt Optionen wie\footnote{Compilerseitiger Schutz vor Buffer Overflows:
\href{http://go.yurichev.com/17133}{wikipedia.org/wiki/Buffer\_overflow\_protection}}:

\begin{lstlisting}
 /RTCs Stack Frame runtime checking
 /GZ Enable stack checks (/RTCs)
\end{lstlisting}

\myindex{x86!\Instructions!RET}
\myindex{Function prologue}
\myindex{Security cookie}
Eine Methode ist eine Zufallszahl zwischen die lokalen Variablen auf dem Stack am Funktionsprolog zu schreiben und
diesen im Funktionsepilog vor dem Beenden der Funktion zu überprüfen.
Wenn der Wert nicht identisch ist, sollte der letzte \RET Befehl nicht ausgeführt werden, sondern das Programm
angehalten werden. Der Prozess wird anhalten, aber das ist deutlich besser als eine Fernattacke auf Ihren Rechner.
    
\newcommand{\CANARYURL}{\href{http://go.yurichev.com/17134}{wikipedia.org/wiki/Domestic\_canary\#Miner.27s\_canary}}

\myindex{Canary}
Die Zufallszahl wird auch \q{canary} (dt. Kanarienvogel) genannt. Der Begriff stammt von den Kanarienvögeln der
Minenarbeiter\footnote{\CANARYURL}, die früher benutzt wurde, um giftige Gase schnell zu erkennen

Kanarienvögel reagieren sehr sensibel auf Grubengase und werden bei Gefahr sehr nervös oder sterben sogar.

Wenn wir unser einfaches Arraybeispiel in \ac{MSVC} mit Optionen RTC1 und RTCs kompilieren~(\myref{arrays_simple})
finden wir einen Aufruf von \TT{@\_RTC\_CheckStackVars@8}, eine Funktion am Ende der Funktion, die prüft, ob der
\q{canary} korrekt ist.

Schauen wir uns an, wie GCC die Sache handhabt.
Betrachten wir ein Beispiel mit \TT{alloca()}~(\myref{alloca}):

\lstinputlisting[style=customc]{patterns/02_stack/04_alloca/2_1.c}
Ohne zusätzliche Optionen fügt GCC 4.7.3 standardmäßig dem Code einen \q{canary} zum Überprüfen hinzu:

\lstinputlisting[caption=GCC 4.7.3,style=customasmx86]{patterns/13_arrays/3_BO_protection/gcc_canary_DE.asm}

\myindex{x86!\Registers!GS}
Der Zufallswert befindet sich in \TT{gs:20}.
Er wird auf den Stack geschrieben und am Ende der Funktion wird der Wert auf dem Stack mit dem korrekten \q{canary} in
\TT{gs:20} verglichen.
Wenn die Werte ungleich sind, wird die Funktion \TT{\_\_stack\_chk\_fail} aufgerufen und wir erkennen in der Konsole in
etwa das Folgende (Ubuntu 13.04 x86):

\begin{lstlisting}
*** buffer overflow detected ***: ./2_1 terminated
======= Backtrace: =========
/lib/i386-linux-gnu/libc.so.6(__fortify_fail+0x63)[0xb7699bc3]
/lib/i386-linux-gnu/libc.so.6(+0x10593a)[0xb769893a]
/lib/i386-linux-gnu/libc.so.6(+0x105008)[0xb7698008]
/lib/i386-linux-gnu/libc.so.6(_IO_default_xsputn+0x8c)[0xb7606e5c]
/lib/i386-linux-gnu/libc.so.6(_IO_vfprintf+0x165)[0xb75d7a45]
/lib/i386-linux-gnu/libc.so.6(__vsprintf_chk+0xc9)[0xb76980d9]
/lib/i386-linux-gnu/libc.so.6(__sprintf_chk+0x2f)[0xb7697fef]
./2_1[0x8048404]
/lib/i386-linux-gnu/libc.so.6(__libc_start_main+0xf5)[0xb75ac935]
======= Memory map: ========
08048000-08049000 r-xp 00000000 08:01 2097586    /home/dennis/2_1
08049000-0804a000 r--p 00000000 08:01 2097586    /home/dennis/2_1
0804a000-0804b000 rw-p 00001000 08:01 2097586    /home/dennis/2_1
094d1000-094f2000 rw-p 00000000 00:00 0          [heap]
b7560000-b757b000 r-xp 00000000 08:01 1048602    /lib/i386-linux-gnu/libgcc_s.so.1
b757b000-b757c000 r--p 0001a000 08:01 1048602    /lib/i386-linux-gnu/libgcc_s.so.1
b757c000-b757d000 rw-p 0001b000 08:01 1048602    /lib/i386-linux-gnu/libgcc_s.so.1
b7592000-b7593000 rw-p 00000000 00:00 0
b7593000-b7740000 r-xp 00000000 08:01 1050781    /lib/i386-linux-gnu/libc-2.17.so
b7740000-b7742000 r--p 001ad000 08:01 1050781    /lib/i386-linux-gnu/libc-2.17.so
b7742000-b7743000 rw-p 001af000 08:01 1050781    /lib/i386-linux-gnu/libc-2.17.so
b7743000-b7746000 rw-p 00000000 00:00 0
b775a000-b775d000 rw-p 00000000 00:00 0
b775d000-b775e000 r-xp 00000000 00:00 0          [vdso]
b775e000-b777e000 r-xp 00000000 08:01 1050794    /lib/i386-linux-gnu/ld-2.17.so
b777e000-b777f000 r--p 0001f000 08:01 1050794    /lib/i386-linux-gnu/ld-2.17.so
b777f000-b7780000 rw-p 00020000 08:01 1050794    /lib/i386-linux-gnu/ld-2.17.so
bff35000-bff56000 rw-p 00000000 00:00 0          [stack]
Aborted (core dumped)
\end{lstlisting}

\myindex{MS-DOS}
gs ist das sogenannte Segmentregister. Diese Register wurden zu Zeiten von MS-DOS und DOS-Erweiterungen häufig
verwendet. Heute ist sein Zweck ein anderer:
\myindex{TLS}
\myindex{Windows!TIB}
Kurz gesagt, zeigt das \TT{gs} Register in Linux stets auf den \ac{TLS}~(\myref{TLS})--hier werden threadspezifische
Informationen gespeichert. In win32 spielt das \TT{fs} Register übrigens die gleiche Rolle und zeigt stets auf
\ac{TIB}\footnote{\href{http://go.yurichev.com/17104}{wikipedia.org/wiki/Win32\_Thread\_Information\_Block}}.

Mehr Informationen finden sich im Quellcode des Linux Kernels (zumindest in der Version 3.11), in\\
\emph{arch/x86/include/asm/stackprotector.h} wird diese Variable in den Kommentaren beschrieben.

\subsubsection{ARM}

\myparagraph{\OptimizingKeilVI (\ThumbMode)}

\lstinputlisting[style=customasmARM]{patterns/04_scanf/1_simple/ARM_IDA.lst}

\myindex{\CLanguageElements!\Pointers}
Damit \scanf Elemente einlesen kann, benötigt die Funktion einen Paramter--einen Pointer vom Typ \Tint.
\Tint hat die Größe 32 Bit, wir benötigen also 4 Byte, um den Wert im Speicher abzulegen, und passt daher genau in ein 32-Bit-Register.
\myindex{IDA!var\_?}
Auf dem Stack wird Platz für die lokalen Variable \GTT{x} reserviert und IDA bezeichnet diese Variable mit \emph{var\_8}. 
Eigentlich ist aber an dieser Stelle gar nicht notwendig, Platz auf dem Stack zu reservieren, da \ac{SP} (\gls{stack pointer} 
bereits auf die Adresse zeigt und auch direkt verwendet werden kann.

Der Wert von \ac{SP} wird also in das \Reg{1} Register kopiert und zusammen mit dem Formatierungsstring an \scanf übergeben.

% TBT here
%\INS{PUSH/POP} instructions behaves differently in ARM than in x86 (it's the other way around).
%They are synonyms to \INS{STM/STMDB/LDM/LDMIA} instructions.
%And \INS{PUSH} instruction first writes a value into the stack, \emph{and then} subtracts \ac{SP} by 4.
%\INS{POP} first adds 4 to \ac{SP}, \emph{and then} reads a value from the stack.
%Hence, after \INS{PUSH}, \ac{SP} points to an unused space in stack.
%It is used by \scanf, and by \printf after.

%\INS{LDMIA} means \emph{Load Multiple Registers Increment address After each transfer}.
%\INS{STMDB} means \emph{Store Multiple Registers Decrement address Before each transfer}.

\myindex{ARM!\Instructions!LDR}
Später wird mithilfe des \INS{LDR} Befehls dieser Wert vom Stack in das \Reg{1} Register verschoben um an \printf übergeben werden zu können.

\myparagraph{ARM64}

\lstinputlisting[caption=\NonOptimizing GCC 4.9.1 ARM64,numbers=left,style=customasmARM]{patterns/04_scanf/1_simple/ARM64_GCC491_O0_DE.s}

Im Stack Frame werden 32 Byte reserviert, was deutlich mehr als benötigt ist. Vielleicht handelt es sich um eine Frage des Aligning (dt. Angleichens) von Speicheradressen.
Der interessanteste Teil ist, im Stack Frame einen Platz für die Variable $x$ zu finden (Zeile 22).
Warum 28? Irgendwie hat der Compiler entschieden die Variable am Ende des Stack Frames anstatt an dessen Beginn abzulegen.
Die Adresse wird an \scanf übergeben; diese Funktion speichert den Userinput an der genannten Adresse im Speicher.
Es handelt sich hier um einen 32-Bit-Wert vom Typ \Tint. 
Der Wert wird in Zeile 27 abgeholt und dann an \printf übergeben.




}
\FR{\subsection{Méthodes de protection contre les débordements de tampon}
\label{subsec:BO_protection}

Il existe quelques méthodes pour protéger contre ce fléau, indépendamment de la négligence
des programmeurs \CCpp.
MSVC possède des options comme\footnote{méthode de protection contre les débordements
de tampons côté compilateur:\href{http://go.yurichev.com/17133}{wikipedia.org/wiki/Buffer\_overflow\_protection}}:

\begin{lstlisting}
 /RTCs Stack Frame runtime checking
 /GZ Enable stack checks (/RTCs)
\end{lstlisting}

\myindex{x86!\Instructions!RET}
\myindex{Function prologue}
\myindex{Security cookie}

Une des méthodes est d'écrire une valeur aléatoire entre les variables locales sur
la pile dans le prologue de la fonction et de la vérifier dans l'épilogue, avant de
sortir de la fonction.
Si la valeur n'est pas la même, ne pas exécuter la dernière instruction \RET, mais
stopper (ou bloquer).
Le processus va s'arrêter, mais c'est mieux qu'une attaque distante sur votre ordinateur.
    
\newcommand{\CANARYURL}{\href{http://go.yurichev.com/17134}{wikipedia.org/wiki/Domestic\_canary\#Miner.27s\_canary}}

\myindex{Canary}

Cette valeur aléatoire est parfois appelé un \q{canari}, c'est lié au canari\footnote{\CANARYURL}
que les mineurs utilisaient dans le passé afin de détecter rapidement les gaz toxiques.

Les canaris sont très sensibles aux gaz, ils deviennent très agités en cas de danger,
et même meurent.

Si nous compilons notre exemple de tableau très simple~(\myref{arrays_simple}) dans
\ac{MSVC} avec les options RTC1 et RTCs, nous voyons un appel à \TT{@\_RTC\_CheckStackVars@8}
une fonction à la fin de la fonction qui vérifie si le \q{canari} est correct.

Voyons comment GCC gère ceci.
Prenons un exemple \TT{alloca()}~(\myref{alloca}):

\lstinputlisting[style=customc]{patterns/02_stack/04_alloca/2_1.c}

Par défaut, sans option supplémentaire, GCC 4.7.3 insère un test de  \q{canari} dans
le code:

\lstinputlisting[caption=GCC 4.7.3,style=customasmx86]{patterns/13_arrays/3_BO_protection/gcc_canary_FR.asm}

\myindex{x86!\Registers!GS}
La valeur aléatoire se trouve en \TT{gs:20}.
Elle est écrite sur la pile et à la fin de la fonction, la valeur sur la pile est
comparée avec le \q{canari} correct dans \TT{gs:20}.
Si les valeurs ne sont pas égales, la fonction \TT{\_\_stack\_chk\_fail} est appelée
et nous voyons dans la console quelque chose comme ça (Ubuntu 13.04 x86):

\begin{lstlisting}
*** buffer overflow detected ***: ./2_1 terminated
======= Backtrace: =========
/lib/i386-linux-gnu/libc.so.6(__fortify_fail+0x63)[0xb7699bc3]
/lib/i386-linux-gnu/libc.so.6(+0x10593a)[0xb769893a]
/lib/i386-linux-gnu/libc.so.6(+0x105008)[0xb7698008]
/lib/i386-linux-gnu/libc.so.6(_IO_default_xsputn+0x8c)[0xb7606e5c]
/lib/i386-linux-gnu/libc.so.6(_IO_vfprintf+0x165)[0xb75d7a45]
/lib/i386-linux-gnu/libc.so.6(__vsprintf_chk+0xc9)[0xb76980d9]
/lib/i386-linux-gnu/libc.so.6(__sprintf_chk+0x2f)[0xb7697fef]
./2_1[0x8048404]
/lib/i386-linux-gnu/libc.so.6(__libc_start_main+0xf5)[0xb75ac935]
======= Memory map: ========
08048000-08049000 r-xp 00000000 08:01 2097586    /home/dennis/2_1
08049000-0804a000 r--p 00000000 08:01 2097586    /home/dennis/2_1
0804a000-0804b000 rw-p 00001000 08:01 2097586    /home/dennis/2_1
094d1000-094f2000 rw-p 00000000 00:00 0          [heap]
b7560000-b757b000 r-xp 00000000 08:01 1048602    /lib/i386-linux-gnu/libgcc_s.so.1
b757b000-b757c000 r--p 0001a000 08:01 1048602    /lib/i386-linux-gnu/libgcc_s.so.1
b757c000-b757d000 rw-p 0001b000 08:01 1048602    /lib/i386-linux-gnu/libgcc_s.so.1
b7592000-b7593000 rw-p 00000000 00:00 0
b7593000-b7740000 r-xp 00000000 08:01 1050781    /lib/i386-linux-gnu/libc-2.17.so
b7740000-b7742000 r--p 001ad000 08:01 1050781    /lib/i386-linux-gnu/libc-2.17.so
b7742000-b7743000 rw-p 001af000 08:01 1050781    /lib/i386-linux-gnu/libc-2.17.so
b7743000-b7746000 rw-p 00000000 00:00 0
b775a000-b775d000 rw-p 00000000 00:00 0
b775d000-b775e000 r-xp 00000000 00:00 0          [vdso]
b775e000-b777e000 r-xp 00000000 08:01 1050794    /lib/i386-linux-gnu/ld-2.17.so
b777e000-b777f000 r--p 0001f000 08:01 1050794    /lib/i386-linux-gnu/ld-2.17.so
b777f000-b7780000 rw-p 00020000 08:01 1050794    /lib/i386-linux-gnu/ld-2.17.so
bff35000-bff56000 rw-p 00000000 00:00 0          [stack]
Aborted (core dumped)
\end{lstlisting}

\myindex{MS-DOS}
gs est ainsi appelé registre de segment. Ces registres étaient beaucoup utilisés
du temps de MS-DOS et des extensions de DOS.
Aujourd'hui, sa fonction est différente.
\myindex{TLS}
\myindex{Windows!TIB}

Dit brièvement, le registre \TT{gs} dans Linux pointe toujours sur le
\ac{TLS}~(\myref{TLS})---des informations spécifiques au thread sont stockées là.
À propos, en win32 le registre \TT{fs} joue le même rôle, pointant sur \ac{TIB}
\footnote{\href{http://go.yurichev.com/17104}{wikipedia.org/wiki/Win32\_Thread\_Information\_Block}}.

Il y a plus d'information dans le code source du noyau Linux (au moins dans la version 3.11),
dans\\
\emph{arch/x86/include/asm/stackprotector.h} cette variable est décrite dans les commentaires.

\subsubsection{ARM: \OptimizingKeilVI (\ARMMode)}
\myindex{\CLanguageElements!switch}

\lstinputlisting[style=customasmARM]{patterns/08_switch/1_few/few_ARM_ARM_O3.asm}

A nouveau, en investiguant ce code, nous ne pouvons pas dire si il y avait un switch()
dans le code source d'origine ou juste un ensemble de déclarations if().

\myindex{ARM!\Instructions!ADRcc}

En tout cas, nous voyons ici des instructions conditionnelles (comme \ADREQ (\emph{Equal}))
qui ne sont exécutées que si $R0=0$, et qui chargent ensuite l'adresse de la chaîne
\emph{<<zero\textbackslash{}n>>} dans \Reg{0}.
\myindex{ARM!\Instructions!BEQ}
L'instruction suivante \ac{BEQ} redirige le flux d'exécution en \TT{loc\_170}, si $R0=0$.

Le lecteur attentif peut se demander si \ac{BEQ} s'exécute correctement puisque \ADREQ
a déjà mis une autre valeur dans le registre \Reg{0}.

Oui, elle s'exécutera correctement, car \ac{BEQ} vérifie les flags mis par l'instruction
\CMP et \ADREQ ne modifie aucun flag.

Les instructions restantes nous sont déjà familières.
Il y a seulement un appel à \printf, à la fin, et nous avons déjà examiné cette
astuce ici~(\myref{ARM_B_to_printf}).
A la fin, il y a trois chemins vers \printf{}.

\myindex{ARM!\Instructions!ADRcc}
\myindex{ARM!\Instructions!CMP}
La dernière instruction, \TT{CMP R0, \#2}, est nécessaire pour vérifier si $a=2$.

Si ce n'est pas vrai, alors \ADRNE charge un pointeur sur la chaîne \emph{<<something unknown \textbackslash{}n>>}
dans \Reg{0}, puisque $a$ a déjà été comparée pour savoir s'elle est égale
à 0 ou 1, et nous sommes sûrs que la variable $a$ n'est pas égale à l'un de
ces nombres, à ce point.
Et si $R0=2$, un pointeur sur la chaîne \emph{<<two\textbackslash{}n>>} sera chargé
par \ADREQ dans \Reg{0}.

\subsubsection{ARM: \OptimizingKeilVI (\ThumbMode)}

\lstinputlisting[style=customasmARM]{patterns/08_switch/1_few/few_ARM_thumb_O3.asm}

% FIXME а каким можно? к каким нельзя? \myref{} ->

Comme il y déjà été dit, il n'est pas possible d'ajouter un prédicat conditionnel
à la plupart des instructions en mode Thumb, donc ce dernier est quelque peu similaire
au code \ac{CISC}-style x86, facilement compréhensible.

\subsubsection{ARM64: GCC (Linaro) 4.9 \NonOptimizing}

\lstinputlisting[style=customasmARM]{patterns/08_switch/1_few/ARM64_GCC_O0_FR.lst}

Le type de la valeur d'entrée est \Tint, par conséquent le registre \RegW{0} est
utilisé pour garder la valeur au lieu du registre complet \RegX{0}.

Les pointeurs de chaîne sont passés à \puts en utilisant la paire d'instructions
\INS{ADRP}/\INS{ADD} comme expliqué dans l'exemple \q{\HelloWorldSectionName}:~\myref{pointers_ADRP_and_ADD}.

\subsubsection{ARM64: GCC (Linaro) 4.9 \Optimizing}

\lstinputlisting[style=customasmARM]{patterns/08_switch/1_few/ARM64_GCC_O3_FR.lst}

Ce morceau de code est mieux optimisé.
L'instruction \TT{CBZ} (\emph{Compare and Branch on Zero} comparer et sauter si zéro)
effectue un saut si \RegW{0} vaut zéro.
Il y a alors un saut direct à \puts au lieu de l'appeler, comme cela a été expliqué
avant:~\myref{JMP_instead_of_RET}.


}
\JPN{\subsection{バッファオーバーフロー保護手法}
\label{subsec:BO_protection}

There are several methods to protect against this scourge, regardless of the \CCpp programmers' negligence.
MSVC has options like\footnote{compiler-side buffer overflow protection methods:
\href{http://go.yurichev.com/17133}{wikipedia.org/wiki/Buffer\_overflow\_protection}}:

このソースコードに対する保護手法はいくつかあり、 \CCpp プログラマの怠慢にもかかわらず、
MSVCにはオプションがあります。\footnote{コンパイラサイドのバッファオーバーフロー保護手法:
\href{http://go.yurichev.com/17133}{wikipedia.org/wiki/Buffer\_overflow\_protection}}

\begin{lstlisting}
 /RTCs スタックフレームの実行時チェック
 /GZ スタックチェックの有効化 (/RTCs)
\end{lstlisting}

\myindex{x86!\Instructions!RET}
\myindex{Function prologue}
\myindex{Security cookie}

手法の1つに関数プロローグでスタックのローカル変数の間にランダムな値を書き込み、
関数を終了する前に関数エピローグでそれをチェックするというものがあります。
値が同じでなければ、最後の命令 \RET を実行せず、停止(ハング)します。
プロセスは停止しますが、遠隔の攻撃者があなたのホストを攻撃するよりはよいことです。

\newcommand{\CANARYURL}{\href{http://go.yurichev.com/17134}{wikipedia.org/wiki/Domestic\_canary\#Miner.27s\_canary}}

\myindex{Canary}

このランダムな値は しばしば \q{カナリア} と呼ばれ、炭鉱労働でのカナリアに関連しています。\footnote{\CANARYURL}
昔、有毒なガスを一早く検知できるよう、炭鉱労働者に使用されていました。

カナリアは炭鉱のガスにとっても敏感で、危機の際に騒ぎ立て、場合によっては死んでしまいました。

とてもシンプルな配列の例を \ac{MSVC} でRTC1とRTCsオプション付きでコンパイルする場合~(\myref{arrays_simple})、
\q{カナリア}が正しいかどうか、関数の最後に \TT{@\_RTC\_CheckStackVars@8} を呼び出すのを見ることができます。

GCCがこれをどのように扱うかを見てみましょう。
\TT{alloca()}~(\myref{alloca})の例を扱いましょう。

\lstinputlisting[style=customc]{patterns/02_stack/04_alloca/2_1.c}

デフォルトでは、追加のオプションなしに、GCC 4.7.3は\q{カナリア}チェックをコードに挿入します。

\lstinputlisting[caption=GCC 4.7.3,style=customasmx86]{patterns/13_arrays/3_BO_protection/gcc_canary_JPN.asm}

ランダム値が\TT{gs:20}に配置されます。
スタックに書かれて、関数の最後でスタックの値が\TT{gs:20}の\q{カナリア}と一致しているか比較します。
値が一致していなければ、
\TT{\_\_stack\_chk\_fail}
関数が呼び出され、ときどき(Ubuntu 13.04 x86):のようなものをコンソールでみることがあります。

\begin{lstlisting}
*** buffer overflow detected ***: ./2_1 terminated
======= Backtrace: =========
/lib/i386-linux-gnu/libc.so.6(__fortify_fail+0x63)[0xb7699bc3]
/lib/i386-linux-gnu/libc.so.6(+0x10593a)[0xb769893a]
/lib/i386-linux-gnu/libc.so.6(+0x105008)[0xb7698008]
/lib/i386-linux-gnu/libc.so.6(_IO_default_xsputn+0x8c)[0xb7606e5c]
/lib/i386-linux-gnu/libc.so.6(_IO_vfprintf+0x165)[0xb75d7a45]
/lib/i386-linux-gnu/libc.so.6(__vsprintf_chk+0xc9)[0xb76980d9]
/lib/i386-linux-gnu/libc.so.6(__sprintf_chk+0x2f)[0xb7697fef]
./2_1[0x8048404]
/lib/i386-linux-gnu/libc.so.6(__libc_start_main+0xf5)[0xb75ac935]
======= Memory map: ========
08048000-08049000 r-xp 00000000 08:01 2097586    /home/dennis/2_1
08049000-0804a000 r--p 00000000 08:01 2097586    /home/dennis/2_1
0804a000-0804b000 rw-p 00001000 08:01 2097586    /home/dennis/2_1
094d1000-094f2000 rw-p 00000000 00:00 0          [heap]
b7560000-b757b000 r-xp 00000000 08:01 1048602    /lib/i386-linux-gnu/libgcc_s.so.1
b757b000-b757c000 r--p 0001a000 08:01 1048602    /lib/i386-linux-gnu/libgcc_s.so.1
b757c000-b757d000 rw-p 0001b000 08:01 1048602    /lib/i386-linux-gnu/libgcc_s.so.1
b7592000-b7593000 rw-p 00000000 00:00 0
b7593000-b7740000 r-xp 00000000 08:01 1050781    /lib/i386-linux-gnu/libc-2.17.so
b7740000-b7742000 r--p 001ad000 08:01 1050781    /lib/i386-linux-gnu/libc-2.17.so
b7742000-b7743000 rw-p 001af000 08:01 1050781    /lib/i386-linux-gnu/libc-2.17.so
b7743000-b7746000 rw-p 00000000 00:00 0
b775a000-b775d000 rw-p 00000000 00:00 0
b775d000-b775e000 r-xp 00000000 00:00 0          [vdso]
b775e000-b777e000 r-xp 00000000 08:01 1050794    /lib/i386-linux-gnu/ld-2.17.so
b777e000-b777f000 r--p 0001f000 08:01 1050794    /lib/i386-linux-gnu/ld-2.17.so
b777f000-b7780000 rw-p 00020000 08:01 1050794    /lib/i386-linux-gnu/ld-2.17.so
bff35000-bff56000 rw-p 00000000 00:00 0          [stack]
Aborted (core dumped)
\end{lstlisting}

\myindex{MS-DOS}
gsはいわゆるセグメントレジスタです。このレジスタは広くMS-DOSやDOS拡張で使用されました。
今日、この機能は異なっています。
\myindex{TLS}
\myindex{Windows!TIB}

簡単に言うと、Linuxでの\TT{gs}レジスタは常に\ac{TLS}~(\myref{TLS})を指し示します。
スレッド固有の情報がそこに保存されます。
ところで、win32では\TT{fs}レジスタは同じ役割を担い、\ac{TIB}を指し示します。
\footnote{\href{http://go.yurichev.com/17104}{wikipedia.org/wiki/Win32\_Thread\_Information\_Block}}

より詳細はLinuxカーネルソースコード\IT{arch/x86/include/asm/stackprotector.h}の中に
コメントとして記述してあるのを見つけられます(少なくとも3.11バージョンには)。

\subsubsection{\OptimizingXcodeIV (\ThumbTwoMode)}

単純な配列の例に戻りましょう(\myref{arrays_simple})。

繰り返しますが、LLVMが\q{カナリア}の正しさをどのようにチェックするのか見ることができます。

% TODO shorten the listing a bit? is full display of unrolled loop necessary?
\lstinputlisting[style=customasmARM]{patterns/13_arrays/3_BO_protection/simple_Xcode_thumb_O3_JPN.asm}

\myindex{Unrolled loop}

まず最初に、見てきたように、LLVMはループを\q{展開し}、LLVMは高速になると結論づけて、
事前に計算されて値はすべて配列に1つ1つ書かれます。
なお、ARMモードでの命令はこれをより高速にする手助けをするかもしれません。
これを見つけるのは宿題にします。

関数の最後で\q{カナリア}の比較を見ます。ローカルスタックのカナリアと \Reg{8}で指し示した正しいものとの。

\myindex{ARM!\Instructions!IT}

それぞれが一致していれば、4命令ブロックが\INS{ITTTT EQ}で実行され、
\Reg{0}に0が書かれ、関数エピローグが終了します。
\q{カナリア}が一致していなければ、ブロックがスキップされ、
\TT{\_\_\_stack\_chk\_fail}関数へのジャンプが実行され、おそらく実行が停止されます。
% TODO1 illustrate this!

}

\EN{\subsection{One more word about arrays}


Now we understand why it is impossible to write something like this in \CCpp code:

\begin{lstlisting}[style=customc]
void f(int size)
{
    int a[size];
...
};
\end{lstlisting}


That's just because the compiler must know the exact array size to allocate space for 
it in the local stack layout on at the compiling stage.

\myindex{\CLanguageElements!C99!variable length arrays}
\myindex{\CStandardLibrary!alloca()}

If you need an array of arbitrary size, allocate it by using \TT{malloc()}, then access the allocated memory block
as an array of variables of the type you need.


Or use the C99 standard feature \InSqBrackets{\CNineNineStd 6.7.5/2},
and it works like \emph{alloca()}~(\myref{alloca}) internally.


It's also possible to use garbage collecting libraries for C.

And there are also libraries supporting smart pointers for C++.

}
\RU{\subsection{Еще немного о массивах}

Теперь понятно, почему нельзя написать в исходном коде на \CCpp что-то вроде:


\begin{lstlisting}[style=customc]
void f(int size)
{
    int a[size];
...
};
\end{lstlisting}

Чтобы выделить место под массив в локальном стеке, 
компилятору нужно знать размер массива, чего он на стадии компиляции, 
разумеется, знать не может.


\myindex{\CLanguageElements!C99!variable length arrays}
\myindex{\CStandardLibrary!alloca()}
Если вам нужен массив произвольной длины, то выделите столько, сколько нужно, через \TT{malloc()}, 
а затем обращайтесь к выделенному блоку байт как к массиву того типа, который вам нужен.


Либо используйте возможность стандарта C99~ \InSqBrackets{\CNineNineStd 6.7.5/2},
и внутри это очень похоже на \emph{alloca()} (\myref{alloca}).


Для работы в с памятью, можно также воспользоваться библиотекой сборщика мусора в Си.

А для языка Си++ есть библиотеки с поддержкой умных указателей.


}
\DE{\subsection{Noch ein Wort zu Arrays}
Wir verstehen nun warum es nicht möglich ist etwas wie das Folgende in \CCpp Code zu schreiben:

\begin{lstlisting}[style=customc]
void f(int size)
{
    int a[size];
...
};
\end{lstlisting}
Das liegt daran, dass der Compiler die exakte Größe des Arrays zur Compilerzeit kennen muss, um Platz auf dem lokalen
Stack zu reservieren.

\myindex{\CLanguageElements!C99!variable length arrays}
\myindex{\CStandardLibrary!alloca()}
Wenn man ein Array beliebiger Größe benötigt, muss es über \TT{malloc()} angelegt werden und dann über den reservieren
Speicherblock als Arrays von Variablen des benötigten Typs angesprochen werden.

Oder man verwendet das C99 Standardfeature \InSqBrackets{\CNineNineStd 6.7.5/2}, dass intern wie
\emph{alloca()}~(\myref{alloca}) arbeitet.

Es ist auch möglich, C-Bibliotheken zu verwenden, die als Garbagecollector fungieren.
Des Weiteren gibt es auch Bibliotheken für C++, die intelligente Pointer unterstützen.}
\FR{\subsection{Encore un mot sur les tableaux}


Maintenant nous comprenons pourquoi il est impossible d'écrire quelque chose comme
ceci en code \CCpp:

\begin{lstlisting}[style=customc]
void f(int size)
{
    int a[size];
...
};
\end{lstlisting}


C'est simplement parce que le compilateur doit connaître la taille exacte du tableau
pour lui allouer de l'espace sur la pile locale lors de l'étape de compilation.

\myindex{\CLanguageElements!C99!variable length arrays}
\myindex{\CStandardLibrary!alloca()}

% TODO: améliorer
Si vous avez besoin d'un tableau de taille arbitraire, il faut l'allouer en utilisant
\TT{malloc()}, puis en accédant aux blocs de mémoire allouée comme un tableau de
variables du type dont vous avez besoin.


Ou utiliser la caractéristique du standart C99 \InSqBrackets{\CNineNineStd 6.7.5/2},
et qui fonctionne comme \emph{alloca()}~(\myref{alloca}) en interne.


Il est aussi possible d'utiliser des bibliothèques de ramasse-miettes pour C.

Et il y a aussi des bibliothèques supportant les pointeurs intelligents pour C++.

}
\JPN{\subsection{配列についてもう少し}

今や \CCpp のコードでこのように書き込むのが不可能なことを理解しています。

\begin{lstlisting}[style=customc]
void f(int size)
{
    int a[size];
...
};
\end{lstlisting}


コンパイラはコンパイル時にローカルスタックレイアウト上の場所を確保するために
正確な配列のサイズを知る必要があります。

\myindex{\CLanguageElements!C99!variable length arrays}
\myindex{\CStandardLibrary!alloca()}

配列の任意のサイズを必要とする場合、\TT{malloc()}を使用して確保し、そして確保したメモリブロックに
必要とする型の変数の配列としてアクセスします。

またはC99標準の機能\InSqBrackets{\CNineNineStd 6.7.5/2}を使用します。
内部で\IT{alloca()}~(\myref{alloca})を使用しているかのように働きます。

C用のガーベッジコレクションライブラリを使用することも可能です。

C++向けにスマートポインタをサポートするライブラリもあります。
}

\EN{\subsection{Array of pointers to strings}
\label{array_of_pointers_to_strings}

Here is an example for an array of pointers.

\lstinputlisting[caption=Get month name,label=get_month1,style=customc]{patterns/13_arrays/45_month_1D/month1_EN.c}

\subsubsection{x64}

\lstinputlisting[caption=\Optimizing MSVC 2013 x64,style=customasmx86]{patterns/13_arrays/45_month_1D/month1_MSVC_2013_x64_Ox.asm}

The code is very simple:

\begin{itemize}

\item
\myindex{x86!\Instructions!MOVSXD}

The first \INS{MOVSXD} instruction copies a 32-bit value from \ECX (where $month$ argument is passed) 
to \RAX with sign-extension (because the $month$ argument is of type \Tint).

The reason for the sign extension is that this 32-bit value is to be used in calculations
with other 64-bit values.

Hence, it has to be promoted to 64-bit%
\footnote{It is somewhat weird, but negative array index could be passed here as $month$
(negative array indices will have been explained later: \myref{negative_array_indices}).
And if this happens, the negative input value of \Tint type is sign-extended correctly 
and the corresponding element before table is picked. 
It is not going to work correctly without sign-extension.}.

\item
Then the address of the pointer table is loaded into \RCX.

\item
Finally, the input value ($month$) is multiplied by 8 and added to the address.
Indeed: we are in a 64-bit environment and all address (or pointers) require exactly 64 bits (or 8 bytes) 
for storage.
Hence, each table element is 8 bytes wide.
And that's why to pick a specific element, $month*8$ bytes has to be skipped from the start.
That's what \MOV does.
In addition, this instruction also loads the element at this address.
For 1, an element would be a pointer to a string that contains \q{February}, etc.

\end{itemize}

\Optimizing GCC 4.9 can do the job even better
\footnote{\q{0+} was left in the listing because GCC assembler output is not tidy enough to eliminate it.
It's \emph{displacement}, and it's zero here.}:

\begin{lstlisting}[caption=\Optimizing GCC 4.9 x64,style=customasmx86]
	movsx	rdi, edi
	mov	rax, QWORD PTR month1[0+rdi*8]
	ret
\end{lstlisting}

\myparagraph{32-bit MSVC}

Let's also compile it in the 32-bit MSVC compiler:

\lstinputlisting[caption=\Optimizing MSVC 2013 x86,style=customasmx86]{patterns/13_arrays/45_month_1D/month1_MSVC_2013_x86_Ox.asm}

The input value does not need to be extended to 64-bit value, so it is used as is.

And it's multiplied by 4, because the table elements are 32-bit (or 4 bytes) wide.

% FIXME1 move to another file
\subsubsection{32-bit ARM}

\myparagraph{ARM in ARM mode}

\lstinputlisting[caption=\OptimizingKeilVI (\ARMMode),style=customasmARM]{patterns/13_arrays/45_month_1D/month1_Keil_ARM_O3.s}

% TODO Fix R1s

The address of the table is loaded in R1.
\myindex{ARM!\Instructions!LDR}

All the rest is done using just one \LDR instruction.

Then input value $month$ is shifted left by 2 (which is the same as multiplying by 4), then added
to R1 (where the address of the table is) and then a table element is loaded from this address.

The 32-bit table element is loaded into R0 from the table.

\myparagraph{ARM in Thumb mode}

The code is mostly the same, but less dense, because the \LSL suffix cannot be specified in the \LDR instruction here:

\begin{lstlisting}[style=customasmARM]
get_month1 PROC
        LSLS     r0,r0,#2
        LDR      r1,|L0.64|
        LDR      r0,[r1,r0]
        BX       lr
        ENDP
\end{lstlisting}

\subsubsection{ARM64}

\lstinputlisting[caption=\Optimizing GCC 4.9 ARM64,style=customasmARM]{patterns/13_arrays/45_month_1D/month1_GCC49_ARM64_O3.s}

\myindex{ARM!\Instructions!ADRP/ADD pair}

The address of the table is loaded in X1 using \ADRP/\ADD pair.

Then corresponding element is picked using just one \LDR, which takes W0 
(the register where input argument $month$ is), shifts it 3 bits to the left (which is the same as multiplying by 8), 
sign-extends it (this is what \q{sxtw} suffix implies) and adds to X0.
Then the 64-bit value is loaded from the table into X0.

\subsubsection{MIPS}

\lstinputlisting[caption=\Optimizing GCC 4.4.5 (IDA),style=customasmMIPS]{patterns/13_arrays/45_month_1D/MIPS_O3_IDA_EN.lst}

\subsubsection{Array overflow}

Our function accepts values in the range of 0..11, but what if 12 is passed?
There is no element in table at this place.

So the function will load some value which happens to be there, and return it.

Soon after, some other function can try to get a text string from this address and may crash.

Let's compile the example in MSVC for win64 and open it in \IDA to see what the linker has placed after the table:

\lstinputlisting[caption=Executable file in IDA,style=customasmx86]{patterns/13_arrays/45_month_1D/MSVC2012_win64_1.lst}

Month names are came right after.

Our program is tiny, so there isn't much data to pack in the data segment, 
so it just the month names.
But it has to be noted that there might be really \emph{anything} that linker has decided to put by chance.

So what if 12 is passed to the function?
The 13th element will be returned.

Let's see how the CPU treats the bytes there as a 64-bit value:

\lstinputlisting[caption=Executable file in IDA,style=customasmx86]{patterns/13_arrays/45_month_1D/MSVC2012_win64_2.lst}

And this is 0x797261756E614A.

Soon after, some other function (presumably, one that processes strings) may try to read bytes at 
this address, expecting a C-string there.

Most likely it is about to crash, because this value doesn't look like a valid address.

\myparagraph{Array overflow protection}

\epigraph{If something can go wrong, it will}{Murphy's Law}

It's a bit naïve to expect that every programmer who use your function or library will never pass
an argument larger than 11.

There exists the philosophy that says \q{fail early and fail loudly} or \q{fail-fast}, 
which teaches to report problems as early as possible and halt.
\myindex{\CStandardLibrary!assert()}

One such method in \CCpp is assertions.

We can modify our program to fail if an incorrect value is passed:

\lstinputlisting[caption=assert() added,style=customc]{patterns/13_arrays/45_month_1D/month1_assert.c}

The assertion macro checks for valid values at every function start and fails if the expression is false.

\lstinputlisting[caption=\Optimizing MSVC 2013 x64,style=customasmx86]{patterns/13_arrays/45_month_1D/MSVC2013_x64_Ox_checked.asm}

In fact, assert() is not a function, but macro. It checks for a condition, then passes also the line number and file
name to another function which reports this information to the user.

Here we see that both file name and condition are encoded in UTF-16.
The line number is also passed (it's 29).

This mechanism is probably the same in all compilers.
Here is what GCC does:

\lstinputlisting[caption=\Optimizing GCC 4.9 x64,style=customasmx86]{patterns/13_arrays/45_month_1D/GCC491_x64_O3_checked.s}

So the macro in GCC also passes the function name for convenience.

Nothing is really free, and this is true for the sanitizing checks as well.

They make your program slower, especially if the assert() macros used in small time-critical functions.

So MSVC, for example, leaves the checks in debug builds, but in release builds they all disappear.
 
Microsoft \gls{Windows NT} kernels come in \q{checked} and \q{free} builds
\footnote{\href{http://go.yurichev.com/17259}{msdn.microsoft.com/en-us/library/windows/hardware/ff543450(v=vs.85).aspx}}.

The first has validation checks (hence, \q{checked}), the second one doesn't (hence, \q{free} of checks).

Of course, \q{checked} kernel works slower because of all these checks, so it is usually used only in debug sessions.

% FIXME: ARM? MIPS?

\subsubsection{Accessing specific character}

An array of pointers to strings can be accessed like this:

\lstinputlisting[style=customc]{patterns/13_arrays/45_month_1D/month2_EN.c}

\dots since \emph{month[3]} expression has a \emph{const char*} type.
And then, 5th character is taken from that expression by adding 4 bytes to its address.

By the way, arguments list passed to \emph{main()} function has the same data type:

\lstinputlisting[style=customc]{patterns/13_arrays/45_month_1D/argv_EN.c}

It's very important to understand, that, despite similar syntax,
this is different from two-dimensional arrays, which we will consider later.

Another important thing to notice: strings to be addressed must be encoded in a system, where each character occupies single
byte, like \ac{ASCII} and extended \ac{ASCII}.
UTF-8 wouldn't work here.

}
\RU{\subsection{Массив указателей на строки}
\label{array_of_pointers_to_strings}

Вот пример массива указателей.

\lstinputlisting[caption=Получить имя месяца,label=get_month1,style=customc]{patterns/13_arrays/45_month_1D/month1_RU.c}

\subsubsection{x64}

\lstinputlisting[caption=\Optimizing MSVC 2013 x64,style=customasmx86]{patterns/13_arrays/45_month_1D/month1_MSVC_2013_x64_Ox.asm}

Код очень простой:

\begin{itemize}

\item
\myindex{x86!\Instructions!MOVSXD}
Первая инструкция \INS{MOVSXD} копирует 32-битное значение из \ECX (где передается аргумент $month$)
в \RAX со знаковым расширением (потому что аргумент $month$ имеет тип \Tint).

Причина расширения в том, что это значение будет использоваться в вычислениях наряду с другими 64-битными
значениями.

Таким образом, оно должно быть расширено до 64-битного
\footnote{Это немного странная вещь, но отрицательный индекс массива может быть передан как $month$ 
(отрицательные индексы массивов будут рассмотрены позже: \myref{negative_array_indices}).
И если так будет, отрицательное значение типа \Tint будет расширено со знаком корректно
и соответствующий элемент перед таблицей будет выбран.
Всё это не будет корректно работать без знакового расширения.}.

\item
Затем адрес таблицы указателей загружается в \RCX.

\item
В конце концов, входное значение ($month$) умножается на 8 и прибавляется к адресу.
Действительно: мы в 64-битной среде и все адреса (или указатели) 
требуют для хранения именно 64 бита (или 8 байт).
Следовательно, каждый элемент таблицы имеет ширину в 8 байт.
Вот почему для выбора элемента под нужным номером нужно пропустить $month*8$ байт от начала.
Это то, что делает \MOV.
Эта инструкция также загружает элемент по этому адресу.
Для 1, элемент будет указателем на строку, содержащую \q{February}, итд.

\end{itemize}

\Optimizing GCC 4.9 может это сделать даже лучше
\footnote{В листинге осталось \q{0+}, потому что вывод ассемблера GCC не так скрупулёзен, чтобы убрать это.
Это \emph{displacement} и он здесь нулевой.}:

\begin{lstlisting}[caption=\Optimizing GCC 4.9 x64,style=customasmx86]
	movsx	rdi, edi
	mov	rax, QWORD PTR month1[0+rdi*8]
	ret
\end{lstlisting}

\myparagraph{32-bit MSVC}

Скомпилируем также в 32-битном компиляторе MSVC:

\lstinputlisting[caption=\Optimizing MSVC 2013 x86,style=customasmx86]{patterns/13_arrays/45_month_1D/month1_MSVC_2013_x86_Ox.asm}

Входное значение не нужно расширять до 64-битного значения, так что оно используется как есть.

И оно умножается на 4, потому что элементы таблицы имеют ширину 32 бита или 4 байта.

% FIXME1 move to another file
\subsubsection{32-битный ARM}

\myparagraph{ARM в режиме ARM}

\lstinputlisting[caption=\OptimizingKeilVI (\ARMMode),style=customasmARM]{patterns/13_arrays/45_month_1D/month1_Keil_ARM_O3.s}

% TODO Fix R1s
Адрес таблицы загружается в R1.

\myindex{ARM!\Instructions!LDR}
Всё остальное делается, используя только одну инструкцию \LDR.

Входное значение $month$ сдвигается влево на 2 (что тоже самое что и умножение на 4), это значение
прибавляется к R1 (где находится адрес таблицы) и затем элемент таблицы загружается по этому адресу.

32-битный элемент таблицы загружается в R0 из таблицы.

\myparagraph{ARM в режиме Thumb}

Код почти такой же, только менее плотный, потому что здесь, в инструкции \LDR, нельзя задать суффикс \LSL:

\begin{lstlisting}[style=customasmARM]
get_month1 PROC
        LSLS     r0,r0,#2
        LDR      r1,|L0.64|
        LDR      r0,[r1,r0]
        BX       lr
        ENDP
\end{lstlisting}

\subsubsection{ARM64}

\lstinputlisting[caption=\Optimizing GCC 4.9 ARM64,style=customasmARM]{patterns/13_arrays/45_month_1D/month1_GCC49_ARM64_O3.s}

\myindex{ARM!\Instructions!ADRP/ADD pair}
Адрес таблицы загружается в X1 используя пару \ADRP/\ADD.

Соответствующий элемент выбирается используя одну инструкцию \LDR, которая берет W0
(регистр, где находится значение входного аргумента $month$), сдвигает его на 3 бита влево
(что то же самое что и умножение на 8),
расширяет его, учитывая знак (это то, что означает суффикс \q{sxtw}) и прибавляет к X0.

Затем 64-битное значение загружается из таблицы в X0.

\subsubsection{MIPS}

\lstinputlisting[caption=\Optimizing GCC 4.4.5 (IDA),style=customasmMIPS]{patterns/13_arrays/45_month_1D/MIPS_O3_IDA_RU.lst}

\subsubsection{Переполнение массива}

Наша функция принимает значения в пределах 0..11, но что будет, если будет передано 12?

В таблице в этом месте нет элемента.
Так что функция загрузит какое-то значение, которое волею случая находится там, и вернет его.

Позже, какая-то другая функция попытается прочитать текстовую строку по этому адресу и, возможно, упадет.

Скомпилируем этот пример в MSVC для win64 и откроем его в \IDA чтобы посмотреть, что линкер расположил
после таблицы:

\lstinputlisting[caption=Исполняемый файл в IDA,style=customasmx86]{patterns/13_arrays/45_month_1D/MSVC2012_win64_1.lst}

Имена месяцев идут сразу после.
Наша программа все-таки крошечная,
так что здесь не так уж много данных (всего лишь названия месяцев) для расположения их в сегменте данных.

Но нужно заметить, что там может быть действительно \emph{что угодно}, что линкер решит там расположить, случайным образом.%

Так что будет если 12 будет передано в функцию?
Вернется 13-й элемент таблицы.
Посмотрим, как CPU обходится с байтами как с 64-битным значением:

\lstinputlisting[caption=Исполняемый файл в IDA,style=customasmx86]{patterns/13_arrays/45_month_1D/MSVC2012_win64_2.lst}

И это 0x797261756E614A.
После этого, какая-то другая функция (вероятно, работающая со строками) попытается загружать байты
по этому адресу, ожидая найти там Си-строку.

И скорее всего упадет, потому что это значение не выглядит как действительный адрес.

\myparagraph{Защита от переполнения массива}

\epigraph{Если какая-нибудь неприятность может случиться, она случается}{Закон Мерфи}

Немного наивно ожидать что всякий программист, кто будет использовать вашу функцию или библиотеку,
никогда не передаст аргумент больше 11.

Существует также хорошая философия \q{fail early and fail loudly} или \q{fail-fast},
которая учит сообщать об ошибках как можно раньше и останавливаться.

\myindex{\CStandardLibrary!assert()}
Один из таких методов в \CCpp это макрос assert().

Мы можем немного изменить нашу программу, чтобы она падала при передаче неверного значения:

\lstinputlisting[caption=assert() добавлен,style=customc]{patterns/13_arrays/45_month_1D/month1_assert.c}

Макрос будет проверять на верные значения во время каждого старта функции и падать если выражение возвращает false.

\lstinputlisting[caption=\Optimizing MSVC 2013 x64,style=customasmx86]{patterns/13_arrays/45_month_1D/MSVC2013_x64_Ox_checked.asm}

На самом деле, assert() это не функция, а макрос. Он проверяет условие и передает также номер строки и название
файла в другую функцию, которая покажет эту информацию пользователю.

Мы видим, что здесь и имя файла и выражение закодировано в UTF-16.

Номер строки также передается (это 29).

Этот механизм, пожалуй, одинаковый во всех компиляторах.

Вот что делает GCC:

\lstinputlisting[caption=\Optimizing GCC 4.9 x64,style=customasmx86]{patterns/13_arrays/45_month_1D/GCC491_x64_O3_checked.s}

Так что макрос в GCC также передает и имя функции, для удобства.

Ничего не бывает бесплатным и проверки на корректность тоже.

Это может замедлить работу вашей программы, особенно если макрос assert() используется в маленькой
критичной ко времени функции.

Так что, например, MSVC оставляет проверки в отладочных сборках, но в окончательных сборках они исчезают.
 
Ядра Microsoft \gls{Windows NT} также идут в виде сборок \q{checked} и \q{free}
\footnote{\href{http://go.yurichev.com/17259}{msdn.microsoft.com/en-us/library/windows/hardware/ff543450(v=vs.85).aspx}}.
В первых есть проверки на корректность аргументов (отсюда \q{checked}), а во вторых~--- нет (отсюда \q{free},
т.е.~\q{свободные} от проверок).

Разумеется, \q{checked}-ядро работает медленнее из-за всех этих проверок, поэтому его обычно используют только на время отладки драйверов, либо самого ядра.

% FIXME: ARM? MIPS?

\subsubsection{Доступ к определенному символу}

К массиву указателей на строки можно обращаться так:

\lstinputlisting[style=customc]{patterns/13_arrays/45_month_1D/month2_RU.c}

\dots так как, выражение \emph{month[3]} имеет тип \emph{const char*}.
И затем, 5-й символ берется из этого выражения прибавлением 4-и байт к его адресу.

Кстати, список аргументов передаваемый в ф-цию \emph{main()} имеет такой же тип:

\lstinputlisting[style=customc]{patterns/13_arrays/45_month_1D/argv_RU.c}

Очень важно понимать, что не смотря на одинаковый синтаксис,
всё это отличается от двухмерных массивов, которые мы будем рассматриать позже.

Еще одна важная вещь, которую нужно отметить: адресуемые строки должны быть закодированы в системе, в которой каждый
символ занимает один байт, как \ac{ASCII} и расширенная \ac{ASCII}.
UTF-8 здесь не будет работать.

}
\DE{\subsubsection{Struct als Menge von Werten}
Um zu veranschaulichen, dass ein struct nur eine Menge von nebeneinanderliegenden Variablen ist, überarbeiten wir unser
Beispiel, indem wir auf die Definition des \emph{tm} structs schauen:\lstref{struct_tm}.

\lstinputlisting[style=customc]{patterns/15_structs/3_tm_linux/as_array/GCC_tm2.c}

\myindex{\CStandardLibrary!localtime\_r()}
Der Pointer auf das Feld \TT{tm\_sec} wird nach \TT{localtime\_r} übergeben, d.h. an das erste Element des structs.

Der Compiler warnt uns:

\begin{lstlisting}[caption=GCC 4.7.3]
GCC_tm2.c: In function 'main':
GCC_tm2.c:11:5: warning: passing argument 2 of 'localtime_r' from incompatible pointer type [enabled by default]
In file included from GCC_tm2.c:2:0:
/usr/include/time.h:59:12: note: expected 'struct tm *' but argument is of type 'int *'
\end{lstlisting}

Trotzdem erzeugt er folgenden Code:

\lstinputlisting[caption=GCC 4.7.3,style=customasmx86]{patterns/15_structs/3_tm_linux/as_array/GCC_tm2.asm}
Dieser Code ist zum vorherigen identisch und es ist unmöglich zu sagen, ob es sich im originalen Quellcode um ein struct
oder nur um eine Menge von Variablen handelt.

Es funktioniert also, ist aber in der Praxis nicht empfehlenswert. 

Nicht optimierende Compiler legen normalerweise Variablen auf dem lokalen Stack in der Reihenfolge an, in der sie in der
Funktion deklariert wurden.

Ein Garantie dafür gibt es freilich nicht.

Andere Compiler könnten an dieser Stelle übrigens davor warnen, dass die Variablen \TT{tm\_year}, \TT{tm\_mon}, \TT{tm\_mday},
\TT{tm\_hour}, \TT{tm\_min} - nicht aber \TT{tm\_sec} - ohne Initialisierung verwendet werden.

Der Compiler weiß nicht, dass diese durch die Funktion \TT{localtime\_r()} befüllt werden.

Wir haben dieses Beispiel ausgewählt, da alle Felder im struct vom Typ \Tint sind.

Es würde nicht funktionieren, wenn die Felder 16 Bit (\TT{WORD}) groß wären, wie im Beispiel des \TT{SYSTEMTIME}
structs---\TT{GetSystemTime()} würde sie falsch befüllen (da die lokalen Variablen auf 32-Bit-Grenzen angeordnet sind).
Mehr dazu im folgenden Abschnitt: \q{\StructurePackingSectionName} (\myref{structure_packing}).

Ein struct ist also nichts als eine Menge von an einer Stelle gespeicherten Variablen.
Man kan sagen, dass das struct ein Befehl an den Compiler ist, diese Variablen an einer Stelle zu halten.
In ganz frühen Versionen von C (vor 1972) gab es übrigens gar keine structs \RitchieDevC.

Dieses Beispiel wird nicht im Debugger gezeigt, da es dem gerade gezeigten entspricht.

\subsubsection{Struct als Array aus 32-Bit-Worten}

\lstinputlisting[style=customc]{patterns/15_structs/3_tm_linux/as_array/GCC_tm3.c}
Wir können einen Pointer auf ein struct in ein Array aus \Tint{}s casten und es funktioniert.
Wir lassen dieses Beispiel zur Systemzeit 23:51:45 26-July-2014 laufen.

\begin{lstlisting}[label=GCC_tm3_output]
0x0000002D (45)
0x00000033 (51)
0x00000017 (23)
0x0000001A (26)
0x00000006 (6)
0x00000072 (114)
0x00000006 (6)
0x000000CE (206)
0x00000001 (1)
\end{lstlisting}
Die Variablen sind hier in der gleichen Reihenfolge, in der die in der Definition des structs aufgezählt
werden:\myref{struct_tm}.

Hier ist der erzeugte Code:

\lstinputlisting[caption=\Optimizing GCC
4.8.1,style=customasmx86]{patterns/15_structs/3_tm_linux/as_array/GCC_tm3_DE.lst}
Tatsächlich: der Platz auf dem lokalen Stack wird zuerst wie in struct und dann wie ein Array behandelt.

Es ist sogar möglich, die Felder des structs über diesen Pointer zu verändern.

Und wiederum ist es zweifellos ein seltsamer Weg die Dinge umzusetzen; er ist für produktiven Code definitiv nicht
empfehlenswert.

\mysubparagraph{\Exercise}
Versuchen Sie als Übung die Monatsnummer zu verändern (um 1 zu erhöhen), indem Sie das struct wie ein Array behandeln.

\subsubsection{Struct als Bytearray}
Wir können sogar noch weiter gehen. Casten wir den Pointer zu einem Bytearray und ziehen einen Dump:

\lstinputlisting[style=customc]{patterns/15_structs/3_tm_linux/as_array/GCC_tm4.c}

\begin{lstlisting}
0x2D 0x00 0x00 0x00 
0x33 0x00 0x00 0x00 
0x17 0x00 0x00 0x00 
0x1A 0x00 0x00 0x00 
0x06 0x00 0x00 0x00 
0x72 0x00 0x00 0x00 
0x06 0x00 0x00 0x00 
0xCE 0x00 0x00 0x00 
0x01 0x00 0x00 0x00 
\end{lstlisting}
Wir haben dieses Beispiel auch zur Systemzeit 23:51:45 26-July-2014 ausgeführt
\footnote{Datum und Uhrzeit sind zu Demonstrationszwecken identisch. Die Bytewerte sind modifiziert.}.
Die Werte sind genau dieselben wie im vorherigen Dump(\myref{GCC_tm3_output}) und natürlich steht das LSB vorne, da es
sich um eine Little-Endian-Architektur handelt(\myref{sec:endianness}). 

\lstinputlisting[caption=\Optimizing GCC
4.8.1,style=customasmx86]{patterns/15_structs/3_tm_linux/as_array/GCC_tm4_DE.lst}
}
\FR{\subsection{Tableau de pointeurs sur des chaînes}
\label{array_of_pointers_to_strings}

Voici un exemple de tableau de pointeurs.\footnote{NDT: attention à l'encodage des
fichiers, en ASCII ou en ISO-8859, un caractère occupe un octet, alors qu'en UTF-8,
notamment, il peut en occuper plusieurs. Par exemple, 'û' est codé \$fb (1 octet)
en ISO-8859 et \$c3\$bb (2 octets) en UTF-8. J'ai donc volontairement mis des caractères
non accentués dans le code.}

\lstinputlisting[caption=Prendre le nom du mois,label=get_month1,style=customc]{patterns/13_arrays/45_month_1D/month1_FR.c}

\subsubsection{x64}

\lstinputlisting[caption=MSVC 2013 \Optimizing x64,style=customasmx86]{patterns/13_arrays/45_month_1D/month1_MSVC_2013_x64_Ox.asm}

Le code est très simple:

\begin{itemize}

\item
\myindex{x86!\Instructions!MOVSXD}

La première instruction \INS{MOVSXD} copie une valeur 32-bit depuis \ECX (où l'argument
$month$ est passé) dans \RAX avec extension du signe (car l'argument $month$ est
de type \Tint).

La raison de l'extension du signe est que cette valeur 32-bit va être utilisée dans
des calculs avec d'autres valeurs 64-bit.

C'est pourquoi il doit être étendu à 64-bit\footnote{C'est parfois bizarre, mais des
indices négatifs de tableau peuvent être passés par $month$ (les indices négatifs
de tableaux sont expliqués plus loin: \myref{negative_array_indices}).
Et si cela arrive, la valeur entrée négative de type \Tint est étendue correctement
et l'élément correspondant avant le tableau est sélectionné.
Ça ne fonctionnera pas correctement sans l'extension du signe.}.

\item
Ensuite l'adresse du pointeur de la table est chargée dans \RCX.

\item
Enfin, la valeur d'entrée ($month$) est multipliée par 8 et ajoutée à l'adresse.
Effectivement: nous sommes dans un environnement 64-bit et toutes les adresses (ou
pointeurs) nécessitent exactement 64 bits (ou 8 octets) pour être stockées.
C'est pourquoi chaque élément de la table a une taille de 8 octets.
Et c'est pourquoi pour prendre un élément spécifique, $month*8$ octets doivent être
passés depuis le début.
C'est ce que fait \MOV.
De plus, cette instruction charge également l'élément à cette adresse.
Pour 1, l'élément sera un pointeur sur la chaîne qui contient \q{février}, etc.

\end{itemize}

GCC 4.9 \Optimizing peut faire encore mieux\footnote{\q{0+} a été laissé dans le
listing car la sortie de l'assembleur GCC n'est pas assez soignée pour l'éliminer.
C'est un \emph{déplacement}, et il vaut zéro ici.}:

\begin{lstlisting}[caption=GCC 4.9 \Optimizing x64,style=customasmx86]
	movsx	rdi, edi
	mov	rax, QWORD PTR month1[0+rdi*8]
	ret
\end{lstlisting}

\myparagraph{MSVC 32-bit}

Compilons-le aussi avec le compilateur MSVC 32-bit:

\lstinputlisting[caption=MSVC 2013 \Optimizing x86,style=customasmx86]{patterns/13_arrays/45_month_1D/month1_MSVC_2013_x86_Ox.asm}

La valeur en entrée n'a pas besoin  d'être étendue sur 64-bit, donc elle est utilisée
telle quelle.

Et elle est multipliée par 4, car les éléments de la table sont larges de 32-bit
(ou 4 octets).

% FIXME1 move to another file
\subsubsection{ARM 32-bit}

\myparagraph{ARM en mode ARM}

\lstinputlisting[caption=\OptimizingKeilVI (\ARMMode),style=customasmARM]{patterns/13_arrays/45_month_1D/month1_Keil_ARM_O3.s}

% TODO Fix R1s

L'adresse de la table est chargée en R1.
\myindex{ARM!\Instructions!LDR}

Tout le reste est effectué en utilisant juste une instruction \LDR.

Puis la valeur en entrée est décalée de 2 vers la gauche (ce qui est la même chose
que multiplier par 4), puis ajoutée à R1 (où se trouve l'adresse de la table) et
enfin un élément de la table est chargé depuis cette adresse.

L'élément 32-bit de la table est chargé dans R1 depuis la table.

\myparagraph{ARM en mode Thumb}

Le code est essentiellement le même, mais moins dense, car le suffixe \LSL ne peut
pas être spécifié dans l'instruction \LDR ici:

\begin{lstlisting}[style=customasmARM]
get_month1 PROC
        LSLS     r0,r0,#2
        LDR      r1,|L0.64|
        LDR      r0,[r1,r0]
        BX       lr
        ENDP
\end{lstlisting}

\subsubsection{ARM64}

\lstinputlisting[caption=GCC 4.9 \Optimizing ARM64,style=customasmARM]{patterns/13_arrays/45_month_1D/month1_GCC49_ARM64_O3.s}

\myindex{ARM!\Instructions!ADRP/ADD pair}

L'adresse de la table est chargée dans X1 en utilisant la paire \ADRP/\ADD.

Puis l'élément correspondant est choisi dans la table en utilisant seulement un \LDR,
qui prend W0 (le registre où l'argument d'entrée $month$ se trouve), le décale de
3 bits vers la gauche (ce qui est la même chose que de le multiplier par 8), étend
son signe (c'est ce que le suffixe \q{sxtw} implique) et l'ajoute à X0.
Enfin la valeur 64-bit est chargée depuis la table dans X0.

\subsubsection{MIPS}

\lstinputlisting[caption=GCC 4.4.5 \Optimizing (IDA),style=customasmMIPS]{patterns/13_arrays/45_month_1D/MIPS_O3_IDA_FR.lst}

\subsubsection{Débordement de tableau}

Notre fonction accepte des valeurs dans l'intervalle 0..11, mais que se passe-t-il
si 12 est passé?
Il n'y a pas d'élément dans la table à cet endroit.

Donc la fonction va charger la valeur qui se trouve là, et la renvoyer.

Peu après, une autre fonction pourrait essayer de lire une chaîne de texte depuis
cette adresse et pourrait planter.

Compilons l'exemple dans MSVC pour win64 et ouvrons le dans \IDA pour voir ce que
l'éditeur de lien à stocker après la table:

\lstinputlisting[caption=Fichier exécutable dans IDA,style=customasmx86]{patterns/13_arrays/45_month_1D/MSVC2012_win64_1.lst}

Les noms des mois se trouvent juste après.

Notre programme est minuscule, il n'y a donc pas beaucoup de données à mettre dans
le segment de données, juste les noms des mois.
Mais il faut noter qu'il peut y avoir ici vraiment \emph{n'importe quoi} que l'éditeur
de lien aurait décidé d'y mettre.

Donc, que se passe-t-il si nous passons 12 à la fonction?
Le 13ème élément va être renvoyé.

Voyons comment le CPU traite les octets en une valeur 64-bit:

\lstinputlisting[caption=Fichier exécutable dans IDA,style=customasmx86]{patterns/13_arrays/45_month_1D/MSVC2012_win64_2.lst}

Et c'est 0x797261756E614A.

Peu après, une autre fonction (supposons, une qui traite des chaînes) pourrait essayer
de lire des octets à cette adresse, y attendant une chaîne-C.

Plus probablement, ça planterait, car cette valeur ne ressemble pas à une adresse
valide.

\myparagraph{Protection contre les débordements de tampon}

\epigraph{Si quelque chose peut mal tourner, ça tournera mal}{Loi de Murphy}

Il est un peu naïf de s'attendre à ce que chaque programmeur qui utilisera votre
fonction ou votre bibliothèque ne passera jamais un argument plus grand que 11.

Il existe une philosophie qui dit \q{échouer tôt et échouer bruyamment} ou \q{échouer rapidement},
qui enseigne de remonter les problèmes le plus tôt possible et d'arrêter.
\myindex{\CStandardLibrary!assert()}

Une telle méthode en \CCpp est les assertions.

Nous pouvons modifier notre programme pour qu'il échoue si une valeur incorrecte
est passée:

\lstinputlisting[caption=assert() ajoutée,style=customc]{patterns/13_arrays/45_month_1D/month1_assert.c}

La macro assertion vérifie que la validité des valeurs à chaque démarrage de fonction
et échoue si l'expression est fausse.

\lstinputlisting[caption=MSVC 2013 x64 \Optimizing,style=customasmx86]{patterns/13_arrays/45_month_1D/MSVC2013_x64_Ox_checked.asm}

En fait, assert() n'est pas une fonction, mais une macro. Elle teste une condition,
puis passe le numéro de ligne et le nom du fichier à une autre fonction qui rapporte
cette information à l'utilisateur.

Ici nous voyons qu'à la fois le nom du fichier et la condition sont encodés en UTF-16.
Le numéro de ligne est aussi passé (c'est 29).

Le mécanisme est sans doute le même dans tous les compilateurs.
Voici ce que fait GCC:

\lstinputlisting[caption=GCC 4.9 x64 \Optimizing,style=customasmx86]{patterns/13_arrays/45_month_1D/GCC491_x64_O3_checked.s}

Donc la macro dans GCC passe aussi le nom de la fonction par commodité.

Rien n'est vraiment gratuit, et c'est également vrai pour les tests de validité.

Ils rendent votre programme plus lent, en particulier si la macro assert() est utilisée
dans des petites fonctions à durée critique.

Donc MSCV, par exemple, laisse les tests dans les compilations debug, mais ils disparaissent
dans celles de release.

Les noyaux de Microsoft \gls{Windows NT} existent en versions \q{checked} et \q{free}.
\footnote{\href{http://go.yurichev.com/17259}{msdn.microsoft.com/en-us/library/windows/hardware/ff543450(v=vs.85).aspx}}.

Le premier a des tests de validation (d'où, \q{checked}), le second n'en a pas (d'où, \q{free/libre} de tests).

Bien sûr, le noyau \q{checked} fonctionne plus lentement à cause de ces tests, donc
il n'est utilisé que pour des sessions de debug.

% FIXME: ARM? MIPS?

\subsubsection{Accéder à un caractère spécifique}

Un tableau de pointeurs sur des chaînes peut être accédé comme ceci\footnote{Lisez
l'avertissement dans la NDT ici \myref{array_of_pointers_to_strings}}:

\lstinputlisting[style=customc]{patterns/13_arrays/45_month_1D/month2_FR.c}

\dots puisque l'expression \emph{month[3]} a un type \emph{const char*}.
Et donc, le 5ème caractère est extrait de cette expression en ajoutant 4 octets à
cette adresse.

À propos, la liste d'arguments passée à la fonction \emph{main()} a le même type de
données:

\lstinputlisting[style=customc]{patterns/13_arrays/45_month_1D/argv_FR.c}

Il est très important de comprendre, que, malgré la syntaxe similaire, c'est différent
d'un tableau à deux dimensions, dont nous allons parler plus tard.

Une autre chose importante à noter: les chaînes considérées doivent être encodées
dans un système où chaque caractère occupe un seul octet, comme l'\ac{ASCII} ou l'\ac{ASCII}
étendu.
UTF-8 ne fonctionnera pas ici.

}
\JPN{\subsection{文字列へのポインタの配列}
\label{array_of_pointers_to_strings}

ここでは、ポインタの配列の例を示します。

\lstinputlisting[caption=Get month name,label=get_month1,style=customc]{patterns/13_arrays/45_month_1D/month1_JPN.c}

\subsubsection{x64}

\lstinputlisting[caption=\Optimizing MSVC 2013 x64,style=customasmx86]{patterns/13_arrays/45_month_1D/month1_MSVC_2013_x64_Ox.asm}

コードはとても単純です。

\begin{itemize}

\item
\myindex{x86!\Instructions!MOVSXD}

最初の\INS{MOVSXD}命令は、 \ECX ( $month$ 引数が渡される)から32ビットの値を
符号拡張付きの \RAX ( $month$ 引数は \Tint 型なので)にコピーします。

符号拡張の理由は、この32ビット値が他の64ビット値との計算に使用されるためです。

したがって、64ビット142に昇格させる必要があります。%
\footnote{やや奇妙ですが、負の配列インデックスはここで $month$ として渡すことができます
(負の配列インデックスは後で説明します:\myref{negative_array_indices})。 
これが起こると、 \Tint 型の負の入力値が正しく符号拡張され、
テーブルの前の対応する要素が選択されます。 符号拡張なしでは正しく動作しません。}

\item
次にポインタテーブルのアドレスが \RCX にロードされます。

\item
最後に、入力値($month$)に8を掛けてアドレスに加算します。 
確かに:私たちは64ビット環境にあり、すべてのアドレス(またはポインタ)は正確に64ビット(または8バイト)の
記憶域を必要とします。 
したがって、各テーブル要素は8バイト幅です。 
それで、なぜ特定の要素 $month*8$ をスキップする必要があるのでしょうか。
これが \MOV が行うことです。 
さらに、この命令はこのアドレスの要素もロードします。 
1の場合、要素は\q{February}などを含む文字列へのポインタになります。

\end{itemize}

\Optimizing GCC 4.9はもっとよく仕事をこなします。
\footnote{GCCアセンブラ出力が排除するのに十分なほど整っていないので、\q{0+}がリストに残っていました。 
それは\IT{変位}であり、ここではゼロです。}

\begin{lstlisting}[caption=\Optimizing GCC 4.9 x64,style=customasmx86]
	movsx	rdi, edi
	mov	rax, QWORD PTR month1[0+rdi*8]
	ret
\end{lstlisting}

\myparagraph{32ビットMSVC}

32ビットMSVCコンパイラでもコンパイルしてみましょう。

\lstinputlisting[caption=\Optimizing MSVC 2013 x86,style=customasmx86]{patterns/13_arrays/45_month_1D/month1_MSVC_2013_x86_Ox.asm}

入力値は64ビットに拡張する必要がないので、そのまま使われます。

そして4倍されます。テーブル要素が32ビット(または4バイト)幅だからです。

% FIXME1 move to another file
\subsubsection{32ビット ARM}

\myparagraph{ARMモードでのARM}

\lstinputlisting[caption=\OptimizingKeilVI (\ARMMode),style=customasmARM]{patterns/13_arrays/45_month_1D/month1_Keil_ARM_O3.s}

% TODO Fix R1s

テーブルのアドレスはR1にロードされます。
\myindex{ARM!\Instructions!LDR}

残りのすべては \LDR 命令1つだけを使って行われます。

入力値 $month$ は2ビット左シフトします(4倍するのと同じです)。それから
R1に加えらえます(テーブルのアドレスの場所)。そしてテーブル要素はこのアドレスからロードされます。

32ビットテーブル要素はテーブルからR0にロードされます。

\myparagraph{ThumbモードでのARM}

コードはほとんど同じですが、より密度が低いです。 \LSL サフィックスは \LDR 命令では特定できないからです。

\begin{lstlisting}[style=customasmARM]
get_month1 PROC
        LSLS     r0,r0,#2
        LDR      r1,|L0.64|
        LDR      r0,[r1,r0]
        BX       lr
        ENDP
\end{lstlisting}

\subsubsection{ARM64}

\lstinputlisting[caption=\Optimizing GCC 4.9 ARM64,style=customasmARM]{patterns/13_arrays/45_month_1D/month1_GCC49_ARM64_O3.s}

\myindex{ARM!\Instructions!ADRP/ADD pair}

テーブルのアドレスは \ADRP/\ADD 命令の組を使ってX1にロードされます。

それから付随する要素 \LDR を使って選ばれて、W0を取ります(入力引数 $month$ の場所のレジスタ)。
左に3ビットシフトします(8倍するのと同じです)。
符号拡張し(\q{sxtw}サフィックスが暗示しています)、X0に加算します。
それから64ビット値がテーブルからX0にロードされます。

\subsubsection{MIPS}

\lstinputlisting[caption=\Optimizing GCC 4.4.5 (IDA),style=customasmMIPS]{patterns/13_arrays/45_month_1D/MIPS_O3_IDA_JPN.lst}

\subsubsection{配列オーバーフロー}

関数は0~11の範囲の値を受け付けますが、12は通すでしょうか?
テーブルにはその場所の要素はありません。

なので関数はそこにたまたまある値をロードしてリターンします。

すぐ後で、他の関数がこのアドレスからテキスト文字列を取得しようとしてクラッシュするかもしれません。

例をwin64用としてMSVCでコンパイルして、テーブルの後にリンカーが何を配置したのかを \IDA で見てみましょう。

\lstinputlisting[caption=IDAでの実行可能ファイル,style=customasmx86]{patterns/13_arrays/45_month_1D/MSVC2012_win64_1.lst}

月の名前がそのあとに来ています。

プログラムは小さいので、データセグメントにパックされるデータは多くありません。
だから単に次の名前が来ています。
しかし注意すべきはリンカーが配置するように決定するのは\IT{どんなものも}ありえます。

だからもし12が関数に渡されたら?
13番目の要素がリターンされます。

CPUがそこにあるバイトを64ビットの値としてどのように扱うかをみてみましょう。

\lstinputlisting[caption=IDAでの実行可能ファイル,style=customasmx86]{patterns/13_arrays/45_month_1D/MSVC2012_win64_2.lst}

0x797261756E614Aです。

すぐ後で、他の関数(おそらく文字列を扱う関数)がこのアドレスでバイトを読み込もうとすると、
C言語の文字列を期待します。

十中八九、クラッシュします。この値は有効なアドレスのようには見えないからです。

\myparagraph{配列オーバーフロー保護}

\epigraph{失敗する可能性のあるものは、失敗する。}{マーフィーの法則}

あなたの関数を使用するプログラマはみな11より大きな値を引数として渡さないと
期待するのはちょっとナイーブです。

問題をできるだけ早く報告し停止することを意味する\q{fail early and fail loudly}
または\q{早く失敗する}という哲学があります。

\myindex{\CStandardLibrary!assert()}

そのような方法の1つに \CCpp のassertionがあります。

不正な値が通ってきたら、失敗するようにプログラムを変更できます。

\lstinputlisting[caption=assert()を追加,style=customc]{patterns/13_arrays/45_month_1D/month1_assert.c}

アサーションマクロは関数の開始時に妥当な値かチェックし、式が偽の場合に失敗します。

\lstinputlisting[caption=\Optimizing MSVC 2013 x64,style=customasmx86]{patterns/13_arrays/45_month_1D/MSVC2013_x64_Ox_checked.asm}

実際、assert() は関数ではなくマクロです。条件をチェックし、
行数とファイル名を他の関数に渡してユーザに情報を報告します。

ファイル名と条件の両方がUTF-16でエンコードされています。
行数も渡されます(29です)。

このメカニズムはおそらくすべてのコンパイラで同じです。
GCCはこのようにします。

\lstinputlisting[caption=\Optimizing GCC 4.9 x64,style=customasmx86]{patterns/13_arrays/45_month_1D/GCC491_x64_O3_checked.s}

GCCのマクロは利便性のために関数名も渡します。

何事もただではできませんが、サニタイズチェックもこれと同様です。

それはプログラムを遅くしますが、特にassert()マクロが小さなタイムクリティカルな関数で使用されると遅くなります。

なのでMSVCでは、例えばデバッグビルドではチェックを残し、リリースビルドでは取り除いたりします。
 
マイクロソフト\gls{Windows NT}カーネルは\q{チェックされた}と\q{フリー}ビルドです。
\footnote{\href{http://go.yurichev.com/17259}{msdn.microsoft.com/en-us/library/windows/hardware/ff543450(v=vs.85).aspx}}.

最初のものは妥当性チェック(\q{チェックされた}なので)があり、もう一つはチェックしていません(チェックが\q{フリー}なので)。

もちろん、 \q{チェックされた}カーネルはこれらのチェックのために遅く動作するので、通常はデバッグセッションでのみ使用されます。

% FIXME: ARM? MIPS?

\subsubsection{特定の文字へのアクセス}

文字列へのポインタの配列はこのようにアクセスできます。

\lstinputlisting[style=customc]{patterns/13_arrays/45_month_1D/month2_JPN.c}

\dots \IT{month[3]}式は\IT{const char*}型をもつので、
5番目の文字列はこのアドレスに4バイトを足した式から取得します。

さて、\IT{main()}関数に渡された引数リストは同じデータ型を持ちます。

\lstinputlisting[style=customc]{patterns/13_arrays/45_month_1D/argv_JPN.c}

似た構文ですが、2次元配列とは異なることを理解することが非常に重要です。
これについては後で検討します。

もう1つの重要なことに注意してください。アドレス指定される文字列は、各文字が\ac{ASCII}や拡張\ac{ASCII}のように1バイトを占めるシステムで
エンコードされなければなりません。 
UTF-8はここでは動作しません。
}

\EN{\subsection{Multidimensional arrays}

Internally, a multidimensional array is essentially the same thing as a linear array.

Since the computer memory is linear, it is an one-dimensional array.
For convenience, this multi-dimensional array can be easily represented as one-dimensional.

For example, this is how the elements of the 3x4 array are placed in one-dimensional array of 12 cells:

% TODO FIXME not clear. First, horizontal would be better. Second, why two columns?
% I'd first show 3x4 with numbered elements (e.g. 32-bit ints) in colored lines,
% then linear with the same numbered elements (and colored blocks)
% then linear with addresses (offsets) - assuming let say 32-bit ints.
\begin{table}[H]
\centering
\begin{tabular}{ | l | l | }
\hline
Offset in memory & array element \\
\hline
0 & [0][0] \\
\hline
1 & [0][1] \\
\hline
2 & [0][2] \\
\hline
3 & [0][3] \\
\hline
4 & [1][0] \\
\hline
5 & [1][1] \\
\hline
6 & [1][2] \\
\hline
7 & [1][3] \\
\hline
8 & [2][0] \\
\hline
9 & [2][1] \\
\hline
10 & [2][2] \\
\hline
11 & [2][3] \\
\hline
\end{tabular}
\caption{Two-dimensional array represented in memory as one-dimensional}
\end{table}

Here is how each cell of 3*4 array are placed in memory:

% TODO coordinates. TikZ?
\begin{table}[H]
\centering
\begin{tabular}{ | l | l | l | l | }
\hline                        
0 & 1 & 2 & 3 \\
\hline  
4 & 5 & 6 & 7 \\
\hline  
8 & 9 & 10 & 11 \\
\hline  
\end{tabular}
\caption{Memory addresses of each cell of two-dimensional array}
\end{table}

\myindex{row-major order}

So, in order to calculate the address of the element we need, we first multiply the first index by
4 (array width) and then add the second index.
That's called \emph{row-major order}, 
and this method of array and matrix representation is used in at least \CCpp and Python. 
The term \emph{row-major order} 
in plain English language means: \q{first, write the elements of the first row, then the second row \dots 
and finally the elements of the last row}.

\myindex{column-major order}
\myindex{Fortran}
Another method for representation is called \emph{column-major order} (the array indices are used in reverse order) 
and it is used at least in Fortran, MATLAB and R. 
\emph{column-major order} term in plain English language means: \q{first, write the elements of the first column, then the second column \dots
and finally the elements of the last column}.

Which method is better?

In general, in terms of performance and cache memory, 
the best scheme for data organization is the one,
in which the elements are accessed sequentially.

So if your function accesses data per row, \emph{row-major order} is better, and vice versa.

% subsubsections
\input{patterns/13_arrays/5_multidimensional/2D_EN}
\subsubsection{Access two-dimensional array as one-dimensional}

We can be easily assured that it's possible to access a two-dimensional array as one-dimensional array in at least two ways:

\lstinputlisting[style=customc]{patterns/13_arrays/5_multidimensional/2D_as_1D_EN.c}

Compile\footnote{This program is to be compiled as a C program, not C++, save it to a file with .c extension to compile it using MSVC}
and run it: it shows correct values.

What MSVC 2013 did is fascinating, all three routines are just the same!

\lstinputlisting[caption=\Optimizing MSVC 2013 x64,style=customasmx86]{patterns/13_arrays/5_multidimensional/2D_as_1D_MSVC_2013_Ox_x64_EN.asm}

GCC also generates equivalent routines, but slightly different:

\lstinputlisting[caption=\Optimizing GCC 4.9 x64,style=customasmx86]{patterns/13_arrays/5_multidimensional/2D_as_1D_GCC49_x64_O3_EN.s}


\subsubsection{Three-dimensional array example}

It's the same for multidimensional arrays.

Now we are going to work with an array of type \Tint: each element requires 4 bytes in memory.

Let's see:

\lstinputlisting[caption=simple example,style=customc]{patterns/13_arrays/5_multidimensional/multi.c}

\myparagraph{x86}

We get (MSVC 2010):

\lstinputlisting[caption=MSVC 2010,style=customasmx86]{patterns/13_arrays/5_multidimensional/multi_msvc_EN.asm}

Nothing special. For index calculation, three input arguments are used 
in the formula $address=600 \cdot 4 \cdot x + 30 \cdot 4 \cdot y + 4z$, to represent the array as multidimensional.
Do not forget that the \Tint type is 32-bit (4 bytes),
so all coefficients must be multiplied by 4.

\lstinputlisting[caption=GCC 4.4.1,style=customasmx86]{patterns/13_arrays/5_multidimensional/multi_gcc_EN.asm}

The GCC compiler does it differently.

For one of the operations in the calculation ($30y$), GCC produces code without multiplication instructions.
This is how it done: 
$(y+y) \ll 4 - (y+y) = (2y) \ll 4 - 2y = 2 \cdot 16 \cdot y - 2y = 32y - 2y = 30y$. 
Thus, for the $30y$ calculation, only one addition operation,
one bitwise shift operation and one subtraction operation are used.
This works faster.

\myparagraph{ARM + \NonOptimizingXcodeIV (\ThumbMode)}

\lstinputlisting[caption=\NonOptimizingXcodeIV (\ThumbMode),style=customasmARM]{patterns/13_arrays/5_multidimensional/multi_Xcode_thumb_O0_EN.asm}

\NonOptimizing LLVM saves all variables in local stack, which is redundant.

The address of the array element is calculated by the formula we already saw.

\myparagraph{ARM + \OptimizingXcodeIV (\ThumbMode)}

\lstinputlisting[caption=\OptimizingXcodeIV (\ThumbMode),style=customasmARM]{patterns/13_arrays/5_multidimensional/multi_Xcode_thumb_O3_EN.asm}

The tricks for replacing multiplication by shift, addition and subtraction which we already saw
are also present here.

\myindex{ARM!\Instructions!RSB}
\myindex{ARM!\Instructions!SUB}
Here we also see a new instruction for us: \RSB (\emph{Reverse Subtract}).

It works just as \SUB, but it swaps its operands with each other before execution.
Why?
\myindex{ARM!Optional operators!LSL}
\SUB and \RSB  are instructions, to the second operand of which shift coefficient may be applied: (\INS{LSL\#4}). 

But this coefficient can be applied only to second operand.

That's fine for commutative operations like addition or multiplication 
(operands may be swapped there without changing the result).

But subtraction is a non-commutative operation, so \RSB exist for these cases.

\myparagraph{MIPS}

\myindex{MIPS!Global Pointer}
My example is tiny, so the GCC compiler decided to put the $a$ array into the 64KiB area 
addressable by the Global Pointer.

\lstinputlisting[caption=\Optimizing GCC 4.4.5 (IDA),style=customasmMIPS]{patterns/13_arrays/5_multidimensional/multi_MIPS_O3_IDA_EN.lst}


\subsubsection{Getting dimensions of multidimensional array}
\myindex{Hex-Rays}

Any string processing function, if an array of characters passed to it, can't deduce a size of the input array.
Likewise, if a function processes 2D array, only one dimension can be deduced.

For example:

\lstinputlisting[style=customc]{patterns/13_arrays/5_multidimensional/dimensions2.c}

... if compiled (by any compiler) and then decompiled by Hex-Rays:

\lstinputlisting[style=customc]{patterns/13_arrays/5_multidimensional/dimensions2_hexrays.c}

There is no way to find a size of the first dimension.
If $x$ value passed is too big, buffer overflow would occur, an element from some random place of memory would be read.

And 3D array:

\lstinputlisting[style=customc]{patterns/13_arrays/5_multidimensional/dimensions3.c}

Hex-Rays:

\lstinputlisting[style=customc]{patterns/13_arrays/5_multidimensional/dimensions3_hexrays.c}

Again, sizes of only two of 3 dimensions can be deduced.



\subsubsection{More examples}

The computer screen is represented as a 2D array, but the video-buffer is a linear 1D array. 
We talk about it here: \myref{Mandelbrot_demo}.

Another example in this book is Minesweeper game: it's field is also two-dimensional array: \ref{minesweeper_winxp}.

}
\RU{\subsection{Многомерные массивы}

Внутри многомерный массив выглядит так же как и линейный.

Ведь память компьютера линейная, это одномерный массив.
Но для удобства этот одномерный массив легко представить как многомерный.

К примеру, вот как элементы массива 3x4 расположены в одномерном массиве из 12 ячеек:

% TODO FIXME not clear. First, horizontal would be better. Second, why two columns?
% I'd first show 3x4 with numbered elements (e.g. 32-bit ints) in colored lines,
% then linear with the same numbered elements (and colored blocks)
% then linear with addresses (offsets) - assuming let say 32-bit ints.
\begin{table}[H]
\centering
\begin{tabular}{ | l | l | }
\hline
Смещение в памяти & элемент массива \\
\hline
0 & [0][0] \\
\hline
1 & [0][1] \\
\hline
2 & [0][2] \\
\hline
3 & [0][3] \\
\hline
4 & [1][0] \\
\hline
5 & [1][1] \\
\hline
6 & [1][2] \\
\hline
7 & [1][3] \\
\hline
8 & [2][0] \\
\hline
9 & [2][1] \\
\hline
10 & [2][2] \\
\hline
11 & [2][3] \\
\hline
\end{tabular}
\caption{Двухмерный массив представляется в памяти как одномерный}
\end{table}

Вот по каким адресам в памяти располагается каждая ячейка двухмерного массива 3*4:

\begin{table}[H]
\centering
\begin{tabular}{ | l | l | l | l | }
\hline                        
0 & 1 & 2 & 3 \\
\hline  
4 & 5 & 6 & 7 \\
\hline  
8 & 9 & 10 & 11 \\
\hline  
\end{tabular}
\caption{Адреса в памяти каждой ячейки двухмерного массива}
\end{table}

\myindex{row-major order}
Чтобы вычислить адрес нужного элемента, сначала умножаем первый индекс (строку) на 4 (ширину массива), 
затем прибавляем второй индекс (столбец).

Это называется \emph{row-major order}, 
и такой способ представления массивов и матриц используется по крайней мере в \CCpp и Python. 
Термин \emph{row-major order} означает по-русски примерно следующее: \q{сначала записываем элементы первой строки, затем второй,~\dots~и~элементы последней 
строки в самом конце}.

\myindex{column-major order}
\myindex{Фортран}
Другой способ представления называется \emph{column-major order} (индексы массива используются в обратном порядке) 
и это используется по крайней мере в Фортране, MATLAB и R. 
Термин \emph{column-major order} означает по-русски
следующее: \q{сначала записываем элементы первого столбца, затем второго,~\dots~и~элементы последнего столбца
в самом конце}.

Какой из способов лучше?
В терминах производительности и кэш-памяти, лучший метод организации данных это тот,
при котором к данным обращаются последовательно.

Так что если ваша функция обращается к данным построчно, то \emph{row-major order} лучше,
и наоборот.

% subsubsections
\input{patterns/13_arrays/5_multidimensional/2D_RU}
\input{patterns/13_arrays/5_multidimensional/2D_as_1D_RU}
\subsubsection{Пример с трехмерным массивом}

То же самое и для многомерных массивов.
На этот раз будем работать с массивом типа \Tint: каждый элемент требует 4 байта в памяти.

Попробуем:

\lstinputlisting[caption=простой пример,style=customc]{patterns/13_arrays/5_multidimensional/multi.c}

\myparagraph{x86}

В итоге (MSVC 2010):

\lstinputlisting[caption=MSVC 2010,style=customasmx86]{patterns/13_arrays/5_multidimensional/multi_msvc_RU.asm}

В принципе, ничего удивительного. В \TT{insert()} для вычисления адреса нужного элемента массива 
три входных аргумента перемножаются по формуле $address=600 \cdot 4 \cdot x + 30 \cdot 4 \cdot y + 4z$, 
чтобы представить массив трехмерным.
Не забывайте также, что тип \Tint 32-битный (4 байта), поэтому все коэффициенты нужно умножить на 4.

\lstinputlisting[caption=GCC 4.4.1,style=customasmx86]{patterns/13_arrays/5_multidimensional/multi_gcc_RU.asm}

Компилятор GCC решил всё сделать немного иначе.
Для вычисления одной из операций ($30y$), GCC создал код, где нет самой операции умножения.

Происходит это так: 
$(y+y) \ll 4 - (y+y) = (2y) \ll 4 - 2y = 2 \cdot 16 \cdot y - 2y = 32y - 2y = 30y$. 
Таким образом, для вычисления $30y$ используется только операция сложения, 
операция битового сдвига и операция вычитания.
Это работает быстрее.

\myparagraph{ARM + \NonOptimizingXcodeIV (\ThumbMode)}

\lstinputlisting[caption=\NonOptimizingXcodeIV (\ThumbMode),style=customasmARM]{patterns/13_arrays/5_multidimensional/multi_Xcode_thumb_O0_RU.asm}

\NonOptimizing LLVM сохраняет все переменные в локальном стеке, хотя это и избыточно.

Адрес элемента массива вычисляется по уже рассмотренной формуле.

\myparagraph{ARM + \OptimizingXcodeIV (\ThumbMode)}

\lstinputlisting[caption=\OptimizingXcodeIV (\ThumbMode),style=customasmARM]{patterns/13_arrays/5_multidimensional/multi_Xcode_thumb_O3_RU.asm}

Тут используются уже описанные трюки для замены умножения на операции сдвига, сложения и вычитания.

\myindex{ARM!\Instructions!RSB}
\myindex{ARM!\Instructions!SUB}
Также мы видим новую для себя инструкцию \RSB (\emph{Reverse Subtract}).
Она работает так же, как и \SUB, только меняет операнды местами.

Зачем?
\myindex{ARM!Optional operators!LSL}
\SUB и \RSB это те инструкции, ко второму операнду которых можно применить коэффициент сдвига, как мы видим и здесь: (\INS{LSL\#4}). 
Но этот коэффициент можно применить только ко второму операнду.

Для коммутативных операций, таких как сложение или умножение, 
операнды можно менять местами и это не влияет на результат.

Но вычитание~--- операция некоммутативная, так что для этих случаев существует инструкция \RSB.

\myparagraph{MIPS}

\myindex{MIPS!Global Pointer}

Мой пример такой крошечный, что компилятор GCC решил разместить массив $a$ в 64KiB-области,
адресуемой при помощи Global Pointer.

\lstinputlisting[caption=\Optimizing GCC 4.4.5 (IDA),style=customasmMIPS]{patterns/13_arrays/5_multidimensional/multi_MIPS_O3_IDA_RU.lst}


\subsubsection{Узнать размеры многомерного массива}
\myindex{Hex-Rays}

Если в ф-цию обработки строки передать массив символов, внутри самой ф-ции невозможно определить размер входного массива.
Точно также, в ф-ции, обрабатывающую двухмерный массив, толко один размер может быть определен.

Например:

\lstinputlisting[style=customc]{patterns/13_arrays/5_multidimensional/dimensions2.c}

... если это скомпилировать (любым компилятором) и затем декомпилировать в Hex-Rays:

\lstinputlisting[style=customc]{patterns/13_arrays/5_multidimensional/dimensions2_hexrays.c}

Нет никакого способа узнать размер первого измерения.
Если переданное значение $x$ слишком большое, произойдет переполнение буфера, и прочитается элемент из какого-то случайного места в памяти.

И трехмерный массив:

\lstinputlisting[style=customc]{patterns/13_arrays/5_multidimensional/dimensions3.c}

Hex-Rays:

\lstinputlisting[style=customc]{patterns/13_arrays/5_multidimensional/dimensions3_hexrays.c}

И снова, можно узнать размеры только двух измерений из трех.



\subsubsection{Ещё примеры}

Компьютерный экран представляет собой двумерный массив, но видеобуфер это линейный
одномерный массив. 
Мы рассматриваем это здесь: \myref{Mandelbrot_demo}.

Еще один пример в этой книге это игра ``Сапер'': её поле это тоже двухмерный массив: \ref{minesweeper_winxp}.

}
\DE{\subsection{Multidimensionale Arrays}
Intern ist ein multidimensionales Array im Prinzip das gleiche wie ein lineares Array.

Da der Speicher eines Rechners linear ist, ist es ein eindimensionales Array.
Zur Vereinfachung kann dieses multidimensionales Array leicht als eindimensional dargestellt werden.

Beispielsweise werden die Elemente eines 3x4 Arrays folgendermaßen in einem eindimensionalen Array aus 12 Zellen
gespeichert:

% TODO FIXME not clear. First, horizontal would be better. Second, why two columns?
% I'd first show 3x4 with numbered elements (e.g. 32-bit ints) in colored lines,
% then linear with the same numbered elements (and colored blocks)
% then linear with addresses (offsets) - assuming let say 32-bit ints.
\begin{table}[H]
\centering
\begin{tabular}{ | l | l | }
\hline
Offset im Speicher & Arrayelement \\
\hline
0 & [0][0] \\
\hline
1 & [0][1] \\
\hline
2 & [0][2] \\
\hline
3 & [0][3] \\
\hline
4 & [1][0] \\
\hline
5 & [1][1] \\
\hline
6 & [1][2] \\
\hline
7 & [1][3] \\
\hline
8 & [2][0] \\
\hline
9 & [2][1] \\
\hline
10 & [2][2] \\
\hline
11 & [2][3] \\
\hline
\end{tabular}
\caption{Zweidimensionales Array in eindimensionaler Speicherdarstellung}
\end{table}

Auf diese Weise wird jede Zellen des 3*4 Arrays im Speicher abgelegt:

% TODO coordinates. TikZ?
\begin{table}[H]
\centering
\begin{tabular}{ | l | l | l | l | }
\hline                        
0 & 1 & 2 & 3 \\
\hline  
4 & 5 & 6 & 7 \\
\hline  
8 & 9 & 10 & 11 \\
\hline  
\end{tabular}
\caption{Speicheradressen jeder Zelle des zweidimensionalen Arrays}
\end{table}

\myindex{row-major order}
Um also die Adresse des benötigten Elements zu berechnen, multiplizieren wir zunächst den ersten Index mit 4 (der
Arraybreite) und addieren dann den zweiten Index.
Dies nennt man \emph{Zeilenordnung} (engl. row-major order) und diese Methode zur Darstellung von Arrays und Matrizen
wird mindestens von \CCpp und Python verwendet.
Der Ausdruck row-major order bedeutet: \q{schreibe zuerst die Elemente der ersten Zeilen, dann die zweite Zeile\dots
und schließlich die Elemente der letzten Zeile}.

\myindex{column-major order}
\myindex{Fortran}
Eine andere Methode zur Darstellung heißt \emph{Spaltenordnung} (engl. column-major order) (die Indizes des Arrays werden
in umgekehrter Reihenfolge verwendet) und wird zumindest in Fortran, MATLAB und R verwendet.
Der Ausdruck column-major oder bedeutet: \q{schreibe zuerst die Elemente der ersten Spalte, dann die zweite Spalte\dots
und schließlich die Elemente der letzten Spalte}.

Welche Method ist besser?

Generel ist hinsichtlich Performance und Cachespeicher die beste Methode der Datenorganisation diejenige, in der auf die
Elemente sequentiell zugegriffen wird.

Wenn eine Funktion auf Daten zeilenweise zugreift, ist Zeilenordnung besser und umgekehrt.

% subsubsections
\input{patterns/13_arrays/5_multidimensional/2D_DE}
\input{patterns/13_arrays/5_multidimensional/2D_as_1D_DE}
\subsubsection{Beispiel: dreidimensionales Array}

Mit multidimensionalen Arrays ist es das gleiche.

Wir werden nun mit einem Array vom Typ \Tint arbeiten: jedes Element benötigt 4 Byte Speicherplatz.

Sehen wir es uns an:

\lstinputlisting[caption=simple example,style=customc]{patterns/13_arrays/5_multidimensional/multi.c}

\myparagraph{x86}

Wir erhalten das Folgende (MSVC 2010):

\lstinputlisting[caption=MSVC 2010,style=customasmx86]{patterns/13_arrays/5_multidimensional/multi_msvc_DE.asm}
Nichts Außergewöhnliches. Zur Berechnung des Index' werden in der Formel $address=600 \cdot 4 \cdot x + 30 \cdot 4 \cdot
y + 4z$ drei Eingabewerte verwendet, um das multidimensionale Array zu repräsentieren.
Vergessen wir nicht, dass der \Tint Typ 32 Bit (4 Byte) breit ist, sodass alle Koeffizienten mit 4 multipliziert werden
müssen.

\lstinputlisting[caption=GCC 4.4.1,style=customasmx86]{patterns/13_arrays/5_multidimensional/multi_gcc_DE.asm}
Der GCC Compiler arbeitet anders.

Für eine der Operationen in der Berechnung ($30y$) produziet GCC Code ohne Multiplikationsbefehle.
Das funktioniert wie folgt:
$(y+y) \ll 4 - (y+y) = (2y) \ll 4 - 2y = 2 \cdot 16 \cdot y - 2y = 32y - 2y = 30y$.

So werden für die $30y$ Berechnung nur ein Addierbefehl, eine bitweiser Verschiebebefehl und ein Subtraktionsbefehl
verwendet. So geht es schneller.

\myparagraph{ARM + \NonOptimizingXcodeIV (\ThumbMode)}

\lstinputlisting[caption=\NonOptimizingXcodeIV
(\ThumbMode),style=customasmARM]{patterns/13_arrays/5_multidimensional/multi_Xcode_thumb_O0_DE.asm}

\NonOptimizing LLVM speichert alle Variablen auf dem lokalen Stack, was redundant ist.

Die Adresse des Arrayelements wird über die eben gezeigte Formel berechnet.

\myparagraph{ARM + \OptimizingXcodeIV (\ThumbMode)}

\lstinputlisting[caption=\OptimizingXcodeIV
(\ThumbMode),style=customasmARM]{patterns/13_arrays/5_multidimensional/multi_Xcode_thumb_O3_DE.asm}
Die Tricks für das Ersetzen der Multiplikation durch Verschieben, Addieren und Subtrahieren, die wir bereits
kennengelernt haben, kommen hier auch vor.

\myindex{ARM!\Instructions!RSB}
\myindex{ARM!\Instructions!SUB}
Hier finden wir auch einen für uns neuen Befehl: \RSB (\emph{Reverse Subtract}).

Er arbeitet genau wie \SUB, aber vertauscht die Operanden vor der Ausführung. Warum?

\myindex{ARM!Optional operators!LSL}
\SUB und \RSB  sind Befehle, bei denen auf den zweiten Operanden eine bitweise Verschiebung angewendet werden kann:
(\INS{LSL\#4}).
Dieser Koeffizient kann aber nur auf den zweiten Operanden angewendet werden.

Das ist günstig für kommutative Operationen wie Addition und Multiplikation (die Operanden können vertauscht werden,
ohne das Ergebnis zu verändern).

Subtraktion dagegen ist nicht kommutativ, weshalb für diese Fälle \RSB existiert.

\myparagraph{MIPS}

\myindex{MIPS!Global Pointer}
Das Beispiel ist sehr klein, sodass der GCC Compiler entschieden hat das Array $a$ im 64KiB Platz abzulegen, um es durch
den globalen Pointer zugreifbar zu machen.

\lstinputlisting[caption=\Optimizing GCC 4.4.5
(IDA),style=customasmMIPS]{patterns/13_arrays/5_multidimensional/multi_MIPS_O3_IDA_DE.lst}


% \input{patterns/13_arrays/5_multidimensional/dimensions_DE}

\subsubsection{Weitere Beispiele}
Der Bildschirm wird als 2D-Array dargestellt, aber der Videopuffer ist ein lineares 1D-Array.
Wir betrachten hier näher: \myref{Mandelbrot_demo}.

Ein anderes Beispiel in diesem Buch ist das Spiel Minesweeper: das Feld ist auch ein zweidimensionales Array:
\ref{minesweeper_winxp}.

}
\FR{\subsection{Tableaux multidimensionnels}

En interne, un tableau multidimensionnel est pratiquement la même chose qu'un tableau
linéaire.

Puisque la mémoire d'un ordinateur est linéaire, c'est un tableau uni-dimensionnel.
Par commodité, ce tableau multidimensionnel peut facilement être représenté comme
un uni-dimensionnel.

Par exemple, voici comment les éléments du tableau 3*4 sont placés dans un tableau
uni-dimensionnel de 12 éléments:

% TODO FIXME not clear. First, horizontal would be better. Second, why two columns?
% I'd first show 3x4 with numbered elements (e.g. 32-bit ints) in colored lines,
% then linear with the same numbered elements (and colored blocks)
% then linear with addresses (offsets) - assuming let say 32-bit ints.
\begin{table}[H]
\centering
\begin{tabular}{ | l | l | }
\hline
Offset en mémoire & élément du tableau \\
\hline
0 & [0][0] \\
\hline
1 & [0][1] \\
\hline
2 & [0][2] \\
\hline
3 & [0][3] \\
\hline
4 & [1][0] \\
\hline
5 & [1][1] \\
\hline
6 & [1][2] \\
\hline
7 & [1][3] \\
\hline
8 & [2][0] \\
\hline
9 & [2][1] \\
\hline
10 & [2][2] \\
\hline
11 & [2][3] \\
\hline
\end{tabular}
\caption{Tableau en deux dimensions représenté en mémoire en une dimension}
\end{table}

Voici comment chacun des éléments du tableau 3*4 sont placés en mémoire:

% TODO coordinates. TikZ?
\begin{table}[H]
\centering
\begin{tabular}{ | l | l | l | l | }
\hline                        
0 & 1 & 2 & 3 \\
\hline  
4 & 5 & 6 & 7 \\
\hline  
8 & 9 & 10 & 11 \\
\hline  
\end{tabular}
\caption{Adresse mémoire de chaque élément d'un tableau à deux dimensions}
\end{table}

\myindex{row-major order}

Donc, afin de calculer l'adresse de l'élément voulu, nous devons d'abord multiplier
le premier index par 4 (largeur du tableau) et puis ajouter le second index.
Ceci est appelé \emph{row-major order} (ordre ligne d'abord),
et c'est la méthode de représentation des tableaux et des matrices au moins en \CCpp
et Python.
Le terme \emph{row-major order} est de l'anglais signifiant: \q{ d'abord, écrire les
éléments de la première ligne, puis ceux de la seconde ligne \dots et enfin les éléments
de la dernière ligne}.

\myindex{column-major order}
\myindex{Fortran}
Une autre méthode de représentation est appelée \emph{column-major order} (ordre colonne
d'abord) (les indices du tableau sont utilisés dans l'ordre inverse) et est utilisé
au moins en ForTran, MATLAB et R.
Le terme \emph{column-major order} est de l'anglais signifiant: \q{ d'abord, écrire les
éléments de la première colonne, puis ceux de la seconde colonne \dots et enfin les
éléments de la dernière colonne}.

Quelle méthode est la meilleure?

En général, en termes de performance et de mémoire cache, le meilleur schéma pour
l'organisation des données est celui dans lequel les éléments sont accédés séquentiellement.

Donc si votre fonction accède les données par ligne, \emph{row-major order} est meilleur,
et vice-versa.

% subsubsections
\input{patterns/13_arrays/5_multidimensional/2D_FR}
\input{patterns/13_arrays/5_multidimensional/2D_as_1D_FR}
\subsubsection{Exemple de tableau à trois dimensions}

C'est la même chose pour des tableaux multidimensionnels.

Nous allons travailler avec des tableaux de type \Tint: chaque élément nécessite
4 octets en mémoire.

Voyons ceci:

\lstinputlisting[caption=simple exemple,style=customc]{patterns/13_arrays/5_multidimensional/multi.c}

\myparagraph{x86}

Nous obtenons (MSVC 2010):

\lstinputlisting[caption=MSVC 2010,style=customasmx86]{patterns/13_arrays/5_multidimensional/multi_msvc_FR.asm}

Rien de particulier. Pour le calcul de l'index, trois arguments en entrée sont utilisés
dans la formule $address=600 \cdot 4 \cdot x + 30 \cdot 4 \cdot y + 4z$, pour représenter
le tableau comme multidimensionnel.
N'oubliez pas que le type \Tint est 32-bit (4 octets), donc tous les coefficients
doivent être multipliés par 4.

\lstinputlisting[caption=GCC 4.4.1,style=customasmx86]{patterns/13_arrays/5_multidimensional/multi_gcc_FR.asm}

Le compilateur GCC fait cela différemment.

Pour une des opérations du calcul ($30y$), GCC produit un code sans instruction de
multiplication. Voici comment il fait:
$(y+y) \ll 4 - (y+y) = (2y) \ll 4 - 2y = 2 \cdot 16 \cdot y - 2y = 32y - 2y = 30y$. 
Ainsi, pour le calcul de $30y$, seulement une addition, un décalage de bit et une
soustraction sont utilisés.
Ceci fonctionne plus vite.

\myparagraph{ARM + \NonOptimizingXcodeIV (\ThumbMode)}

\lstinputlisting[caption=\NonOptimizingXcodeIV (\ThumbMode),style=customasmARM]{patterns/13_arrays/5_multidimensional/multi_Xcode_thumb_O0_FR.asm}

LLVM \NonOptimizing sauve toutes les variables dans la pile locale, ce qui est redondant.

L'adresse de l'élément du tableau est calculée par la formule vue précédemment.

\myparagraph{ARM + \OptimizingXcodeIV (\ThumbMode)}

\lstinputlisting[caption=\OptimizingXcodeIV (\ThumbMode),style=customasmARM]{patterns/13_arrays/5_multidimensional/multi_Xcode_thumb_O3_FR.asm}

L'astuce de remplacer la multiplication par des décalage, addition et soustraction
que nous avons déjà vue est aussi utilisée ici.

\myindex{ARM!\Instructions!RSB}
\myindex{ARM!\Instructions!SUB}
Ici, nous voyons aussi une nouvelle instruction: \RSB (\emph{Reverse Subtract}).

Elle fonctionne comme \SUB, mais échange ses opérandes l'un avec l'autre avant l'exécution.
Pourquoi?
\myindex{ARM!Optional operators!LSL}
\SUB et \RSB sont des instructions auxquelles un coefficient de décalage peut être
appliqué au second opérande: (\INS{LSL\#4}).

Mais ce coefficient ne peut être appliqué qu'au second opérande.

C'est bien pour des opérations commutatives comme l'addition ou la multiplication
(les opérandes peuvent être échangés sans changer le résultat).

Mais la soustraction est une opération non commutative, donc \RSB existe pour ces
cas.

\myparagraph{MIPS}

\myindex{MIPS!Global Pointer}
Mon exemple est minuscule, donc le compilateur GCC a décidé de mettre le tableau
$a$ dans la zone de 64KiB adressable par le Global Pointer.

\lstinputlisting[caption=GCC 4.4.5 \Optimizing (IDA),style=customasmMIPS]{patterns/13_arrays/5_multidimensional/multi_MIPS_O3_IDA_FR.lst}


%\input{patterns/13_arrays/5_multidimensional/dimensions_FR}

\subsubsection{Plus d'exemples}

L'écran de l'ordinateur est représenté comme un tableau 2D, mais le buffer vidéo
est un tableau linéaire 1D.
Nous en parlons ici: \myref{Mandelbrot_demo}.

Un autre exemple dans ce livre est le jeu Minesweeper: son champ est aussi un tableau
à deux dimensions: \ref{minesweeper_winxp}.

}
\JPN{\subsection{多次元配列}

内部的には、多次元配列は本質的には一次元の配列と同じです。

コンピュータメモリは一次元なので、メモリは一次元配列です。
便宜上、多次元配列は一次元として表現可能です。

例えば、3x4の配列の要素が12のセルの1次元配列にどのように配置されるかを示します。

% TODO FIXME not clear. First, horizontal would be better. Second, why two columns?
% I'd first show 3x4 with numbered elements (e.g. 32-bit ints) in colored lines,
% then linear with the same numbered elements (and colored blocks)
% then linear with addresses (offsets) - assuming let say 32-bit ints.
\begin{table}[H]
\centering
\begin{tabular}{ | l | l | }
\hline
Offset in memory & array element \\
\hline
0 & [0][0] \\
\hline
1 & [0][1] \\
\hline
2 & [0][2] \\
\hline
3 & [0][3] \\
\hline
4 & [1][0] \\
\hline
5 & [1][1] \\
\hline
6 & [1][2] \\
\hline
7 & [1][3] \\
\hline
8 & [2][0] \\
\hline
9 & [2][1] \\
\hline
10 & [2][2] \\
\hline
11 & [2][3] \\
\hline
\end{tabular}
\caption{1次元配列としてメモリ上で表現される2次元配列}
\end{table}

3*4配列の各セルがメモリ上でどう配置されるかを示します。

% TODO coordinates. TikZ?
\begin{table}[H]
\centering
\begin{tabular}{ | l | l | l | l | }
\hline                        
0 & 1 & 2 & 3 \\
\hline  
4 & 5 & 6 & 7 \\
\hline  
8 & 9 & 10 & 11 \\
\hline  
\end{tabular}
\caption{2次元配列の各セルのメモリアドレス}
\end{table}

\myindex{row-major order}

したがって、必要な要素のアドレスを計算するには、まず最初のインデックスに
4(配列の幅)を掛けてから2番目のインデックスを追加します。
これは\IT{行優先順位}と呼ばれ、配列と行列表現のこの方法は、少なくとも \CCpp とPythonで使用されます。
単純な英単語の\IT{行優先順位}は、\q{最初に、最初の行の要素を書き、次に2番目の行 \dots 
最後に最後の行の要素を書き込む}という意味です。

\myindex{column-major order}
\myindex{Fortran}
表現のもう1つの方法は、\IT{列優先順位}(配列の添字は逆順で使用されます)と呼ばれ、
少なくともFortran、MATLAB、およびRで使用されます。
\IT{列優先順位}は、単純な英語では、\q{最初に、最初の列の要素を書き込み、次に2番目の列を \dots
最後に最後の列の要素を書き込む}となります。

どの方法が良いでしょうか?

一般に、パフォーマンスとキャッシュメモリの観点からは、
データ編成のための最良の方法は、要素が順次アクセスされる方法です。

したがって、関数が行ごとにデータにアクセスする場合は、\IT{行優先順位}が優れていて、逆もまた同様です。

% subsections
\subsubsection{2次元配列の例}

\Tchar 型の配列で作業していきます。これは、各要素がメモリ上に1バイトしか必要ないことを意味します。

\myparagraph{行を埋める例}
\myindex{\olly}

2行目を0~3の値で埋めてみましょう。

\lstinputlisting[caption=行を埋める例,style=customc]{patterns/13_arrays/5_multidimensional/two1_JPN.c}

3つの行はすべて赤でマークしてあります。
2行目は0,1,2と3の値を持っています。

\begin{figure}[H]
\centering
\includegraphics[width=0.6\textwidth]{patterns/13_arrays/5_multidimensional/olly_2D_1.png}
\caption{\olly: 配列が埋められる}
\end{figure}

\myparagraph{列を埋める例}
\myindex{\olly}

3列目を値0~2で埋めてみましょう。

\lstinputlisting[caption=列を埋める例,style=customc]{patterns/13_arrays/5_multidimensional/two2_JPN.c}

3つの行はここでも赤でマークしてあります。

各行の3番目の値が0,1と2で書かれています。

\begin{figure}[H]
\centering
\includegraphics[width=0.6\textwidth]{patterns/13_arrays/5_multidimensional/olly_2D_2.png}
\caption{\olly: 配列が埋められる}
\end{figure}


\subsubsection{2次元配列を1次元配列としてアクセスする}

少なくとも2つの方法で、2次元配列を1次元配列としてアクセスすることが可能だといえます。

\lstinputlisting[style=customc]{patterns/13_arrays/5_multidimensional/2D_as_1D_JPN.c}

コンパイルして実行してください。\footnote{プログラムはC++ではなく、Cプログラムとしてコンパイルされます。.c拡張子でファイルを保存してMSVCでコンパイルします}
正しい値を表示します。

MSVC 2013の結果は興味部会です。3つのルーチンはすべて同じです!

\lstinputlisting[caption=\Optimizing MSVC 2013 x64,style=customasmx86]{patterns/13_arrays/5_multidimensional/2D_as_1D_MSVC_2013_Ox_x64_JPN.asm}

GCCも同じルーチンを生成しますが、少し異なります。

\lstinputlisting[caption=\Optimizing GCC 4.9 x64,style=customasmx86]{patterns/13_arrays/5_multidimensional/2D_as_1D_GCC49_x64_O3_JPN.s}


\subsubsection{3次元配列の例}

多次元配列でも同じです。

\Tint 型の配列で作業していきます。各要素はメモリ上で4バイト必要とします。

見てみましょう。

\lstinputlisting[caption=単純な例,style=customc]{patterns/13_arrays/5_multidimensional/multi.c}

\myparagraph{x86}

MSVC 2010の結果

\lstinputlisting[caption=MSVC 2010,style=customasmx86]{patterns/13_arrays/5_multidimensional/multi_msvc_JPN.asm}

特別なことはありません。インデックスの計算では、式 $address=600 \cdot 4 \cdot x + 30 \cdot 4 \cdot y + 4z$ 
では3つの入力引数が使用され、配列を多次元として表現しています。
\Tint 型は32ビット(4バイト)なので、
係数は4倍する必要があることを忘れないでください。

\lstinputlisting[caption=GCC 4.4.1,style=customasmx86]{patterns/13_arrays/5_multidimensional/multi_gcc_JPN.asm}

GCCコンパイラは異なります。

計算での演算において($30y$)、GCCは乗算命令を使わないコードを生成します。
このようにします。
$(y+y) \ll 4 - (y+y) = (2y) \ll 4 - 2y = 2 \cdot 16 \cdot y - 2y = 32y - 2y = 30y$. 
従って、 $30y$ の計算には、加算命令が1つだけです。
ビットシフト演算と減算が使用されます。
これはより高速です。

\myparagraph{ARM + \NonOptimizingXcodeIV (\ThumbMode)}

\lstinputlisting[caption=\NonOptimizingXcodeIV (\ThumbMode),style=customasmARM]{patterns/13_arrays/5_multidimensional/multi_Xcode_thumb_O0_JPN.asm}

\NonOptimizing LLVMは変数すべてをローカルスタックに保存しますが、冗長です。

配列の要素のアドレスはすでに見た式によって計算されます。

\myparagraph{ARM + \OptimizingXcodeIV (\ThumbMode)}

\lstinputlisting[caption=\OptimizingXcodeIV (\ThumbMode),style=customasmARM]{patterns/13_arrays/5_multidimensional/multi_Xcode_thumb_O3_JPN.asm}

既に見たシフト、加減算による乗算を置き換えるためのトリックもここにあります。

\myindex{ARM!\Instructions!RSB}
\myindex{ARM!\Instructions!SUB}
新しい命令を見てみます:\RSB (\IT{Reverse Subtract})

単純に \SUB として機能しますが、実行前にオペランドをスワップします。
なぜでしょう?
\myindex{ARM!Optional operators!LSL}
\SUB および \RSB は、シフト係数が適用される第2のオペランド(\INS{LSL\#4})への命令です。

ただし、この係数は第2オペランドにのみ適用されます。

これは、加算や乗算のような可換的な(交換可能な)演算の場合は問題ありません。
(結果を変更せずにオペランドを入れ替えてもかまいません)

しかし、減算は非可換的な演算なので、 \RSB が存在します。

\myparagraph{MIPS}

\myindex{MIPS!Global Pointer}
私の例はとても小さいので、GCCコンパイラはグローバルポインタによってアドレス可能な64KiB領域に
配列を配置することに決めました。

\lstinputlisting[caption=\Optimizing GCC 4.4.5 (IDA),style=customasmMIPS]{patterns/13_arrays/5_multidimensional/multi_MIPS_O3_IDA_JPN.lst}



\subsubsection{More examples}

コンピュータ画面は2D配列として表現されますが、ビデオバッファは1次元配列です。
これについてはこちらで:\myref{Mandelbrot_demo}

本書での他の例としてはマインスイーパーゲームがあります。そのフィールドは2次元配列です:\ref{minesweeper_winxp}
}

\EN{\subsubsection{ARM}

\myparagraph{\NonOptimizingKeilVI (\ARMMode)}

\lstinputlisting[label=Keil_number_sign,style=customasmARM]{patterns/09_loops/simple/ARM/Keil_ARM_O0.asm}

Iteration counter $i$ is to be stored in the \Reg{4} register.
The \INS{MOV R4, \#2} instruction just initializes $i$.
The \INS{MOV R0, R4} and \INS{BL printing\_function} instructions
compose the body of the loop, the first instruction preparing the argument for 
\ttf function and the second calling the function.
\myindex{ARM!\Instructions!ADD}
The \INS{ADD R4, R4, \#1} instruction just adds 1 to the $i$ variable at each iteration.
\myindex{ARM!\Instructions!CMP}
\myindex{ARM!\Instructions!BLT}
\INS{CMP R4, \#0xA} compares $i$ with \TT{0xA} (10). 
The next instruction \INS{BLT} (\emph{Branch Less Than}) jumps if $i$ is less than 10.
Otherwise, 0 is to be written into \Reg{0} (since our function returns 0)
and function execution finishes.

\myparagraph{\OptimizingKeilVI (\ThumbMode)}

\lstinputlisting[style=customasmARM]{patterns/09_loops/simple/ARM/Keil_thumb_O3.asm}

Practically the same.

\myparagraph{\OptimizingXcodeIV (\ThumbTwoMode)}
\label{ARM_unrolled_loops}

\lstinputlisting[style=customasmARM]{patterns/09_loops/simple/ARM/xcode_thumb_O3.asm}

In fact, this was in my \ttf function:

\begin{lstlisting}[style=customc]
void printing_function(int i)
{
    printf ("%d\n", i);
};
\end{lstlisting}

\myindex{Unrolled loop}
\myindex{Inline code}
So, LLVM not just \emph{unrolled} the loop, 
but also \emph{inlined} my 
very simple function \ttf,
and inserted its body 8 times instead of calling it. 

This is possible when the function is so simple (like mine) and when it is not called too much (like here).

\myparagraph{ARM64: \Optimizing GCC 4.9.1}

\lstinputlisting[caption=\Optimizing GCC 4.9.1,style=customasmARM]{patterns/09_loops/simple/ARM/ARM64_GCC491_O3_EN.s}

\myparagraph{ARM64: \NonOptimizing GCC 4.9.1}

\lstinputlisting[caption=\NonOptimizing GCC 4.9.1 -fno-inline,style=customasmARM]{patterns/09_loops/simple/ARM/ARM64_GCC491_O0_EN.s}
}
\RU{\mysection{Функция toupper()}
\myindex{\CStandardLibrary!toupper()}

Еще одна очень востребованная функция конвертирует символ из строчного в заглавный, если нужно:

\lstinputlisting[style=customc]{\CURPATH/toupper.c}

Выражение \TT{'a'+'A'} оставлено в исходном коде для удобства чтения, 
конечно, оно соптимизируется

\footnote{Впрочем, если быть дотошным, вполне могут до сих пор существовать компиляторы,
которые не оптимизируют подобное и оставляют в коде.}.

\ac{ASCII}-код символа \q{a} это 97 (или 0x61), и 65 (или 0x41) для символа \q{A}.

Разница (или расстояние) между ними в \ac{ASCII}-таблица это 32 (или 0x20).

Для лучшего понимания, читатель может посмотреть на стандартную 7-битную таблицу \ac{ASCII}:

\begin{figure}[H]
\centering
\includegraphics[width=0.7\textwidth]{ascii.png}
\caption{7-битная таблица \ac{ASCII} в Emacs}
\end{figure}

\subsection{x64}

\subsubsection{Две операции сравнения}

\NonOptimizing MSVC прямолинеен: код проверят, находится ли входной символ в интервале [97..122]
(или в интервале [`a'..`z'] ) и вычитает 32 в таком случае.

Имеется также небольшой артефакт компилятора:

\lstinputlisting[caption=\NonOptimizing MSVC 2013 (x64),numbers=left,style=customasmx86]{\CURPATH/MSVC_2013_x64_RU.asm}

Важно отметить что (на строке 3) входной байт загружается в 64-битный слот локального стека.

Все остальные биты ([8..63]) не трогаются, т.е. содержат случайный шум (вы можете увидеть его в отладчике).
% TODO add debugger example

Все инструкции работают только с байтами, так что всё нормально.

Последняя инструкция \TT{MOVZX} на строке 15 берет байт из локального стека и расширяет его 
до 32-битного \Tint, дополняя нулями.

\NonOptimizing GCC делает почти то же самое:

\lstinputlisting[caption=\NonOptimizing GCC 4.9 (x64),style=customasmx86]{\CURPATH/GCC_49_x64_O0.s}

\subsubsection{Одна операция сравнения}
\label{toupper_one_comparison}

\Optimizing MSVC работает лучше, он генерирует только одну операцию сравнения:

\lstinputlisting[caption=\Optimizing MSVC 2013 (x64),style=customasmx86]{\CURPATH/MSVC_2013_Ox_x64.asm}

Уже было описано, как можно заменить две операции сравнения на одну: \myref{one_comparison_instead_of_two}.

Мы бы переписал это на \CCpp так:

\begin{lstlisting}[style=customc]
int tmp=c-97;

if (tmp>25)
        return c;
else
        return c-32;
\end{lstlisting}

Переменная \emph{tmp} должна быть знаковая.

При помощи этого, имеем две операции вычитания в случае конверсии плюс одну операцию сравнения.

В то время как оригинальный алгоритм использует две операции сравнения плюс одну операцию вычитания.

\Optimizing GCC 
даже лучше, он избавился от переходов (а это хорошо: \myref{branch_predictors}) используя инструкцию CMOVcc:

\lstinputlisting[caption=\Optimizing GCC 4.9 (x64),numbers=left,style=customasmx86,label=toupper_GCC_O3]{\CURPATH/GCC_49_x64_O3.s}

На строке 3 код готовит уже сконвертированное значение заранее, как если бы конверсия всегда происходила.

На строке 5 это значение в EAX заменяется нетронутым входным значением, если конверсия не нужна.
И тогда это значение (конечно, неверное), просто выбрасывается.

Вычитание с упреждением это цена, которую компилятор платит за отсутствие условных переходов.

\subsection{ARM}

\Optimizing Keil для режима ARM также генерирует только одну операцию сравнения:

\lstinputlisting[caption=\OptimizingKeilVI (\ARMMode),style=customasmARM]{\CURPATH/Keil_ARM_O3.s}

\myindex{ARM!\Instructions!SUBcc}
\myindex{ARM!\Instructions!ANDcc}

\INS{SUBLS} и \INS{ANDLS} исполняются только если значение \Reg{1} меньше чем 0x19 (или равно).
Они и делают конверсию.

\Optimizing Keil для режима Thumb также генерирует только одну операцию сравнения:

\lstinputlisting[caption=\OptimizingKeilVI (\ThumbMode),style=customasmARM]{\CURPATH/Keil_thumb_O3.s}

\myindex{ARM!\Instructions!LSLS}
\myindex{ARM!\Instructions!LSLR}

Последние две инструкции \INS{LSLS} и \INS{LSRS} работают как \INS{AND reg, 0xFF}:
это аналог \CCpp-выражения $(i<<24)>>24$.

Очевидно, Keil для режима Thumb решил, что две 2-байтных инструкции это короче чем код, загружающий
константу 0xFF плюс инструкция AND.

\subsubsection{GCC для ARM64}

\lstinputlisting[caption=\NonOptimizing GCC 4.9 (ARM64),style=customasmARM]{\CURPATH/GCC_49_ARM64_O0.s}

\lstinputlisting[caption=\Optimizing GCC 4.9 (ARM64),style=customasmARM]{\CURPATH/GCC_49_ARM64_O3.s}

\subsection{Используя битовые операции}
\label{toupper_bit}

Учитывая тот факт, что 5-й бит (считая с 0-его) всегда присутствует после проверки, вычитание его это просто
сброс этого единственного бита, но точно такого же эффекта можно достичть при помощи обычного применения операции
``И'' (\myref{AND_OR_as_SUB_ADD}).

И даже проще, с исключающим ИЛИ:

\lstinputlisting[style=customc]{\CURPATH/toupper2.c}

Код близок к тому, что сгенерировал оптимизирующий GCC для предыдущего примера (\myref{toupper_GCC_O3}):

\lstinputlisting[caption=\Optimizing GCC 5.4 (x86),style=customasmx86]{\CURPATH/toupper2_GCC540_x86_O3.s}

\dots но используется \INS{XOR} вместо \INS{SUB}.

Переворачивание 5-го бита это просто перемещение \textit{курсора} в таблице \ac{ASCII} вверх/вниз на 2 ряда.

Некоторые люди говорят, что буквы нижнего/верхнего регистра были расставлены в \ac{ASCII}-таблице таким манером намеренно,
потому что:

\begin{framed}
\begin{quotation}
Very old keyboards used to do Shift just by toggling the 32 or 16 bit, depending on the key; this is why the relationship between small and capital letters in ASCII is so regular, and the relationship between numbers and symbols, and some pairs of symbols, is sort of regular if you squint at it.
\end{quotation}
\end{framed}

( Eric S. Raymond, \url{http://www.catb.org/esr/faqs/things-every-hacker-once-knew/} )

Следовательно, мы можем написать такой фрагмент кода, который просто меняет регистр букв:

\lstinputlisting[style=customc]{\CURPATH/flip_EN.c}

\subsection{Итог}

Все эти оптимизации компиляторов очень популярны в наше время и практикующий
reverse engineer обычно часто видит такие варианты кода.
}
\DE{\subsubsection{Struct als Menge von Werten}
Um zu veranschaulichen, dass ein struct nur eine Menge von nebeneinanderliegenden Variablen ist, überarbeiten wir unser
Beispiel, indem wir auf die Definition des \emph{tm} structs schauen:\lstref{struct_tm}.

\lstinputlisting[style=customc]{patterns/15_structs/3_tm_linux/as_array/GCC_tm2.c}

\myindex{\CStandardLibrary!localtime\_r()}
Der Pointer auf das Feld \TT{tm\_sec} wird nach \TT{localtime\_r} übergeben, d.h. an das erste Element des structs.

Der Compiler warnt uns:

\begin{lstlisting}[caption=GCC 4.7.3]
GCC_tm2.c: In function 'main':
GCC_tm2.c:11:5: warning: passing argument 2 of 'localtime_r' from incompatible pointer type [enabled by default]
In file included from GCC_tm2.c:2:0:
/usr/include/time.h:59:12: note: expected 'struct tm *' but argument is of type 'int *'
\end{lstlisting}

Trotzdem erzeugt er folgenden Code:

\lstinputlisting[caption=GCC 4.7.3,style=customasmx86]{patterns/15_structs/3_tm_linux/as_array/GCC_tm2.asm}
Dieser Code ist zum vorherigen identisch und es ist unmöglich zu sagen, ob es sich im originalen Quellcode um ein struct
oder nur um eine Menge von Variablen handelt.

Es funktioniert also, ist aber in der Praxis nicht empfehlenswert. 

Nicht optimierende Compiler legen normalerweise Variablen auf dem lokalen Stack in der Reihenfolge an, in der sie in der
Funktion deklariert wurden.

Ein Garantie dafür gibt es freilich nicht.

Andere Compiler könnten an dieser Stelle übrigens davor warnen, dass die Variablen \TT{tm\_year}, \TT{tm\_mon}, \TT{tm\_mday},
\TT{tm\_hour}, \TT{tm\_min} - nicht aber \TT{tm\_sec} - ohne Initialisierung verwendet werden.

Der Compiler weiß nicht, dass diese durch die Funktion \TT{localtime\_r()} befüllt werden.

Wir haben dieses Beispiel ausgewählt, da alle Felder im struct vom Typ \Tint sind.

Es würde nicht funktionieren, wenn die Felder 16 Bit (\TT{WORD}) groß wären, wie im Beispiel des \TT{SYSTEMTIME}
structs---\TT{GetSystemTime()} würde sie falsch befüllen (da die lokalen Variablen auf 32-Bit-Grenzen angeordnet sind).
Mehr dazu im folgenden Abschnitt: \q{\StructurePackingSectionName} (\myref{structure_packing}).

Ein struct ist also nichts als eine Menge von an einer Stelle gespeicherten Variablen.
Man kan sagen, dass das struct ein Befehl an den Compiler ist, diese Variablen an einer Stelle zu halten.
In ganz frühen Versionen von C (vor 1972) gab es übrigens gar keine structs \RitchieDevC.

Dieses Beispiel wird nicht im Debugger gezeigt, da es dem gerade gezeigten entspricht.

\subsubsection{Struct als Array aus 32-Bit-Worten}

\lstinputlisting[style=customc]{patterns/15_structs/3_tm_linux/as_array/GCC_tm3.c}
Wir können einen Pointer auf ein struct in ein Array aus \Tint{}s casten und es funktioniert.
Wir lassen dieses Beispiel zur Systemzeit 23:51:45 26-July-2014 laufen.

\begin{lstlisting}[label=GCC_tm3_output]
0x0000002D (45)
0x00000033 (51)
0x00000017 (23)
0x0000001A (26)
0x00000006 (6)
0x00000072 (114)
0x00000006 (6)
0x000000CE (206)
0x00000001 (1)
\end{lstlisting}
Die Variablen sind hier in der gleichen Reihenfolge, in der die in der Definition des structs aufgezählt
werden:\myref{struct_tm}.

Hier ist der erzeugte Code:

\lstinputlisting[caption=\Optimizing GCC
4.8.1,style=customasmx86]{patterns/15_structs/3_tm_linux/as_array/GCC_tm3_DE.lst}
Tatsächlich: der Platz auf dem lokalen Stack wird zuerst wie in struct und dann wie ein Array behandelt.

Es ist sogar möglich, die Felder des structs über diesen Pointer zu verändern.

Und wiederum ist es zweifellos ein seltsamer Weg die Dinge umzusetzen; er ist für produktiven Code definitiv nicht
empfehlenswert.

\mysubparagraph{\Exercise}
Versuchen Sie als Übung die Monatsnummer zu verändern (um 1 zu erhöhen), indem Sie das struct wie ein Array behandeln.

\subsubsection{Struct als Bytearray}
Wir können sogar noch weiter gehen. Casten wir den Pointer zu einem Bytearray und ziehen einen Dump:

\lstinputlisting[style=customc]{patterns/15_structs/3_tm_linux/as_array/GCC_tm4.c}

\begin{lstlisting}
0x2D 0x00 0x00 0x00 
0x33 0x00 0x00 0x00 
0x17 0x00 0x00 0x00 
0x1A 0x00 0x00 0x00 
0x06 0x00 0x00 0x00 
0x72 0x00 0x00 0x00 
0x06 0x00 0x00 0x00 
0xCE 0x00 0x00 0x00 
0x01 0x00 0x00 0x00 
\end{lstlisting}
Wir haben dieses Beispiel auch zur Systemzeit 23:51:45 26-July-2014 ausgeführt
\footnote{Datum und Uhrzeit sind zu Demonstrationszwecken identisch. Die Bytewerte sind modifiziert.}.
Die Werte sind genau dieselben wie im vorherigen Dump(\myref{GCC_tm3_output}) und natürlich steht das LSB vorne, da es
sich um eine Little-Endian-Architektur handelt(\myref{sec:endianness}). 

\lstinputlisting[caption=\Optimizing GCC
4.8.1,style=customasmx86]{patterns/15_structs/3_tm_linux/as_array/GCC_tm4_DE.lst}
}
\FR{\subsection{Méthodes de protection contre les débordements de tampon}
\label{subsec:BO_protection}

Il existe quelques méthodes pour protéger contre ce fléau, indépendamment de la négligence
des programmeurs \CCpp.
MSVC possède des options comme\footnote{méthode de protection contre les débordements
de tampons côté compilateur:\href{http://go.yurichev.com/17133}{wikipedia.org/wiki/Buffer\_overflow\_protection}}:

\begin{lstlisting}
 /RTCs Stack Frame runtime checking
 /GZ Enable stack checks (/RTCs)
\end{lstlisting}

\myindex{x86!\Instructions!RET}
\myindex{Function prologue}
\myindex{Security cookie}

Une des méthodes est d'écrire une valeur aléatoire entre les variables locales sur
la pile dans le prologue de la fonction et de la vérifier dans l'épilogue, avant de
sortir de la fonction.
Si la valeur n'est pas la même, ne pas exécuter la dernière instruction \RET, mais
stopper (ou bloquer).
Le processus va s'arrêter, mais c'est mieux qu'une attaque distante sur votre ordinateur.
    
\newcommand{\CANARYURL}{\href{http://go.yurichev.com/17134}{wikipedia.org/wiki/Domestic\_canary\#Miner.27s\_canary}}

\myindex{Canary}

Cette valeur aléatoire est parfois appelé un \q{canari}, c'est lié au canari\footnote{\CANARYURL}
que les mineurs utilisaient dans le passé afin de détecter rapidement les gaz toxiques.

Les canaris sont très sensibles aux gaz, ils deviennent très agités en cas de danger,
et même meurent.

Si nous compilons notre exemple de tableau très simple~(\myref{arrays_simple}) dans
\ac{MSVC} avec les options RTC1 et RTCs, nous voyons un appel à \TT{@\_RTC\_CheckStackVars@8}
une fonction à la fin de la fonction qui vérifie si le \q{canari} est correct.

Voyons comment GCC gère ceci.
Prenons un exemple \TT{alloca()}~(\myref{alloca}):

\lstinputlisting[style=customc]{patterns/02_stack/04_alloca/2_1.c}

Par défaut, sans option supplémentaire, GCC 4.7.3 insère un test de  \q{canari} dans
le code:

\lstinputlisting[caption=GCC 4.7.3,style=customasmx86]{patterns/13_arrays/3_BO_protection/gcc_canary_FR.asm}

\myindex{x86!\Registers!GS}
La valeur aléatoire se trouve en \TT{gs:20}.
Elle est écrite sur la pile et à la fin de la fonction, la valeur sur la pile est
comparée avec le \q{canari} correct dans \TT{gs:20}.
Si les valeurs ne sont pas égales, la fonction \TT{\_\_stack\_chk\_fail} est appelée
et nous voyons dans la console quelque chose comme ça (Ubuntu 13.04 x86):

\begin{lstlisting}
*** buffer overflow detected ***: ./2_1 terminated
======= Backtrace: =========
/lib/i386-linux-gnu/libc.so.6(__fortify_fail+0x63)[0xb7699bc3]
/lib/i386-linux-gnu/libc.so.6(+0x10593a)[0xb769893a]
/lib/i386-linux-gnu/libc.so.6(+0x105008)[0xb7698008]
/lib/i386-linux-gnu/libc.so.6(_IO_default_xsputn+0x8c)[0xb7606e5c]
/lib/i386-linux-gnu/libc.so.6(_IO_vfprintf+0x165)[0xb75d7a45]
/lib/i386-linux-gnu/libc.so.6(__vsprintf_chk+0xc9)[0xb76980d9]
/lib/i386-linux-gnu/libc.so.6(__sprintf_chk+0x2f)[0xb7697fef]
./2_1[0x8048404]
/lib/i386-linux-gnu/libc.so.6(__libc_start_main+0xf5)[0xb75ac935]
======= Memory map: ========
08048000-08049000 r-xp 00000000 08:01 2097586    /home/dennis/2_1
08049000-0804a000 r--p 00000000 08:01 2097586    /home/dennis/2_1
0804a000-0804b000 rw-p 00001000 08:01 2097586    /home/dennis/2_1
094d1000-094f2000 rw-p 00000000 00:00 0          [heap]
b7560000-b757b000 r-xp 00000000 08:01 1048602    /lib/i386-linux-gnu/libgcc_s.so.1
b757b000-b757c000 r--p 0001a000 08:01 1048602    /lib/i386-linux-gnu/libgcc_s.so.1
b757c000-b757d000 rw-p 0001b000 08:01 1048602    /lib/i386-linux-gnu/libgcc_s.so.1
b7592000-b7593000 rw-p 00000000 00:00 0
b7593000-b7740000 r-xp 00000000 08:01 1050781    /lib/i386-linux-gnu/libc-2.17.so
b7740000-b7742000 r--p 001ad000 08:01 1050781    /lib/i386-linux-gnu/libc-2.17.so
b7742000-b7743000 rw-p 001af000 08:01 1050781    /lib/i386-linux-gnu/libc-2.17.so
b7743000-b7746000 rw-p 00000000 00:00 0
b775a000-b775d000 rw-p 00000000 00:00 0
b775d000-b775e000 r-xp 00000000 00:00 0          [vdso]
b775e000-b777e000 r-xp 00000000 08:01 1050794    /lib/i386-linux-gnu/ld-2.17.so
b777e000-b777f000 r--p 0001f000 08:01 1050794    /lib/i386-linux-gnu/ld-2.17.so
b777f000-b7780000 rw-p 00020000 08:01 1050794    /lib/i386-linux-gnu/ld-2.17.so
bff35000-bff56000 rw-p 00000000 00:00 0          [stack]
Aborted (core dumped)
\end{lstlisting}

\myindex{MS-DOS}
gs est ainsi appelé registre de segment. Ces registres étaient beaucoup utilisés
du temps de MS-DOS et des extensions de DOS.
Aujourd'hui, sa fonction est différente.
\myindex{TLS}
\myindex{Windows!TIB}

Dit brièvement, le registre \TT{gs} dans Linux pointe toujours sur le
\ac{TLS}~(\myref{TLS})---des informations spécifiques au thread sont stockées là.
À propos, en win32 le registre \TT{fs} joue le même rôle, pointant sur \ac{TIB}
\footnote{\href{http://go.yurichev.com/17104}{wikipedia.org/wiki/Win32\_Thread\_Information\_Block}}.

Il y a plus d'information dans le code source du noyau Linux (au moins dans la version 3.11),
dans\\
\emph{arch/x86/include/asm/stackprotector.h} cette variable est décrite dans les commentaires.

\subsubsection{ARM: \OptimizingKeilVI (\ARMMode)}
\myindex{\CLanguageElements!switch}

\lstinputlisting[style=customasmARM]{patterns/08_switch/1_few/few_ARM_ARM_O3.asm}

A nouveau, en investiguant ce code, nous ne pouvons pas dire si il y avait un switch()
dans le code source d'origine ou juste un ensemble de déclarations if().

\myindex{ARM!\Instructions!ADRcc}

En tout cas, nous voyons ici des instructions conditionnelles (comme \ADREQ (\emph{Equal}))
qui ne sont exécutées que si $R0=0$, et qui chargent ensuite l'adresse de la chaîne
\emph{<<zero\textbackslash{}n>>} dans \Reg{0}.
\myindex{ARM!\Instructions!BEQ}
L'instruction suivante \ac{BEQ} redirige le flux d'exécution en \TT{loc\_170}, si $R0=0$.

Le lecteur attentif peut se demander si \ac{BEQ} s'exécute correctement puisque \ADREQ
a déjà mis une autre valeur dans le registre \Reg{0}.

Oui, elle s'exécutera correctement, car \ac{BEQ} vérifie les flags mis par l'instruction
\CMP et \ADREQ ne modifie aucun flag.

Les instructions restantes nous sont déjà familières.
Il y a seulement un appel à \printf, à la fin, et nous avons déjà examiné cette
astuce ici~(\myref{ARM_B_to_printf}).
A la fin, il y a trois chemins vers \printf{}.

\myindex{ARM!\Instructions!ADRcc}
\myindex{ARM!\Instructions!CMP}
La dernière instruction, \TT{CMP R0, \#2}, est nécessaire pour vérifier si $a=2$.

Si ce n'est pas vrai, alors \ADRNE charge un pointeur sur la chaîne \emph{<<something unknown \textbackslash{}n>>}
dans \Reg{0}, puisque $a$ a déjà été comparée pour savoir s'elle est égale
à 0 ou 1, et nous sommes sûrs que la variable $a$ n'est pas égale à l'un de
ces nombres, à ce point.
Et si $R0=2$, un pointeur sur la chaîne \emph{<<two\textbackslash{}n>>} sera chargé
par \ADREQ dans \Reg{0}.

\subsubsection{ARM: \OptimizingKeilVI (\ThumbMode)}

\lstinputlisting[style=customasmARM]{patterns/08_switch/1_few/few_ARM_thumb_O3.asm}

% FIXME а каким можно? к каким нельзя? \myref{} ->

Comme il y déjà été dit, il n'est pas possible d'ajouter un prédicat conditionnel
à la plupart des instructions en mode Thumb, donc ce dernier est quelque peu similaire
au code \ac{CISC}-style x86, facilement compréhensible.

\subsubsection{ARM64: GCC (Linaro) 4.9 \NonOptimizing}

\lstinputlisting[style=customasmARM]{patterns/08_switch/1_few/ARM64_GCC_O0_FR.lst}

Le type de la valeur d'entrée est \Tint, par conséquent le registre \RegW{0} est
utilisé pour garder la valeur au lieu du registre complet \RegX{0}.

Les pointeurs de chaîne sont passés à \puts en utilisant la paire d'instructions
\INS{ADRP}/\INS{ADD} comme expliqué dans l'exemple \q{\HelloWorldSectionName}:~\myref{pointers_ADRP_and_ADD}.

\subsubsection{ARM64: GCC (Linaro) 4.9 \Optimizing}

\lstinputlisting[style=customasmARM]{patterns/08_switch/1_few/ARM64_GCC_O3_FR.lst}

Ce morceau de code est mieux optimisé.
L'instruction \TT{CBZ} (\emph{Compare and Branch on Zero} comparer et sauter si zéro)
effectue un saut si \RegW{0} vaut zéro.
Il y a alors un saut direct à \puts au lieu de l'appeler, comme cela a été expliqué
avant:~\myref{JMP_instead_of_RET}.


}
\JPN{\subsection{2次元配列としての文字列のパック}

月の名前を返す関数を再考してみましょう:\lstref{get_month1}

月の名前の文字列へのポインタを準備するには少なくともメモリロード演算が1つ必要です。

メモリロード演算を取り除くことは可能でしょうか?

実際できます。文字列のリストを2次元配列として表現すれば。

\lstinputlisting[style=customc]{patterns/13_arrays/55_month_2D/month2_JPN.c}

このような結果を得ました。

\lstinputlisting[caption=\Optimizing MSVC 2013 x64,style=customasmx86]{patterns/13_arrays/55_month_2D/MSVC2013_x64_Ox_JPN.asm}

メモリアクセスは全くありません。

この関数でやっていることは、月の名前の最初の文字のポインタを計算することです:
$pointer\_to\_the\_table + month * 10$.

\LEA 命令も2つあります。 いくつかの \MUL と \MOV 命令として機能します。

配列の幅は10バイトです。

実際、ここでの最も長い文字列、\q{September}、は9バイトで、加えて0終端して10バイトです。

月の名前の残りはゼロで埋められて、月の名前は同じ領域(10バイト)を占有します。

従って、関数はより早く機能します。文字列の開始アドレスが簡単に計算できるためです。

\Optimizing GCC 4.9はより短くなります。

\begin{lstlisting}[caption=\Optimizing GCC 4.9 x64,style=customasmx86]
	movsx	rdi, edi
	lea	rax, [rdi+rdi*4]
	lea	rax, month2[rax+rax]
	ret
\end{lstlisting}

\LEA は10倍するためにここでも使用されます。

最適化されていないコンパイラは、異なる方法で乗算を行います。

\lstinputlisting[caption=\NonOptimizing GCC 4.9 x64,style=customasmx86]{patterns/13_arrays/55_month_2D/x64_GCC49_O0_JPN.asm}

\NonOptimizing MSVCは単に \IMUL 命令を使用します。

\myindex{x86!\Instructions!IMUL}

\lstinputlisting[caption=\NonOptimizing MSVC 2013 x64,style=customasmx86]{patterns/13_arrays/55_month_2D/MSVC2013_x64_JPN.asm}

\myindex{\CompilerAnomaly}
\label{MSVC2013_anomaly}

しかし、奇妙なことが1つあります。なぜ、0で乗算し、最終結果に0を加算するのでしょうか?

これはコンパイラのコードジェネレータの癖のように見えますが、コンパイラのテストでは検出されませんでした。
(結局のところ、結果のコードは正しく動作します)
% класс!
%
このようなコードを意図的に検討することで、読者がそのようなコンパイラ成果物に困惑すべきでないときが
あることを理解するでしょう。

\subsubsection{32ビットARM}

\Optimizing Keil 
Thumbモードでは、乗算命令\INS{MULS}を使用します。

\lstinputlisting[caption=\OptimizingKeilVI (\ThumbMode),style=customasmARM]{patterns/13_arrays/55_month_2D/Keil_O3_thumb_JPN.asm}

ARMモードでの \Optimizing Keil は加算とシフト命令を使用します。

\lstinputlisting[caption=\OptimizingKeilVI (\ARMMode),style=customasmARM]{patterns/13_arrays/55_month_2D/Keil_O3_ARM_JPN.asm}

\subsubsection{ARM64}

\lstinputlisting[caption=\Optimizing GCC 4.9 ARM64,style=customasmARM]{patterns/13_arrays/55_month_2D/GCC49_ARM64_JPN.asm}

\myindex{ARM!\Instructions!SXTW}
\myindex{ARM!\Instructions!ADRP/ADD pair}

\INS{SXTW}は32ビット入力値を64ビットにし、X0に保存する、符号拡張のために使用されます。

\ADRP/\ADD の命令の組はテーブルのアドレスをロードするために使用されます。

\ADD 命令には乗算に役立つ \LSL サフィックスもあります。

\subsubsection{MIPS}
\lstinputlisting[caption=\Optimizing GCC 4.4.5 (IDA),style=customasmMIPS]{patterns/13_arrays/55_month_2D/MIPS_O3_IDA_JPN.lst}

\subsubsection{\Conclusion{}}

これはテキスト文字列を保存するための昔ながらの技術です。
あなたは、たとえば、 \oracle でそれを見つけることができます。
現代のコンピュータで実行する価値があるかどうかは言い難いですが、
配列の良い例であるため、この本に追加されました。
}

\EN{\subsection{\Conclusion{}}

An array is a pack of values in memory located adjacently.

It's true for any element type, including structures.

Access to a specific array element is just a calculation of its address.

\myindex{Hex-Rays}
So, a pointer to an array and address of a first element---is the same thing.
This is why \TT{ptr[0]} and \TT{*ptr} expressions are equivalent in \CCpp.
It's interesting to note that Hex-Rays often replaces the first by the second.
It does so when it have no idea that it works with pointer to the whole array,
and thinks that this is a pointer to single variable.
}
\RU{\subsection{\Conclusion{}}

Массив это просто набор значений в памяти, расположенных рядом друг с другом.

Это справедливо для любых типов элементов, включая структуры.

Доступ к определенному элементу массива это просто вычисление его адреса.

\myindex{Hex-Rays}
Итак, указатель на массив и адрес первого элемента --- это одно и то же.
Вот почему выражения \TT{ptr[0]} и \TT{*ptr} в \CCpp равноценны.
Любопытно что Hex-Rays часто заменяет первое вторым.
Он делает это в тех случаях, когда не знает, что имеет дело с указателем на целый массив,
и думает, что это указатель только на одну переменную.

}
\DE{\subsection{\Conclusion{}}

Ein Array ist eine Ansammlung von Werten, die im Speicher nebeneinander angeordnet sind.

Dies gilt für alle Elementtypen und sogar für Structs.

Der Zugriff auf ein spezielles Element des Arrays entspricht lediglich eine Berechnung von dessen Adresse.

\myindex{Hex-Rays}
% TBT

}
\FR{\subsection{\Conclusion{}}

Un tableau est un ensemble de données adjacentes en mémoire.

C'est vrai pour tout type d'élément, structures incluses.

Pour accéder à un élément spécifique d'un tableau, il suffit de calculer son adresse.

\myindex{Hex-Rays}
Donc, un pointeur sur un tableau et l'adresse de son premier élément---sont la même
chose.
C'est pourquoi les expressions \TT{ptr[0]} et \TT{*ptr} sont équivalentes en \CCpp.
Il est intéressant de noter que Hex-Rays remplace souvent la première par la seconde.
Il procède ainsi lorsqu'il n'a aucune idée qu'il travaille avec un pointeur sur
le tableau complet et pense que c'est un pointeur sur une seule variable.
}
\JPN{\subsection{\Conclusion{}}

配列は、隣り合って配置されたメモリ内の値の束です。

構造体を含むあらゆる要素種別に当てはまります。

特定の配列要素へのアクセスは、そのアドレスの計算に過ぎません。
}

\myindex{Hex-Rays}

\RU{\mysection{Кстати}
Итак, указатель на массив и адрес первого элемента --- это одно и то же.
Вот почему выражения \TT{ptr[0]} и \TT{*ptr} в \CCpp равноценны.
Любопытно что Hex-Rays часто заменяет первое вторым.
Он делает это в тех случаях, когда не знает, что имеет дело с указателем на целый массив,
и думает, что это указатель только на одну переменную.}%
\EN{\mysection{By the way}
So, pointer to an array and address of a first element---is the same thing.
This is why \TT{ptr[0]} and \TT{*ptr} expressions are equivalent in \CCpp.
It's interesting to note that Hex-Rays often replaces the first by the second.
It does so when it have no idea that it works with pointer to the whole array,
and thinks that this is a pointer to single variable.}
\DEph{}
\FR{\mysection{À propos}
Donc, un pointeur sur un tableau et l'adresse de son premier élément---sont la même
chose.
C'est pourquoi les expressions \TT{ptr[0]} et \TT{*ptr} sont équivalentes en \CCpp.
Il est intéressant de noter que Hex-Rays remplace souvent la première par la seconde.
Il procède ainsi lorsqu'il n'a aucune idée qu'il travaille avec un pointeur sur
le tableau complet et pense que c'est un pointeur sur une seule variable.}
\JPN{\mysection{ところで}
したがって、最初の要素の配列とアドレスへのポインタは同じことです。 
このため、\TT{ptr[0]}と\TT{*ptr}の式は \CCpp で同等です。 
Hex-Raysはしばしば最初のものを2番目のものに置き換えることは興味深いことです。 
これは、配列全体へのポインタで動作するかどうかわからないときに行い、
これが単一変数へのポインタであると考えます。}
\input{patterns/13_arrays/exercises}
