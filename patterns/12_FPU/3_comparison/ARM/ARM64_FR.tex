\subsubsection{ARM64}

\myparagraph{GCC (Linaro) 4.9 \Optimizing}

\lstinputlisting[style=customasmARM]{patterns/12_FPU/3_comparison/ARM/ARM64_GCC_O3_FR.lst}

L'ARM64 \ac{ISA} possède des instructions FPU qui mettent les flags CPU \ac{APSR}
au lieu de \ac{FPSCR}, par commodité.
Le \ac{FPU} n'est plus un device séparé (au moins, logiquement).
\myindex{ARM!\Instructions!FCMPE}
Ici, nous voyons \INS{FCMPE}. Ceci compare les deux valeurs passées dans \RegD{0}
et \RegD{1} (qui sont le premier et le second argument de la fonction) et met les
flags \ac{APSR} (N, Z, C, V).

\myindex{ARM!\Instructions!FCSEL}
\INS{FCSEL} (\emph{Floating Conditional Select} (sélection de flottant conditionnelle)
copie la valeur de \RegD{0} ou \RegD{1} dans \RegD{0} suivant le résultat de la comparaison
(\GTT{GT}---Greater Than), et de nouveau, il utilise les flags dans le registre \ac{APSR}
au lieu de \ac{FPSCR}.

Ceci est bien plus pratique, comparé au jeu d'instructions des anciens CPUs.

Si la condition est vraie (\GTT{GT}), alors la valeur de \RegD{0} est copiée dans
\RegD{0} (i.e., il ne se passe rien).
Si la condition n'est pas vraie, la valeur de \RegD{1} est copiée dans \RegD{0}.

\myparagraph{GCC (Linaro) 4.9 \NonOptimizing}

\lstinputlisting[style=customasmARM]{patterns/12_FPU/3_comparison/ARM/ARM64_GCC_FR.lst}

GCC sans optimisation est plus verbeux.

Tout d'abord, la fonction sauve la valeur de ses arguments en entrée dans la pile
locale (\emph{Register Save Area}, espace de sauvegarde des registres).
Ensuite, le code recharge ces valeurs dans les registres \RegX{0}/\RegX{1} et finalement
les copie dans \RegD{0}/\RegD{1} afin de les comparer en utilisant \INS{FCMPE}.
Beaucoup de code redondant, mais c'est ainsi que fonctionne les compilateurs sans
optimisation.
\INS{FCMPE} compare les valeurs et met les flags du registre \ac{APSR}.
À ce moment, le compilateur ne pense pas encore à l'instruction plus commode \INS{FCSEL},
donc il procède en utilisant de vieilles méthodes:
en utilisant l'instruction \INS{BLE} (\emph{Branch if Less than or Equal} branchement si
inférieur ou égal).
Dans le premier cas ($a>b$), la valeur de $a$ est chargée dans \RegX{0}.
Dans les autres cas ($a<=b$), la valeur de $b$ est chargée dans \RegX{0}.
Enfin, la valeur dans \RegX{0} est copiée dans \RegD{0}, car la valeur de retour
doit être dans ce registre.

\mysubparagraph{\Exercise}

À titre d'exercice, vous pouvez essayer d'optimiser ce morceau de code manuellement
en supprimant les instructions redondantes et sans en introduire de nouvelles (incluant
\INS{FCSEL}).

\myparagraph{GCC (Linaro) 4.9 \Optimizing---float}

Ré-écrivons cet exemple en utilisant des \Tfloat à la place de \Tdouble.

\begin{lstlisting}[style=customc]
float f_max (float a, float b)
{
	if (a>b)
		return a;

	return b;
};
\end{lstlisting}

\lstinputlisting[style=customasmARM]{patterns/12_FPU/3_comparison/ARM/ARM64_GCC_O3_float_FR.lst}

C'est le même code, mais des S-registres sont utilisés à la place de D-registres.
C'est parce que les nombres de type \Tfloat sont passés dans des S-registres de 32-bit
(qui sont en fait la partie basse des D-registres 64-bit).

