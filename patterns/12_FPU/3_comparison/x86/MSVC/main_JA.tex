\myparagraph{\NonOptimizing MSVC}

MSVC 2010は以下のコードを生成します。

\lstinputlisting[caption=\NonOptimizing MSVC 2010,style=customasmx86]{patterns/12_FPU/3_comparison/x86/MSVC/MSVC_JA.asm}

\myindex{x86!\Instructions!FLD}

\FLD は\GTT{\_b}を\ST{0}にロードします。

\label{Czero_etc}
\newcommand{\Czero}{\GTT{C0}\xspace}
\newcommand{\Ctwo}{\GTT{C2}\xspace}
\newcommand{\Cthree}{\GTT{C3}\xspace}
\newcommand{\CThreeBits}{\Cthree/\Ctwo/\Czero}

\myindex{x86!\Instructions!FCOMP}

\FCOMP は\ST{0}の値と\GTT{\_a}の値を比較し、
それに応じてFPUステータスワードレジスタの \CThreeBits ビットを設定します。
これは、FPUの現在の状態を反映する16ビットのレジスタです。

ビットがセットされると、 \FCOMP 命令はスタックから1つの変数もポップします。
これは、値を比較してスタックを同じ状態にしておく \FCOM とは区別されます。

残念ながら、インテルP6 
\footnote{インテルP6はPentium Pro、Pentium IIなどです。}
より前のCPUには、 \CThreeBits ビットをチェックする条件付きジャンプ命令はありません。
おそらく、それは歴史の問題です。(思い起こしてみてください:FPUは過去に別のチップでした)

インテルP6で始まる最新のCPUは、\FCOMI/\FCOMIP/\FUCOMI/\FUCOMIP 命令を持っていて、
同じことをしますが、 \ZF/\PF/\CF CPUフラグを変更します。

\myindex{x86!\Instructions!FNSTSW}

\FNSTSW 命令は状態レジスタであるFPUを \AX にコピーします。 
\CThreeBits ビットは14/10/8の位置に配置され、
\AX レジスタの同じ位置にあり、 \AX{}~---\AH{} の上位部分に配置されます。

\begin{itemize}
\item この例では $b>a$ の場合、 \CThreeBits ビットは0,0,0と設定します。
\item $a>b$ の場合、ビットは0,0,1です。
\item $a=b$ の場合、ビットは1,0,0です。
\item 結果が順序付けられていない場合(エラーの場合)、セットされたビットは1,1,1,1です。
\end{itemize}
% TODO: table here?

これは、 \CThreeBits ビットが \AX レジスタにどのように配置されるかを示しています。

\input{C3_in_AX}

これは、 \CThreeBits ビットが \AH レジスタにどのように配置されるかを示しています。

\input{C3_in_AH}

\INS{test ah, 5}\footnote{5=101b}の実行後、
\Czero と \Ctwo ビット(0と2の位置)のみが考慮され、他のビットはすべて無視されます。

\label{parity_flag}
\myindex{x86!\Registers!\Flags!Parity flag}

さて、\emph{パリティーフラグ}と注目すべきもう1つの歴史的基礎についてお話しましょう。

このフラグは、最後の計算結果の1の数が偶数の場合は1に設定され、奇数の場合は0に設定されます。

Wikipedia\footnote{\href{http://go.yurichev.com/17131}{wikipedia.org/wiki/Parity\_flag}}を見てみましょう:

\begin{framed}
\begin{quotation}
パリティフラグをテストする一般的な理由の1つに、無関係なFPUフラグをチェックすることがあります。 FPUには4つの条件フラグ
(C0~C3)がありますが、直接テストすることはできず、最初にフラグレジスタにコピーする必要があります。 
これが起こると、C0はキャリーフラグに、C2はパリティフラグに、C3はゼロフラグに置かれます。 
C2フラグは、例えば比較できない浮動小数点値(NaNまたはサポートされていない形式)がFUCOM命令と比較されます。
\end{quotation}
\end{framed}

Wikipediaで述べられているように、パリティフラグはFPUコードで使用されることがあります。

\myindex{x86!\Instructions!JP}

\Czero と \Ctwo の両方が0に設定されている場合、 \PF フラグは1に設定されます。その場合、
後続の \JP (\emph{jump if PF==1})が実行されます。 
いろいろな場合の \CThreeBits の値を思い出すと、
条件ジャンプ \JP は、 $b>a$ または $a=b$ の場合に実行されます。
(\INS{test ah, 5}命令によってクリアされているので、 \Cthree ビットはここでは考慮されていません)

それ以降はすべて簡単です。 
条件付きジャンプが実行された場合、
\FLD は\ST{0}の\GTT{\_b}の値をロードし、
実行されていなければ\GTT{\_a}の値をロードします。

\mysubparagraph{\Ctwo? のチェックは?}

\Ctwo フラグはエラー(\gls{NaN}など)の場合に設定されますが、コードではチェックされません。

プログラマがFPUエラーを気にする場合は、チェックを追加する必要があります。

\clearpage
\mysubparagraph{最初の \olly の例: a=1.2 と b=3.4}
\myindex{\olly}

\olly で例をロードしてみましょう。

\begin{figure}[H]
\centering
\myincludegraphics{patterns/12_FPU/3_comparison/x86/MSVC/olly1_1.png}
\caption{\olly: 最初の \FLD が実行される}
\label{fig:FPU_comparison_case1_olly1}
\end{figure}

関数の引数:$a=1.2$ と $b=3.4$ (スタックで見ることができます。32ビット値の2組のペアです)
$b$ (3.4) は \ST{0}にすでにロードされています。
そして \FCOMP が実行されます。
\olly は次の \FCOMP の引数を表示します。スタックにあります。

\clearpage
\FCOMP が実行されます。

\begin{figure}[H]
\centering
\myincludegraphics{patterns/12_FPU/3_comparison/x86/MSVC/olly1_2.png}
\caption{\olly: \FCOMP が実行される}
\label{fig:FPU_comparison_case1_olly2}
\end{figure}

\ac{FPU}条件フラグの状態を見ることができます。すべて0です。
ポップされた値は\ST{7}に反映されます。この理由については前に書きました:
\myref{FPU_is_rather_circular_buffer}.

\clearpage
\FNSTSW が実行されます。
\begin{figure}[H]
\centering
\myincludegraphics{patterns/12_FPU/3_comparison/x86/MSVC/olly1_3.png}
\caption{\olly: \FNSTSW が実行される}
\label{fig:FPU_comparison_case1_olly3}
\end{figure}

\GTT{AX}レジスタが0であるのが見えます。実際、条件フラグはすべてゼロです。
( \olly は \FNSTSW 命令を\INS{FSTSW}としてディスアセンブルします。これは同義語です)

\clearpage
\TEST が実行されます。

\begin{figure}[H]
\centering
\myincludegraphics{patterns/12_FPU/3_comparison/x86/MSVC/olly1_4.png}
\caption{\olly: \TEST が実行される}
\label{fig:FPU_comparison_case1_olly4}
\end{figure}

\GTT{PF}フラグが1にセットされます。

実際、0にセットされるビットの数は0で0は偶数です。
\olly は\INS{JP}を\ac{JPE}としてディスアセンブルします。これは同義語です。
そして、実行されます。

\clearpage
\ac{JPE}が実行され、 \FLD は $b$ (3.4)の値を \ST{0}にロードします。

\begin{figure}[H]
\centering
\myincludegraphics{patterns/12_FPU/3_comparison/x86/MSVC/olly1_5.png}
\caption{\olly: 次の \FLD が実行される}
\label{fig:FPU_comparison_case1_olly5}
\end{figure}

関数が終了します。

\clearpage
\mysubparagraph{次の \olly の例: a=5.6 と b=-4}

例を \olly にロードしてみましょう。

\begin{figure}[H]
\centering
\myincludegraphics{patterns/12_FPU/3_comparison/x86/MSVC/olly2_1.png}
\caption{\olly: 最初の \FLD が実行される}
\label{fig:FPU_comparison_case2_olly1}
\end{figure}

関数の引数:$a=5.6$ と $b=-4$
$b$ (-4) はすでに \ST{0}にロードされています。
\FCOMP は実行されます。
\olly は次の \FCOMP の引数を表示します。スタックにあります。

\clearpage
\FCOMP が実行されます。

\begin{figure}[H]
\centering
\myincludegraphics{patterns/12_FPU/3_comparison/x86/MSVC/olly2_2.png}
\caption{\olly: \FCOMP が実行される}
\label{fig:FPU_comparison_case2_olly2}
\end{figure}

\ac{FPU}条件フラグの状態を見ることができます。 \Czero を除いてすべてゼロです。

\clearpage
\FNSTSW が実行されます。

\begin{figure}[H]
\centering
\myincludegraphics{patterns/12_FPU/3_comparison/x86/MSVC/olly2_3.png}
\caption{\olly: \FNSTSW が実行される}
\label{fig:FPU_comparison_case2_olly3}
\end{figure}

\GTT{AX}レジスタが\GTT{0x100}であるのが見えます。 \Czero フラグは8番目のビットです。

\clearpage
\TEST が実行されます。

\begin{figure}[H]
\centering
\myincludegraphics{patterns/12_FPU/3_comparison/x86/MSVC/olly2_4.png}
\caption{\olly: \TEST が実行される}
\label{fig:FPU_comparison_case2_olly4}
\end{figure}

\GTT{PF}フラグがクリアされます。
実際、

\GTT{0x100}にセットされるビットの数は1で、1は奇数です。
\ac{JPE}はスキップされます。

\clearpage
\ac{JPE}は実行されず、 \FLD は $a$ (5.6)の値を\ST{0}にロードします。

\begin{figure}[H]
\centering
\myincludegraphics{patterns/12_FPU/3_comparison/x86/MSVC/olly2_5.png}
\caption{\olly: 次の \FLD が実行される}
\label{fig:FPU_comparison_case2_olly5}
\end{figure}

関数が終了します。

