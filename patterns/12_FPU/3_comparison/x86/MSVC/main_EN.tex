\myparagraph{\NonOptimizing MSVC}

MSVC 2010 generates the following:

\lstinputlisting[caption=\NonOptimizing MSVC 2010,style=customasmx86]{patterns/12_FPU/3_comparison/x86/MSVC/MSVC_EN.asm}

\myindex{x86!\Instructions!FLD}

So, \FLD loads \GTT{\_b} into \ST{0}.

\label{Czero_etc}
\newcommand{\Czero}{\GTT{C0}\xspace}
\newcommand{\Ctwo}{\GTT{C2}\xspace}
\newcommand{\Cthree}{\GTT{C3}\xspace}
\newcommand{\CThreeBits}{\Cthree/\Ctwo/\Czero}

\myindex{x86!\Instructions!FCOMP}

\FCOMP compares the value in \ST{0} with what is in \GTT{\_a} 
and sets \CThreeBits bits in FPU status word register, accordingly. 
This is a 16-bit register that reflects the current state of the FPU.

After the bits are set, the \FCOMP instruction also pops one variable from the stack. 
This is what distinguishes it from \FCOM, which is just compares values, leaving the stack in the same state.

Unfortunately, CPUs before Intel P6
\footnote{Intel P6 is Pentium Pro, Pentium II, etc.} don't have any conditional 
jumps instructions which check the \CThreeBits bits. 
Perhaps, it is a matter of history (recall: FPU was a separate chip in past).\\
Modern CPU starting at Intel P6 have \FCOMI/\FCOMIP/\FUCOMI/\FUCOMIP 
instructions~---which do the same, but modify the \ZF/\PF/\CF CPU flags.

\myindex{x86!\Instructions!FNSTSW}

The \FNSTSW instruction copies FPU the status word register to \AX. 
\CThreeBits bits are placed at positions 14/10/8, 
they are at the same positions in the \AX register and all they are placed in the high part of \AX{}~---\AH{}.

\begin{itemize}
\item If $b>a$ in our example, then \CThreeBits bits are to be set as following: 0, 0, 0.
\item If $a>b$, then the bits are: 0, 0, 1.
\item If $a=b$, then the bits are: 1, 0, 0.
\item

If the result is unordered (in case of error), then the set bits are: 1, 1, 1.
\end{itemize}
% TODO: table here?

This is how \CThreeBits bits are located in the \AX register:

\input{C3_in_AX}

This is how \CThreeBits bits are located in the \AH register:

\input{C3_in_AH}

After the execution of \INS{test ah, 5}\footnote{5=101b}, 
only \Czero and \Ctwo bits (on 0 and 2 position) are considered, all other bits are just
ignored.

\label{parity_flag}
\myindex{x86!\Registers!\Flags!Parity flag}

Now let's talk about the \emph{parity flag}, another notable historical rudiment.

This flag is set to 1 if the number of ones in the result of the last calculation is even, and to 0 if it is odd.

Let's look into Wikipedia\footnote{\href{http://go.yurichev.com/17131}{wikipedia.org/wiki/Parity\_flag}}:

\begin{framed}
\begin{quotation}
One common reason to test the parity flag actually has nothing to do with parity. The FPU has four condition flags 
(C0 to C3), but they cannot be tested directly, and must instead be first copied to the flags register. 
When this happens, C0 is placed in the carry flag, C2 in the parity flag and C3 in the zero flag. 
The C2 flag is set when e.g. incomparable floating point values (NaN or unsupported format) are compared 
with the FUCOM instructions.
\end{quotation}
\end{framed}

As noted in Wikipedia, the parity flag used sometimes in FPU code, let's see how.

\myindex{x86!\Instructions!JP}

The \PF flag is to be set to 1 if both \Czero and \Ctwo are set to 0 or both are 1, in which case
the subsequent \JP (\emph{jump if PF==1}) is triggering. 
If we recall the values of \CThreeBits for various cases,
we can see that the conditional jump 
\JP is triggering in two cases: if $b>a$ or $a=b$ 
(\Cthree bit is not considered here, since it has been cleared by the \INS{test ah, 5} instruction).

It is all simple after that. 
If the conditional jump has been triggered, 
\FLD loads the value of \GTT{\_b} 
in \ST{0}, and if it hasn't been triggered, the value of \GTT{\_a} is loaded there.

\mysubparagraph{And what about checking \Ctwo?}

The \Ctwo flag is set in case of error (\gls{NaN}, etc.), but our code doesn't check it.

If the programmer cares about FPU errors, he/she must add additional checks.

\clearpage
\mysubparagraph{\olly}
\myindex{\olly}

Let's try this example in \olly.
The input value of the function (2) is loaded into \EAX: 

\begin{figure}[H]
\centering
\myincludegraphics{patterns/08_switch/2_lot/olly1.png}
\caption{\olly: function's input value is loaded in \EAX}
\label{fig:switch_lot_olly1}
\end{figure}

\clearpage
The input value is checked, is it bigger than 4? 
If not, the \q{default} jump is not taken:
\begin{figure}[H]
\centering
\myincludegraphics{patterns/08_switch/2_lot/olly2.png}
\caption{\olly: 2 is no bigger than 4: no jump is taken}
\label{fig:switch_lot_olly2}
\end{figure}

\clearpage
Here we see a jumptable:

\begin{figure}[H]
\centering
\myincludegraphics{patterns/08_switch/2_lot/olly3.png}
\caption{\olly: calculating destination address using jumptable}
\label{fig:switch_lot_olly3}
\end{figure}

Here we've clicked \q{Follow in Dump} $\rightarrow$ \q{Address constant}, so now we see the \emph{jumptable} in the data window.
These are 5 32-bit values\footnote{They are underlined by \olly because
these are also FIXUPs: \myref{subsec:relocs}, we are going to come back to them later}.
\ECX is now 2, so the third element (can be indexed as 2\footnote{About indexing, see also: \ref{arrays_at_one}}) of the table is to be used.
It's also possible to click \q{Follow in Dump} $\rightarrow$ 
\q{Memory address} and \olly will show the element addressed by the \JMP instruction. 
That's \TT{0x010B103A}.

\clearpage
After the jump we are at \TT{0x010B103A}: the code printing \q{two} will now be executed:

\begin{figure}[H]
\centering
\myincludegraphics{patterns/08_switch/2_lot/olly4.png}
\caption{\olly: now we at the \emph{case:} label}
\label{fig:switch_lot_olly4}
\end{figure}

