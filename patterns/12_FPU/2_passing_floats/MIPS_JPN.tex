\subsubsection{MIPS}

\lstinputlisting[caption=\Optimizing GCC 4.4.5 (IDA),style=customasmMIPS]{patterns/12_FPU/2_passing_floats/MIPS_O3_IDA_JPN.lst}

そしてここでも、\INS{LUI}は \Tdouble の32ビット部分を \$V0 にロードしています。 
そして、これを理解するのは難しいです。

\myindex{MIPS!\Instructions!MFC1}

ここで私たちの新しい指示は\INS{MFC1}です(\q{Coprocessor 1から移動})。 
FPUはコプロセッサ番号1なので、命令名に\q{1}が入ります。 
この命令は、コプロセッサのレジスタからCPU(\ac{GPR})のレジスタに値を転送します。 
したがって、最後に\TT{pow()}の結果はレジスタ \$A3 と \$A2 に移動され、
\printf はこのレジスタのペアから64ビットのdouble値をとります。
