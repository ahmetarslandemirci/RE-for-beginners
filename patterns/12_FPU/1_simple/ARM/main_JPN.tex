\subsubsection{ARM: \OptimizingXcodeIV (\ARMMode)}

Until ARM got standardized floating point support, several processor manufacturers added their own 
instructions extensions.
Then, VFP (\IT{Vector Floating Point}) was standardized.

One important difference from x86 is that in ARM, there
is no stack, you work just with registers.

\lstinputlisting[label=ARM_leaf_example10,caption=\OptimizingXcodeIV (\ARMMode),style=customasmARM]{patterns/12_FPU/1_simple/ARM/Xcode_ARM_O3_JPN.asm}

\myindex{ARM!D-\registers{}}
\myindex{ARM!S-\registers{}}

So, we see here new some registers used, with D prefix.

These are 64-bit registers, there are 32 of them, and they can be used both for floating-point numbers 
(double) but also for SIMD (it is called NEON here in ARM).

There are also 32 32-bit S-registers, intended to be used for single precision 
floating pointer numbers (float).

It is easy to memorize: D-registers are for double precision numbers, while
S-registers---for single precision numbers.
More about it: \myref{ARM_VFP_registers}.

Both constants (3.14 and 4.1) are stored in memory in IEEE 754 format.

\myindex{ARM!\Instructions!VLDR}
\myindex{ARM!\Instructions!VMOV}
\INS{VLDR} and \INS{VMOV}, as it can be easily deduced, are analogous to the \INS{LDR} and \MOV instructions,
but they work with D-registers.

It has to be noted that these instructions, just like the D-registers, are intended not only for
floating point numbers, 
but can be also used for SIMD (NEON) operations and this will also be shown soon.

The arguments are passed to the function in a common way, via the R-registers, however
each number that has double precision has a size of 64 bits, so two R-registers are needed to pass each one.

\INS{VMOV D17, R0, R1} at the start, composes two 32-bit values from \Reg{0} and \Reg{1} into one 64-bit value
and saves it to \GTT{D17}.

\INS{VMOV R0, R1, D16} is the inverse operation: what has been in \GTT{D16} 
is split in two registers, \Reg{0} and \Reg{1}, because a double-precision number 
that needs 64 bits for storage, is returned in \Reg{0} and \Reg{1}.

\myindex{ARM!\Instructions!VDIV}
\myindex{ARM!\Instructions!VMUL}
\myindex{ARM!\Instructions!VADD}
\INS{VDIV}, \INS{VMUL} and \INS{VADD}, 
are instruction for processing floating point numbers that compute \gls{quotient}, 
\gls{product} and sum, respectively.

The code for Thumb-2 is same.

\subsubsection{ARM: \OptimizingKeilVI (\ThumbMode)}

\lstinputlisting[style=customasmARM]{patterns/12_FPU/1_simple/ARM/Keil_O3_thumb_JPN.asm}

Keil generated code for a processor without FPU or NEON support.

The double-precision floating-point numbers are passed via generic R-registers,
and instead of FPU-instructions, service library functions are called\\
(like \GTT{\_\_aeabi\_dmul}, \GTT{\_\_aeabi\_ddiv}, \GTT{\_\_aeabi\_dadd})
which emulate multiplication, division and addition for floating-point numbers.

Of course, that is slower than FPU-coprocessor, but it's still better than nothing.

By the way, similar FPU-emulating libraries were very popular in the x86 world when coprocessors were rare
and expensive, and were installed only on expensive computers.

\myindex{ARM!soft float}
\myindex{ARM!armel}
\myindex{ARM!armhf}
\myindex{ARM!hard float}

The FPU-coprocessor emulation is called \IT{soft float} or \IT{armel} (\IT{emulation}) in the ARM world, 
while using the coprocessor's FPU-instructions is called \IT{hard float} or \IT{armhf}.

\iffalse
% TODO разобраться...
\myindex{Raspberry Pi}

For example, the Linux kernel for Raspberry Pi is compiled in two variants.

In the \IT{soft float} case, arguments are passed via R-registers, and in the \IT{hard float} case---via D-registers.

And that is what stops you from using armhf-libraries from armel-code or vice versa,
and that is why all the code in Linux distributions must be compiled according to a single convention.
\fi

\subsubsection{ARM64: \Optimizing GCC (Linaro) 4.9}

Very compact code:

\lstinputlisting[caption=\Optimizing GCC (Linaro) 4.9,style=customasmARM]{patterns/12_FPU/1_simple/ARM/ARM64_GCC_O3_JPN.s}

\subsubsection{ARM64: \NonOptimizing GCC (Linaro) 4.9}

\lstinputlisting[caption=\NonOptimizing GCC (Linaro) 4.9,style=customasmARM]{patterns/12_FPU/1_simple/ARM/ARM64_GCC_O0_JPN.s}

\NonOptimizing GCC is more verbose.

There is a lot of unnecessary value shuffling, including some clearly redundant code 
(the last two \INS{FMOV} instructions). Probably, GCC 4.9 is not yet good in generating ARM64 code.

What is worth noting is that ARM64 has 64-bit registers, and the D-registers are 64-bit ones as well.

So the compiler is free to save values of type \Tdouble in \ac{GPR}s instead of the local stack.
This isn't possible on 32-bit CPUs.

And again, as an exercise, you can try to optimize this function manually, without introducing
new instructions like \INS{FMADD}.
