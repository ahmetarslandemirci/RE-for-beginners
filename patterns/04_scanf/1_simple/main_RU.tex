\subsection{Простой пример}

\lstinputlisting[style=customc]{patterns/04_scanf/1_simple/ex1.c}

Использовать \scanf в наши времена для того, чтобы спросить у пользователя что-то --- не самая хорошая идея.
Но так мы проиллюстрируем передачу указателя на переменную типа \Tint.

\subsubsection{Об указателях}
\myindex{\CLanguageElements!\Pointers}

Это одна из фундаментальных вещей в программировании.
Часто большой массив, структуру или объект передавать в другую функцию путем копирования данных невыгодно, а передать адрес массива, структуры или объекта куда проще.
Например, если вы собираетесь вывести в консоль текстовую строку, достаточно только передать её адрес в ядро \ac{OS}.

К тому же, если вызываемая функция (\gls{callee}) должна изменить что-то в этом большом массиве или структуре, то возвращать её полностью так же абсурдно.
Так что самое простое, что можно сделать, это передать в функцию-\gls{callee} адрес массива или структуры, и пусть \gls{callee} что-то там изменит.

Указатель в \CCpp --- это просто адрес какого-либо места в памяти.

\myindex{x86-64}
В x86 адрес представляется в виде 32-битного числа (т.е. занимает 4 байта), а в x86-64 как 64-битное число (занимает 8 байт).
Кстати, отсюда негодование некоторых людей, связанное с переходом на x86-64 --- на этой архитектуре все указатели занимают в 2 раза больше места, в том числе и в ``дорогой'' кэш-памяти.

% TODO ... а делать разные версии memcpy для разных типов - абсурд
\myindex{\CStandardLibrary!memcpy()}
При некотором упорстве можно работать только с безтиповыми указателями (\TT{void*}), например, стандартная функция Си \TT{memcpy()},
копирующая блок из одного места памяти в другое принимает на вход 2 указателя типа \TT{void*}, потому что нельзя заранее предугадать, какого типа блок вы собираетесь копировать.
Для копирования тип данных не важен, важен только размер блока.

Также указатели широко используются, когда функции нужно вернуть более одного значения
(мы ещё вернемся к этому в будущем
~(\myref{label_pointers})
).

Функция \emph{scanf()}---это как раз такой случай.

Помимо того, что этой функции нужно показать, сколько значений было прочитано успешно, ей ещё и нужно вернуть сами значения.

Тип указателя в \CCpp нужен только для проверки типов на стадии компиляции.

Внутри, в скомпилированном коде, никакой информации о типах указателей нет вообще.
% TODO это сильно затрудняет декомпиляцию

\EN{\subsubsection{x86}

\myparagraph{MSVC}

Here is what we get after compiling with MSVC 2010:

\lstinputlisting[style=customasmx86]{patterns/04_scanf/1_simple/ex1_MSVC_EN.asm}

\TT{x} is a local variable.

According to the \CCpp standard it must be visible only in this function and not from any other external scope. 
Traditionally, local variables are stored on the stack. 
There are probably other ways to allocate them, but in x86 that is the way it is.

\myindex{x86!\Instructions!PUSH}
The goal of the instruction following the function prologue, \TT{PUSH ECX}, is not to save the \ECX state 
(notice the absence of corresponding \TT{POP ECX} at the function's end).

In fact it allocates 4 bytes on the stack for storing the \TT{x} variable.

\label{stack_frame}
\myindex{\Stack!Stack frame}
\myindex{x86!\Registers!EBP}
\TT{x} is to be accessed with the assistance of the \TT{\_x\$} macro (it equals to -4) and the \EBP register pointing to the current frame.

Over the span of the function's execution, \EBP is pointing to the current \gls{stack frame}
making it possible to access local variables and function arguments via \TT{EBP+offset}.

\myindex{x86!\Registers!ESP}
It is also possible to use \ESP for the same purpose, although that is not very convenient since it changes frequently.
The value of the \EBP could be perceived as a \emph{frozen state} of the value in \ESP at the start of the function's execution.

% FIXME1 это уже было в 02_stack?
Here is a typical \gls{stack frame} layout in 32-bit environment:

\begin{center}
\begin{tabular}{ | l | l | }
\hline
\dots & \dots \\
\hline
EBP-8 & local variable \#2, \MarkedInIDAAs{} \TT{var\_8} \\
\hline
EBP-4 & local variable \#1, \MarkedInIDAAs{} \TT{var\_4} \\
\hline
EBP & saved value of \EBP \\
\hline
EBP+4 & return address \\
\hline
EBP+8 & \argument \#1, \MarkedInIDAAs{} \TT{arg\_0} \\
\hline
EBP+0xC & \argument \#2, \MarkedInIDAAs{} \TT{arg\_4} \\
\hline
EBP+0x10 & \argument \#3, \MarkedInIDAAs{} \TT{arg\_8} \\
\hline
\dots & \dots \\
\hline
\end{tabular}
\end{center}

The \scanf function in our example has two arguments.

The first one is a pointer to the string containing \TT{\%d} and the second is the address of the \TT{x} variable.

\myindex{x86!\Instructions!LEA}
First, the \TT{x} variable's address is loaded into the \EAX register by the \\
\TT{lea eax, DWORD PTR \_x\$[ebp]} instruction.

\LEA stands for \emph{load effective address}, and is often used for forming an address ~(\myref{sec:LEA}).

We could say that in this case \LEA simply stores the sum of the \EBP register value and the \TT{\_x\$} macro in the \EAX register.

This is the same as \INS{lea eax, [ebp-4]}.

So, 4 is being subtracted from the \EBP register value and the result is loaded in the \EAX register.
Next the \EAX register value is pushed into the stack and \scanf is being called.

\printf is being called after that with its first argument --- a pointer to the string:
\TT{You entered \%d...\textbackslash{}n}.

The second argument is prepared with: \TT{mov ecx, [ebp-4]}.
The instruction stores the \TT{x} variable value and not its address, in the \ECX register.

Next the value in the \ECX is stored on the stack and the last \printf is being called.

\EN{\input{patterns/04_scanf/1_simple/olly_EN}}
\RU{\input{patterns/04_scanf/1_simple/olly_RU}}
\IT{\input{patterns/04_scanf/1_simple/olly_IT}}
\DE{\input{patterns/04_scanf/1_simple/olly_DE}}
\FR{\input{patterns/04_scanf/1_simple/olly_FR}}
\JA{\input{patterns/04_scanf/1_simple/olly_JA}}


\myparagraph{GCC}

Let's try to compile this code in GCC 4.4.1 under Linux:

\lstinputlisting[style=customasmx86]{patterns/04_scanf/1_simple/ex1_GCC.asm}

\myindex{puts() instead of printf()}
GCC replaced the \printf call with call to \puts. The reason for this was explained in ~(\myref{puts}).

% TODO: rewrite
%\RU{Почему \scanf переименовали в \TT{\_\_\_isoc99\_scanf}, я честно говоря, пока не знаю.}
%\EN{Why \scanf is renamed to \TT{\_\_\_isoc99\_scanf}, I do not know yet.}
% 
% Apparently it has to do with the ISO c99 standard compliance. By default GCC allows specifying a standard to adhere to.
% For example if you compile with -std=c89 the outputted assmebly file will contain scanf and not __isoc99__scanf. I guess current GCC version adhares to c99 by default.
% According to my understanding the two implementations differ in the set of suported modifyers (See printf man page)

As in the MSVC example---the arguments are placed on the stack using the \MOV instruction.

\myparagraph{By the way}

This simple example is a demonstration of the fact that compiler translates
list of expressions in \CCpp-block into sequential list of instructions.
There are nothing between expressions in \CCpp, and so in resulting machine code, 
there are nothing between, control flow slips from one expression to the next one.

}
\RU{\subsubsection{x86}

\myparagraph{MSVC}

Что получаем на ассемблере, компилируя в MSVC 2010:

\lstinputlisting[style=customasmx86]{patterns/04_scanf/1_simple/ex1_MSVC_RU.asm}

Переменная \TT{x} является локальной.

По стандарту \CCpp она доступна только из этой же функции и нигде более. 
Так получилось, что локальные переменные располагаются в стеке. 
Может быть, можно было бы использовать и другие варианты, но в x86 это традиционно так.

\myindex{x86!\Instructions!PUSH}
Следующая после пролога инструкция \TT{PUSH ECX} не ставит своей целью сохранить 
значение регистра \ECX. 
(Заметьте отсутствие соответствующей инструкции \TT{POP ECX} в конце функции).

Она на самом деле выделяет в стеке 4 байта для хранения \TT{x} в будущем.

\label{stack_frame}
\myindex{\Stack!Стековый фрейм}
\myindex{x86!\Registers!EBP}
Доступ к \TT{x} будет осуществляться при помощи объявленного макроса \TT{\_x\$} (он равен -4) и регистра \EBP указывающего на текущий фрейм.

Во всё время исполнения функции \EBP указывает на текущий \glslink{stack frame}{фрейм} и через \TT{EBP+смещение}
можно получить доступ как к локальным переменным функции, так и аргументам функции.

\myindex{x86!\Registers!ESP}
Можно было бы использовать \ESP, но он во время исполнения функции часто меняется, а это не удобно. 
Так что можно сказать, что \EBP это \emph{замороженное состояние} \ESP на момент начала исполнения функции.

% FIXME1 это уже было в 02_stack?
Разметка типичного стекового \glslink{stack frame}{фрейма} в 32-битной среде:

\begin{center}
\begin{tabular}{ | l | l | }
\hline
\dots & \dots \\
\hline
EBP-8 & локальная переменная \#2, \MarkedInIDAAs{} \TT{var\_8} \\
\hline
EBP-4 & локальная переменная \#1, \MarkedInIDAAs{} \TT{var\_4} \\
\hline
EBP & сохраненное значение \EBP \\
\hline
EBP+4 & адрес возврата \\
\hline
EBP+8 & \argument \#1, \MarkedInIDAAs{} \TT{arg\_0} \\
\hline
EBP+0xC & \argument \#2, \MarkedInIDAAs{} \TT{arg\_4} \\
\hline
EBP+0x10 & \argument \#3, \MarkedInIDAAs{} \TT{arg\_8} \\
\hline
\dots & \dots \\
\hline
\end{tabular}
\end{center}

У функции \scanf в нашем примере два аргумента.

Первый~--- указатель на строку, содержащую \TT{\%d} и второй~--- адрес переменной \TT{x}.

\myindex{x86!\Instructions!LEA}
Вначале адрес \TT{x} помещается в регистр \EAX при помощи инструкции \TT{lea eax, DWORD PTR \_x\$[ebp]}.

Инструкция \LEA означает \emph{load effective address}, и часто используется для формирования адреса чего-либо ~(\myref{sec:LEA}).

Можно сказать, что в данном случае \LEA просто помещает в \EAX результат суммы значения в регистре \EBP и макроса \TT{\_x\$}.

Это тоже что и \INS{lea eax, [ebp-4]}.

Итак, от значения \EBP отнимается 4 и помещается в \EAX.
Далее значение \EAX заталкивается в стек и вызывается \scanf.

После этого вызывается \printf. Первый аргумент вызова строка:
\TT{You entered \%d...\textbackslash{}n}.

Второй аргумент: \INS{mov ecx, [ebp-4]}.
Эта инструкция помещает в \ECX не адрес переменной \TT{x}, а её значение.

Далее значение \ECX заталкивается в стек и вызывается \printf.

\EN{\input{patterns/04_scanf/1_simple/olly_EN}}
\RU{\input{patterns/04_scanf/1_simple/olly_RU}}
\IT{\input{patterns/04_scanf/1_simple/olly_IT}}
\DE{\input{patterns/04_scanf/1_simple/olly_DE}}
\FR{\input{patterns/04_scanf/1_simple/olly_FR}}
\JA{\input{patterns/04_scanf/1_simple/olly_JA}}


\myparagraph{GCC}

Попробуем тоже самое скомпилировать в Linux при помощи GCC 4.4.1:

\lstinputlisting[style=customasmx86]{patterns/04_scanf/1_simple/ex1_GCC.asm}

\myindex{puts() вместо printf()}
GCC заменил первый вызов \printf на \puts. Почему это было сделано, 
уже было описано ранее~(\myref{puts}).

% TODO: rewrite
%\RU{Почему \scanf переименовали в \TT{\_\_\_isoc99\_scanf}, я честно говоря, пока не знаю.}
%\EN{Why \scanf is renamed to \TT{\_\_\_isoc99\_scanf}, I do not know yet.}
% 
% Apparently it has to do with the ISO c99 standard compliance. By default GCC allows specifying a standard to adhere to.
% For example if you compile with -std=c89 the outputted assmebly file will contain scanf and not __isoc99__scanf. I guess current GCC version adhares to c99 by default.
% According to my understanding the two implementations differ in the set of suported modifyers (See printf man page)


Далее всё как и прежде~--- параметры заталкиваются через стек при помощи \MOV.

\myparagraph{Кстати}

Этот простой пример иллюстрирует то обстоятельство, что компилятор преобразует
список выражений в \CCpp-блоке просто в последовательный набор инструкций.
Между выражениями в \CCpp ничего нет, и в итоговом машинном коде между ними тоже ничего нет, 
управление переходит от одной инструкции к следующей за ней.

}
\PTBR{\subsubsection{MSVC: x86}

Aqui está o a saída em assembly (MSVC 2010):

\lstinputlisting[style=customasmx86]{patterns/04_scanf/3_checking_retval/ex3_MSVC_x86.asm}

\myindex{x86!\Registers!EAX}
A função que chamou (\main) precisa do resultado da função chamada (\scanf),
então a função chamada retorna esse valor no registrador \EAX.

\myindex{x86!\Instructions!CMP}
Nós verificamos com a ajuda da instrução \TT{CMP EAX, 1} (\emph{CoMParar}). Em outras palavras, comparamos o valor em \EAX com 1.

\myindex{x86!\Instructions!JNE}
O jump condicional \JNE está logo depois da instrução \CMP. \JNE significa \emph{Jump if Not Equal} ou seja, ela desvia se o valor não for igual ao comparado.

Então, se o valor em \EAX não é 1, a \ac{CPU} vai passar a execução para o endereço contido no operando de \JNE, no nosso caso \TT{\$LN2@main}.
Passando a execução para esse endereço resulta na \ac{CPU} executando \printf com o argumento \TT{What you entered? Huh?}.
Mas se tudo estiver correto, o jump condicional não será efetuado e outra chamada do \printf é executada, com dois argumentos: \TT{`You entered \%d...'} e o valor de \TT{x}.

\myindex{x86!\Instructions!XOR}
\myindex{\CLanguageElements!return}
Como nesse caso o segundo \printf() não tem que ser executado, tem um \JMP precedendo ele (jump incondicional).
Ele passa a execução para o ponto depois do segundo \printf e logo antes de \TT{XOR EAX, EAX}, que implementa \TT{return 0}.

% FIXME internal \ref{} to x86 flags instead of wikipedia
\myindex{x86!\Registers!\Flags}
Então, podemos dizer que comparar um valor com outro é geralmente realizado através do par de instruções \CMP/\Jcc, onde \emph{cc} é código condicional.
\CMP compara dois valores e altera os registros da \ac{CPU} (flags)\footnote{\ac{TBT}: x86 flags, see also: \href{http://go.yurichev.com/17120}{wikipedia}.}.
\Jcc checa esses registro e decide passar a execução para o endereço específico contido no operando ou não.

\myindex{x86!\Instructions!CMP}
\myindex{x86!\Instructions!SUB}
\myindex{x86!\Instructions!JNE}
\myindex{x86!\Registers!ZF}
\label{CMPandSUB}
Isso pode parecer meio paradoxal, mas a instrução \CMP é na verdade \SUB (subtrair).
Todo o conjunto de instruções aritiméticas alteram os registros da \ac{CPU}, não só \CMP.
Se compararmos 1 e 1, $1-1$ é 0 então \ZF (zero flag) será acionado (significando que o último resultado foi zero).
Em nenhuma outra circunstância \ZF pode ser acionado, exceto quando os operandos forem iguais.
\JNE verifica somente o ZF e desvia só não estiver acionado.
\JNE é na verdade um sinônimo para \JNZ (jump se não zero).
\JNE e \JNZ são traduzidos no mesmo código de operação.
Então, a instrução CMP pode ser substituida com a instrução \SUB e quase tudo estará certo, com a diferença de que \SUB altera o valor do primeiro operando.
\CMP é \SUB sem salvar o resultado, mas afetando os registros da \ac{CPU}.

\subsubsection{MSVC: x86: IDA}

\PTBRph{}

% TODO translate: \input{patterns/04_scanf/3_checking_retval/olly_PTBR.tex}

\clearpage
\subsubsection{MSVC: x86 + Hiew}
\myindex{Hiew}

Esse exemplo também pode ser usado como uma maneira simples de exemplificar o patch de arquivos executáveis.
Nós podemos tentar rearranjar o executável de forma que o programa sempre imprima a saída, não importando o que inserirmos.

Assumindo que o executavel está compilado com a opção \TT{/MD}\footnote{isso também é chamada ``linkagem dinâmica''}
(\TT{MSVCR*.DLL}), nós vemos a função main no começo da seção \TT{.text}.
Vamos abrir o executável no Hiew e procurar o começo da seção \TT{.text} (Enter, F8, F6, Enter, Enter).

Nós chegamos a isso:

\begin{figure}[H]
\centering
\myincludegraphics{patterns/04_scanf/3_checking_retval/hiew_1.png}
\caption{\PTBRph{}}
\label{fig:scanf_ex3_hiew_1}
\end{figure}

Hiew encontra strings em \ac{ASCIIZ} e as exibe, como faz com os nomes de funções importadas.

\clearpage
Mova o cursor para o endereço \TT{.00401027} (onde a instrução \TT{JNZ}, que temos de evitar, está localizada), aperte F3 e então digite \q{9090} (que significa dois \ac{NOP}s):

\begin{figure}[H]
\centering
\myincludegraphics{patterns/04_scanf/3_checking_retval/hiew_2.png}
\caption{PTBRph{}}
\label{fig:scanf_ex3_hiew_2}
\end{figure}

Então aperte F9 (atualizar). Agora o executável está salvo no disco. Ele executará da maneira que nós desejávamos.

Duas instruções \ac{NOP} não é a abordagem mais estética.
Outra maneira de rearranjar essa instrução é somente escrever um 0 no operando da instrução jump,
então \INS{JNZ} só avançará para a próxima instrução.

Nós poderíamos também ter feito o oposto: mudado o primeiro byte com \TT{EB} e deixa o segundo byte como está.
Nós teriamos um jump incondicional que é sempre deviado.
Nesse caso, a mensagem de erro seria mostrada todas as vezes, não importando a entrada.

}
\IT{\subsubsection{x86}

\myparagraph{MSVC}

Questo e' cio' che si ottiene dopo la compilazione con MSVC 2010:

\lstinputlisting[style=customasmx86]{patterns/04_scanf/1_simple/ex1_MSVC_EN.asm}

\TT{x} e' una variabile locale.

In base allo standard \CCpp deve essere visibile soltanto in questa funzione e non in altri ambiti (esterni alla funzione).
Tradizionalmente, le variabili locali sono memorizzate sullo stack. 
Ci sono probabilmente altri modi per allocarle, ma in x86 e' cosi'.

\myindex{x86!\Instructions!PUSH}
Lo scopo dell'istruzione che segue il prologo della funzione, \TT{PUSH ECX}, non e' quello di salvare lo stato di \ECX  
(si noti infatti l'assenza della corrispondente istruzione \TT{POP ECX} alla fine della funzione).

Infatti alloca 4 byte sullo stack per memorizzare la variabile \TT{x}.

\label{stack_frame}
\myindex{\Stack!Stack frame}
\myindex{x86!\Registers!EBP}
\TT{x} sara' acceduta con l'aiuto della macro \TT{\_x\$} (che e' uguale a -4) ed il registro \EBP che punta al frame corrente.

Durante l'esecuzione delle funziona, \EBP punta allo \gls{stack frame} corrente 
rendendo possibile accedere alle variabili locali ed agli argomenti della funzione attraverso \TT{EBP+offset}.

\myindex{x86!\Registers!ESP}
E' anche possibile usare \ESP per lo stesso scopo, tuttavia non e' molto conveniente poiche' cambia di frequente.
Il valore di \EBP puo' essere pensato come uno \emph{stato congelato} del valore in \ESP all'inizio dell'esecuzione della funzione.

% FIXME1 это уже было в 02_stack?
Questo e' un tipico layout di uno \gls{stack frame} in un ambiente a 32-bit:

\begin{center}
\begin{tabular}{ | l | l | }
\hline
\dots & \dots \\
\hline
EBP-8 & local variable \#2, \MarkedInIDAAs{} \TT{var\_8} \\
\hline
EBP-4 & local variable \#1, \MarkedInIDAAs{} \TT{var\_4} \\
\hline
EBP & saved value of \EBP \\
\hline
EBP+4 & return address \\
\hline
EBP+8 & \argument \#1, \MarkedInIDAAs{} \TT{arg\_0} \\
\hline
EBP+0xC & \argument \#2, \MarkedInIDAAs{} \TT{arg\_4} \\
\hline
EBP+0x10 & \argument \#3, \MarkedInIDAAs{} \TT{arg\_8} \\
\hline
\dots & \dots \\
\hline
\end{tabular}
\end{center}

La funzione \scanf nel nostro esempio ha due argomenti.
Il primo e' un puntatore alla stringa contenente \TT{\%d} e il secondo e' l'indirizzo della variabile \TT{x}.

\myindex{x86!\Instructions!LEA}
Per prima cosa l'indirizzo della variabile \TT{x} e' caricato nel registro \EAX dall'istruzione \TT{lea eax, DWORD PTR \_x\$[ebp]}.

\LEA sta per \emph{load effective address}, ed e' spesso usata per formare un indirizzo ~(\myref{sec:LEA}).

Potremmo dire che in questo caso \LEA memorizza semplicemente la somma del valore nel registro \EBP e della macro \TT{\_x\$} nel registro \EAX.

E' l'equivalente di \INS{lea eax, [ebp-4]}.

Quindi, 4 viene sottratto dal valore del registro \EBP ed il risultato e' memorizzato nel registro \EAX.
Successivamente il registro \EAX e' messo sullo stack (push) e \scanf viene chiamata.

\printf viene chiamata subito dopo con il suo primo argomento --- un puntatore alla stringa:
\TT{You entered \%d...\textbackslash{}n}.

Il secondo argomento e' preparato con: \TT{mov ecx, [ebp-4]}.
L'istruzione memorizza il valore della variabile \TT{x},  non il suo indirizzo, nel registro \ECX.

Successivamente il valore in \ECX e' memorizzato sullo stack e l'ultima \printf viene chiamata.

\EN{\input{patterns/04_scanf/1_simple/olly_EN}}
\RU{\input{patterns/04_scanf/1_simple/olly_RU}}
\IT{\input{patterns/04_scanf/1_simple/olly_IT}}
\DE{\input{patterns/04_scanf/1_simple/olly_DE}}
\FR{\input{patterns/04_scanf/1_simple/olly_FR}}
\JA{\input{patterns/04_scanf/1_simple/olly_JA}}


\myparagraph{GCC}

Proviamo a compilare questo codice con GCC 4.4.1 su Linux:

\lstinputlisting[style=customasmx86]{patterns/04_scanf/1_simple/ex1_GCC.asm}

\myindex{puts() instead of printf()}
GCC ha sostituito la chiamata a \printf con \puts. La ragione per cui cio' avviene e' stata spiegata in ~(\myref{puts}).

% TODO: rewrite
%\RU{Почему \scanf переименовали в \TT{\_\_\_isoc99\_scanf}, я честно говоря, пока не знаю.}
%\EN{Why \scanf is renamed to \TT{\_\_\_isoc99\_scanf}, I do not know yet.}
% 
% Apparently it has to do with the ISO c99 standard compliance. By default GCC allows specifying a standard to adhere to.
% For example if you compile with -std=c89 the outputted assmebly file will contain scanf and not __isoc99__scanf. I guess current GCC version adhares to c99 by default.
% According to my understanding the two implementations differ in the set of suported modifyers (See printf man page)

Come nell'esempio compilato con MSVC ---gli argomenti sono messi sullo stack utilizzando l'istruzione \MOV.


}
\DE{\subsubsection{x86}

\myparagraph{MSVC}
Den folgenden Code erhalten wie nach dem Kompilieren mit MSVC 2010:

\lstinputlisting[style=customasmx86]{patterns/04_scanf/1_simple/ex1_MSVC_DE.asm}

\TT{x} ist eine lokale Variable.

Gemäß dem \CCpp-Standard darf diese nur innerhalb dieser Funktion sichtbar sein und nicht aus einem anderen, äußeren Scope.
Traditionell werden lokale Variablen auf dem Stack gespeichert.
Es gibt möglicherweise andere Wege sie anzulegen, aber in x86 geschieht es auf diese Weise.


\myindex{x86!\Instructions!PUSH}
Das Ziel des Befehls direkt nach dem Funktionsprolog, \TT{PUSH ECX}), ist es nicht, den Status von \ECX zu sichern
(man beachte, dass Fehlen eines entsprechenden \TT{POP ECX} im Funktionsepilog).
Tatsächlich reserviert der Befehl 4 Byte auf dem Stack, um die Variable $x$ speichern zu können.

\label{stack_frame}
\myindex{\Stack!Stack frame}
\myindex{x86!\Registers!EBP}
Auf \TT{x} wird mithilfe des \TT{\_x\$} Makros (es entspricht -4) und des \EBP Registers, das auf den aktuellen Stack Frame zeigt, zugegriffen. 
Während der Dauer der Funktionsausführung zeigt \EBP auf den aktuellen \glslink{stack frame}{Stack Frame}, wodurch mittels \TT{EBP+offset} auf lokalen Variablen und Funktionsargumente zugegriffen werden kann.

\TT{x} is to be accessed with the assistance of the \TT{\_x\$} macro (it equals to -4) and the \EBP register pointing to the current frame.

\myindex{x86!\Registers!ESP}
Es ist auch möglich, das \ESP Register zu diesem Zweck zu verwenden, aber dies ist ungebräuchlich, da es sich häufig verändert.
Der Wert von \EBP kann als eingefrorener Wert des Wertes von \ESP zu Beginn der Funktionsausführung verstanden werden.

It is also possible to use \ESP for the same purpose, although that is not very convenient since it changes frequently.
The value of the \EBP could be perceived as a \emph{frozen state} of the value in \ESP at the start of the function's execution.

% FIXME1 это уже было в 02_stack?
Hier ist ein typisches Layour eines Stack Frames in einer 32-Bit-Umgebung:

\begin{center}
\begin{tabular}{ | l | l | }
\hline
\dots & \dots \\
\hline
EBP-8 & local variable \#2, \MarkedInIDAAs{} \TT{var\_8} \\
\hline
EBP-4 & local variable \#1, \MarkedInIDAAs{} \TT{var\_4} \\
\hline
EBP & saved value of \EBP \\
\hline
EBP+4 & return address \\
\hline
EBP+8 & \argument \#1, \MarkedInIDAAs{} \TT{arg\_0} \\
\hline
EBP+0xC & \argument \#2, \MarkedInIDAAs{} \TT{arg\_4} \\
\hline
EBP+0x10 & \argument \#3, \MarkedInIDAAs{} \TT{arg\_8} \\
\hline
\dots & \dots \\
\hline
\end{tabular}
\end{center}
Die Funktion \scanf in unserem Beispiel hat zwei Argumente.

Das erste ist ein Pointer auf den String \TT{\%d} und das zweite ist die Adresse der Variablen \TT{x}.

\myindex{x86!\Instructions!LEA}
Zunächst wird die Adresse der Variablen $x$ durch den Befehl \\
\TT{lea eax, DWORD PTR \_x\$[ebp]} in das \EAX Register geladen.

\LEA steht für \emph{load effective address} und wird häufig benutzt, um eine Adresse zu erstellen ~(\myref{sec:LEA}).
In diesem Fall speichert \LEA einfach die Summe des \EBP Registers und des \TT{\_\$} Makros im Register \EAX.
Dies entspricht dem Befehl \INS{lea eax, [ebp-4]}.

Es wird also 4 von Wert in \EBP abgezogen und das Ergebnis in das Register \EAX geladen.
Danach wird der Wert in \EAX auf dem Stack abgelegt und \scanf wird aufgerufen.

Anschließend wird \printf mit einem Argument aufgerufen--einen Pointer auf den String:
\TT{You entered \%d...\textbackslash{}n}.

Das zweite Argument wird mit \TT{mov ecx, [ebp-4]} vorbereitet.
Dieser Befehl speichert den Wert der Variablen $x$ (nicht seine Adresse) im Register \ECX.

Schließlich wird der Wert in \ECX auf dem Stack gespeichert und das letzte \printf wird aufgerufen.

\EN{\input{patterns/04_scanf/1_simple/olly_EN}}
\RU{\input{patterns/04_scanf/1_simple/olly_RU}}
\IT{\input{patterns/04_scanf/1_simple/olly_IT}}
\DE{\input{patterns/04_scanf/1_simple/olly_DE}}
\FR{\input{patterns/04_scanf/1_simple/olly_FR}}
\JA{\input{patterns/04_scanf/1_simple/olly_JA}}


\myparagraph{GCC}

Kompilieren wir diesen Code mit GCC 4.4.1 unter Linux:

\lstinputlisting[style=customasmx86]{patterns/04_scanf/1_simple/ex1_GCC.asm}

\myindex{puts() instead of printf()}
GCC ersetzt den Aufruf von \printf durch einen Aufruf von \puts. Der Grund hierfür wurde bereits in ~(\myref{puts}) erklärt.

% TODO: rewrite
%\RU{Почему \scanf переименовали в \TT{\_\_\_isoc99\_scanf}, я честно говоря, пока не знаю.}
%\EN{Why \scanf is renamed to \TT{\_\_\_isoc99\_scanf}, I do not know yet.}
% 
% Apparently it has to do with the ISO c99 standard compliance. By default GCC allows specifying a standard to adhere to.
% For example if you compile with -std=c89 the outputted assmebly file will contain scanf and not __isoc99__scanf. I guess current GCC version adhares to c99 by default.
% According to my understanding the two implementations differ in the set of suported modifyers (See printf man page)
Genau wie im MSVC Beispiel werden die Argumente mithilfe des Befehls \MOV auf dem Stack abgelegt.

\myparagraph{By the way}
Dieses einfache Beispiel ist übrigens eine Demonstration der Tatsache, dass der Compiler eine Liste von Ausdrücken in einem
\CCpp-Block in eine sequentielle Liste von Befehlen übersetzt.
Es gibt nichts zwischen zwei \CCpp-Anweisungen und genauso verhält es sich auch im Maschinencode.
Der Control Flow geht von einem Ausdruck direkt an den folgenden über.
}
\FR{\subsubsection{x86}

\myparagraph{MSVC}

Voici ce que l'on obtient après avoir compilé avec MSVC 2010:

\lstinputlisting[style=customasmx86]{patterns/04_scanf/1_simple/ex1_MSVC_FR.asm}

\TT{x} est une variable locale.

D'après le standard \CCpp elle ne doit être visible que dans cette fonction et dans
aucune autre portée.
Traditionnellement, les variables locales sont stockées sur la pile.
Il y a probablement d'autres moyens de les allouer, mais en x86, c'est la façon de faire.

\myindex{x86!\Instructions!PUSH}
Le but de l'instruction suivant le prologue de la fonction, \TT{PUSH ECX}, n'est
pas de sauver l'état de \ECX (noter l'absence d'un \TT{POP ECX} à la fin de la
fonction).

En fait, cela alloue 4 octets sur la pile pour stocker la variable \TT{x}.

\label{stack_frame}
\myindex{\Stack!Stack frame}
\myindex{x86!\Registers!EBP}
\TT{x} est accédée à l'aide de la macro \TT{\_x\$} (qui vaut -4) et du registre \EBP
qui pointe sur la structure de pile courante.

Pendant la durée de l'exécution de la fonction, \EBP pointe sur la \glslink{stack frame}{structure locale de pile}
courante, rendant possible l'accès aux variables locales et aux arguments de la
fonction via \TT{EBP+offset}.

\myindex{x86!\Registers!ESP}
Il est aussi possible d'utiliser \ESP dans le même but, bien que ça ne soit pas
très commode, car il change fréquemment.
La valeur de \EBP peut être perçue comme un \emph{état figé} de la valeur de \ESP
au début de l'exécution de la fonction.

% FIXME1 это уже было в 02_stack?
Voici une \glslink{stack frame}{structure de pile} typique dans un environnement 32-bit:

\begin{center}
\begin{tabular}{ | l | l | }
\hline
\dots & \dots \\
\hline
EBP-8 & variable locale \#2, \MarkedInIDAAs{} \TT{var\_8} \\
\hline
EBP-4 & variable locale \#1, \MarkedInIDAAs{} \TT{var\_4} \\
\hline
EBP & valeur sauvée de \EBP \\
\hline
EBP+4 & adresse de retour \\
\hline
EBP+8 & \argument \#1, \MarkedInIDAAs{} \TT{arg\_0} \\
\hline
EBP+0xC & \argument \#2, \MarkedInIDAAs{} \TT{arg\_4} \\
\hline
EBP+0x10 & \argument \#3, \MarkedInIDAAs{} \TT{arg\_8} \\
\hline
\dots & \dots \\
\hline
\end{tabular}
\end{center}

La fonction \scanf de notre exemple a deux arguments.

Le premier est un pointeur sur la chaîne contenant \TT{\%d} et le second est l'adresse
de la variable \TT{x}.

\myindex{x86!\Instructions!LEA}
Tout d'abord, l'adresse de la variable \TT{x} est chargée dans le registre \EAX
par l'instruction \\ \TT{lea eax, DWORD PTR \_x\$[ebp]}.

\LEA signifie \emph{load effective address} (charger l'adresse effective) et est souvent
utilisée pour composer une adresse ~(\myref{sec:LEA}).

Nous pouvons dire que dans ce cas, \LEA stocke simplement la somme de la valeur du
registre \EBP et de la macro \TT{\_x\$} dans le registre \EAX.

C'est la même chose que \INS{lea eax, [ebp-4]}.

Donc, 4 est soustrait de la valeur du registre \EBP et le résultat est chargé dans
le registre \EAX.
Ensuite, la valeur du registre \EAX est poussée sur la pile et \scanf est appelée.

\printf est appelée ensuite avec son premier argument --- un pointeur sur la chaîne:
\TT{You entered \%d...\textbackslash{}n}.

Le second argument est préparé avec: \TT{mov ecx, [ebp-4]}.
L'instruction stocke la valeur de la variable \TT{x} et non son adresse, dans le
registre \ECX.

Puis, la valeur de \ECX est stockée sur la pile et le dernier appel à \printf
est effectué.

\EN{\input{patterns/04_scanf/1_simple/olly_EN}}
\RU{\input{patterns/04_scanf/1_simple/olly_RU}}
\IT{\input{patterns/04_scanf/1_simple/olly_IT}}
\DE{\input{patterns/04_scanf/1_simple/olly_DE}}
\FR{\input{patterns/04_scanf/1_simple/olly_FR}}
\JA{\input{patterns/04_scanf/1_simple/olly_JA}}


\myparagraph{GCC}

Compilons ce code avec GCC 4.4.1 sous Linux:

\lstinputlisting[style=customasmx86]{patterns/04_scanf/1_simple/ex1_GCC.asm}

\myindex{puts() instead of printf()}
GCC a remplacé l'appel à \printf avec un appel à \puts. La raison de cela a été
expliquée dans ~(\myref{puts}).

% TODO: rewrite
%\RU{Почему \scanf переименовали в \TT{\_\_\_isoc99\_scanf}, я честно говоря, пока не знаю.}
%\EN{Why \scanf is renamed to \TT{\_\_\_isoc99\_scanf}, I do not know yet.}
% 
% Apparently it has to do with the ISO c99 standard compliance. By default GCC allows specifying a standard to adhere to.
% For example if you compile with -std=c89 the outputted assmebly file will contain scanf and not __isoc99__scanf. I guess current GCC version adhares to c99 by default.
% According to my understanding the two implementations differ in the set of suported modifyers (See printf man page)

Comme dans l'exemple avec MSVC---les arguments sont placés dans la pile avec l'instruction
\MOV.

\myparagraph{À propos}

Ce simple exemple est la démonstration du fait que le compilateur traduit
une liste d'expression en bloc-\CCpp en une liste séquentielle d'instructions.
% TODO FIXME: better translation / clarify ?
Il n'y a rien entre les expressions en \CCpp, et le résultat en code machine,
il n'y a rien entre le déroulement du flux de contrôle d'une expression à la suivante.
}
\PL{\subsubsection{x86}

\myparagraph{MSVC}

Tutaj znajduje się wynik kompilacji programu w MSVC 2010:

\lstinputlisting[style=customasmx86]{patterns/04_scanf/1_simple/ex1_MSVC_EN.asm}

\TT{x} jest zmienną lokalną.

Według standardu \CCpp zmienna lokalna może być widoczna tylko w konkretnej funkcji. Tradycyjnie zmienne lokalne są przechowywane na stosie. Prawdopodnie są inne moliwości przechowywania tych zmiennych, ale tak akurat jest w x86.

\myindex{x86!\Instructions!PUSH}
Zadaniem instrukcji rozpoczynającej funkcję, \TT{PUSH ECX}, nie jest zapisanie stanu \ECX (można zauważyć brak odpowiadającej instrukcji POP ECX na końcu funkcji).

Tak naprawdę instrukcja ta alokuje 4 bajty na stosie do przechowania zmiennej x.

\label{stack_frame}
\myindex{\Stack!Stack frame}
\myindex{x86!\Registers!EBP}
Dostęp do \TT{x} odbywa się z asystującym makrem \TT{\_x\$} (równym -4) i rejestrem \EBP rwskazującym bieżącą ramkę.

We fragmencie wykonującym funkcje, \EBP wskazuje bieżącą \gls{stack frame}
umożliwiając dostęp do zmiennych lokalnych i argumentów funkcji poprzez \TT{EBP+offset}.

\myindex{x86!\Registers!ESP}
Możliwe jest także użycie \ESP w takim samym celu, le nie jest to zbyt wygodne, ponieważ wartość tego rejestru często się zmienia.
Wartość \EBP może być postrzegana jako \emph{frozen state} wartości w \ESP z początku wykonania funkcji.

% FIXME1 это уже было в 02_stack?
Tutaj znajduje się typowa ramka stosu w układzie środowiska 32-bitowego:

\begin{center}
\begin{tabular}{ | l | l | }
\hline
\dots & \dots \\
\hline
EBP-8 & zmienna lokalna \#2, \MarkedInIDAAs{} \TT{var\_8} \\
\hline
EBP-4 & zmienna lokalna \#1, \MarkedInIDAAs{} \TT{var\_4} \\
\hline
EBP & zapisana wartość \EBP \\
\hline
EBP+4 & adres powrotu \\
\hline
EBP+8 & \argument \#1, \MarkedInIDAAs{} \TT{arg\_0} \\
\hline
EBP+0xC & \argument \#2, \MarkedInIDAAs{} \TT{arg\_4} \\
\hline
EBP+0x10 & \argument \#3, \MarkedInIDAAs{} \TT{arg\_8} \\
\hline
\dots & \dots \\
\hline
\end{tabular}
\end{center}

Funkcja \scanf w naszym przykładzie ma dwa argumenty.

Pierwszy jest wskaźnikiem na string \TT{\%d} a drugi jest adresem zmiennej \TT{x}.

\myindex{x86!\Instructions!LEA}
Na początku adres zmiennej \TT{x} jest ładowany do rejestru \EAX przy pomocy instrukcji \\
\TT{lea eax, DWORD PTR \_x\$[ebp]}.

\LEA oznacza \emph{load effective address} i jest często używana do formowania adresów ~(\myref{sec:LEA}).

Można powiedzieć, że w tym przypadku \LEA po prostu umieszcza sumę rejestru \EBP i makra \TT{\_x\$} w rejestrze \EAX.

To jest to samo co \INS{lea eax, [ebp-4]}.

Więc od rejestru \EBP jest odejmowane 4 i wynik zostaje umieszczony w rejestrze \EAX.
Następnie wartość rejestru \EAX jest odkładana na stosie i funkcja \scanf zostaje wywołana.

\printf wywołuje się z pierwszym argumentem- wskaźnikiem na string:
\TT{You entered \%d...\textbackslash{}n}.

Drugi argument jest przygotowywany za pomocą: \TT{mov ecx, [ebp-4]}.
Instrukcja kopiuje zmienną \TT{x} (nie jej adres) do rejestru \ECX.

Następnie wartość z \ECX jest odkładana na stos, a na koniec zostaje wywołana funkcja  \printf.

\EN{\input{patterns/04_scanf/1_simple/olly_EN}}
\RU{\input{patterns/04_scanf/1_simple/olly_RU}}
\IT{\input{patterns/04_scanf/1_simple/olly_IT}}
\DE{\input{patterns/04_scanf/1_simple/olly_DE}}
\FR{\input{patterns/04_scanf/1_simple/olly_FR}}
\JA{\input{patterns/04_scanf/1_simple/olly_JA}}


\myparagraph{GCC}

Tak wygląda skompilowany kod w GCC 4.4.1 w systemie Linux:

\lstinputlisting[style=customasmx86]{patterns/04_scanf/1_simple/ex1_GCC.asm}

\myindex{puts() instead of printf()}
GCC zamienia wywołanie funkcji \printf na wywołanie funkcji \puts. Powód tego został wyjaśniony w ~(\myref{puts}).

% TODO: rewrite
%\RU{Почему \scanf переименовали в \TT{\_\_\_isoc99\_scanf}, я честно говоря, пока не знаю.}
%\EN{Why \scanf is renamed to \TT{\_\_\_isoc99\_scanf}, I do not know yet.}
% 
% Apparently it has to do with the ISO c99 standard compliance. By default GCC allows specifying a standard to adhere to.
% For example if you compile with -std=c89 the outputted assmebly file will contain scanf and not __isoc99__scanf. I guess current GCC version adhares to c99 by default.
% According to my understanding the two implementations differ in the set of suported modifyers (See printf man page)

Jak w przykładzie MSVC---argumenty funkcji są umieszczane na stosie przy użyciu instrukcji \MOV.

\myparagraph{By the way}

Ten prosty przykład pokazuje jak faktycznie kompilatory tłumaczą
listy wyrażeń w \CCpp-block na sekwencyjne listy instrukcji.
Nie ma nic pomiędzy wyrażeniami w \CCpp a wynikowym kodem maszynowym.
}
\JA{\subsubsection{x86}

\myparagraph{MSVC}

MSVC 2010でコンパイルした後に得られるものは次のとおりです。

\lstinputlisting[style=customasmx86]{patterns/04_scanf/1_simple/ex1_MSVC_JA.asm}

\TT{x}はローカル変数です。

\CCpp 標準によれば、この関数でのみ表示でき、他の外部スコープでは表示できません。
従来、ローカル変数はスタックに格納されていました。
それらを割り当てる方法はおそらく他にもありますが、それはx86の方法です。

\myindex{x86!\Instructions!PUSH}
関数プロローグ、\TT{PUSH ECX}に続く命令の目的は、 \ECX 状態を保存することではありません
(関数の最後に対応する\TT{POP ECX}が存在しないことに注意してください)。

実際、\TT{x}変数を格納するためにスタックに4バイトを割り当てます。

\label{stack_frame}
\myindex{\Stack!Stack frame}
\myindex{x86!\Registers!EBP}
\TT{x}は、\TT{\_x\$} マクロ (-4に等しい)と現在のフレームを指す \EBP レジスタの助けを借りてアクセスされます。

関数の実行の範囲にわたって、 \EBP は現在の\gls{stack frame}を指しており、 \TT{EBP+オフセット}
を介してローカル変数と関数引数にアクセスすることができます。

\myindex{x86!\Registers!ESP}
同じ目的で \ESP を使用することもできますが、 \ESP は頻繁に変更されるためあまり便利ではありません。 
\EBP の値は、関数の実行開始時に \ESP の値が固定された状態として認識される可能性があります。

% FIXME1 это уже было в 02_stack?
32ビット環境での典型的な\gls{stack frame}レイアウトを次に示します。

\begin{center}
\begin{tabular}{ | l | l | }
\hline
\dots & \dots \\
\hline
EBP-8 & local variable \#2, \MarkedInIDAAs{} \TT{var\_8} \\
\hline
EBP-4 & local variable \#1, \MarkedInIDAAs{} \TT{var\_4} \\
\hline
EBP & saved value of \EBP \\
\hline
EBP+4 & return address \\
\hline
EBP+8 & \argument \#1, \MarkedInIDAAs{} \TT{arg\_0} \\
\hline
EBP+0xC & \argument \#2, \MarkedInIDAAs{} \TT{arg\_4} \\
\hline
EBP+0x10 & \argument \#3, \MarkedInIDAAs{} \TT{arg\_8} \\
\hline
\dots & \dots \\
\hline
\end{tabular}
\end{center}

この例の \scanf 関数には2つの引数があります。

最初のものは\TT{\%d}を含む文字列へのポインタで、2番目のものは\TT{x}変数のアドレスです。

\myindex{x86!\Instructions!LEA}
最初に、\TT{x}変数のアドレスが\TT{lea eax, DWORD PTR \_x\$[ebp]}命令によって \EAX レジスタにロードされます。

\LEA は\emph{ロード実効アドレス}の略で、アドレスを形成するためによく使用されます(~(\myref{sec:LEA}))。

この場合、\LEA は単に \EBP レジスタ値と\TT{\_x\$}マクロの合計を \EAX レジスタに格納すると言うことができます。

これは\INS{lea eax, [ebp-4]}と同じです。

したがって、 \EBP レジスタ値から4が減算され、その結果が \EAX レジスタにロードされます。
次に、 \EAX レジスタの値がスタックにプッシュされ、 \scanf が呼び出されます。

\printf は最初の引数で呼び出されています。文字列へのポインタ:
\TT{You entered \%d...\textbackslash{}n}

2番目の引数は\TT{mov ecx, [ebp-4]}で準備されています。
命令は、 \ECX レジスタにそのアドレスではなく\TT{x}変数値を格納します。

次に、 \ECX の値がスタックに格納され、最後の \printf が呼び出されます。

\EN{\input{patterns/04_scanf/1_simple/olly_EN}}
\RU{\input{patterns/04_scanf/1_simple/olly_RU}}
\IT{\input{patterns/04_scanf/1_simple/olly_IT}}
\DE{\input{patterns/04_scanf/1_simple/olly_DE}}
\FR{\input{patterns/04_scanf/1_simple/olly_FR}}
\JA{\input{patterns/04_scanf/1_simple/olly_JA}}


\myparagraph{GCC}

Linux上のGCC 4.4.1でこのコードをコンパイルしようとしましょう。

\lstinputlisting[style=customasmx86]{patterns/04_scanf/1_simple/ex1_GCC.asm}

\myindex{puts() instead of printf()}
GCCは \printf 呼び出しを \puts の呼び出しで置き換えました。 この理由は、~(\myref{puts})で説明されました。

% TODO: rewrite
%\RU{Почему \scanf переименовали в \TT{\_\_\_isoc99\_scanf}, я честно говоря, пока не знаю.}
%\EN{Why \scanf is renamed to \TT{\_\_\_isoc99\_scanf}, I do not know yet.}
% 
% Apparently it has to do with the ISO c99 standard compliance. By default GCC allows specifying a standard to adhere to.
% For example if you compile with -std=c89 the outputted assmebly file will contain scanf and not __isoc99__scanf. I guess current GCC version adhares to c99 by default.
% According to my understanding the two implementations differ in the set of suported modifyers (See printf man page)

MSVCの例のように、引数は \MOV 命令を使用してスタックに配置されます。

\myparagraph{ところで}

ところで、この単純な例は、コンパイラが \CCpp ブロックの式のリストを命令の連続したリストに
変換するという事実のデモンストレーションです。
\CCpp の式の間には何もないので、結果のマシンコードには、
ある式から次の式への制御フローの間には何もありません。
}

\EN{\subsection{x64: \Optimizing MSVC 2013}

\lstinputlisting[caption=\Optimizing MSVC 2013 x64,style=customasmx86]{\CURPATH/MSVC2013_x64_Ox_EN.asm}

First, MSVC inlined the \strlen{} function code, because it concluded this 
is to be faster than the usual \strlen{} work + the cost of calling it and returning from it.
This is called inlining: \myref{inline_code}.

\myindex{x86!\Instructions!OR}
\myindex{\CStandardLibrary!strlen()}
\label{using_OR_instead_of_MOV}
The first instruction of the inlined \strlen{} is\\
\TT{OR RAX, 0xFFFFFFFFFFFFFFFF}. 

MSVC often uses \TT{OR} instead of \TT{MOV RAX, 0xFFFFFFFFFFFFFFFF}, because resulting opcode is shorter.

And of course, it is equivalent: all bits are set, and a number with all bits set is $-1$ 
in two's complement arithmetic: \myref{sec:signednumbers}.

Why would the $-1$ number be used in \strlen{}, one might ask.
Due to optimizations, of course.
Here is the code that MSVC generated:

\lstinputlisting[caption=Inlined \strlen{} by MSVC 2013 x64,style=customasmx86]{\CURPATH/strlen_MSVC_EN.asm}

Try to write shorter if you want to initialize the counter at 0!
OK, let' try:

\lstinputlisting[caption=Our version of \strlen{},style=customasmx86]{\CURPATH/my_strlen_EN.asm}

We failed. We have to use additional \INS{JMP} instruction!

So what the MSVC 2013 compiler did is to move the \TT{INC} instruction to the place before 
the actual character loading.

If the first character is 0, that's OK, \RAX is 0 at this moment, 
so the resulting string length is 0.

The rest in this function seems easy to understand.

\subsection{x64: \NonOptimizing GCC 4.9.1}

\lstinputlisting[style=customasmx86]{\CURPATH/GCC491_x64_O0_EN.asm}

Comments are added by the author of the book.

After the execution of \strlen{}, the control is passed to the L2 label, 
and there two clauses are checked, one after another.

\myindex{\CLanguageElements!Short-circuit}
The second will never be checked, if the first one (\emph{str\_len==0}) is false 
(this is \q{short-circuit}).

Now let's see this function in short form:

\begin{itemize}
\item First for() part (call to \strlen{})
\item goto L2
\item L5: for() body. goto exit, if needed
\item for() third part (decrement of str\_len)
\item L2: 
for() second part: check first clause, then second. goto loop body begin or exit.
\item L4: // exit
\item return s
\end{itemize}

\subsection{x64: \Optimizing GCC 4.9.1}
\label{string_trim_GCC_x64_O3}

\lstinputlisting[style=customasmx86]{\CURPATH/GCC491_x64_O3_EN.asm}

Now this is more complex.

The code before the loop's body start is executed only once, but it has the \CRLF{} 
characters check too!
What is this code duplication for?

The common way to implement the main loop is probably this:

\begin{itemize}
\item (loop start) check for 
\CRLF{} characters, make decisions
\item store zero character
\end{itemize}

But GCC has decided to reverse these two steps. 

Of course, \emph{store zero character} cannot be first step, so another check is needed:

\begin{itemize}
\item workout first character. match it to \CRLF{}, exit if character is not \CRLF{}

\item (loop begin) store zero character

\item check for \CRLF{} characters, make decisions
\end{itemize}

Now the main loop is very short, which is good for latest \ac{CPU}s.

The code doesn't use the str\_len variable, but str\_len-1.
So this is more like an index in a buffer.

Apparently, GCC notices that the str\_len-1 statement is used twice.

So it's better to allocate a variable which always holds a value that's smaller than 
the current string length by one, 
and decrement it (this is the same effect as decrementing the str\_len variable).
}
\RU{\subsubsection{x64: 8 аргументов}

\myindex{x86-64}
\label{example_printf8_x64}
Для того чтобы посмотреть, как остальные аргументы будут передаваться через стек, 
изменим пример ещё раз, 
увеличив количество передаваемых аргументов до 9 
(строка формата \printf и 8 переменных типа \Tint):

\lstinputlisting[style=customc]{patterns/03_printf/2.c}

\myparagraph{MSVC}

Как уже было сказано ранее, первые 4 аргумента в Win64 передаются в регистрах \RCX, \RDX, \Reg{8}, \Reg{9}, а остальные~--- через стек.
Здесь мы это и видим.
Впрочем, инструкция \PUSH не используется, вместо неё при помощи \MOV значения сразу записываются в стек.

\lstinputlisting[caption=MSVC 2012 x64,style=customasmx86]{patterns/03_printf/x86/2_MSVC_x64_RU.asm}

Наблюдательный читатель может спросить, почему для значений типа \Tint отводится 8 байт, ведь нужно только 4?
Да, это нужно запомнить: для значений всех типов более коротких чем 64-бита, отводится 8 байт.
Это сделано для удобства: так всегда легко рассчитать адрес того или иного аргумента.
К тому же, все они расположены по выровненным адресам в памяти.
В 32-битных средах точно также: для всех типов резервируется 4 байта в стеке.

% also for local variables?

\myparagraph{GCC}

В *NIX-системах для x86-64 ситуация похожая, вот только первые 6 аргументов передаются через
\RDI, \RSI, \RDX, \RCX, \Reg{8}, \Reg{9}.
Остальные~--- через стек.
GCC генерирует код, записывающий указатель на строку в \EDI вместо \RDI~--- 
это мы уже рассмотрели чуть раньше: \myref{hw_EDI_instead_of_RDI}.

Почему перед вызовом \printf очищается регистр \EAX мы уже рассмотрели ранее \myref{SysVABI_input_EAX}.

\lstinputlisting[caption=\Optimizing GCC 4.4.6 x64,style=customasmx86]{patterns/03_printf/x86/2_GCC_x64_RU.s}

\myparagraph{GCC + GDB}
\myindex{GDB}

Попробуем этот пример в \ac{GDB}.

\begin{lstlisting}
$ gcc -g 2.c -o 2
\end{lstlisting}

\begin{lstlisting}
$ gdb 2
GNU gdb (GDB) 7.6.1-ubuntu
...
Reading symbols from /home/dennis/polygon/2...done.
\end{lstlisting}

\begin{lstlisting}[caption=ставим точку останова на \printf{,} запускаем]
(gdb) b printf
Breakpoint 1 at 0x400410
(gdb) run
Starting program: /home/dennis/polygon/2 

Breakpoint 1, __printf (format=0x400628 "a=%d; b=%d; c=%d; d=%d; e=%d; f=%d; g=%d; h=%d\n") at printf.c:29
29	printf.c: No such file or directory.
\end{lstlisting}

В регистрах \RSI/\RDX/\RCX/\Reg{8}/\Reg{9} 
всё предсказуемо.
А \RIP содержит адрес самой первой инструкции функции \printf{}.

\begin{lstlisting}
(gdb) info registers
rax            0x0	0
rbx            0x0	0
rcx            0x3	3
rdx            0x2	2
rsi            0x1	1
rdi            0x400628	4195880
rbp            0x7fffffffdf60	0x7fffffffdf60
rsp            0x7fffffffdf38	0x7fffffffdf38
r8             0x4	4
r9             0x5	5
r10            0x7fffffffdce0	140737488346336
r11            0x7ffff7a65f60	140737348263776
r12            0x400440	4195392
r13            0x7fffffffe040	140737488347200
r14            0x0	0
r15            0x0	0
rip            0x7ffff7a65f60	0x7ffff7a65f60 <__printf>
...
\end{lstlisting}

\begin{lstlisting}[caption=смотрим на строку формата]
(gdb) x/s $rdi
0x400628:	"a=%d; b=%d; c=%d; d=%d; e=%d; f=%d; g=%d; h=%d\n"
\end{lstlisting}

Дампим стек на этот раз с командой x/g --- \emph{g} означает \emph{giant words}, т.е. 64-битные слова.

\begin{lstlisting}
(gdb) x/10g $rsp
0x7fffffffdf38:	0x0000000000400576	0x0000000000000006
0x7fffffffdf48:	0x0000000000000007	0x00007fff00000008
0x7fffffffdf58:	0x0000000000000000	0x0000000000000000
0x7fffffffdf68:	0x00007ffff7a33de5	0x0000000000000000
0x7fffffffdf78:	0x00007fffffffe048	0x0000000100000000
\end{lstlisting}

Самый первый элемент стека, как и в прошлый раз, это \ac{RA}.
Через стек также передаются 3 значения: 6, 7, 8.
Видно, что 8 передается с неочищенной старшей 32-битной частью: \GTT{0x00007fff00000008}.
Это нормально, ведь передаются числа типа \Tint, а они 32-битные.
Так что в старшей части регистра или памяти стека остался \q{случайный мусор}.

\ac{GDB} показывает всю функцию \main, если попытаться посмотреть, куда вернется управление после исполнения \printf{}.

\begin{lstlisting}[style=customasmx86]
(gdb) set disassembly-flavor intel
(gdb) disas 0x0000000000400576
Dump of assembler code for function main:
   0x000000000040052d <+0>:	push   rbp
   0x000000000040052e <+1>:	mov    rbp,rsp
   0x0000000000400531 <+4>:	sub    rsp,0x20
   0x0000000000400535 <+8>:	mov    DWORD PTR [rsp+0x10],0x8
   0x000000000040053d <+16>:	mov    DWORD PTR [rsp+0x8],0x7
   0x0000000000400545 <+24>:	mov    DWORD PTR [rsp],0x6
   0x000000000040054c <+31>:	mov    r9d,0x5
   0x0000000000400552 <+37>:	mov    r8d,0x4
   0x0000000000400558 <+43>:	mov    ecx,0x3
   0x000000000040055d <+48>:	mov    edx,0x2
   0x0000000000400562 <+53>:	mov    esi,0x1
   0x0000000000400567 <+58>:	mov    edi,0x400628
   0x000000000040056c <+63>:	mov    eax,0x0
   0x0000000000400571 <+68>:	call   0x400410 <printf@plt>
   0x0000000000400576 <+73>:	mov    eax,0x0
   0x000000000040057b <+78>:	leave  
   0x000000000040057c <+79>:	ret    
End of assembler dump.
\end{lstlisting}

Заканчиваем исполнение \printf, исполняем инструкцию обнуляющую \EAX, удостоверяемся что в регистре \EAX именно ноль.
\RIP указывает сейчас на инструкцию \INS{LEAVE}, т.е. предпоследнюю в функции \main{}.

\begin{lstlisting}
(gdb) finish
Run till exit from #0  __printf (format=0x400628 "a=%d; b=%d; c=%d; d=%d; e=%d; f=%d; g=%d; h=%d\n") at printf.c:29
a=1; b=2; c=3; d=4; e=5; f=6; g=7; h=8
main () at 2.c:6
6		return 0;
Value returned is $1 = 39
(gdb) next
7	};
(gdb) info registers
rax            0x0	0
rbx            0x0	0
rcx            0x26	38
rdx            0x7ffff7dd59f0	140737351866864
rsi            0x7fffffd9	2147483609
rdi            0x0	0
rbp            0x7fffffffdf60	0x7fffffffdf60
rsp            0x7fffffffdf40	0x7fffffffdf40
r8             0x7ffff7dd26a0	140737351853728
r9             0x7ffff7a60134	140737348239668
r10            0x7fffffffd5b0	140737488344496
r11            0x7ffff7a95900	140737348458752
r12            0x400440	4195392
r13            0x7fffffffe040	140737488347200
r14            0x0	0
r15            0x0	0
rip            0x40057b	0x40057b <main+78>
...
\end{lstlisting}
}
\PTBR{\input{patterns/04_scanf/1_simple/x64_PTBR}}
\IT{\subsubsection{x64}

\myindex{x86-64}
La situazione e' simile, con l'unica differenza che, per il passaggio degli argomenti, i registri sono usati al posto dello stack.

\myparagraph{MSVC}

\lstinputlisting[caption=MSVC 2012 x64,style=customasmx86]{patterns/04_scanf/1_simple/ex1_MSVC_x64_EN.asm}

\myparagraph{GCC}

\lstinputlisting[caption=\Optimizing GCC 4.4.6 x64,style=customasmx86]{patterns/04_scanf/1_simple/ex1_GCC_x64_EN.s}

}
\DE{\subsection{x64}

\myindex{x86-64}

Die Geschichte bei x86-64 Funktions Argumenten ist ein wenig anders (zumindest für die ersten vier bis sechs)
sie werden über die Register übergeben z.b. der \gls{callee} liest direkt aus den Registern anstatt vom Stack 
zu lesen.

\subsubsection{MSVC}

\Optimizing MSVC:

\lstinputlisting[caption=\Optimizing MSVC 2012 x64,style=customasmx86]{patterns/05_passing_arguments/x64_MSVC_Ox_EN.asm}

Wie wir sehen können, die compact Funktion \ttf nimmt alle Argumente aus den Registern.

Die \LEA Instruktion wird hier für Addition benutzt,
scheinbar hat der Compiler die Instruktion für schneller befunden als
die \TT{ADD} Instruktion.

\myindex{x86!\Instructions!LEA}

\LEA wird auch benutzt in der \main Funktion um das erste und das dritte \ttf Argument vor zu bereiten.
Der Compiler muss entschieden haben das dies schneller abgearbeitet wird als die Werte in die Register 
zu laden mit der \MOV Instruktion.

Lasst uns einen Blick auf nicht optimierte MSVC Ausgabe werfen:

\lstinputlisting[caption=MSVC 2012 x64,style=customasmx86]{patterns/05_passing_arguments/x64_MSVC_IDA_EN.asm}

Es sieht ein bisschen wie ein Puzzle aus, weil alle drei Argumente aus den Registern auf dem Stack
gespeichert werden aus irgend einem Grund.

\myindex{Shadow space}
\label{shadow_space}
Dies bezeichnet man als \q{shadow space}

\footnote{\href{http://go.yurichev.com/17256}{MSDN}}: 
So wird sich wahrscheinlich jede Win64 EXE verhalten und alle vier Register Werte auf dem Stack speichern.

Das wird aus zwei Gründen so gemacht:

1) Es ist ziemlich übertrieben ein ganzes Register (oder gar vier Register) zu Reservieren für eine
Argument Übergabe, also werden die Argumente über den Stack zugänglich gemacht.
2) Der Debugger weiß immer wo die Funktions Argumente zu finden sind bei einem breakpoint\footnote{\href{http://go.yurichev.com/17257}{MSDN}}.


Also, so können größere Funktionen ihre Eingabe Argumente im \q{shadows space} speichern wenn die Funktion
auf die Argumente während der Laufzeit zugreifen will, kleinere Funktionen (wie unsere) zeigen dieses Verhalten 
nicht. 

Es liegt in der Verantwortung vom \gls{caller} den \q{shadow space} auf dem Stack zu allozieren.

\subsubsection{GCC}

Optimierter GCC generiert mehr oder minder verständlichen Code:

\lstinputlisting[caption=\Optimizing GCC 4.4.6 x64,style=customasmx86]{patterns/05_passing_arguments/x64_GCC_O3_EN.s}

\NonOptimizing GCC:

\lstinputlisting[caption=GCC 4.4.6 x64,style=customasmx86]{patterns/05_passing_arguments/x64_GCC_EN.s}

\myindex{Shadow space}

Bei System V *NIX Systemen (\SysVABI) ist kein \q{shadow space} nötig, aber der \gls{callee} will vielleicht
seine Argumente irgendwo speichern im Fall das keine oder zu wenig Register frei sind.

\subsubsection{GCC: uint64\_t statt int}

Unser Beispiel funktioniert mit 32-Bit \Tint, weshalb auch 32-Bit Register Bereiche benutzt werden (mit dem Präfix \TT{E-}).

Es lassen sich auch ohne Probleme 64-Bit Werte benutzen:

\lstinputlisting{patterns/05_passing_arguments/ex64.c}

\lstinputlisting[caption=\Optimizing GCC 4.4.6 x64,style=customasmx86]{patterns/05_passing_arguments/ex64_GCC_O3_IDA_EN.asm}

Der Code ist der gleiche, aber diesmal werden die \emph{full size} 64-Bit Register benutzt (mit dem \TT{R-} Präfix).

}
\FR{\subsection{x64: MSVC 2013 \Optimizing}

\lstinputlisting[caption=MSVC 2013 x64 \Optimizing,style=customasmx86]{\CURPATH/MSVC2013_x64_Ox_FR.asm}

Tout d'abord, MSVC a inliné le code la fonction \strlen{}, car il en a conclus que
ceci était plus rapide que le \strlen{} habituel + le coût de l'appel et du retour.
Ceci est appelé de l'inlining: \myref{inline_code}.

\myindex{x86!\Instructions!OR}
\myindex{\CStandardLibrary!strlen()}
\label{using_OR_instead_of_MOV}
La première instruction de \strlen{} mis en ligne est\\
\TT{OR RAX, 0xFFFFFFFFFFFFFFFF}. 

MSVC utilise souvent \TT{OR} au lieu de \TT{MOV RAX, 0xFFFFFFFFFFFFFFFF}, car l'opcode
résultant est plus court.

Et bien sûr, c'est équivalent: tous les bits sont mis à 1, et un nombre avec tous
les bits mis vaut $-1$ en complément à 2:\myref{sec:signednumbers}.

On peut se demander pourquoi le nombre $-1$ est utilisé dans \strlen{}.
À des fins d'optimisation, bien sûr.
Voici le code que MSVC a généré:

\lstinputlisting[caption=Inlined \strlen{} by MSVC 2013 x64,style=customasmx86]{\CURPATH/strlen_MSVC_FR.asm}

Essayez d'écrite plus court si vous voulez initialiser le compteur à 0!
OK, essayons:

\lstinputlisting[caption=Our version of \strlen{},style=customasmx86]{\CURPATH/my_strlen_FR.asm}

Nous avons échoué. Nous devons utilisé une instruction \INS{JMP} additionnelle!

Donc, ce que le compilateur de MSVC 2013 a fait, c'est de déplacer l'instruction
\TT{INC} avant le chargement du caractère courant.

Si le premier caractère est 0, c'est OK, \RAX contient 0 à ce moment, donc la longueur
de la chaîne est 0.

Le reste de cette fonction semble facile à comprendre.

\subsection{x64: GCC 4.9.1 \NonOptimizing}

\lstinputlisting[style=customasmx86]{\CURPATH/GCC491_x64_O0_FR.asm}

Les commentaires ont été ajoutés par l'auteur du livre.

Après l'exécution de \strlen{}, le contrôle est passé au label L2, et ici deux clauses
sont vérifiées, l'une après l'autre.

\myindex{\CLanguageElements!Short-circuit}
La seconde ne sera jamais vérifiée, si la première (\emph{str\_len==0}) est fausse
(ceci est un \q{short-circuit} (court-circuit)).

Maintenant regardons la forme courte de cette fonction:

\begin{itemize}
\item Première partie de for() (appel à \strlen{})
\item goto L2
\item L5: corps de for(). sauter à la fin, si besoin
\item troisième partie de for() (décrémenter str\_len)
\item L2: 
deuxième partie de for(): vérifier la première clause, puis la seconde. sauter au
début du corps de la boucle ou sortir.
\item L4: // sortir
\item renvoyer s
\end{itemize}

\subsection{x64: GCC 4.9.1 \Optimizing}
\label{string_trim_GCC_x64_O3}

\lstinputlisting[style=customasmx86]{\CURPATH/GCC491_x64_O3_FR.asm}

Maintenant, c'est plus complexe.

Le code avant le début du corps de la boucle est exécuté une seule fois, mais il contient
le test des caractères \CRLF{} aussi!
À quoi sert cette duplication du code?

La façon courante d'implémenter la boucle principale est sans doute ceci:

\begin{itemize}
\item (début de la boucle) tester la présence des caractères \CRLF{}, décider
\item stocker le caractère zéro
\end{itemize}

Mais GCC a décidé d'inverser ces deux étapes.

Bien sûr,  \emph{stocker le caractère zéro} ne peut pas être la première étape, donc
un autre test est nécessaire:

\begin{itemize}
\item traiter le premier caractère. matcher avec \CRLF{}, sortir si le caractère
n'est pas \CRLF{}

\item (début de la boucle) stocker le caractère zéro

\item tester la présence des caractères \CRLF{}, décider
\end{itemize}

Maintenant la boucle principale est très courte, ce qui est bon pour les derniers
\ac{CPU}s.

Le code n'utilise pas la variable str\_len, mais str\_len-1.
Donc c'est plus comme un index dans un buffer.

Apparemment, GCC a remarqué que l'expression str\_len-1 est utilisée deux fois.

Donc, c'est mieux d'allouer une variable qui contient toujours une valeur qui est
plus petite que la longueur actuelle de la chaîne de un, et la décrémente (ceci a
le même effet que de décrémenter la variable str\_len).
}
\JA{\subsubsection{x64}

\myindex{x86-64}
ここの画像は、スタックではなくレジスタが引数の受け渡しに使用されるという違いと似ています。

\myparagraph{MSVC}

\lstinputlisting[caption=MSVC 2012 x64,style=customasmx86]{patterns/04_scanf/1_simple/ex1_MSVC_x64_EN.asm}

\myparagraph{GCC}

\lstinputlisting[caption=\Optimizing GCC 4.4.6 x64,style=customasmx86]{patterns/04_scanf/1_simple/ex1_GCC_x64_EN.s}

}

\EN{\subsubsection{ARM}

\myparagraph{\OptimizingKeilVI (\ThumbMode)}

\lstinputlisting[style=customasmARM]{patterns/04_scanf/1_simple/ARM_IDA.lst}

\myindex{\CLanguageElements!\Pointers}

In order for \scanf to be able to read item it needs a parameter---pointer to an \Tint.
\Tint is 32-bit, so we need 4 bytes to store it somewhere in memory, and it fits exactly in a 32-bit register.
\myindex{IDA!var\_?}
A place for the local variable \GTT{x} is allocated in the stack and \IDA
has named it \emph{var\_8}. It is not necessary, however, to allocate a such since \ac{SP} (\gls{stack pointer}) is already pointing to that space and it can be used directly.

So, \ac{SP}'s value is copied to the \Reg{1} register and, together with the format-string, passed to \scanf.

\INS{PUSH/POP} instructions behaves differently in ARM than in x86 (it's the other way around).
They are synonyms to \INS{STM/STMDB/LDM/LDMIA} instructions.
And \INS{PUSH} instruction first writes a value into the stack, \emph{and then} subtracts \ac{SP} by 4.
\INS{POP} first adds 4 to \ac{SP}, \emph{and then} reads a value from the stack.
Hence, after \INS{PUSH}, \ac{SP} points to an unused space in stack.
It is used by \scanf, and by \printf after.

\INS{LDMIA} means \emph{Load Multiple Registers Increment address After each transfer}.
\INS{STMDB} means \emph{Store Multiple Registers Decrement address Before each transfer}.

\myindex{ARM!\Instructions!LDR}
Later, with the help of the \INS{LDR} instruction, this value is moved from the stack to the \Reg{1} register in order to be passed to \printf.

\myparagraph{ARM64}

\lstinputlisting[caption=\NonOptimizing GCC 4.9.1 ARM64,numbers=left,style=customasmARM]{patterns/04_scanf/1_simple/ARM64_GCC491_O0_EN.s}

There is 32 bytes are allocated for stack frame, which is bigger than it needed. Perhaps some memory aligning issue?
The most interesting part is finding space for the $x$ variable in the stack frame (line 22).
Why 28? Somehow, compiler decided to place this variable at the end of stack frame instead of beginning.
The address is passed to \scanf, which just stores the user input value in the memory at that address.
This is 32-bit value of type \Tint.
The value is fetched at line 27 and then passed to \printf.

}
\RU{\subsubsection{ARM}

\myparagraph{\OptimizingKeilVI (\ThumbMode)}

\lstinputlisting[style=customasmARM]{patterns/04_scanf/1_simple/ARM_IDA.lst}

\myindex{\CLanguageElements!\Pointers}
Чтобы \scanf мог вернуть значение, ему нужно передать указатель на переменную типа \Tint.
\Tint~--- 32-битное значение, для его хранения нужно только 4 байта, и оно помещается в 32-битный регистр.

\myindex{IDA!var\_?}
Место для локальной переменной \GTT{x} выделяется в стеке, \IDA наименовала её \emph{var\_8}. 
Впрочем, место для неё выделять не обязательно, т.к. \glslink{stack pointer}{указатель стека} \ac{SP} уже указывает на место, 
свободное для использования.
Так что значение указателя \ac{SP} копируется в регистр \Reg{1}, и вместе с format-строкой, 
передается в \scanf.

Инструкции \INS{PUSH/POP} в ARM работают иначе, чем в x86 (тут всё наоборот).
Это синонимы инструкций \INS{STM/STMDB/LDM/LDMIA}.
И инструкция \INS{PUSH} в начале записывает в стек значение, \emph{затем} вычитает 4 из \ac{SP}.
\INS{POP} в начале прибавляет 4 к \ac{SP}, \emph{затем} читает значение из стека.
Так что после \INS{PUSH}, \ac{SP} указывает на неиспользуемое место в стеке.
Его и использует \scanf, а затем и \printf.

\INS{LDMIA} означает \emph{Load Multiple Registers Increment address After each transfer}.
\INS{STMDB} означает \emph{Store Multiple Registers Decrement address Before each transfer}.

\myindex{ARM!\Instructions!LDR}
Позже, при помощи инструкции \INS{LDR}, это значение перемещается из стека в регистр \Reg{1}, чтобы быть переданным в \printf.

\myparagraph{ARM64}

\lstinputlisting[caption=\NonOptimizing GCC 4.9.1 ARM64,numbers=left,style=customasmARM]{patterns/04_scanf/1_simple/ARM64_GCC491_O0_RU.s}

Под стековый фрейм выделяется 32 байта, что больше чем нужно. Может быть, это связано с выравниваем по границе памяти?
Самая интересная часть~--- это поиск места под переменную $x$ в стековом фрейме (строка 22).
Почему 28? Почему-то, компилятор решил расположить эту переменную в конце стекового фрейма, а не в начале.
Адрес потом передается в \scanf, которая просто сохраняет значение, введенное пользователем, в памяти по этому адресу.
Это 32-битное значение типа \Tint.
Значение загружается в строке 27 и затем передается в \printf.

}
\IT{\subsubsection{ARM}

\myparagraph{\OptimizingKeilVI (\ThumbMode)}

\lstinputlisting[style=customasmARM]{patterns/04_scanf/1_simple/ARM_IDA.lst}

\myindex{\CLanguageElements!\Pointers}

Affinche' \scanf possa leggere l'input, necessita di un parametro ---puntatore ad un \Tint.
\Tint e' 32-bit, quindi servono 4 byte per memorizzarlo da qualche parte in memoria, e entra perfettamente in un registro a 32-bit.
\myindex{IDA!var\_?}
Uno spazio per la variabile locale \GTT{x} e' allocato nello stack e \IDA
lo ha chiamato \emph{var\_8}. Non e' comunque necessario allocarlo in questo modo poiche' \ac{SP} (\gls{stack pointer}) punta gia' a quella posizione e puo' essere usato direttamente.

Successivamente il valore di \ac{SP} e' copiato nel registro \Reg{1} e sono passati, insieme alla format-string, a \scanf.

% TBT here
%\INS{PUSH/POP} instructions behaves differently in ARM than in x86 (it's the other way around).
%They are synonyms to \INS{STM/STMDB/LDM/LDMIA} instructions.
%And \INS{PUSH} instruction first writes a value into the stack, \emph{and then} subtracts \ac{SP} by 4.
%\INS{POP} first adds 4 to \ac{SP}, \emph{and then} reads a value from the stack.
%Hence, after \INS{PUSH}, \ac{SP} points to an unused space in stack.
%It is used by \scanf, and by \printf after.

%\INS{LDMIA} means \emph{Load Multiple Registers Increment address After each transfer}.
%\INS{STMDB} means \emph{Store Multiple Registers Decrement address Before each transfer}.

\myindex{ARM!\Instructions!LDR}
Questo valore, con l'aiuto dell'istruzione \INS{LDR} , viene poi spostato dallo stakc al registro \Reg{1} per essere passato a \printf.

\myparagraph{ARM64}

\lstinputlisting[caption=\NonOptimizing GCC 4.9.1 ARM64,numbers=left,style=customasmARM]{patterns/04_scanf/1_simple/ARM64_GCC491_O0_EN.s}

Ci sono 32 byte allocati per lo stack frame, che e' piu' grande del necessario. Forse a causa di meccanismi di allineamento della memoria?
La parte piu' interessante e' quella in cui trova spazio per la variabile $x$ nello stack frame (riga 22).
Perche' 28? Il compilatore ha in qualche modo deciso di piazzare questa variabile alla fine dello stack frame anziche' all'inizio.
L'indirizzo e' passato a \scanf, che memorizzera' il valore immesso dall'utente nella memoria a quell'indirizzo.
Si tratta di un valore a 32-bit di tipo \Tint.
Il valore e' recuperato successivamente a riga 27 e passato a \printf.

}
\DE{\subsubsection{ARM}

\myparagraph{\OptimizingKeilVI (\ThumbMode)}

\lstinputlisting[style=customasmARM]{patterns/04_scanf/1_simple/ARM_IDA.lst}

\myindex{\CLanguageElements!\Pointers}
Damit \scanf Elemente einlesen kann, benötigt die Funktion einen Paramter--einen Pointer vom Typ \Tint.
\Tint hat die Größe 32 Bit, wir benötigen also 4 Byte, um den Wert im Speicher abzulegen, und passt daher genau in ein 32-Bit-Register.
\myindex{IDA!var\_?}
Auf dem Stack wird Platz für die lokalen Variable \GTT{x} reserviert und IDA bezeichnet diese Variable mit \emph{var\_8}. 
Eigentlich ist aber an dieser Stelle gar nicht notwendig, Platz auf dem Stack zu reservieren, da \ac{SP} (\gls{stack pointer} 
bereits auf die Adresse zeigt und auch direkt verwendet werden kann.

Der Wert von \ac{SP} wird also in das \Reg{1} Register kopiert und zusammen mit dem Formatierungsstring an \scanf übergeben.

% TBT here
%\INS{PUSH/POP} instructions behaves differently in ARM than in x86 (it's the other way around).
%They are synonyms to \INS{STM/STMDB/LDM/LDMIA} instructions.
%And \INS{PUSH} instruction first writes a value into the stack, \emph{and then} subtracts \ac{SP} by 4.
%\INS{POP} first adds 4 to \ac{SP}, \emph{and then} reads a value from the stack.
%Hence, after \INS{PUSH}, \ac{SP} points to an unused space in stack.
%It is used by \scanf, and by \printf after.

%\INS{LDMIA} means \emph{Load Multiple Registers Increment address After each transfer}.
%\INS{STMDB} means \emph{Store Multiple Registers Decrement address Before each transfer}.

\myindex{ARM!\Instructions!LDR}
Später wird mithilfe des \INS{LDR} Befehls dieser Wert vom Stack in das \Reg{1} Register verschoben um an \printf übergeben werden zu können.

\myparagraph{ARM64}

\lstinputlisting[caption=\NonOptimizing GCC 4.9.1 ARM64,numbers=left,style=customasmARM]{patterns/04_scanf/1_simple/ARM64_GCC491_O0_DE.s}

Im Stack Frame werden 32 Byte reserviert, was deutlich mehr als benötigt ist. Vielleicht handelt es sich um eine Frage des Aligning (dt. Angleichens) von Speicheradressen.
Der interessanteste Teil ist, im Stack Frame einen Platz für die Variable $x$ zu finden (Zeile 22).
Warum 28? Irgendwie hat der Compiler entschieden die Variable am Ende des Stack Frames anstatt an dessen Beginn abzulegen.
Die Adresse wird an \scanf übergeben; diese Funktion speichert den Userinput an der genannten Adresse im Speicher.
Es handelt sich hier um einen 32-Bit-Wert vom Typ \Tint. 
Der Wert wird in Zeile 27 abgeholt und dann an \printf übergeben.


}
\FR{\subsubsection{ARM}

\myparagraph{\OptimizingKeilVI (\ThumbMode)}

\lstinputlisting[style=customasmARM]{patterns/04_scanf/1_simple/ARM_IDA.lst}

\myindex{\CLanguageElements!\Pointers}

Afin que \scanf puisse lire l'item, elle a besoin d'un paramètre---un pointeur sur un \Tint.
Le type \Tint est 32-bit, donc nous avons besoin de 4 octets pour le stocker quelque
part en mémoire, et il tient exactement dans un registre 32-bit.
\myindex{IDA!var\_?}
De l'espace pour la variable locale \GTT{x} est allouée sur la pile et \IDA l'a
nommée \emph{var\_8}. Il n'est toutefois pas nécessaire de définir cette macro, puisque
le \ac{SP} (\glslink{stack pointer}{pointeur de pile}) pointe déjà sur cet espace et
peut être utilisé directement.

Donc, la valeur de \ac{SP} est copiée dans la registre \Reg{1} et, avec la chaîne
de format, passée à \scanf.

Les instructions \INS{PUSH/POP} se comportent différemment en ARM et en x86 (c'est l'inverse)
Il y a des sysnonymes aux instructions \INS{STM/STMDB/LDM/LDMIA}.
Et l'instruction \INS{PUSH} écrit d'abord une valeur sur la pile, \emph{et ensuite}
soustrait 4 de \ac{SP}.
De ce fait, après \INS{PUSH}, \ac{SP} pointe sur de l'espace inutilisé sur la pile.
Il est utilisé par \scanf, et après par \printf.

\INS{LDMIA} signifie \emph{Load Multiple Registers Increment address After each transfer}
(charge plusieurs registres incrémente l'adresse après chaque transfert).
\INS{STMDB} signifie \emph{Store Multiple Registers Decrement address Before each transfer}
(socke plusieurs registres décrémente l'adresse avant chaque transfert).

\myindex{ARM!\Instructions!LDR}
Plus tard, avec l'aide de l'instruction \INS{LDR}, cette valeur est copiée depuis
la pile vers le registre \Reg{1} afin de la passer à \printf.

\myparagraph{ARM64}

\lstinputlisting[caption=GCC 4.9.1 ARM64 \NonOptimizing,numbers=left,style=customasmARM]{patterns/04_scanf/1_simple/ARM64_GCC491_O0_FR.s}

Il y a 32 octets alloués pour la structure de pile, ce qui est plus que nécessaire.
Peut-être dans un soucis d'alignement de mémoire?
La partie la plus intéressante est de trouver de l'espace pour la variable $x$ dans
la structure de pile (ligne 22).
Pourquoi 28? Pour une certaine raison, le compilateur a décidé de stocker cette
variable à la fin de la structure de pile locale au lieu du début.
L'adresse est passée à \scanf, qui stocke l'entrée de l'utilisateur en mémoire à
cette adresse.
Il s'agit d'une valeur sur 32-bit de type \Tint.
La valeur est prise à la ligne 27 puis passée à \printf.

}
\JA{\subsubsection{ARM}

\myparagraph{\OptimizingKeilVI (\ThumbMode)}

\lstinputlisting[style=customasmARM]{patterns/04_scanf/1_simple/ARM_IDA.lst}

\myindex{\CLanguageElements!\Pointers}

\scanf がitemを読み込むためには、 \Tint へのparameter.pointerが必要です。 
\Tint は32ビットなので、メモリのどこかに格納するには4バイトが必要で、32ビットのレジスタに正確に収まります。 
\myindex{IDA!var\_?}
ローカル変数\GTT{x}の場所がスタックに割り当てられ、 
\IDA の名前は\emph{var\_8}です。 ただし、\ac{SP}(\gls{stack pointer})がすでにその領域を指しているため、その領域を直接割り当てることはできません。 

\INS{PUSH/POP}命令は、ARMとx86とでは動作が異なります(これは逆です)。 
これらは\INS{STM/STMDB/LDM/LDMIA}命令の同義語です。 
そして、\INS{PUSH}命令は最初に値をスタックに書き込み、\emph{次に} \ac{SP} を4で減算します。
\INS{POP}は最初に\ac{SP}に4を加算してから、スタックから値を読み取ります。 
したがって、\INS{PUSH}後、\ac{SP}はスタック内の未使用スペースを指します。 
それは \scanf によって、そして後に \printf によって使用されます。

\INS{LDMIA} は \emph{Load Multiple Registers Increment address After each transfer}の略です。
\INS{STMDB} は \emph{Store Multiple Registers Decrement address Before each transfer}の略です。

したがって、\ac{SP}の値は\Reg{1}レジスタにコピーされ、フォーマット文字列とともに \scanf に渡されます。 
その後、\INS{LDR}命令の助けを借りて、この値はスタックから\Reg{1}レジスタに移動され、 \printf に渡されます。

\myparagraph{ARM64}

\lstinputlisting[caption=\NonOptimizing GCC 4.9.1 ARM64,numbers=left,style=customasmARM]{patterns/04_scanf/1_simple/ARM64_GCC491_O0_JA.s}

スタックフレームには32バイトが割り当てられており、必要なサイズよりも大きくなっています。 たぶんメモリのアラインメントの問題でしょうか? 
最も興味深いのはスタックフレーム内の$x$変数のためのスペースを見つけることです(22行目)。 
なぜ28なのでしょう? 何らかの理由で、コンパイラは、この変数をスタックフレームの最後に置きます。 
アドレスは \scanf に渡され、\scanf はユーザ入力値をそのアドレスのメモリに格納するだけです。 
これは \Tint 型の32ビット値です。 
値は27行目から取得され、 \printf に渡されます。
}

\EN{\subsubsection{ARM}

\myparagraph{\NonOptimizingKeilVI (\ARMMode)}

\lstinputlisting[label=Keil_number_sign,style=customasmARM]{patterns/09_loops/simple/ARM/Keil_ARM_O0.asm}

Iteration counter $i$ is to be stored in the \Reg{4} register.
The \INS{MOV R4, \#2} instruction just initializes $i$.
The \INS{MOV R0, R4} and \INS{BL printing\_function} instructions
compose the body of the loop, the first instruction preparing the argument for 
\ttf function and the second calling the function.
\myindex{ARM!\Instructions!ADD}
The \INS{ADD R4, R4, \#1} instruction just adds 1 to the $i$ variable at each iteration.
\myindex{ARM!\Instructions!CMP}
\myindex{ARM!\Instructions!BLT}
\INS{CMP R4, \#0xA} compares $i$ with \TT{0xA} (10). 
The next instruction \INS{BLT} (\emph{Branch Less Than}) jumps if $i$ is less than 10.
Otherwise, 0 is to be written into \Reg{0} (since our function returns 0)
and function execution finishes.

\myparagraph{\OptimizingKeilVI (\ThumbMode)}

\lstinputlisting[style=customasmARM]{patterns/09_loops/simple/ARM/Keil_thumb_O3.asm}

Practically the same.

\myparagraph{\OptimizingXcodeIV (\ThumbTwoMode)}
\label{ARM_unrolled_loops}

\lstinputlisting[style=customasmARM]{patterns/09_loops/simple/ARM/xcode_thumb_O3.asm}

In fact, this was in my \ttf function:

\begin{lstlisting}[style=customc]
void printing_function(int i)
{
    printf ("%d\n", i);
};
\end{lstlisting}

\myindex{Unrolled loop}
\myindex{Inline code}
So, LLVM not just \emph{unrolled} the loop, 
but also \emph{inlined} my 
very simple function \ttf,
and inserted its body 8 times instead of calling it. 

This is possible when the function is so simple (like mine) and when it is not called too much (like here).

\myparagraph{ARM64: \Optimizing GCC 4.9.1}

\lstinputlisting[caption=\Optimizing GCC 4.9.1,style=customasmARM]{patterns/09_loops/simple/ARM/ARM64_GCC491_O3_EN.s}

\myparagraph{ARM64: \NonOptimizing GCC 4.9.1}

\lstinputlisting[caption=\NonOptimizing GCC 4.9.1 -fno-inline,style=customasmARM]{patterns/09_loops/simple/ARM/ARM64_GCC491_O0_EN.s}
}
\RU{\mysection{Функция toupper()}
\myindex{\CStandardLibrary!toupper()}

Еще одна очень востребованная функция конвертирует символ из строчного в заглавный, если нужно:

\lstinputlisting[style=customc]{\CURPATH/toupper.c}

Выражение \TT{'a'+'A'} оставлено в исходном коде для удобства чтения, 
конечно, оно соптимизируется

\footnote{Впрочем, если быть дотошным, вполне могут до сих пор существовать компиляторы,
которые не оптимизируют подобное и оставляют в коде.}.

\ac{ASCII}-код символа \q{a} это 97 (или 0x61), и 65 (или 0x41) для символа \q{A}.

Разница (или расстояние) между ними в \ac{ASCII}-таблица это 32 (или 0x20).

Для лучшего понимания, читатель может посмотреть на стандартную 7-битную таблицу \ac{ASCII}:

\begin{figure}[H]
\centering
\includegraphics[width=0.7\textwidth]{ascii.png}
\caption{7-битная таблица \ac{ASCII} в Emacs}
\end{figure}

\subsection{x64}

\subsubsection{Две операции сравнения}

\NonOptimizing MSVC прямолинеен: код проверят, находится ли входной символ в интервале [97..122]
(или в интервале [`a'..`z'] ) и вычитает 32 в таком случае.

Имеется также небольшой артефакт компилятора:

\lstinputlisting[caption=\NonOptimizing MSVC 2013 (x64),numbers=left,style=customasmx86]{\CURPATH/MSVC_2013_x64_RU.asm}

Важно отметить что (на строке 3) входной байт загружается в 64-битный слот локального стека.

Все остальные биты ([8..63]) не трогаются, т.е. содержат случайный шум (вы можете увидеть его в отладчике).
% TODO add debugger example

Все инструкции работают только с байтами, так что всё нормально.

Последняя инструкция \TT{MOVZX} на строке 15 берет байт из локального стека и расширяет его 
до 32-битного \Tint, дополняя нулями.

\NonOptimizing GCC делает почти то же самое:

\lstinputlisting[caption=\NonOptimizing GCC 4.9 (x64),style=customasmx86]{\CURPATH/GCC_49_x64_O0.s}

\subsubsection{Одна операция сравнения}
\label{toupper_one_comparison}

\Optimizing MSVC работает лучше, он генерирует только одну операцию сравнения:

\lstinputlisting[caption=\Optimizing MSVC 2013 (x64),style=customasmx86]{\CURPATH/MSVC_2013_Ox_x64.asm}

Уже было описано, как можно заменить две операции сравнения на одну: \myref{one_comparison_instead_of_two}.

Мы бы переписал это на \CCpp так:

\begin{lstlisting}[style=customc]
int tmp=c-97;

if (tmp>25)
        return c;
else
        return c-32;
\end{lstlisting}

Переменная \emph{tmp} должна быть знаковая.

При помощи этого, имеем две операции вычитания в случае конверсии плюс одну операцию сравнения.

В то время как оригинальный алгоритм использует две операции сравнения плюс одну операцию вычитания.

\Optimizing GCC 
даже лучше, он избавился от переходов (а это хорошо: \myref{branch_predictors}) используя инструкцию CMOVcc:

\lstinputlisting[caption=\Optimizing GCC 4.9 (x64),numbers=left,style=customasmx86,label=toupper_GCC_O3]{\CURPATH/GCC_49_x64_O3.s}

На строке 3 код готовит уже сконвертированное значение заранее, как если бы конверсия всегда происходила.

На строке 5 это значение в EAX заменяется нетронутым входным значением, если конверсия не нужна.
И тогда это значение (конечно, неверное), просто выбрасывается.

Вычитание с упреждением это цена, которую компилятор платит за отсутствие условных переходов.

\subsection{ARM}

\Optimizing Keil для режима ARM также генерирует только одну операцию сравнения:

\lstinputlisting[caption=\OptimizingKeilVI (\ARMMode),style=customasmARM]{\CURPATH/Keil_ARM_O3.s}

\myindex{ARM!\Instructions!SUBcc}
\myindex{ARM!\Instructions!ANDcc}

\INS{SUBLS} и \INS{ANDLS} исполняются только если значение \Reg{1} меньше чем 0x19 (или равно).
Они и делают конверсию.

\Optimizing Keil для режима Thumb также генерирует только одну операцию сравнения:

\lstinputlisting[caption=\OptimizingKeilVI (\ThumbMode),style=customasmARM]{\CURPATH/Keil_thumb_O3.s}

\myindex{ARM!\Instructions!LSLS}
\myindex{ARM!\Instructions!LSLR}

Последние две инструкции \INS{LSLS} и \INS{LSRS} работают как \INS{AND reg, 0xFF}:
это аналог \CCpp-выражения $(i<<24)>>24$.

Очевидно, Keil для режима Thumb решил, что две 2-байтных инструкции это короче чем код, загружающий
константу 0xFF плюс инструкция AND.

\subsubsection{GCC для ARM64}

\lstinputlisting[caption=\NonOptimizing GCC 4.9 (ARM64),style=customasmARM]{\CURPATH/GCC_49_ARM64_O0.s}

\lstinputlisting[caption=\Optimizing GCC 4.9 (ARM64),style=customasmARM]{\CURPATH/GCC_49_ARM64_O3.s}

\subsection{Используя битовые операции}
\label{toupper_bit}

Учитывая тот факт, что 5-й бит (считая с 0-его) всегда присутствует после проверки, вычитание его это просто
сброс этого единственного бита, но точно такого же эффекта можно достичть при помощи обычного применения операции
``И'' (\myref{AND_OR_as_SUB_ADD}).

И даже проще, с исключающим ИЛИ:

\lstinputlisting[style=customc]{\CURPATH/toupper2.c}

Код близок к тому, что сгенерировал оптимизирующий GCC для предыдущего примера (\myref{toupper_GCC_O3}):

\lstinputlisting[caption=\Optimizing GCC 5.4 (x86),style=customasmx86]{\CURPATH/toupper2_GCC540_x86_O3.s}

\dots но используется \INS{XOR} вместо \INS{SUB}.

Переворачивание 5-го бита это просто перемещение \textit{курсора} в таблице \ac{ASCII} вверх/вниз на 2 ряда.

Некоторые люди говорят, что буквы нижнего/верхнего регистра были расставлены в \ac{ASCII}-таблице таким манером намеренно,
потому что:

\begin{framed}
\begin{quotation}
Very old keyboards used to do Shift just by toggling the 32 or 16 bit, depending on the key; this is why the relationship between small and capital letters in ASCII is so regular, and the relationship between numbers and symbols, and some pairs of symbols, is sort of regular if you squint at it.
\end{quotation}
\end{framed}

( Eric S. Raymond, \url{http://www.catb.org/esr/faqs/things-every-hacker-once-knew/} )

Следовательно, мы можем написать такой фрагмент кода, который просто меняет регистр букв:

\lstinputlisting[style=customc]{\CURPATH/flip_EN.c}

\subsection{Итог}

Все эти оптимизации компиляторов очень популярны в наше время и практикующий
reverse engineer обычно часто видит такие варианты кода.
}
\IT{\subsubsection{MIPS}

Nello stack locale viene allocato spazio per la variabile $x$ , a cui viene fatto riferimento come $\$sp+24$.
\myindex{MIPS!\Instructions!LW}

Il suo indirizzo è passato a \scanf, il valore immesso dall'utente è caricato usand l'istruzione \INS{LW} (\q{Load Word}) ed è infine passato a \printf.

\lstinputlisting[caption=\Optimizing GCC 4.4.5 (\assemblyOutput),style=customasmMIPS]{patterns/04_scanf/1_simple/MIPS/ex1.O3_EN.s}

IDA mostra il layout dello stack nel modo seguente:

\lstinputlisting[caption=\Optimizing GCC 4.4.5 (IDA),style=customasmMIPS]{patterns/04_scanf/1_simple/MIPS/ex1.O3.IDA_EN.lst}

% TODO non-optimized version?
}
\DE{\subsubsection{Struct als Menge von Werten}
Um zu veranschaulichen, dass ein struct nur eine Menge von nebeneinanderliegenden Variablen ist, überarbeiten wir unser
Beispiel, indem wir auf die Definition des \emph{tm} structs schauen:\lstref{struct_tm}.

\lstinputlisting[style=customc]{patterns/15_structs/3_tm_linux/as_array/GCC_tm2.c}

\myindex{\CStandardLibrary!localtime\_r()}
Der Pointer auf das Feld \TT{tm\_sec} wird nach \TT{localtime\_r} übergeben, d.h. an das erste Element des structs.

Der Compiler warnt uns:

\begin{lstlisting}[caption=GCC 4.7.3]
GCC_tm2.c: In function 'main':
GCC_tm2.c:11:5: warning: passing argument 2 of 'localtime_r' from incompatible pointer type [enabled by default]
In file included from GCC_tm2.c:2:0:
/usr/include/time.h:59:12: note: expected 'struct tm *' but argument is of type 'int *'
\end{lstlisting}

Trotzdem erzeugt er folgenden Code:

\lstinputlisting[caption=GCC 4.7.3,style=customasmx86]{patterns/15_structs/3_tm_linux/as_array/GCC_tm2.asm}
Dieser Code ist zum vorherigen identisch und es ist unmöglich zu sagen, ob es sich im originalen Quellcode um ein struct
oder nur um eine Menge von Variablen handelt.

Es funktioniert also, ist aber in der Praxis nicht empfehlenswert. 

Nicht optimierende Compiler legen normalerweise Variablen auf dem lokalen Stack in der Reihenfolge an, in der sie in der
Funktion deklariert wurden.

Ein Garantie dafür gibt es freilich nicht.

Andere Compiler könnten an dieser Stelle übrigens davor warnen, dass die Variablen \TT{tm\_year}, \TT{tm\_mon}, \TT{tm\_mday},
\TT{tm\_hour}, \TT{tm\_min} - nicht aber \TT{tm\_sec} - ohne Initialisierung verwendet werden.

Der Compiler weiß nicht, dass diese durch die Funktion \TT{localtime\_r()} befüllt werden.

Wir haben dieses Beispiel ausgewählt, da alle Felder im struct vom Typ \Tint sind.

Es würde nicht funktionieren, wenn die Felder 16 Bit (\TT{WORD}) groß wären, wie im Beispiel des \TT{SYSTEMTIME}
structs---\TT{GetSystemTime()} würde sie falsch befüllen (da die lokalen Variablen auf 32-Bit-Grenzen angeordnet sind).
Mehr dazu im folgenden Abschnitt: \q{\StructurePackingSectionName} (\myref{structure_packing}).

Ein struct ist also nichts als eine Menge von an einer Stelle gespeicherten Variablen.
Man kan sagen, dass das struct ein Befehl an den Compiler ist, diese Variablen an einer Stelle zu halten.
In ganz frühen Versionen von C (vor 1972) gab es übrigens gar keine structs \RitchieDevC.

Dieses Beispiel wird nicht im Debugger gezeigt, da es dem gerade gezeigten entspricht.

\subsubsection{Struct als Array aus 32-Bit-Worten}

\lstinputlisting[style=customc]{patterns/15_structs/3_tm_linux/as_array/GCC_tm3.c}
Wir können einen Pointer auf ein struct in ein Array aus \Tint{}s casten und es funktioniert.
Wir lassen dieses Beispiel zur Systemzeit 23:51:45 26-July-2014 laufen.

\begin{lstlisting}[label=GCC_tm3_output]
0x0000002D (45)
0x00000033 (51)
0x00000017 (23)
0x0000001A (26)
0x00000006 (6)
0x00000072 (114)
0x00000006 (6)
0x000000CE (206)
0x00000001 (1)
\end{lstlisting}
Die Variablen sind hier in der gleichen Reihenfolge, in der die in der Definition des structs aufgezählt
werden:\myref{struct_tm}.

Hier ist der erzeugte Code:

\lstinputlisting[caption=\Optimizing GCC
4.8.1,style=customasmx86]{patterns/15_structs/3_tm_linux/as_array/GCC_tm3_DE.lst}
Tatsächlich: der Platz auf dem lokalen Stack wird zuerst wie in struct und dann wie ein Array behandelt.

Es ist sogar möglich, die Felder des structs über diesen Pointer zu verändern.

Und wiederum ist es zweifellos ein seltsamer Weg die Dinge umzusetzen; er ist für produktiven Code definitiv nicht
empfehlenswert.

\mysubparagraph{\Exercise}
Versuchen Sie als Übung die Monatsnummer zu verändern (um 1 zu erhöhen), indem Sie das struct wie ein Array behandeln.

\subsubsection{Struct als Bytearray}
Wir können sogar noch weiter gehen. Casten wir den Pointer zu einem Bytearray und ziehen einen Dump:

\lstinputlisting[style=customc]{patterns/15_structs/3_tm_linux/as_array/GCC_tm4.c}

\begin{lstlisting}
0x2D 0x00 0x00 0x00 
0x33 0x00 0x00 0x00 
0x17 0x00 0x00 0x00 
0x1A 0x00 0x00 0x00 
0x06 0x00 0x00 0x00 
0x72 0x00 0x00 0x00 
0x06 0x00 0x00 0x00 
0xCE 0x00 0x00 0x00 
0x01 0x00 0x00 0x00 
\end{lstlisting}
Wir haben dieses Beispiel auch zur Systemzeit 23:51:45 26-July-2014 ausgeführt
\footnote{Datum und Uhrzeit sind zu Demonstrationszwecken identisch. Die Bytewerte sind modifiziert.}.
Die Werte sind genau dieselben wie im vorherigen Dump(\myref{GCC_tm3_output}) und natürlich steht das LSB vorne, da es
sich um eine Little-Endian-Architektur handelt(\myref{sec:endianness}). 

\lstinputlisting[caption=\Optimizing GCC
4.8.1,style=customasmx86]{patterns/15_structs/3_tm_linux/as_array/GCC_tm4_DE.lst}
}
\FR{\subsection{Méthodes de protection contre les débordements de tampon}
\label{subsec:BO_protection}

Il existe quelques méthodes pour protéger contre ce fléau, indépendamment de la négligence
des programmeurs \CCpp.
MSVC possède des options comme\footnote{méthode de protection contre les débordements
de tampons côté compilateur:\href{http://go.yurichev.com/17133}{wikipedia.org/wiki/Buffer\_overflow\_protection}}:

\begin{lstlisting}
 /RTCs Stack Frame runtime checking
 /GZ Enable stack checks (/RTCs)
\end{lstlisting}

\myindex{x86!\Instructions!RET}
\myindex{Function prologue}
\myindex{Security cookie}

Une des méthodes est d'écrire une valeur aléatoire entre les variables locales sur
la pile dans le prologue de la fonction et de la vérifier dans l'épilogue, avant de
sortir de la fonction.
Si la valeur n'est pas la même, ne pas exécuter la dernière instruction \RET, mais
stopper (ou bloquer).
Le processus va s'arrêter, mais c'est mieux qu'une attaque distante sur votre ordinateur.
    
\newcommand{\CANARYURL}{\href{http://go.yurichev.com/17134}{wikipedia.org/wiki/Domestic\_canary\#Miner.27s\_canary}}

\myindex{Canary}

Cette valeur aléatoire est parfois appelé un \q{canari}, c'est lié au canari\footnote{\CANARYURL}
que les mineurs utilisaient dans le passé afin de détecter rapidement les gaz toxiques.

Les canaris sont très sensibles aux gaz, ils deviennent très agités en cas de danger,
et même meurent.

Si nous compilons notre exemple de tableau très simple~(\myref{arrays_simple}) dans
\ac{MSVC} avec les options RTC1 et RTCs, nous voyons un appel à \TT{@\_RTC\_CheckStackVars@8}
une fonction à la fin de la fonction qui vérifie si le \q{canari} est correct.

Voyons comment GCC gère ceci.
Prenons un exemple \TT{alloca()}~(\myref{alloca}):

\lstinputlisting[style=customc]{patterns/02_stack/04_alloca/2_1.c}

Par défaut, sans option supplémentaire, GCC 4.7.3 insère un test de  \q{canari} dans
le code:

\lstinputlisting[caption=GCC 4.7.3,style=customasmx86]{patterns/13_arrays/3_BO_protection/gcc_canary_FR.asm}

\myindex{x86!\Registers!GS}
La valeur aléatoire se trouve en \TT{gs:20}.
Elle est écrite sur la pile et à la fin de la fonction, la valeur sur la pile est
comparée avec le \q{canari} correct dans \TT{gs:20}.
Si les valeurs ne sont pas égales, la fonction \TT{\_\_stack\_chk\_fail} est appelée
et nous voyons dans la console quelque chose comme ça (Ubuntu 13.04 x86):

\begin{lstlisting}
*** buffer overflow detected ***: ./2_1 terminated
======= Backtrace: =========
/lib/i386-linux-gnu/libc.so.6(__fortify_fail+0x63)[0xb7699bc3]
/lib/i386-linux-gnu/libc.so.6(+0x10593a)[0xb769893a]
/lib/i386-linux-gnu/libc.so.6(+0x105008)[0xb7698008]
/lib/i386-linux-gnu/libc.so.6(_IO_default_xsputn+0x8c)[0xb7606e5c]
/lib/i386-linux-gnu/libc.so.6(_IO_vfprintf+0x165)[0xb75d7a45]
/lib/i386-linux-gnu/libc.so.6(__vsprintf_chk+0xc9)[0xb76980d9]
/lib/i386-linux-gnu/libc.so.6(__sprintf_chk+0x2f)[0xb7697fef]
./2_1[0x8048404]
/lib/i386-linux-gnu/libc.so.6(__libc_start_main+0xf5)[0xb75ac935]
======= Memory map: ========
08048000-08049000 r-xp 00000000 08:01 2097586    /home/dennis/2_1
08049000-0804a000 r--p 00000000 08:01 2097586    /home/dennis/2_1
0804a000-0804b000 rw-p 00001000 08:01 2097586    /home/dennis/2_1
094d1000-094f2000 rw-p 00000000 00:00 0          [heap]
b7560000-b757b000 r-xp 00000000 08:01 1048602    /lib/i386-linux-gnu/libgcc_s.so.1
b757b000-b757c000 r--p 0001a000 08:01 1048602    /lib/i386-linux-gnu/libgcc_s.so.1
b757c000-b757d000 rw-p 0001b000 08:01 1048602    /lib/i386-linux-gnu/libgcc_s.so.1
b7592000-b7593000 rw-p 00000000 00:00 0
b7593000-b7740000 r-xp 00000000 08:01 1050781    /lib/i386-linux-gnu/libc-2.17.so
b7740000-b7742000 r--p 001ad000 08:01 1050781    /lib/i386-linux-gnu/libc-2.17.so
b7742000-b7743000 rw-p 001af000 08:01 1050781    /lib/i386-linux-gnu/libc-2.17.so
b7743000-b7746000 rw-p 00000000 00:00 0
b775a000-b775d000 rw-p 00000000 00:00 0
b775d000-b775e000 r-xp 00000000 00:00 0          [vdso]
b775e000-b777e000 r-xp 00000000 08:01 1050794    /lib/i386-linux-gnu/ld-2.17.so
b777e000-b777f000 r--p 0001f000 08:01 1050794    /lib/i386-linux-gnu/ld-2.17.so
b777f000-b7780000 rw-p 00020000 08:01 1050794    /lib/i386-linux-gnu/ld-2.17.so
bff35000-bff56000 rw-p 00000000 00:00 0          [stack]
Aborted (core dumped)
\end{lstlisting}

\myindex{MS-DOS}
gs est ainsi appelé registre de segment. Ces registres étaient beaucoup utilisés
du temps de MS-DOS et des extensions de DOS.
Aujourd'hui, sa fonction est différente.
\myindex{TLS}
\myindex{Windows!TIB}

Dit brièvement, le registre \TT{gs} dans Linux pointe toujours sur le
\ac{TLS}~(\myref{TLS})---des informations spécifiques au thread sont stockées là.
À propos, en win32 le registre \TT{fs} joue le même rôle, pointant sur \ac{TIB}
\footnote{\href{http://go.yurichev.com/17104}{wikipedia.org/wiki/Win32\_Thread\_Information\_Block}}.

Il y a plus d'information dans le code source du noyau Linux (au moins dans la version 3.11),
dans\\
\emph{arch/x86/include/asm/stackprotector.h} cette variable est décrite dans les commentaires.

\subsubsection{ARM: \OptimizingKeilVI (\ARMMode)}
\myindex{\CLanguageElements!switch}

\lstinputlisting[style=customasmARM]{patterns/08_switch/1_few/few_ARM_ARM_O3.asm}

A nouveau, en investiguant ce code, nous ne pouvons pas dire si il y avait un switch()
dans le code source d'origine ou juste un ensemble de déclarations if().

\myindex{ARM!\Instructions!ADRcc}

En tout cas, nous voyons ici des instructions conditionnelles (comme \ADREQ (\emph{Equal}))
qui ne sont exécutées que si $R0=0$, et qui chargent ensuite l'adresse de la chaîne
\emph{<<zero\textbackslash{}n>>} dans \Reg{0}.
\myindex{ARM!\Instructions!BEQ}
L'instruction suivante \ac{BEQ} redirige le flux d'exécution en \TT{loc\_170}, si $R0=0$.

Le lecteur attentif peut se demander si \ac{BEQ} s'exécute correctement puisque \ADREQ
a déjà mis une autre valeur dans le registre \Reg{0}.

Oui, elle s'exécutera correctement, car \ac{BEQ} vérifie les flags mis par l'instruction
\CMP et \ADREQ ne modifie aucun flag.

Les instructions restantes nous sont déjà familières.
Il y a seulement un appel à \printf, à la fin, et nous avons déjà examiné cette
astuce ici~(\myref{ARM_B_to_printf}).
A la fin, il y a trois chemins vers \printf{}.

\myindex{ARM!\Instructions!ADRcc}
\myindex{ARM!\Instructions!CMP}
La dernière instruction, \TT{CMP R0, \#2}, est nécessaire pour vérifier si $a=2$.

Si ce n'est pas vrai, alors \ADRNE charge un pointeur sur la chaîne \emph{<<something unknown \textbackslash{}n>>}
dans \Reg{0}, puisque $a$ a déjà été comparée pour savoir s'elle est égale
à 0 ou 1, et nous sommes sûrs que la variable $a$ n'est pas égale à l'un de
ces nombres, à ce point.
Et si $R0=2$, un pointeur sur la chaîne \emph{<<two\textbackslash{}n>>} sera chargé
par \ADREQ dans \Reg{0}.

\subsubsection{ARM: \OptimizingKeilVI (\ThumbMode)}

\lstinputlisting[style=customasmARM]{patterns/08_switch/1_few/few_ARM_thumb_O3.asm}

% FIXME а каким можно? к каким нельзя? \myref{} ->

Comme il y déjà été dit, il n'est pas possible d'ajouter un prédicat conditionnel
à la plupart des instructions en mode Thumb, donc ce dernier est quelque peu similaire
au code \ac{CISC}-style x86, facilement compréhensible.

\subsubsection{ARM64: GCC (Linaro) 4.9 \NonOptimizing}

\lstinputlisting[style=customasmARM]{patterns/08_switch/1_few/ARM64_GCC_O0_FR.lst}

Le type de la valeur d'entrée est \Tint, par conséquent le registre \RegW{0} est
utilisé pour garder la valeur au lieu du registre complet \RegX{0}.

Les pointeurs de chaîne sont passés à \puts en utilisant la paire d'instructions
\INS{ADRP}/\INS{ADD} comme expliqué dans l'exemple \q{\HelloWorldSectionName}:~\myref{pointers_ADRP_and_ADD}.

\subsubsection{ARM64: GCC (Linaro) 4.9 \Optimizing}

\lstinputlisting[style=customasmARM]{patterns/08_switch/1_few/ARM64_GCC_O3_FR.lst}

Ce morceau de code est mieux optimisé.
L'instruction \TT{CBZ} (\emph{Compare and Branch on Zero} comparer et sauter si zéro)
effectue un saut si \RegW{0} vaut zéro.
Il y a alors un saut direct à \puts au lieu de l'appeler, comme cela a été expliqué
avant:~\myref{JMP_instead_of_RET}.


}
\JA{\subsubsection{MIPS}

ローカルスタック内の場所は$x$変数に割り当てられ、$\$sp+24$ と呼ばれます。
\myindex{MIPS!\Instructions!LW}

そのアドレスは \scanf に渡され、ユーザー入力値は\INC{LW}(\q{Load Word})を使用してロードされます。
そしてそれから \printf に渡されます。

\lstinputlisting[caption=\Optimizing GCC 4.4.5 (\assemblyOutput),style=customasmMIPS]{patterns/04_scanf/1_simple/MIPS/ex1.O3_JA.s}

IDAはスタックレイアウトを次のように表示します。

\lstinputlisting[caption=\Optimizing GCC 4.4.5 (IDA),style=customasmMIPS]{patterns/04_scanf/1_simple/MIPS/ex1.O3.IDA_JA.lst}

% TODO non-optimized version?
}

