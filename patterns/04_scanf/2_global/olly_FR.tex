\clearpage
\subsubsection{MSVC: x86 + \olly}
\myindex{\olly}

Les choses sont encore plus simple ici:

\begin{figure}[H]
\centering
\myincludegraphics{patterns/04_scanf/2_global/ex2_olly_1.png}
\caption{\olly: après l'exécution de \scanf}
\label{fig:scanf_ex2_olly_1}
\end{figure}

La variable se trouve dans le segment de données.
Après que l'instruction \PUSH (pousser l'adresse de $x$) ait été exécutée,
l'adresse apparaît dans la fenêtre de la pile. Cliquer droit sur cette ligne
et choisir \q{Follow in dump}. % TODO olly French ?
La variable va apparaître dans la fenêtre de la mémoire sur la gauche.
Après que nous ayons entré 123 dans la console, \TT{0x7B} apparaît dans la fenêtre
de la mémoire (voir les régions surlignées dans la copie d'écran).

Mais pourquoi est-ce que le premier octet est \TT{7B}?
Logiquement, Il devrait y avoir \TT{00 00 00 7B} ici.
La cause de ceci est référé comme \gls{endianness}, et x86 utilise \emph{little-endian}.
Cela implique que l'octet le plus faible poids est écrit en premier, et le plus fort
en dernier.
Voir à ce propos: \myref{sec:endianness}.
Revenons à l'exemple, la valeur 32-bit est chargée depuis son adresse mémoire
dans \EAX et passée à \printf.

L'adresse mémoire de $x$ est \TT{0x00C53394}.

\clearpage
Dans \olly nous pouvons examiner l'espace mémoire du processus  (Alt-M) et nous
pouvons voir que cette adresse se trouve dans le segment PE \TT{.data} de notre
programme:

\label{olly_memory_map_example}
\begin{figure}[H]
\centering
\myincludegraphics{patterns/04_scanf/2_global/ex2_olly_2.png}
\caption{\olly: espace mémoire du processus}
\label{fig:scanf_ex2_olly_2}
\end{figure}

