\clearpage
\myparagraph{MSVC \Optimizing + \olly}
\myindex{\olly}

Nous pouvons essayer cet exemple (optimisé) dans \olly. Voici la première itération:

\begin{figure}[H]
\centering
\myincludegraphics{patterns/10_strings/1_strlen/olly1.png}
\caption{\olly: début de la première itération}
\label{fig:strlen_olly_1}
\end{figure}

Nous voyons qu'\olly a trouvé une boucle et, par facilité, a mis ses instructions
entre crochets.
En cliquant sur le bouton droit sur \EAX, nous pouvons choisir \q{Follow in Dump}
et la fenêtre de la mémoire se déplace jusqu'à la bonne adresse.
Ici, nous voyons la chaîne \q{hello!} en mémoire.
Il y a au moins un zéro après cette dernière et ensuite des données aléatoires.

Si \olly voit un registre contenant une adresse valide, qui pointe sur une chaîne,
il montre cette chaîne.

\clearpage
Appuyons quelques fois sur F8 (\stepover), pour aller jusqu'au début du corps de
la boucle:

\begin{figure}[H]
\centering
\myincludegraphics{patterns/10_strings/1_strlen/olly2.png}
\caption{\olly: début de la seconde itération}
\label{fig:strlen_olly_2}
\end{figure}

Nous voyons qu'\EAX contient l'adresse du second caractère de la chaîne.

\clearpage

Nous devons appuyons un certain nombre de fois sur F8 afin de sortir de la boucle:

\begin{figure}[H]
\centering
\myincludegraphics{patterns/10_strings/1_strlen/olly3.png}
\caption{\olly: calcul de la différence entre les pointeurs}
\label{fig:strlen_olly_3}
\end{figure}

% FIXME:
Nous voyons qu'\EAX contient l'adresse de l'octet à zéro situé juste après la chaîne.
Entre temps, \EDX n'a pas changé, donc il pointe sur le début de la chaîne.

La différence entre ces deux valeurs est maintenant calculée.

\clearpage
L'instruction \SUB vient juste d'être effectuée:

\begin{figure}[H]
\centering
\myincludegraphics{patterns/10_strings/1_strlen/olly4.png}
\caption{\olly: maintenant décrémenter \EAX}
\label{fig:strlen_olly_4}
\end{figure}

La différence entre les deux pointeurs est maintenant dans le registre \EAX---7.
Effectivement, la longueur de la chaîne \q{hello!} est 6, mais avec l'octet à zéro
inclus---7.
Mais \TT{strlen()} doit renvoyer le nombre de caractère non-zéro dans la chaîne.
Donc la décrémentation est effectuée et ensuite la fonction sort.
