\subsection{GCC---もう1つ}
\label{use_parts_of_C_strings}

\emph{匿名の}C文字列が\emph{const}型(\myref{string_is_const_char})を持ち、
定数セグメントに割り当てられたC文字列が不変であることが保証されているという事実は興味深い結果をもたらします:
コンパイラは文字列の特定の部分を使用するかもしれません。

下記の例で試してみましょう。

\begin{lstlisting}[style=customc]
#include <stdio.h>

int f1()
{
	printf ("world\n");
}

int f2()
{
	printf ("hello world\n");
}

int main()
{
	f1();
	f2();
}
\end{lstlisting}

一般的な \CCpp{} コンパイラ(MSVCを含む)は2つの文字列を割り当てますが、GCC 4.8.1の動作を見てみましょう。

\lstinputlisting[caption=GCC 4.8.1 + IDA,style=customasmx86]{patterns/01_helloworld/two_strings.asm}

確かに、\q{hello world}という文字列を印刷すると、これらの2つの単語はメモリに隣接して配置され、\GTT{f2()}関数から呼び出される \puts 関数はこの文字列が分割されていることを認識しません。 
実際、分割されていません。 このリストには、\q{仮想的に}分けられています。

\puts が\GTT{f1()}から呼び出されると、\q{world}文字列とNULLバイトを使用します。 \puts はこの文字列の前に何かがあることを認識していません!

この巧妙なトリックは、少なくともGCCでよく使用され、メモリを節約できます。 これは\emph{文字列インターン}に似ています。

別の関連する例がここにあります:\myref{strstr_example}
