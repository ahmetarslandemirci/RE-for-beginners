\subsubsection{\NonOptimizingKeilVI (\ARMMode)}

Let's start by compiling our example in Keil:

\begin{lstlisting}
armcc.exe --arm --c90 -O0 1.c 
\end{lstlisting}

\myindex{\IntelSyntax}
The \emph{armcc} compiler produces assembly listings in Intel-syntax, but it has high-level ARM-processor related macros
\footnote{e.g. ARM mode lacks \PUSH/\POP instructions}, 
but it is more important for us to see the instructions \q{as is} so let's see the compiled result in \IDA.

\begin{lstlisting}[caption=\NonOptimizingKeilVI (\ARMMode) \IDA,style=customasmARM]
.text:00000000             main
.text:00000000 10 40 2D E9    STMFD   SP!, {R4,LR}
.text:00000004 1E 0E 8F E2    ADR     R0, aHelloWorld ; "hello, world"
.text:00000008 15 19 00 EB    BL      __2printf
.text:0000000C 00 00 A0 E3    MOV     R0, #0
.text:00000010 10 80 BD E8    LDMFD   SP!, {R4,PC}

.text:000001EC 68 65 6C 6C+aHelloWorld  DCB "hello, world",0    ; DATA XREF: main+4
\end{lstlisting}

In the example, we can easily see each instruction has a size of 4 bytes.
Indeed, we compiled our code for ARM mode, not for Thumb.

\myindex{ARM!\Instructions!STMFD}
\myindex{ARM!\Instructions!POP}
The very first instruction, \INS{STMFD SP!, \{R4,LR\}}\footnote{\ac{STMFD}}, 
works as an x86 \PUSH instruction, writing the values of two registers (\Reg{4} and \ac{LR}) into the stack.

Indeed, in the output listing from the \emph{armcc} compiler, for the sake of simplification, 
actually shows the \INS{PUSH \{r4,lr\}} instruction.
But that is not quite precise. The \PUSH instruction is only available in Thumb mode.
So, to make things less confusing, we're doing this in \IDA.

This instruction first \glspl{decrement} the \ac{SP} so it points to the place in the stack
that is free for new entries, then it saves the values of the \Reg{4} and \ac{LR} registers at the address
stored in the modified \ac{SP}.

This instruction (like the \PUSH instruction in Thumb mode) is able to save several register values at once which can be very useful.
By the way, this has no equivalent in x86.
It can also be noted that the \TT{STMFD} instruction is a generalization 
of the \PUSH instruction (extending its features), since it can work with any register, not just with \ac{SP}.
In other words, \TT{STMFD} may be used for storing a set of registers at the specified memory address.

\myindex{\PICcode}
\myindex{ARM!\Instructions!ADR}
The \INS{ADR R0, aHelloWorld}
instruction adds or subtracts the value in the \ac{PC} register to the offset where the \TT{hello, world} string is located.
How is the \TT{PC} register used here, one might ask?
This is called \q{\PICcode}\footnote{Read more about it in relevant section~(\myref{sec:PIC})}.

Such code can be executed at a non-fixed address in memory.
In other words, this is \ac{PC}-relative addressing.
The \INS{ADR} instruction takes into account the difference between the address of this instruction and the address where the string is located.
This difference (offset) is always to be the same, no matter at what address our code is loaded by the \ac{OS}.
That's why all we need is to add the address of the current instruction (from \ac{PC}) in order to get the absolute memory address of our C-string.

\myindex{ARM!\Registers!Link Register}
\myindex{ARM!\Instructions!BL}
\INS{BL \_\_2printf}\footnote{Branch with Link} instruction calls the \printf function. 
Here's how this instruction works: 

\begin{itemize}
\item store the address following the \INS{BL} instruction (\TT{0xC}) into the \ac{LR};
\item then pass the control to \printf by writing its address into the \ac{PC} register.
\end{itemize}

When \printf finishes its execution it must have information about where it needs to return the control to.
That's why each function passes control to the address stored in the \ac{LR} register.

That is a difference between \q{pure} \ac{RISC}-processors like ARM and \ac{CISC}-processors like x86,
where the return address is usually stored on the stack.
Read more about this in next section~(\myref{sec:stack}).

By the way, an absolute 32-bit address or offset cannot be encoded in the 32-bit \TT{BL} instruction because
it only has space for 24 bits.
As we may recall, all ARM-mode instructions have a size of 4 bytes (32 bits).
Hence, they can only be located on 4-byte boundary addresses.
This implies that the last 2 bits of the instruction address (which are always zero bits) may be omitted.
In summary, we have 26 bits for offset encoding. This is enough to encode $current\_PC \pm{} \approx{}32M$.

\myindex{ARM!\Instructions!MOV}
Next, the \INS{MOV R0, \#0}\footnote{Meaning MOVe} instruction just writes 0 into the \Reg{0} register.
That's because our C-function returns 0 and the return value is to be placed in the \Reg{0} register.

\myindex{ARM!\Registers!Link Register}
\myindex{ARM!\Instructions!LDMFD}
\myindex{ARM!\Instructions!POP}
The last instruction \INS{LDMFD SP!, {R4,PC}}\footnote{\ac{LDMFD} is an inverse instruction of \ac{STMFD}}.
It loads values from the stack (or any other memory place) in order to save them into \Reg{4} and \ac{PC}, and \glslink{increment}{increments} the \gls{stack pointer} \ac{SP}.
It works like \POP here.\\
N.B. The very first instruction \TT{STMFD} saved the \Reg{4} and \ac{LR} registers pair on the stack, but \Reg{4} and \ac{PC} are \emph{restored} during the \TT{LDMFD} execution.

As we already know, the address of the place where each function must return control to is usually saved in the \ac{LR} register.
The very first instruction saves its value in the stack because the same register will be used by our
\main function when calling \printf.
In the function's end, this value can be written directly to the \ac{PC} register, thus passing control to where our function has been called.

Since \main is usually the primary function in \CCpp,
the control will be returned to the \ac{OS} loader or to a point in a \ac{CRT},
or something like that.

All that allows omitting the \INS{BX LR} instruction at the end of the function.

\myindex{ARM!DCB}
\TT{DCB} is an assembly language directive defining an array of bytes or ASCII strings, akin to the DB directive 
in the x86-assembly language.

