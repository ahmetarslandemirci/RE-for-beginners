\subsubsection{\NonOptimizingKeilVI (\ARMMode)}

Beginnen wir mit dem Kompilieren des Beispiels mit Keil:

\begin{lstlisting}
armcc.exe --arm --c90 -O0 1.c 
\end{lstlisting}

\myindex{\IntelSyntax}
Der \emph{armcc}-Compiler erstellt Assembler-Quelltext im Intel-Syntax, hat aber High-Level-Makros
bezüglich der ARM-Prozessoren\footnote{d.h. der ARM-Mode hat keine \PUSH/\POP-Anweisungen}.
Es ist hier wichtig die \q{richtigen} Anweisungen zu sehen, deswegen ist hier das Ergebnis mit
\IDA kompiliert.

\begin{lstlisting}[caption=\NonOptimizingKeilVI (\ARMMode) \IDA,style=customasmARM]
.text:00000000             main
.text:00000000 10 40 2D E9    STMFD   SP!, {R4,LR}
.text:00000004 1E 0E 8F E2    ADR     R0, aHelloWorld ; "hello, world"
.text:00000008 15 19 00 EB    BL      __2printf
.text:0000000C 00 00 A0 E3    MOV     R0, #0
.text:00000010 10 80 BD E8    LDMFD   SP!, {R4,PC}

.text:000001EC 68 65 6C 6C+aHelloWorld  DCB "hello, world",0    ; DATA XREF: main+4
\end{lstlisting}

Im ersten Beispiel ist zu erkennen, dass jede Anweisung 4 Byte groß ist.
Tatsächlich wurde der Code für den ARM- und nicht den Thumb-Mode erstellt.

\myindex{ARM!\Instructions!STMFD}
\myindex{ARM!\Instructions!POP}
Die erste Anweisung, \INS{STMFD SP!, \{R4,LR\}}\footnote{\ac{STMFD}}, arbeitet wie eine x86-\PUSH-Anweisung
um die Werte der beiden Register (\Reg{4} and \ac{LR}) auf den Stack zu legen.

Die Ausgabe des \emph{armcc}-Compilers, zeigt, aus Gründen der Einfachheit, die \INS{PUSH \{r4,lr\}}-Anweisung.
Dies ist nicht vollständig präzise. Die \PUSH-Anweisung ist nur im Thumb-Mode verfügbar.
Um die Dinge nicht zu verwirrend zu machen, wird der Code in \IDA kompiliert.

Die Anweisung dekrementiert zunächst den \ac{SP}, so dass er auf den Bereich im Stack zeigt, der
für neue Einträge frei ist. Anschließend werden die Werte der Register \Reg{4} und \ac{LR} an der Adresse
gespeichert auf den der (modifizierte) \ac{SP} zeigt.

Diese Anweisungen (wie \PUSH im Thumb-Mode) ist in der Lage mehrere Register-Werte auf einmal zu speichern,
was sehr nützlich sein kann. Übrigens: in x86 gibt es dazu kein Äquivalent.
Außerdem ist erwähnenswert, dass die \TT{STMFD}-Anweisung eine Generalisierung der \PUSH-Anweisung
(ohne deren Eigenschaften) ist, weil sie auf jedes Register angewandt werden kann und nicht nur auf \ac{SP}.
Mit anderen Worten kann \TT{STMFD} genutzt werden um eine Reihen von Registern an einer angegebenen
Speicher-Adresse zu sichern.

\myindex{\PICcode}
\myindex{ARM!\Instructions!ADR}
Die \INS{ADR R0, aHelloWorld}-Anweisung addiert oder subtrahiert den Wert im \ac{PC}-Register zum Offset
an dem die \TT{hello, world}-Zeichenkette ist.
Man kann sich nun fragen, wie das \TT{PC}-Register hier genutzt wird.
Dies wird \q{\PICcode}\footnote{mehr darüber in der entsprechenden Sektion~(\myref{sec:PIC})} genannt.

Code dieser Art kann an nicht-festen Adressen im Speicher ausgeführt werden.
Mit anderen Worten: dies ist \ac{PC}-relative Adressierung.
Die \INS{ADR}-Anweisung berücksichtigt den Unterschied zwischen der Adresse dieser Anweisung und der Adresse
an dem die Zeichenkette gespeichert ist.
Der Unterschied (Offset) ist immer gleich, egal an welcher Adresse der Code vom \ac{OS} geladen wurden.
Dementsprechend ist alles was gemacht werden muss, die Adresse der aktuellen Anweisung (vom \ac{PC})
zu addieren um die absolute Speicheradresse der Zeichenkette zu bekommen.

\myindex{ARM!\Registers!Link Register}
\myindex{ARM!\Instructions!BL}
\INS{BL \_\_2printf}\footnote{Branch with Link}-Anweisung ruft die \printf-Funktion auf. 
Die Anweisung funktioniert wie folgt:

\begin{itemize}
\item Speichere die Adresse hinter der \INS{BL}-Anweisung (\TT{0xC}) in \ac{LR};
\item anschließend wird übergebe die Kontrolle an \printf indem dessen Adresse ins \ac{PC}-Register geschrieben wird.
\end{itemize}

Wenn \printf die Ausführung beendet, müssen Informationen vorliegen, wo die Ausführung weitergehen soll.
Das ist der Grund warum jede Funktion die Kontrolle an die Adresse, gespeichert im \ac{LR}-Register übergibt.

Dies ist ein Unterschied zwischen einem \q{reinem} \ac{RISC}-Prozessor wie ARM und \ac{CISC}-Prozessoren wie x86,
bei denen die Rücksprungadresse in der Regel auf dem Stack gespeichert wird.
Mehr dazu ist im nächsten Abschnitt zu lesen~(\myref{sec:stack}).

Übrigens eine absolute 32-Bit-Adresse oder -Offset kann nicht in einer 32-Bit-\TT{BL}-Anweisung kodiert werden,
weil diese nur für 24 Bit Platz bietet. Wie bereits erwähnt haben alle ARM-Mode-Anweisungen eine Größe
von 4 Byte (32 Bit). Aus diesem Grund können diese nur an 4-Byte-Grenzen des Speichers platziert werden.
Dies heißt auch, das die letzten zwei Bit der Anweisungsadresse (die immer Null sind) entfallen können.
Zusammenfassend, stehen 26 Bit für die Offset-Kodierung zur Verfügung. Dies ist genug für
$current\_PC \pm{} \approx{}32M$.

\myindex{ARM!\Instructions!MOV}
Als nächstes schreibt die Anweisung \INS{MOV R0, \#0}\footnote{das heißt MOVe} lediglich 0 in
das \Reg{0}-Register weil der Rückgabewert hier gespeichert wird und die gezeigte C-Funktion 0
als Argument für die return-Anweisung hat.

\myindex{ARM!\Registers!Link Register}
\myindex{ARM!\Instructions!LDMFD}
\myindex{ARM!\Instructions!POP}
Die letzte Anweisung \INS{LDMFD SP!, {R4,PC}}\footnote{\ac{LDMFD} ist eine inverse Anweisung von \ac{STMFD}}
lädt die Werte nacheinander vom Stack (oder eine andere Speicheradresse) um sie in die Register \Reg{4} und \ac{PC}
zu sichern. Außerdem wird der Stack Pointer \ac{SP} inkrementiert. Hier arbeitet der Befehl wie \POP.

Die erste Anweisung \TT{STMFD} sichert das Register-Paar \Reg{4} und \ac{LR} auf dem Stack, jedoch werden \Reg{4} und \ac{PC}
während der Ausführung von \TT{LDMFD} \emph{wiederhergestellt}.

Wie bereits bekannt, wird die Adresse die nach der Ausführung einer Funktion angesprungen wird in dem \ac{LR}-Register gesichert.
Die allererste Anweisung sichert diese Wert auf dem Stack weil das gleiche Register von der \main-Funktion genutzt wird,
wenn \printf aufgerufen wird.
Am Ende der Funktion kann dieser Wert direkt in das \ac{PC}-Register geschrieben werden und so die Ausführung an der
Stelle fortgesetzt werden an der die Funktion aufgerufen wurde.

Da \main in der Regel die erste Funktion in \CCpp ist, wird die Kontrolle an das \ac{OS} oder einen Punkt in der
\ac{CRT} übergeben.

All dies erlaubt das Auslassen der \INS{BX LR}-Anweisung am Ende der Funktion.

\myindex{ARM!DCB}
\TT{DCB} ist eine Assemblerdirektive die ein Array von Bytes oder ASCII anlegt, ähnlich der DB-Direktive
in der x86-Assembler-Sprache.
