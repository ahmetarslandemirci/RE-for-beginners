\subsubsection{\NonOptimizingKeilVI (\ARMMode)}

Na początek skompilujmy nasz przykład w Keil:

\begin{lstlisting}
armcc.exe --arm --c90 -O0 1.c 
\end{lstlisting}

\myindex{\IntelSyntax}
Kompilator \emph{armcc} generuje listing w asemblerze w formacie Intel.
Ten listing zawiera niektóre wysokopoziomowe makra, powiązane z ARM
\footnote{na przykład, korzysta on z instrukcji \PUSH/\POP, których nie ma w trybie ARM},
dlatego zobaczmy jak wygląda skompilowany kod w \IDA.

\begin{lstlisting}[caption=\NonOptimizingKeilVI (\ARMMode) \IDA,style=customasmARM]
.text:00000000             main
.text:00000000 10 40 2D E9    STMFD   SP!, {R4,LR}
.text:00000004 1E 0E 8F E2    ADR     R0, aHelloWorld ; "hello, world"
.text:00000008 15 19 00 EB    BL      __2printf
.text:0000000C 00 00 A0 E3    MOV     R0, #0
.text:00000010 10 80 BD E8    LDMFD   SP!, {R4,PC}

.text:000001EC 68 65 6C 6C+aHelloWorld  DCB "hello, world",0    ; DATA XREF: main+4
\end{lstlisting}

W przykładzie wyżej można dostrzec, że każda instrukcja ma rozmiar 4 bajty.
Rzeczywiście, przecież kompilowaliśmy nasz kod dla trybu ARM, a nie Thumb.

\myindex{ARM!\Instructions!STMFD}
\myindex{ARM!\Instructions!POP}
Pierwsza instrukcja, \INS{STMFD SP!, \{R4,LR\}}\footnote{\ac{STMFD}},
działa jak instrukcja \PUSH w x86: odkłada wartości dwóch rejestrów (\Reg{4} i \ac{LR}) na stos.
W listingu w asemblerze kompilator \emph{armcc}, żeby uprościć sprawę pokazuje tu instrukcję
\INS{PUSH \{r4,lr\}}.
Nie jest to dokładnie to co chcieliibyśmy zobaczyć. Instrukcja \PUSH jest dostępna tylko w trybie Thumb, dlatego,
 zaproponowałem pracować z kompilatorem \IDA.

Ta instrukcja zmniejsza \ac{SP}, żeby on wskazywał na miejsce na stosie, dostępne do zapisu nowych wartości, następnie zapisuje wartości rejestrów \Reg{4} i \ac{LR} 
pod adres w pamięci, na który wskazuje zmodyfikowany rejestr \ac{SP}.

Ta instrukcja, tak samo jak i instrukcja \PUSH w trybie Thumb, pozwala na odkładanie na stos kilku wartości rejestrów na raz, co może się okazać bardzo wygodne.
A propos, w x86 czegoś takiego nie ma.
Należy zauważyć, że \TT{STMFD}~--- jest generalizacją instrukcji \PUSH (czyli rozszerza jej możliwości), dlatego że może operować na różnych rejestrach, a nie tylko na \ac{SP}.
Inaczej mówiąc, z \TT{STMFD} można korzystać przy zapisie zestawu rejestrów we wskazanym miejscu w pamięci.

\myindex{\PICcode}
\myindex{ARM!\Instructions!ADR}
Instrukcja \INS{ADR R0, aHelloWorld} dodaje lub odejmuje zawartość rejestru \ac{PC} do przesunięcia, w którym jest przechowywana linia
\TT{hello, world}.
Jaki jest związek z \ac{PC}? Jest to tzw \q{\PICcode}
\footnote{będzie to szerzej omówione w kolejnym rozdziale ~(\myref{sec:PIC})}.
Nie jest on przywiązany do jakiegokolwiek miejsca w pamięci.
Inaczej mówiąc, jest to \ac{PC}-względne adresowanie.
W opcode instrukcji \TT{ADR} jest zapisana różnica między adresem tej instrukcji a miejscem, w którym jest przechowywana linia.
Ta różnica zawsze jest stała, niezależnie od tego w którym miejscu został załadowany \ac{OS} nasz kod.
Dlatego, wszystko czego potrzebujemy~--- to dodanie adresu bieżącej instrukcji (z \ac{PC}), żeby otrzymać bieżący adres bezwględny linii.

\myindex{ARM!\Registers!Link Register}
\myindex{ARM!\Instructions!BL}
Instrukcja \INS{BL \_\_2printf}\footnote{Branch with Link} wywołuje f-cje \printf.
Działanie tej f-cji składa się z 2 kroków:

\begin{itemize}
\item zapisać adres występujący po instrukcji \INS{BL} (\TT{0xC}) do rejestru \ac{LR};
\item przekazać zarządzanie do \printf, zapisując adres tej f-cji do \ac{PC}.
\end{itemize}

Kiedy f-cja \printf zakończy działanie, procesor musi wiedzieć, gdzie zwrócić zarządzanie, dlatego kończąc pracę, każda funkcja przekazuje zarządzanie pod adres zapisany w rejestrze \ac{LR}.

Na tym polega główna różnica między \ac{RISC}-procesorami typu ARM i \ac{CISC}-procesorami typu x86,
gdzie adres powrotu zwykle jest odkładany na stos ~(\myref{sec:stack}).

Dodatkowo, nie jest możliwe zakodowanie 32-bitowego adresu bezwzględnego (lub przesunięcia) w 32-bitowej instrukcji \INS{BL}, jako że ona ma miejsce tylko dla 24-ch bitów.
Z powodu, że wszystkie instrukcje w trybie ARM mają długość 4 bajty (32 bity) i instrukcje mogą się znajdować tylko pod adresem wielokrotnym 4, to ostatnich 2 bitów (które są zawsze zerowe) można nie kodować.
W wyniku mamy 26 bitów, za pomocą których można zakodować $current\_PC \pm{} \approx{}32M$.

\myindex{ARM!\Instructions!MOV}
Następna instrukcja \INS{MOV R0, \#0}\footnote{oznacza MOV}
po prostu zapisuje 0 do rejestru \Reg{0}.
To dlatego, że nasza f-cja zwraca 0, a wartość zwróconą każda f-cja zostawia w \Reg{0}.

\myindex{ARM!\Registers!Link Register}
\myindex{ARM!\Instructions!LDMFD}
\myindex{ARM!\Instructions!POP}
Ostatnia instrukcja \INS{LDMFD SP!, {R4,PC}}\footnote{\ac{LDMFD}~--- jest instrukcją względem działania odwrotną do \ac{STMFD}}.
Ona ładuje ze stosu (lub każdego innego miejsca w pamięci) wartości do zapisu w \Reg{4} i \ac{PC}, zwiększając \glslink{stack pointer}{wskaźnik stosu} \ac{SP}.
Tutaj działa analogicznie do \POP.\\
N.B. Pierwsza instrukcja \TT{STMFD} odłożyła na stos \Reg{4} i \ac{LR}, a \emph{przywracane} w wyniku działania \TT{LDMFD} są rejestry \Reg{4} i \ac{PC}.

Jak już wiemy, do rejestru \ac{LR} zwykle jest zapisywany adres miejsca w pamięci, pod który f-cja będzie zwracała zarządzanie.
Pierwsza isntrukcja odkłada tę wartość na stos, dlatego że nasza f-cja \main później sama będzie korzystała z tego rejstru w momencie wywołania f-cji \printf.
Czyli, na końcu f-cji, tę wartość można było od razu zapisać do \ac{PC}, w ten sposób przekazując zarządzanie tam, skąd nasza f-cja była wywołana.

Jako że, z reguły, f-cja \main jest funkcją główną w \CCpp, zarządzanie zostanie zwróconę do \ac{OS}, lub gdzieś do \ac{CRT} 
lub coś w tym stylu.

Wszystko t pozwala pozbyć się ręcznego \INS{BX LR} na samym końcu f-cji.

\myindex{ARM!DCB}
\TT{DCB}~--- dyrektywa asemblera, opisująca tablice bajtów bądź ciągów ASCII, analogiczna dyrektywa w asemblerze x86 to DB.


