\mysection[線形合同生成器]{擬似乱数生成器としての線形合同生成器}
\myindex{\CStandardLibrary!rand()}
\label{LCG_simple}

おそらく、線形合同ジェネレータは、乱数を生成するための最も簡単な方法です。

今日では\footnote{メルセンヌツイスターの方がいいです}選択されませんが、とても単純です
(1回の乗算、1回の加算とAND演算)。
これを例として使用できます。

\lstinputlisting[style=customc]{patterns/145_LCG/rand_JPN.c}

2つの関数があります:最初のものは内部状態を初期化するために使用され、2つ目は
擬似乱数を生成するために呼び出されます。

アルゴリズムでは2つの定数が使用されていることがわかります。 
それらは[William H. Press and Saul A. Teukolsky and William T. Vetterling and Brian P. Flannery, \IT{Numerical Recipes}, (2007)]
から取られています。

\TT{\#define} \CCpp 命令文を使ってそれらを定義しましょう。 これはマクロです。

\CCpp マクロと定数の違いは、すべてのマクロが \CCpp プリプロセッサでその値に置換され、
変数と異なりメモリを使用しないことです。

対照的に、定数は読み取り専用変数です。

定数変数のポインタ(またはアドレス)を取ることは可能ですが、マクロではできません。

C標準の\TT{my\_rand()}は0から32767の範囲の値を返さなければならないため、
最後のAND演算が必要です。

32ビットの擬似乱数値を取得する場合は、最後のAND演算を省略してください。

\subsection{x86}

\lstinputlisting[caption=\Optimizing MSVC 2013,style=customasmx86]{patterns/145_LCG/rand_MSVC_2013_x86_Ox.asm}

ここでは、両方の定数がコードに埋め込まれています。 
割り当てられたメモリはありません。

\TT{my\_srand()}関数は入力値を内部の
\TT{rand\_state}変数にコピーするだけです。

\TT{my\_rand()}はそれを受け取り、次の\TT{rand\_state}を計算し、それを切り取り、EAXレジスタに残します。

最適化されていないバージョンはより冗長です。

\lstinputlisting[caption=\NonOptimizing MSVC 2013,style=customasmx86]{patterns/145_LCG/rand_MSVC_2013_x86.asm}

\subsection{x64}

x64のバージョンはほとんど同じで、64ビットではなく32ビットのレジスタを使用しています。
(ここで \Tint 値を使用しているためです)

しかし、\TT{my\_srand()}は入力引数をスタックからではなく \ECX レジスタから取ります:

\lstinputlisting[caption=\Optimizing MSVC 2013 x64,style=customasmx86]{patterns/145_LCG/rand_MSVC_2013_x64_Ox_JPN.asm}

GCCコンパイラはほとんど同じコードを生成します。

\subsection{32ビットARM}

\lstinputlisting[caption=\OptimizingKeilVI (\ARMMode),style=customasmARM]{patterns/145_LCG/rand.s_Keil_ARM_O3_JPN.s}

32ビット定数をARM命令に埋め込むことはできないため、Keilはそれらを外部に配置して追加する必要があります。 
興味深いことに、0x7FFF定数も埋め込むことはできません。
Keilがやっているのは、\TT{rand\_state}を17ビット左にシフトし、右に17ビットシフトすることです。 
これは、 \CCpp の $(rand\_state \ll 17) \gg 17$ 命令文に似ています。
それは役に立たない操作だと思われますが、それは17ビットをクリアして15ビットをそのままにして、これが結局のところ私たちの目標です。\\
\\
Thumbモードの \Optimizing Keil はほとんど同じコードが生成します。

\subsection{MIPS}

\lstinputlisting[caption=\Optimizing GCC 4.4.5 (IDA),style=customasmMIPS]{patterns/145_LCG/MIPS_O3_IDA_JPN.lst}

おっと、ここでは1つの定数(0x3C6EF35Fまたは1013904223)しか表示されません。 
もう1つはどこでしょうか(1664525)?

1664525による乗算は、シフトと加算だけを使用して実行されるようです! 
この仮定を確認してみましょう:

\lstinputlisting[style=customc]{patterns/145_LCG/test.c}

\lstinputlisting[caption=\Optimizing GCC 4.4.5 (IDA),style=customasmMIPS]{patterns/145_LCG/test_O3_MIPS.lst}

本当に!

\subsubsection{MIPSの再配置}

また、メモリやストアから実際にメモリにロードする操作がどのように機能するかにも焦点を当てます。

ここのリストはIDAによって作成され、IDAはいくつかの詳細を隠しています。

objdumpを2回実行します:逆アセンブルされたリストと再配置リストを取得します。

\lstinputlisting[caption=\Optimizing GCC 4.4.5 (objdump)]{patterns/145_LCG/MIPS_O3_objdump.txt}

\TT{my\_srand()}関数の2つの再配置を考えてみましょう。

最初のアドレス0は\TT{R\_MIPS\_HI16}のタイプを持ち、
アドレス8の2番目のアドレスは\TT{R\_MIPS\_LO16}のタイプです。

つまり、.bssセグメントの先頭のアドレスは、0(アドレスの上位部分)および
8(アドレスの下位部分)のアドレスに書き込まれることを意味します。

\TT{rand\_state}変数は、.bssセグメントの先頭にあります。

したがって、命令 \LUI と \SW のオペランドにはゼロがあります。何もまだ存在しないからです。
コンパイラは何をそこに書き込んだらいいかわかりません。

リンカがこれを修正し、アドレスの上位部分が \LUI のオペランドに書き込まれ、
アドレスの下位部分が \SW のオペランドに書き込まれます。

\SW はアドレスの下位部分とレジスタ \$V0 にあるものを合計します(上位部分はそこにあります)。

これは my\_rand() 関数の場合と同じです。 R\_MIPS\_HI16 再配置は、リンカに.bssセグメントアドレスの上位部分を 
\LUI 命令に書き込むように指示します。

したがって、rand\_state 変数アドレスの上位部分はレジスタ \$V1 に存在します。

アドレス0x10にある \LW 命令は、上位部分と下位部分を合計し、rand\_state
変数の値を \$V0 にロードします。

アドレス0x54にある \SW 命令は、加算を再度行い、新しい値をrand\_stateグローバル変数に
格納します。

IDAは、ロード中に再配置を処理するため、これらの詳細は隠していますが、それらを念頭に置いておく必要があります。

% TODO add example of compiled binary, GDB example, etc...


\subsection{スレッドセーフ版の例}

この例のスレッドセーフ版は、後で説明します:\myref{LCG_TLS}
