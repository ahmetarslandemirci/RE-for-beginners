\mysection[Линейный конгруэнтный генератор]{Линейный конгруэнтный генератор как генератор псевдослучайных чисел}
\myindex{\CStandardLibrary!rand()}
\label{LCG_simple}

Линейный конгруэнтный генератор, пожалуй, самый простой способ генерировать псевдослучайные числа.

Он не в почете в наше время\footnote{Вихрь Мерсенна куда лучше}, но он настолько прост
(только одно умножение, одно сложение и одна операция \q{И}),
что мы можем использовать его в качестве примера.

\lstinputlisting[style=customc]{patterns/145_LCG/rand_RU.c}

Здесь две функции: одна используется для инициализации внутреннего состояния, а вторая
вызывается собственно для генерации псевдослучайных чисел.

Мы видим, что в алгоритме применяются две константы.
Они взяты из
[William H. Press and Saul A. Teukolsky and William T. Vetterling and Brian P. Flannery, \emph{Numerical Recipes}, (2007)].
Определим их используя выражение \CCpp \TT{\#define}. Это макрос.

Разница между макросом в \CCpp и константой в том, что все макросы заменяются на значения препроцессором
\CCpp и они не занимают места в памяти как переменные.

А константы, напротив, это переменные только для чтения.

Можно взять указатель (или адрес) переменной-константы, но это невозможно сделать с макросом.

Последняя операция \q{И} нужна, потому что согласно стандарту Си \TT{my\_rand()} должна возвращать значение в пределах
0..32767.

Если вы хотите получать 32-битные псевдослучайные значения, просто уберите последнюю операцию \q{И}.

\subsection{x86}

\lstinputlisting[caption=\Optimizing MSVC 2013,style=customasmx86]{patterns/145_LCG/rand_MSVC_2013_x86_Ox.asm}

Вот мы это и видим: обе константы встроены в код.

Память для них не выделяется.
Функция \TT{my\_srand()} просто копирует входное значение во внутреннюю переменную \TT{rand\_state}.

\TT{my\_rand()} берет её, вычисляет следующее состояние \TT{rand\_state}, 
обрезает его и оставляет в регистре EAX.

Неоптимизированная версия побольше:

\lstinputlisting[caption=\NonOptimizing MSVC 2013,style=customasmx86]{patterns/145_LCG/rand_MSVC_2013_x86.asm}

\subsection{x64}

Версия для x64 почти такая же, и использует 32-битные регистры вместо 64-битных
(потому что мы работаем здесь с переменными типа \Tint).

Но функция \TT{my\_srand()} берет входной аргумент из регистра \ECX, а не из стека:

\lstinputlisting[caption=\Optimizing MSVC 2013 x64,style=customasmx86]{patterns/145_LCG/rand_MSVC_2013_x64_Ox_RU.asm}

GCC делает почти такой же код.

\subsection{32-bit ARM}

\lstinputlisting[caption=\OptimizingKeilVI (\ARMMode),style=customasmARM]{patterns/145_LCG/rand.s_Keil_ARM_O3_RU.s}

В ARM инструкцию невозможно встроить 32-битную константу, так что Keil-у приходится размещать их отдельно и дополнительно загружать.
Вот еще что интересно: константу 0x7FFF также нельзя встроить.
Поэтому Keil сдвигает \TT{rand\_state} влево на 17 бит и затем сдвигает вправо на 17 бит.
Это аналогично \CCpp{}-выражению $(rand\_state \ll 17) \gg 17$.
Выглядит как бессмысленная операция, но тем не менее, что она делает это очищает старшие 17 бит, оставляя младшие 15 бит нетронутыми, и это наша цель, в конце концов. \\
\\
\Optimizing Keil для режима Thumb делает почти такой же код.

\subsubsection{MIPS}

\lstinputlisting[caption=\Optimizing GCC 4.4.5 (IDA),style=customasmMIPS]{patterns/08_switch/1_few/MIPS_O3_IDA_RU.lst}

\myindex{MIPS!\Instructions!JR}

Функция всегда заканчивается вызовом \puts, так что здесь мы видим переход на \puts (\INS{JR}: \q{Jump Register})
вместо перехода с сохранением \ac{RA} (\q{jump and link}).

Мы говорили об этом ранее: \myref{JMP_instead_of_RET}.

\myindex{MIPS!Load delay slot}
Мы также часто видим NOP-инструкции после \INS{LW}.
Это \q{load delay slot}: ещё один \emph{delay slot} в MIPS.
\myindex{MIPS!\Instructions!LW}
Инструкция после \INS{LW} может исполняться в тот момент, когда \INS{LW} загружает значение из памяти.

Впрочем, следующая инструкция не должна использовать результат \INS{LW}.

Современные MIPS-процессоры ждут, если следующая инструкция использует результат \INS{LW}, так что всё это уже
устарело, но GCC всё еще добавляет NOP-ы для более старых процессоров.

Вообще, это можно игнорировать.



\subsection{Версия этого примера для многопоточной среды}

Версия примера для многопоточной среды будет рассмотрена позже: \myref{LCG_TLS}.

