\mysection[Générateur congruentiel linéaire]{Générateur congruentiel linéaire comme générateur de nombres pseudo-aléatoires}
\myindex{\CStandardLibrary!rand()}
\label{LCG_simple}

Peut-être que le générateur congruentiel linéaire est le moyen le plus simple possible
de générer des nombres aléatoires.

Ce n'est plus très utilisé aujourd'hui\footnote{Le twister de Mersenne est meilleur.},
mais il est si simple (juste une multiplication, une addition et une opération AND)
que nous pouvons l'utiliser comme un exemple.

\lstinputlisting[style=customc]{patterns/145_LCG/rand_FR.c}

Il y a deux fonctions: la première est utilisée pour initialiser l'état interne,
et la seconde est appelée pour générer un nombre pseudo-aléatoire.

Nous voyons que deux constantes sont utilisées dans l'algorithme.
Elles proviennent de
[William H. Press and Saul A. Teukolsky and William T. Vetterling and Brian P. Flannery, \emph{Numerical Recipes}, (2007)].

Définissons-les en utilisant la déclaration \CCpp \TT{\#define}. C'est une macro.

La différence entre une macro \CCpp et une constante est que toutes les macros sont
remplacées par leur valeur par le pré-processeur \CCpp, et qu'elles n'utilisent pas
de mémoire, contrairement aux variables.

Par contre, une constante est une variable en lecture seule.

Il est possible de prendre un pointeur (ou une adresse) d'une variable constante,
mais c'est impossible de faire ça avec une macro.

La dernière opération AND est nécessaire car d'après le standard C \TT{my\_rand()}
doit renvoyer une valeur dans l'intervalle 0..32767.

Si vous voulez obtenir des valeurs pseudo-aléatoires 32-bit, il suffit d'omettre
la dernière opération AND.

\subsection{x86}

\lstinputlisting[caption=MSVC 2013 \Optimizing,style=customasmx86]{patterns/145_LCG/rand_MSVC_2013_x86_Ox.asm}

Nous les voyons ici: les deux constantes sont intégrées dans le code.
Il n'y a pas de mémoire allouée pour elles.

La fonction \TT{my\_srand()} copie juste sa valeur en entrée dans la variable \TT{rand\_state}
interne.

\TT{my\_rand()} la prend, calcule le \TT{rand\_state} suivant, le coupe et le laisse
dans le registre \EAX.

La version non optimisée est plus verbeuse:

\lstinputlisting[caption=MSVC 2013 \NonOptimizing,style=customasmx86]{patterns/145_LCG/rand_MSVC_2013_x86.asm}

\subsection{x64}

La version x64 est essentiellement la même et utilise des registres 32-bit au lieu
de 64-bit (car nous travaillons avec des valeurs de type \Tint ici).

Mais \TT{my\_srand()} prend son argument en entrée dans le registre \ECX plutôt que
sur la pile:

\lstinputlisting[caption=MSVC 2013 x64 \Optimizing,style=customasmx86]{patterns/145_LCG/rand_MSVC_2013_x64_Ox_FR.asm}

Le compilateur GCC génère en grande partie le même code.

\subsection{ARM 32-bit}

\lstinputlisting[caption=\OptimizingKeilVI (\ARMMode),style=customasmARM]{patterns/145_LCG/rand.s_Keil_ARM_O3_FR.s}

Il n'est pas possible d'intégrer une constante 32-bit dans des instructions ARM,
donc Keil doit les stocker à l'extérieur et en outre les charger.
Une chose intéressante est qu'il n'est pas possible non plus d'intégrer la constante
0x7FFF.
Donc ce que fait Keil est de décaler \TT{rand\_state} vers la gauche de 17 bits et
ensuite la décale de 17 bits vers la droite.
Ceci est analogue à la déclaration $(rand\_state \ll 17) \gg 17$ en \CCpp.
Il semble que ça soit une opération inutile, mais ce qu'elle fait est de mettre à
zéro les 17 bits hauts, laissant les 15 bits bas inchangés, et c'est notre but après
tout.\\
\\
Keil \Optimizing pour le mode Thumb génère essentiellement le même code.

\subsubsection{MIPS}

\lstinputlisting[caption=\Optimizing GCC 4.4.5 (IDA),style=customasmMIPS]{patterns/10_strings/1_strlen/MIPS_O3_IDA_FR.lst}

\myindex{MIPS!\Instructions!NOR}
\myindex{MIPS!\Pseudoinstructions!NOT}

Il manque en MIPS une instruction \NOT, mais il y a \NOR qui correspond à l'opération
\TT{OR~+~NOT}.

Cette opération est largement utilisée en électronique digitale\footnote{NOR est
appelé \q{porte universelle}}.
Par exemple, l'Apollo Guidance Computer (ordinateur de guidage Apollo) utilisé dans
le programme Apollo, a été construit en utilisant seulement 5600 portes NOR:
[Jens Eickhoff, \emph{Onboard Computers, Onboard Software and Satellite Operations: An Introduction}, (2011)].
Mais l'élément NOT n'est pas très populaire en programmation informatique.

Donc, l'opération NOT est implémentée ici avec \TT{NOR~DST,~\$ZERO,~SRC}.

D'après le chapitre sur les fondamentaux \myref{sec:signednumbers} nous savons qu'une
inversion des bits d'un nombre signé est la même chose que changer son signe et soustraire
1 du résultat.

Donc ce que \NOT fait ici est de prendre la valeur de $str$ et de la transformer
en $-str-1$.
L'opération d'addition qui suit prépare le résultat.


\subsection{Version thread-safe de l'exemple}

La version thread-safe de l'exemple sera montrée plus tard: \myref{LCG_TLS}.
