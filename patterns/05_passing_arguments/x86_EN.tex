\subsection{x86}

\subsubsection{MSVC}

Here is what we get after compilation (MSVC 2010 Express):

\lstinputlisting[label=src:passing_arguments_ex_MSVC_cdecl,caption=MSVC 2010 Express,style=customasmx86]{patterns/05_passing_arguments/msvc_EN.asm}

\myindex{x86!\Registers!EBP}

What we see is that the \main function pushes 3 numbers onto the stack and calls \TT{f(int,int,int).} 

Argument access inside \ttf is organized with the help of macros like:\\
\TT{\_a\$ = 8}, 
in the same way as local variables, but with positive offsets (addressed with \emph{plus}).
So, we are addressing the \emph{outer} side of the \gls{stack frame} by adding the \TT{\_a\$} macro to the value in the \EBP register.

\myindex{x86!\Instructions!IMUL}
\myindex{x86!\Instructions!ADD}

Then the value of $a$ is stored into \EAX. After \IMUL instruction execution, the value in \EAX is 
a \gls{product} of the value in \EAX and the content of \TT{\_b}.

After that, \ADD adds the value in \TT{\_c} to \EAX.

The value in \EAX does not need to be moved: it is already where it must be.
On returning to \gls{caller}, it takes the \EAX value and uses it as an argument to \printf.

\clearpage
\mysubparagraph{\olly}
\myindex{\olly}

Let's try this example in \olly.
The input value of the function (2) is loaded into \EAX: 

\begin{figure}[H]
\centering
\myincludegraphics{patterns/08_switch/2_lot/olly1.png}
\caption{\olly: function's input value is loaded in \EAX}
\label{fig:switch_lot_olly1}
\end{figure}

\clearpage
The input value is checked, is it bigger than 4? 
If not, the \q{default} jump is not taken:
\begin{figure}[H]
\centering
\myincludegraphics{patterns/08_switch/2_lot/olly2.png}
\caption{\olly: 2 is no bigger than 4: no jump is taken}
\label{fig:switch_lot_olly2}
\end{figure}

\clearpage
Here we see a jumptable:

\begin{figure}[H]
\centering
\myincludegraphics{patterns/08_switch/2_lot/olly3.png}
\caption{\olly: calculating destination address using jumptable}
\label{fig:switch_lot_olly3}
\end{figure}

Here we've clicked \q{Follow in Dump} $\rightarrow$ \q{Address constant}, so now we see the \emph{jumptable} in the data window.
These are 5 32-bit values\footnote{They are underlined by \olly because
these are also FIXUPs: \myref{subsec:relocs}, we are going to come back to them later}.
\ECX is now 2, so the third element (can be indexed as 2\footnote{About indexing, see also: \ref{arrays_at_one}}) of the table is to be used.
It's also possible to click \q{Follow in Dump} $\rightarrow$ 
\q{Memory address} and \olly will show the element addressed by the \JMP instruction. 
That's \TT{0x010B103A}.

\clearpage
After the jump we are at \TT{0x010B103A}: the code printing \q{two} will now be executed:

\begin{figure}[H]
\centering
\myincludegraphics{patterns/08_switch/2_lot/olly4.png}
\caption{\olly: now we at the \emph{case:} label}
\label{fig:switch_lot_olly4}
\end{figure}


\subsubsection{GCC}

Let's compile the same in GCC 4.4.1 and see the results in \IDA:

\lstinputlisting[caption=GCC 4.4.1,style=customasmx86]{patterns/05_passing_arguments/gcc_EN.asm}

The result is almost the same with some minor differences discussed earlier.

The \gls{stack pointer} is not set back after the two function calls(f and printf), 
because the penultimate \TT{LEAVE} (\myref{x86_ins:LEAVE}) 
instruction takes care of this at the end.
