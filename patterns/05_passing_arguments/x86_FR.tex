\subsection{x86}

\subsubsection{MSVC}

Voici ce que l'on obtient après compilation (MSVC 2010 Express) :

\lstinputlisting[label=src:passing_arguments_ex_MSVC_cdecl,caption=MSVC 2010 Express,style=customasmx86]{patterns/05_passing_arguments/msvc_FR.asm}

\myindex{x86!\Registers!EBP}

Ce que l'on voit, c'est que la fonction \main pousse 3 nombres sur la pile et appelle
\TT{f(int,int,int)}.

L'accès aux arguments à l'intérieur de \ttf est organisé à l'aide de macros
comme:\\
\TT{\_a\$ = 8},
de la même façon que pour les variables locales, mais avec des offsets positifs
(accédés avec \emph{plus}).
Donc, nous accédons à la partie \emph{hors} de la \glslink{stack frame}{structure locale de pile}
en ajoutant la macro \TT{\_a\$} à la valeur du registre \EBP.

\myindex{x86!\Instructions!IMUL}
\myindex{x86!\Instructions!ADD}

Ensuite, la valeur de $a$ est stockée dans \EAX. Après l'exécution de l'instruction
\IMUL, la valeur de \EAX est le \glslink{product}{produit} de la valeur de \EAX
et du contenu de \TT{\_b}.

Après cela, \ADD ajoute la valeur dans \TT{\_c} à \EAX.

La valeur dans \EAX n'a pas besoin d'être déplacée/copiée : elle est déjà là
où elle doit être.
Lors du retour dans la fonction \glslink{caller}{appelante}, elle prend la valeur dans
\EAX et l'utilise comme argument pour \printf.

\clearpage
\subsubsection{MSVC: x86 + \olly}
\myindex{\olly}

Les choses sont encore plus simple ici:

\begin{figure}[H]
\centering
\myincludegraphics{patterns/04_scanf/2_global/ex2_olly_1.png}
\caption{\olly: après l'exécution de \scanf}
\label{fig:scanf_ex2_olly_1}
\end{figure}

La variable se trouve dans le segment de données.
Après que l'instruction \PUSH (pousser l'adresse de $x$) ait été exécutée,
l'adresse apparaît dans la fenêtre de la pile. Cliquer droit sur cette ligne
et choisir \q{Follow in dump}. % TODO olly French ?
La variable va apparaître dans la fenêtre de la mémoire sur la gauche.
Après que nous ayons entré 123 dans la console, \TT{0x7B} apparaît dans la fenêtre
de la mémoire (voir les régions surlignées dans la copie d'écran).

Mais pourquoi est-ce que le premier octet est \TT{7B}?
Logiquement, Il devrait y avoir \TT{00 00 00 7B} ici.
La cause de ceci est référé comme \gls{endianness}, et x86 utilise \emph{little-endian}.
Cela implique que l'octet le plus faible poids est écrit en premier, et le plus fort
en dernier.
Voir à ce propos: \myref{sec:endianness}.
Revenons à l'exemple, la valeur 32-bit est chargée depuis son adresse mémoire
dans \EAX et passée à \printf.

L'adresse mémoire de $x$ est \TT{0x00C53394}.

\clearpage
Dans \olly nous pouvons examiner l'espace mémoire du processus  (Alt-M) et nous
pouvons voir que cette adresse se trouve dans le segment PE \TT{.data} de notre
programme:

\label{olly_memory_map_example}
\begin{figure}[H]
\centering
\myincludegraphics{patterns/04_scanf/2_global/ex2_olly_2.png}
\caption{\olly: espace mémoire du processus}
\label{fig:scanf_ex2_olly_2}
\end{figure}



\subsubsection{GCC}

Compilons le même code avec GCC 4.4.1 et regardons le résultat dans \IDA :

\lstinputlisting[caption=GCC 4.4.1,style=customasmx86]{patterns/05_passing_arguments/gcc_FR.asm}

Le résultat est presque le même, avec quelques différences mineures discutées
précédemment.

Le \glslink{stack pointer}{pointeur de pile} n'est pas remis après les deux appels
de fonction (f et printf), car la pénultième instruction \TT{LEAVE} (\myref{x86_ins:LEAVE})
s'en occupe à la fin.
