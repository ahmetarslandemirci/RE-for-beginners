\subsubsection{x86}

Компилируется это все в:

\lstinputlisting[caption=MSVC 2012 /GS- /Ob0,label=src:struct_packing_4,numbers=left,style=customasmx86]{patterns/15_structs/4_packing/packing_RU.asm}

Кстати, мы передаем всю структуру, но в реальности, как видно, структура в начале копируется
во временную структуру (выделение места под нее в стеке происходит в строке 10,
а все 4 поля, по одному, копируются в строках 12 \ldots\ 19), 
затем передается только указатель на нее (или адрес).

Структура копируется, потому что неизвестно, будет ли функция \ttf модифицировать структуру или нет.
И если да, то структура внутри \main должна остаться той же.

Мы могли бы использовать указатели на \CCpp, и итоговый код был бы почти такой же,
только копирования не было бы.

Мы видим здесь что адрес каждого поля в структуре выравнивается по 4-байтной границе. 
Так что каждый \Tchar здесь занимает те же 4 байта что и \Tint. Зачем? 
Затем что процессору удобнее обращаться по таким адресам и кэшировать данные из памяти.

Но это не экономично по размеру данных.

Попробуем скомпилировать тот же исходник с опцией (\TT{/Zp1}) 
(\emph{/Zp[n] pack structures on n-byte boundary}).

\lstinputlisting[caption=MSVC 2012 /GS- /Zp1,label=src:struct_packing_1,numbers=left,style=customasmx86]{patterns/15_structs/4_packing/packing_msvc_Zp1_RU.asm}

Теперь структура занимает 10 байт и все \Tchar занимают по байту. Что это дает? 
Экономию места. Недостаток ~--- процессор будет обращаться к этим полям не так эффективно 
по скорости, как мог бы.

\label{short_struct_copying_using_MOV}
Структура так же копируется в \main. Но не по одному полю, а 10 байт, при помощи трех
пар \MOV.

Почему не 4?
Компилятор рассудил, что будет лучше скопировать 10 байт
при помощи 3 пар \MOV, чем копировать два 32-битных слова и два байта при помощи 4 пар \MOV.

Кстати, подобная реализация копирования при помощи \MOV взамен вызова функции \TT{memcpy()}, например, это
очень распространенная практика, потому что это в любом случае работает быстрее чем вызов \TT{memcpy()} ---
если речь идет о коротких блоках, конечно: \myref{copying_short_blocks}.

Как нетрудно догадаться, если структура используется много в каких исходниках и объектных файлах, 
все они должны быть откомпилированы с одним и тем же соглашением об упаковке структур.

\newcommand{\FNURLMSDNZP}{\footnote{\href{http://go.yurichev.com/17067}
{MSDN: Working with Packing Structures}}}
\newcommand{\FNURLGCCPC}{\footnote{\href{http://go.yurichev.com/17068}
{Structure-Packing Pragmas}}}

Помимо ключа MSVC \TT{/Zp}, указывающего, по какой границе упаковывать поля структур, есть также 
опция компилятора \TT{\#pragma pack}, её можно указывать прямо в исходнике. 
Это справедливо и для MSVC\FNURLMSDNZP и GCC\FNURLGCCPC{}.

Давайте теперь вернемся к \TT{SYSTEMTIME}, которая состоит из 16-битных полей. 
Откуда наш компилятор знает что их надо паковать по однобайтной границе?

В файле \TT{WinNT.h} попадается такое:

\begin{lstlisting}[caption=WinNT.h,style=customc]
#include "pshpack1.h"
\end{lstlisting}

И такое:

\begin{lstlisting}[caption=WinNT.h,style=customc]
#include "pshpack4.h"                   // 4 byte packing is the default
\end{lstlisting}

Сам файл PshPack1.h выглядит так:

\lstinputlisting[caption=PshPack1.h,style=customc]{patterns/15_structs/4_packing/tmp1.c}

Собственно, так и задается компилятору, как паковать объявленные после \TT{\#pragma pack} структуры.

\clearpage
\myparagraph{MSVC + \olly}
\myindex{\olly}

2 пары 32-битных слов обведены в стеке красным.
Каждая пара --- это числа двойной точности в формате IEEE 754, переданные из \main.

Видно, как первая \FLD загружает значение 1,2 из стека и помещает в регистр \ST{0}:

\begin{figure}[H]
\centering
\myincludegraphics{patterns/12_FPU/1_simple/olly1.png}
\caption{\olly: первая \FLD исполнилась}
\label{fig:FPU_simple_olly_1}
\end{figure}

Из-за неизбежных ошибок конвертирования числа из 64-битного IEEE 754 в 80-битное (внутреннее в FPU),
мы видим здесь 1,1999\ldots, что очень близко к 1,2.

Прямо сейчас \EIP указывает на следующую инструкцию (\FDIV), загружающую константу двойной точности 
из памяти.

Для удобства, \olly показывает её значение: 3,14.

\clearpage
Трассируем дальше. 
\FDIV исполнилась, теперь \ST{0} содержит 0,382\ldots
(\gls{quotient}):

\begin{figure}[H]
\centering
\myincludegraphics{patterns/12_FPU/1_simple/olly2.png}
\caption{\olly: \FDIV исполнилась}
\label{fig:FPU_simple_olly_2}
\end{figure}

\clearpage
Третий шаг: вторая \FLD 
исполнилась, загрузив в \ST{0} 3,4 (мы видим приближенное число 3,39999\ldots): 

\begin{figure}[H]
\centering
\myincludegraphics{patterns/12_FPU/1_simple/olly3.png}
\caption{\olly: вторая \FLD исполнилась}
\label{fig:FPU_simple_olly_3}
\end{figure}

В это время \gls{quotient} \emph{провалилось} 
в \ST{1}.
\EIP указывает на следующую инструкцию: \FMUL. 
Она загружает константу 4,1 из памяти, так что \olly тоже показывает её здесь.

\clearpage
Затем: \FMUL исполнилась, теперь в \ST{0} произведение:

\begin{figure}[H]
\centering
\myincludegraphics{patterns/12_FPU/1_simple/olly4.png}
\caption{\olly: \FMUL исполнилась}
\label{fig:FPU_simple_olly_4}
\end{figure}

\clearpage
Затем: \FADDP исполнилась, теперь в \ST{0} сумма, а \ST{1} очистился:

\begin{figure}[H]
\centering
\myincludegraphics{patterns/12_FPU/1_simple/olly5.png}
\caption{\olly: \FADDP исполнилась}
\label{fig:FPU_simple_olly_5}
\end{figure}

Сумма остается в \ST{0} потому что функция возвращает результат своей работы через \ST{0}.

Позже \main возьмет это значение оттуда.

Мы также видим кое-что необычное: значение 13,93\ldots теперь находится в \ST{7}.

Почему?

\label{FPU_is_rather_circular_buffer}
Мы читали в этой книге, что регистры в \ac{FPU} представляют собой стек: \myref{FPU_is_stack}. 
Но это упрощение.
Представьте, если бы \emph{в железе} было бы так, как описано. Тогда при каждом заталкивании (или выталкивании) в стек,
все остальные 7 значений нужно было бы передвигать (или копировать) в соседние регистры, 
а это слишком затратно.

Так что в реальности у
\ac{FPU} есть просто 8 регистров и указатель (называемый \GTT{TOP}), содержащий номер регистра,
который в текущий момент является \q{вершиной стека}.

При заталкивании значения в стек регистр \GTT{TOP} меняется, и указывает на свободный регистр. 
Затем значение записывается в этот регистр.

При выталкивании значения из стека процедура обратная. Однако освобожденный регистр не обнуляется
(наверное, можно было бы сделать, чтобы обнулялся, но это лишняя работа и работало бы медленнее).
Так что это мы здесь и видим. 
Можно сказать, что \FADDP сохранила сумму, а затем вытолкнула один элемент.

Но в реальности, эта инструкция сохранила сумму и затем передвинула регистр \GTT{TOP}.

Было бы ещё точнее сказать, что регистры \ac{FPU} представляют собой кольцевой буфер.


