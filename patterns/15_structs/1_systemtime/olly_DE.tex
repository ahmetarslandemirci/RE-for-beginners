\clearpage
\subsubsection{\olly}
\myindex{\olly}
Kompilieren wir dieses Beispiel in MSVC 2010 mit \TT{/GS- /MD} und laden es in \olly.

Öffnen wir die Fenster für Daten und Stack an der Adresse, die als erstes Argument der Funktion \TT{GetSystemTime()}
übergeben wird und warten, bis das Programm an dieser Stelle ist. Wir sehen das folgende:

\begin{figure}[H]
\centering
\myincludegraphics{patterns/15_structs/1_systemtime/olly_systemtime1.png}
\caption{\olly: \TT{GetSystemTime()} wurde gerade ausgeführt}
\label{fig:struct_olly_1}
\end{figure}
Die Systemzeit, die diese Ausführung der Funktion auf meinem Computer liefert, ist 9. Dezember 2014, 22:29:52:

\lstinputlisting[caption=\printf output]{patterns/15_structs/1_systemtime/console.txt}
Wir sehen also diese 16 Byte im Datenfenster:
 
\begin{lstlisting}
DE 07 0C 00 02 00 09 00 16 00 1D 00 34 00 D4 03
\end{lstlisting}
Je zwei Byte repräsentieren ein Feld des structs. 
Da die \glslink{endianness}{Endianess} hier \emph{little Endian} ist, finden wir das niederwertige Byte zuerst und danach das
höherwertige.

Es werden also die folgenden Werte aktuell im Speicher gehalten:

\begin{center}
\begin{tabular}{ | l | l | l | }
\hline
\headercolor{} Hexadezimalzahl & 
\headercolor{} Dezimalzahl & 
\headercolor{} Feldname \\
\hline
0x07DE & 2014	& wYear \\
\hline
0x000C & 12	& wMonth \\
\hline
0x0002 & 2	& wDayOfWeek \\
\hline
0x0009 & 9	& wDay \\
\hline
0x0016 & 22	& wHour \\
\hline
0x001D & 29	& wMinute \\
\hline
0x0034 & 52	& wSecond \\
\hline	
0x03D4 & 980	& wMilliseconds \\
\hline
\end{tabular}
\end{center}
Wir finden die gleichen Werte im Stackfenster, aber sie werden als 32-Bit-Werte gruppiert.

Die Funktion \printf nimmt sich die Werte, die sie braucht und schreibt sie in die Konsole.

Manche Werte werden von \printf nicht ausgegeben (\TT{wDayOfWeek} und \TT{wMilliseconds}), aber sie sind im Speicher
jederzeit für uns verfügbar.
