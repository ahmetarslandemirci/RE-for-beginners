\subsubsection{Structure comme un ensemble de valeurs}

Afin d'illustrer le fait qu'une structure n'est qu'une collection de variables située côte-à-côte, 
retravaillons notre exemple sur la base de la définition de la structure \emph{tm}: \lstref{struct_tm}.

\lstinputlisting[style=customc]{patterns/15_structs/3_tm_linux/as_array/GCC_tm2.c}

\myindex{\CStandardLibrary!localtime\_r()}
N.B. Le pointeur vers le champ \TT{tm\_sec} est passé comme argument de la fonction 
\TT{localtime\_r}, en tant que premier élément de la \q{structure}.

Le compilateur nous alerte:

\begin{lstlisting}[caption=GCC 4.7.3]
GCC_tm2.c: In function 'main':
GCC_tm2.c:11:5: warning: passing argument 2 of 'localtime_r' from incompatible pointer type [enabled by default]
In file included from GCC_tm2.c:2:0:
/usr/include/time.h:59:12: note: expected 'struct tm *' but argument is of type 'int *'
\end{lstlisting}

Mais il génère cependant un fragment exécutable correspondant au code assembleur suivant:

\lstinputlisting[caption=GCC 4.7.3,style=customasmx86]{patterns/15_structs/3_tm_linux/as_array/GCC_tm2.asm}

Ce code est similaire à ce que nous avons déjà vue et il n'est pas possible de dire si le code 
source original contenait une structure ou un groupe de variables.

Et cela fonctionne. Mais encore une fois ce n'est pas une bonne pratique. 

En règle générale les compilateurs en l'absence d'optimisation allouent les variables sur la pile 
dans le même ordre que celui dans lequel elles ont été déclarées dans le code source. Pour autant, 
ce n'est pas une garantie.

Par ailleurs certains compilateurs peuvent vous avertir que les variables \TT{tm\_year}, 
\TT{tm\_mon}, \TT{tm\_mday}, \TT{tm\_hour}, \TT{tm\_min} n'ont pas été initialisées avant leur 
utilisation, mais resteront muets au sujet de \TT{tm\_sec}

Le compilateur lui non plus ne sait pas qu'ils sont appelés à être initialisés par la fonction 
\TT{localtime\_r()}.

Nous avons chois cet exemple car tous les champs de la structure sont de type \Tint.%

Tout ceci ne fonctionnerait pas sir les champs de la structure étaient des \TT{WORD} de 16 bits, tel 
que dans le cas de la structure \TT{SYSTEMTIME} structure---\TT{GetSystemTime()} les initialiserait 
de manière erronée (puisque les variables locales sont alignées sur des frontières de 32bits).
Vous en saurez plus à ce sujet dans la prochaine section: 
\q{\StructurePackingSectionName} (\myref{structure_packing}).

Une structure n'est donc qu'un groupe de variables disposées côte-à-côte en mémoire. Nous pouvons 
dire que la structure est une instruction adressée au compilateur et l'obligeant à conserver le 
groupement des variables.
Cela étant dans les toutes premières versions du langage C (avant 1972), la notion de structure 
n'existait pas encore \RitchieDevC.

Pas d'exemple de débogage ici. Le comportement est toujours le même.

\subsubsection{Une structure sous forme de table de 32 bits}

\lstinputlisting[style=customc]{patterns/15_structs/3_tm_linux/as_array/GCC_tm3.c}

Nous n'avons qu'à utiliser l'opérateur \emph{cast} pour transformer notre pointeur vers une structure 
en un tableau de \Tint{}'s. Et cela fonctionne !
Nous avons exécuté l'exemple à 23h51m45s le 26 juillet 2014.

\begin{lstlisting}[label=GCC_tm3_output]
0x0000002D (45)
0x00000033 (51)
0x00000017 (23)
0x0000001A (26)
0x00000006 (6)
0x00000072 (114)
0x00000006 (6)
0x000000CE (206)
0x00000001 (1)
\end{lstlisting}

Les variables sont dans le même ordre que celui dans lequel elles apparaissent dans la définition 
de la structure: \myref{struct_tm}.

Nous avons effectué la compilation avec:

\lstinputlisting[caption=\Optimizing GCC 4.8.1,style=customasmx86]{patterns/15_structs/3_tm_linux/as_array/GCC_tm3_FR.lst}

En fait, l'espace dans la pile est tout d'abord traité comme une structure, puis ensuite comme un 
tableau.

Le pointeur sur le tableau permet même de modifier les champs de la structure.

Et encore une fois cette manière de procéder est extrêmement douteuse et pas du tout recommandée 
pour l'écriture d'un code qui atterrira en production.

\mysubparagraph{\Exercise}

Tentez de modifier (en l'augmentant de 1) le numéro du mois, en traitant la structure comme s'il 
s'agissait d'un tableau.

\subsubsection{Une structure sous forme d'un tableau d'octets}

Nous pouvons aller plus loin.
Utilisons l'opérateur \emph{cast} pour transformer le pointeur en un tableau d'octets, puis affichons 
son contenu:

\lstinputlisting[style=customc]{patterns/15_structs/3_tm_linux/as_array/GCC_tm4.c}

\begin{lstlisting}
0x2D 0x00 0x00 0x00 
0x33 0x00 0x00 0x00 
0x17 0x00 0x00 0x00 
0x1A 0x00 0x00 0x00 
0x06 0x00 0x00 0x00 
0x72 0x00 0x00 0x00 
0x06 0x00 0x00 0x00 
0xCE 0x00 0x00 0x00 
0x01 0x00 0x00 0x00 
\end{lstlisting}

Cet exemple a été exécuté à 23h51m45s le 26 juillet 2014
\footnote{Les dates et heures sont les mêmes dans tous les exemples. Elles ont été éditées 
pour la clarté de la démonstration.}.
Les valeurs sont identiques à celles du précédent affichage (\myref{GCC_tm3_output}), et bien 
entendu l'octet de poids faible figure en premier puisque nous sommes sur une architecture de type 
little-endian (\myref{sec:endianness}).

\lstinputlisting[caption=\Optimizing GCC 4.8.1,style=customasmx86]{patterns/15_structs/3_tm_linux/as_array/GCC_tm4_FR.lst}
