\subsubsection{Linux}

Let's take the \TT{tm} structure from \TT{time.h} in Linux for example:

\lstinputlisting[style=customc]{patterns/15_structs/3_tm_linux/GCC_tm.c}

Let's compile it in GCC 4.4.1:

\lstinputlisting[caption=GCC 4.4.1,style=customasmx86]{patterns/15_structs/3_tm_linux/GCC_tm_EN.asm}

Somehow, \IDA did not write the local variables' names in the local stack.
But since we already are experienced reverse engineers :-) we may do it without this information in 
this simple example.

\myindex{x86!\Instructions!LEA}

Please also pay attention to the \TT{lea edx, [eax+76Ch]}~---this instruction just adds \TT{0x76C} (1900) to value in \EAX,
but doesn't modify any flags. See also the relevant section about \LEA{}~(\myref{sec:LEA}).

\myparagraph{GDB}

Let's try to load the example into GDB
\footnote{The \emph{date} result is slightly corrected for demonstration purposes.
Of course, it's not possible to run GDB that quickly, in the same second.}:

\lstinputlisting[caption=GDB]{patterns/15_structs/3_tm_linux/GCC_tm_GDB.txt}

We can easily find our structure in the stack.
First, let's see how it's defined in \emph{time.h}:

\begin{lstlisting}[caption=time.h, label=struct_tm,style=customc]
struct tm
{
  int	tm_sec;
  int	tm_min;
  int	tm_hour;
  int	tm_mday;
  int	tm_mon;
  int	tm_year;
  int	tm_wday;
  int	tm_yday;
  int	tm_isdst;
};
\end{lstlisting}

Pay attention that
32-bit \Tint is used here instead of WORD in SYSTEMTIME.
So, each field occupies 32-bit.

Here are the fields of our structure in the stack:

\begin{lstlisting}
0xbffff0dc:	0x080484c3	0x080485c0	0x000007de	0x00000000
0xbffff0ec:	0x08048301	0x538c93ed	0x00000025 sec	0x0000000a min
0xbffff0fc:	0x00000012 hour	0x00000002 mday	0x00000005 mon 	0x00000072 year
0xbffff10c:	0x00000001 wday	0x00000098 yday	0x00000001 isdst0x00002a30
0xbffff11c:	0x0804b090	0x08048530	0x00000000	0x00000000
\end{lstlisting}

Or as a table:

\begin{center}
\begin{tabular}{ | l | l | l | }
\hline
\headercolor{} Hexadecimal number & 
\headercolor{} decimal number & 
\headercolor{} field name \\
\hline
0x00000025 & 37 	& tm\_sec \\
\hline
0x0000000a & 10 	& tm\_min \\
\hline
0x00000012 & 18 	& tm\_hour \\	
\hline
0x00000002 & 2 		& tm\_mday \\	
\hline
0x00000005 & 5 		& tm\_mon \\	
\hline
0x00000072 & 114 	& tm\_year \\
\hline
0x00000001 & 1 		& tm\_wday \\	
\hline
0x00000098 & 152 	& tm\_yday \\	
\hline
0x00000001 & 1 		& tm\_isdst \\
\hline
\end{tabular}
\end{center}

Just like SYSTEMTIME (\myref{sec:SYSTEMTIME}), 

there are also other fields available that are not used, like tm\_wday, tm\_yday, tm\_isdst.
