\mysection{\Stack}
\label{sec:stack}
\myindex{\Stack}

Stos w informatyce jest jedną z najbardziej fundamentalnych struktur danych
\footnote{\href{http://go.yurichev.com/17119}{wikipedia.org/wiki/Call\_stack}}.
\ac{AKA} \ac{LIFO}.

Technicznie rzecz biorąc, jest to tylko blok pamięci w pamięci procesora + rejestr \ESP w x86 lub \RSP w x64, lub \ac{SP} w ARM, który wskazuje obszar gdzieś w granicach tego bloku.

\myindex{ARM!\Instructions!PUSH}
\myindex{ARM!\Instructions!POP}
\myindex{x86!\Instructions!PUSH}
\myindex{x86!\Instructions!POP}
Najczęsciej wykorzystywanymi instrukcjami do operowania na stosie są \PUSH i \POP (w x86 i Thumb-trybie ARM). 
\PUSH zmniejsza \ESP/\RSP/\ac{SP} o 4 w trybie 32-bitowym (lub o 8 w 64-bitowym),
następnie zapisuje pod adres, na który wskazuje \ESP/\RSP/\ac{SP}, zawartość swojego operandu.

\POP jest odwrotną operacją- najpierw zdejmuje ze \glslink{stack pointer}{wskaźnika stosu} wartość i umieszcza ją do operandu 
(który często jest rejestrem), a następnie zwiększa wskaźnik o 4 (lub 8).

Przed alokacją pamięci na stosie \glslink{stack pointer}{rejestr-wskaźnik} wskazuje na koniec stosu.
Koniec stosu znajduje się na początku zaalokowanego bloku pamięci, przeznaczonego na stos. Może zabrzmieć to dziwnie, ale tak to działa.
\PUSH zmniejsza \glslink{stack pointer}{rejestr-wskaźnik}, а \POP~--- zwiększa.

W procesorze ARM jest wsparcie dla stosów zarówno rosnących w dół, jak i rosnącyh w górę.

\myindex{ARM!\Instructions!STMFD}
\myindex{ARM!\Instructions!LDMFD}
\myindex{ARM!\Instructions!STMED}
\myindex{ARM!\Instructions!LDMED}
\myindex{ARM!\Instructions!STMFA}
\myindex{ARM!\Instructions!LDMFA}
\myindex{ARM!\Instructions!STMEA}
\myindex{ARM!\Instructions!LDMEA}

Na przykład, instrukcje \ac{STMFD}/\ac{LDMFD}, \ac{STMED}/\ac{LDMED} są przeznaczone dla stosu malejącego (rośnie w dół, zaczynając od adresów wysokich, do adresów niskich).\\
Natomiast instrukcje \ac{STMFA}/\ac{LDMFA}, \ac{STMEA}/\ac{LDMEA} są przeznaczone dla stosu rosnącego (rośnie w górę, zaczynając od niskich adresów, kończąc na adresach wysokich).

% It might be worth mentioning that STMED and STMEA write first,
% and then move the pointer,
% and that LDMED and LDMEA move the pointer first, and then read.
% In other words, ARM not only lets the stack grow in a non-standard direction,
% but also in a non-standard order.
% Maybe this can be in the glossary, which would explain why E stands for "empty".

\subsection{Dlaczego stos rośnie wstecznie?}
\label{stack_grow_backwards}

Intuicyjnie moglibyśmy pomyśleć, że, jak i każda inna struktura danych, stos mogłby rosnąć w górę, tzn. w kierunku zwiększenia adresów.

Powód, dlaczego stos rośnie w dół, jest najprawdobodobniej historyczny.
Kiedy komputery były duże i zajmowały cały pokój, można było bardzo łatwo rozdzielić segment na dwa obszary: dla \glslink{heap}{kopca} i dla stosu.
Z góry nie było wiadomo, jak duża może być \glslink{heap}{sterta} lub stos, dlatego takie rozwiązanie było najbardziej logiczne.

\input{patterns/02_stack/stack_and_heap}

W \RitchieThompsonUNIX można przeczytać:

\begin{framed}
\begin{quotation}
The user-core part of an image is divided into three logical segments. The program text segment begins at location 0 in the virtual address space. During execution, this segment is write-protected and a single copy of it is shared among all processes executing the same program. At the first 8K byte boundary above the program text segment in the virtual address space begins a nonshared, writable data segment, the size of which may be extended by a system call. Starting at the highest address in the virtual address space is a stack segment, which automatically grows downward as the hardware's stack pointer fluctuates.
\end{quotation}
\end{framed}

To trochę przypomina podejście studenta,
który pisze dwie osobne lektury w jednym zeszycie:
pierwsza lektura jest pisana jak zwykle od początku zeszytu, a druga jest pisana od końca zeszytu.
Lektury mogą się "spotkać" gdzieś na środku zeszytu, jeśli zabraknie miejsca.

% I think if we want to expand on this analogy,
% one might remember that the line number increases as as you go down a page.
% So when you decrease the address when pushing to the stack, visually,
% the stack does grow upwards.
% Of course, the problem is that in most human languages,
% just as with computers,
% we write downwards, so this direction is what makes buffer overflows so messy.

\subsection{Do jakich celów służy stos?}

% subsections
\subsubsection{Zapisywanie adresu powrotu}

\myparagraph{x86}

\myindex{x86!\Instructions!CALL}
Przy wywołaniu funkcji przez \CALL najpierw na stos jest odkładany adres, wskazujący na miejsce po 
instrukcji \CALL, następnie robi się przejście bezwzględne (prawie jak \TT{JMP}) pod adres, zapisany w operandzie.

\myindex{x86!\Instructions!PUSH}
\myindex{x86!\Instructions!JMP}
\CALL~--- jest analogiczny do pary instrukcji \INS{PUSH address\_after\_call / JMP}.

\myindex{x86!\Instructions!RET}
\myindex{x86!\Instructions!POP}
\RET zdejmuje ze stosu wartość i przekazuje zarządzanie pod ten adres~--- 
jest to analogiczne do działania pary instrukcji \TT{POP tmp / JMP tmp}.

\myindex{\Stack!\MLStackOverflow}
\myindex{\Recursion}
Bardzo łatwo przepełnić stos, poprzez rekurencję:

\begin{lstlisting}[style=customc]
void f()
{
	f();
};
\end{lstlisting}

MSVC 2008 uprzedza:

\begin{lstlisting}
c:\tmp6>cl ss.cpp /Fass.asm
Microsoft (R) 32-bit C/C++ Optimizing Compiler Version 15.00.21022.08 for 80x86
Copyright (C) Microsoft Corporation.  All rights reserved.

ss.cpp
c:\tmp6\ss.cpp(4) : warning C4717: 'f' : recursive on all control paths, function will cause runtime stack overflow
\end{lstlisting}

\dots ale, niemniej jednak, tworzy potrzebny kod:

\lstinputlisting[style=customasmx86]{patterns/02_stack/1.asm}

\dots do tego, jeśli optymalizacja jest wyłączona (\TT{\Ox}), to będzie ciekawiej, bez przepełnienia stosu, 
ale będzie działało \emph{poprawnie}\footnote{ironia}:

\lstinputlisting[style=customasmx86]{patterns/02_stack/2.asm}

GCC 4.4.1 generuje taki sam kod w obu przypadkach, chociaż i nie wydaje stosownego komunikatu.

\myparagraph{ARM}

\myindex{ARM!\Registers!Link Register}
Programy dla ARM również korzystają ze stosu do zapisywania \ac{RA}, gdzie trzeba wrócić, ale trochę w inny sposób.
Jak już było wspomniane w sekcji \q{\HelloWorldSectionName}~(\myref{sec:hw_ARM}),
\ac{RA} jest zapisywany do rejestru \ac{LR} (\gls{link register}).
Ale jeśli wynika potrzeba wywołania jeszcze jakiejś funkcji i trzeba skorzystać z rejestru \ac{LR} jeszcze
raz, to jego zawartość najlepiej gdzieś zapisać.
\myindex{Function prologue}
\myindex{ARM!\Instructions!PUSH}
\myindex{ARM!\Instructions!POP}

Zwykle to się odbywa w prologu funkcji, często widzimy tam instrukcje w stylu \INS{PUSH \{R4-R7,LR\}}, a w epilogu
\INS{POP \{R4-R7,PC\}}~--- w ten sposób są zapisywane rejestry, z których będzie korzystała bieżąca funkcja, w tym rejestr \ac{LR}.

\myindex{ARM!Leaf function}
Niemniej jednak, jeśli jakaś funkcja nie wywołuje żadnych innych funkcji w trakcie swojej roboty, według terminologii \ac{RISC} jest ona nazywana
\emph{\gls{leaf function}}\footnote{\href{http://go.yurichev.com/17064}{infocenter.arm.com/help/index.jsp?topic=/com.arm.doc.faqs/ka13785.html}}. 
Wskutek tego, \q{leaf}-funkcja nie zapisuje rejestru \ac{LR} (dlatego że ona go nie zmienia).
A jeśli funkcja jest niewielkich rozmiarów, korzysta z małej ilości rejestrów, to może nie korzystać ze stosu w ogóle.
W ten sposób, w ARM możliwe jest wywoływanie małych leaf-funkcji nie korzystając ze stosu.
jest to szybsze niż w starych x86, dlatego że nie korzysta się z pamięci zewnętrznej do stosu
\footnote{Kiedyś, bardzo dawno temu, na PDP-11 i VAX na wykonanie instrukcjii CALL (wywołanie innych funkcji) mogło być zatracone
nawet 50\% czasu (przwdopodobnie przez pracę z pamięcią zewnętrzną),
dlatego było uważane, że dużo małych funkcji to \glslink{anti-pattern}
\InSqBrackets{\TAOUP Chapter 4, Part II}.}.
Również to może być też korzystne, kiedy pamięć pod stos jeszcze nie jest zaalokowana, lub jest niedostępna,

kilka przykładów takich funkcji:
\myref{ARM_leaf_example1}, \myref{ARM_leaf_example2}, 
\myref{ARM_leaf_example3}, \myref{ARM_leaf_example4}, \myref{ARM_leaf_example5},
\myref{ARM_leaf_example6}, \myref{ARM_leaf_example7}, \myref{ARM_leaf_example10}.



\subsubsection{Przekazywanie argumentów funkcji}

Najbardziej powszechny sposób na przekazywanie parametrów funkcji w x86 to \q{cdecl}:

\begin{lstlisting}[style=customasmx86]
push arg3
push arg2
push arg1
call f
add esp, 12 ; 4*3=12
\end{lstlisting}

Wywoływana funkcja otrzymuje swoje parametry również przez wskaźnik stosu.

W konsekwensji tego, zawartość stosu przed wykonaniem pierwszej instrukcji funkcji wygląda w ten sposób \ttf{}:

\begin{center}
\begin{tabular}{ | l | l | }
\hline
ESP & adres powrotu \\
\hline
ESP+4 & \argument \#1, \MarkedInIDAAs{} \TT{arg\_0} \\
\hline
ESP+8 & \argument \#2, \MarkedInIDAAs{} \TT{arg\_4} \\
\hline
ESP+0xC & \argument \#3, \MarkedInIDAAs{} \TT{arg\_8} \\
\hline
\dots & \dots \\
\hline
\end{tabular}
\end{center}

Patrz również w odpowiednim rodziale o innych sposobach przekazywania argumentów przez stos ~(\myref{sec:callingconventions}).

\par A propos, funkcja wywoływana nie posiada informacji o ilości argumentów przekazywanych do niej.
Funkcje w C o zmiennej ilości parametrów (jak np. \printf) wyznaczają ich ilość za pomocą specjalnych specyfikatorów (rozpoczynających się z \%).

Jeśli napisać coś w stylu:

\begin{lstlisting}
printf("%d %d %d", 1234);
\end{lstlisting}

\printf wyprowadzi 1234, następnie jeszcze dwie liczby losowe\footnote{Tak na prawdę nie są one losowe, patrz: \myref{noise_in_stack}}, który przypadkowo okazały się na stosie obok.

\par
Właśnie dlatego nie jest to ważne jak zapiszemy f-cję \main{}:\\
jak \main{}, \TT{main(int argc, char *argv[])}\\
lub \TT{main(int argc, char *argv[], char *envp[])}.

W rzeczywistości, \ac{CRT}-kod wywołuje \main mniej więcej w ten sposób:
	
\begin{lstlisting}[style=customasmx86]
push envp
push argv
push argc
call main
...
\end{lstlisting}

Jeśli zadeklarujecie \main bez argumentów, one, jednak, są obecne na stosie, lecz nie są wykorzystywane.
Jeśli zadeklarujecie \main jako \TT{main(int argc, char *argv[])}, 
to można korzystać z pierwszych dwóch argumentów, a trzeci zostanie dla funkcji \q{niewidocznym}.
Co więcej, można nawet zadeklarować \TT{main(int argc)}, i to zadziała.

\myparagraph{Alternatywne sposoby na przekazywanie argumentów}

Warto zauważyć,że, generalnie, nie ma narzutu na przekazywanie argumentów przez stos, nie jest to wymogiem formalnym.
Można robić to zupełnie inaczej, nie korzystając ze stosu w ogóle.

W pewnym sensie, popularną metodą wśród początkującyh jest przekazywanie argumentów przez zmienne globalne, na przykład:

\lstinputlisting[caption=Kod w asemblerze,style=customasmx86]{patterns/02_stack/global_args.asm}

Ale ta metoda posiada dużą wadę: funkcja \emph{do\_something()} nie może wywołać sama siebie poprzez rekurencję (lub za pomocą innej funkcji),
dlatego że wtedy będzie trzeba wymazać własne argumenty.
Ta sama historia ze zmiennymi lokalnymi: jeśli przechowywać je w zmiennych globalnych, funkcja nie będzie nogła wywołać sama siebie.
Do tego, ta metoda nie jest biezpieczna dla środowiska wielowątkowego\footnote{Przy poprawnej realizacji,
każdy wątek będzie miał własny stos lokalny ze swoimi argumentami/zmiennymi.}.
Metoda przechowywania podobnej informacji na stosie wszystko znacznie upraszcza ---
on może przechowywać tyle argumentów funkcji/zmiennych,
ile się w nim zmieści.

W \InSqBrackets{\TAOCPvolI{}, 189} można przeczytać o jeszcze bardziej dziwnych metodach przekazywania argumentów funkcji, które były bardzo wygodne na
 IBM System/360.

\myindex{MS-DOS}
\myindex{x86!\Instructions!INT}

W MS-DOS istniała metoda przekazywania argumentów przez rejestry, na przykład, ten fragment kodu dla starego 16-bitowego MS-DOS
wyprintuje ``Hello, world!'':

\begin{lstlisting}[style=customasmx86]
mov  dx, msg      ; adres powiadomienia
mov  ah, 9        ; §9 oznacza funkcję "wyprowadzenie linii"§
int  21h          ; DOS "syscall"

mov  ah, 4ch      ; §funkcja zakończenia programu
int  21h          ; DOS "syscall"

msg  db 'Hello, World!\$'
\end{lstlisting}

\myindex{fastcall}
Jest to bardzo podobne do metody \myref{fastcall}.
I jeszcze do sposobu robienia syscall w Linux (\myref{linux_syscall}) i Windows.

\myindex{x86!\Flags!CF}
Jeżeli f-cja w MS-DOS zwraca boolean (tzn jeden bit, zwule sygnalizujący o błędzie wewnętrznym),
to często była wykorzystywana flaga \TT{CF}.

Na przykład:

\begin{lstlisting}[style=customasmx86]
mov ah, 3ch       ; stworzyć plik
lea dx, filename
mov cl, 1
int 21h
jc  error
mov file_handle, ax
...
error:
...
\end{lstlisting}

W przypadku wystąpienia błędu, flaga \TT{CF} zostaje ustawiona.
W innym przypadku, handle stworzonego pliku jest zwracany do \TT{AX}.

Ta metoda dotychczasowo jest wykorzystywana przez programistów asemblera.
W kodach wyjściowych Windows Research Kernel (który jest bardzo podobny do Windows 2003) możemy znaleźć coś takiego\\
(plik \emph{base/ntos/ke/i386/cpu.asm}):

\begin{lstlisting}[style=customasmx86]
        public  Get386Stepping
Get386Stepping  proc

        call    MultiplyTest            ; Perform multiplication test
        jnc     short G3s00             ; if nc, muttest is ok
        mov     ax, 0
        ret
G3s00:
        call    Check386B0              ; Check for B0 stepping
        jnc     short G3s05             ; if nc, it's B1/later
        mov     ax, 100h                ; It is B0/earlier stepping
        ret

G3s05:
        call    Check386D1              ; Check for D1 stepping
        jc      short G3s10             ; if c, it is NOT D1
        mov     ax, 301h                ; It is D1/later stepping
        ret

G3s10:
        mov     ax, 101h                ; assume it is B1 stepping
        ret

	...

MultiplyTest    proc

        xor     cx,cx                   ; 64K times is a nice round number
mlt00:  push    cx
        call    Multiply                ; does this chip's multiply work?
        pop     cx
        jc      short mltx              ; if c, No, exit
        loop    mlt00                   ; if nc, YEs, loop to try again
        clc
mltx:
        ret

MultiplyTest    endp
\end{lstlisting}




\input{patterns/02_stack/03_local_vars_PL}
\EN{\subsubsection{x86: alloca() function}
\label{alloca}
\myindex{\CStandardLibrary!alloca()}

\newcommand{\AllocaSrcPath}{C:\textbackslash{}Program Files (x86)\textbackslash{}Microsoft Visual Studio 10.0\textbackslash{}VC\textbackslash{}crt\textbackslash{}src\textbackslash{}intel}

It is worth noting the \TT{alloca()} function
\footnote{In MSVC, the function implementation can be found in \TT{alloca16.asm} and \TT{chkstk.asm} in \\
\TT{\AllocaSrcPath{}}}.
This function works like \TT{malloc()}, but allocates memory directly on the stack.
% page break added to prevent "\vref on page boundary" error. it may be dropped in future.
The allocated memory chunk does not have to be freed via a \TT{free()} function call, \\
since the function epilogue (\myref{sec:prologepilog}) returns \ESP back to its initial state and 
the allocated memory is just \emph{dropped}.
It is worth noting how \TT{alloca()} is implemented.
In simple terms, this function just shifts \ESP downwards toward the stack bottom by the number of bytes you need and sets \ESP as a pointer to the \emph{allocated} block.

Let's try:

\lstinputlisting[style=customc]{patterns/02_stack/04_alloca/2_1.c}

\TT{\_snprintf()} function works just like \printf, but instead of dumping the result into \gls{stdout} (e.g., to terminal or 
console), it writes it to the \TT{buf} buffer. Function \puts copies the contents of \TT{buf} to \gls{stdout}. Of course, these two
function calls might be replaced by one \printf call, but we have to illustrate small buffer usage.

\myparagraph{MSVC}

Let's compile (MSVC 2010):

\lstinputlisting[caption=MSVC 2010,style=customasmx86]{patterns/02_stack/04_alloca/2_2_msvc.asm}

\myindex{Compiler intrinsic}
The sole \TT{alloca()} argument is passed via \EAX (instead of pushing it into the stack)
\footnote{It is because alloca() is rather a compiler intrinsic (\myref{sec:compiler_intrinsic}) than a normal function.
One of the reasons we need a separate function instead of just a couple of instructions in the code,
is because the \ac{MSVC} alloca() implementation also has code which reads from the memory just allocated, in order to let the \ac{OS} map
physical memory to this \ac{VM} region.
After the \TT{alloca()} call, \ESP points to the block of 600 bytes and we can use it as memory for the \TT{buf} array.}.

\myparagraph{GCC + \IntelSyntax}

GCC 4.4.1 does the same without calling external functions:

\lstinputlisting[caption=GCC 4.7.3,style=customasmx86]{patterns/02_stack/04_alloca/2_1_gcc_intel_O3_EN.asm}

\myparagraph{GCC + \ATTSyntax}

Let's see the same code, but in AT\&T syntax:

\lstinputlisting[caption=GCC 4.7.3,style=customasmx86]{patterns/02_stack/04_alloca/2_1_gcc_ATT_O3.s}

\myindex{\ATTSyntax}
The code is the same as in the previous listing.

By the way, \INS{movl \$3, 20(\%esp)} corresponds to
\INS{mov DWORD PTR [esp+20], 3} in Intel-syntax.
In the AT\&T syntax, the register+offset format of addressing memory looks like
\TT{offset(\%{register})}.

}
\FR{\subsubsection{x86: alloca() function}
\label{alloca}
\myindex{\CStandardLibrary!alloca()}

\newcommand{\AllocaSrcPath}{C:\textbackslash{}Program Files (x86)\textbackslash{}Microsoft Visual Studio 10.0\textbackslash{}VC\textbackslash{}crt\textbackslash{}src\textbackslash{}intel}

Intéressons-nous à la fonction \TT{alloca()}
\footnote{Avec MSVC, l'implémentation de cette fonction peut être trouvée dans les fichiers \TT{alloca16.asm} et \TT{chkstk.asm} dans \\
\TT{\AllocaSrcPath{}}}

Cette fonction fonctionne comme \TT{malloc()}, mais alloue de la mémoire directement sur la pile.
% page break added to prevent "\vref on page boundary" error. it may be dropped in future.
L'espace de mémoire ne doit pas être libéré via un appel à la fonction \TT{free()},
puisque l'épilogue de fonction (\myref{sec:prologepilog}) remet \ESP à son état initial
ce qui va automatiquement libérer cet espace mémoire.

Intéressons-nous à l'implémentation d'\TT{alloca()}.
Cette fonction décale simplement \ESP du nombre d'octets demandé vers le bas de la
pile et définit \ESP comme un pointeur vers la mémoire \emph{allouée}.

Essayons :

\lstinputlisting[style=customc]{patterns/02_stack/04_alloca/2_1.c}

La fonction \TT{\_snprintf()} fonctionne comme \printf, mais au lieu d'afficher le
résultat sur la \glslink{stdout}{sortie standard} (ex., dans un terminal ou une console), il l'écrit dans
le buffer \TT{buf}. La fonction \puts copie le contenu de \TT{buf} dans la \glslink{stdout}{sortie standard}.
Évidemment, ces deux appels de fonctions peuvent être remplacés par un seul appel à
la fonction \printf, mais nous devons illustrer l'utilisation de petit buffer.

\myparagraph{MSVC}

Compilons (MSVC 2010) :

\lstinputlisting[caption=MSVC 2010,style=customasmx86]{patterns/02_stack/04_alloca/2_2_msvc.asm}

\myindex{Compiler intrinsic}
Le seul argument d'\TT{alloca()} est passé via \EAX (au lieu de le mettre sur la pile)
\footnote{C'est parce que alloca() est plutôt une fonctionnalité intrinsèque du compilateur (\myref{sec:compiler_intrinsic}) qu'une fonction normale. Une des raisons pour laquelle nous avons besoin d'une fonction séparée au lieu de quelques instructions dans le code, est parce que l'implémentation d'alloca() par \ac{MSVC} a également du code qui lit depuis la mémoire récemment allouée pour laisser l'\ac{OS} mapper la mémoire physique vers la \ac{VM}. Aprés l'appel à la fonction \TT{alloca()}, \ESP pointe sur un bloc de 600 octets que nous pouvons utiliser pour le tableau \TT{buf}.}.

\myparagraph{GCC + \IntelSyntax}

GCC 4.4.1 fait la même chose sans effectuer d'appel à des fonctions externes :

\lstinputlisting[caption=GCC 4.7.3,style=customasmx86]{patterns/02_stack/04_alloca/2_1_gcc_intel_O3_FR.asm}

\myparagraph{GCC + \ATTSyntax}

Voyons le même code mais avec la syntaxe AT\&T :

\lstinputlisting[caption=GCC 4.7.3,style=customasmx86]{patterns/02_stack/04_alloca/2_1_gcc_ATT_O3.s}

\myindex{\ATTSyntax}
Le code est le même que le précédent.

Au fait, \INS{movl \$3, 20(\%esp)} correspond à
\INS{mov DWORD PTR [esp+20], 3} avec la syntaxe intel.
Dans la syntaxe AT\&T, le format registre+offset pour l'adressage mémoire
ressemble à \TT{offset(\%{register})}.
}
\RU{\subsubsection{x86: Функция alloca()}
\label{alloca}
\myindex{\CStandardLibrary!alloca()}

\newcommand{\AllocaSrcPath}{C:\textbackslash{}Program Files (x86)\textbackslash{}Microsoft Visual Studio 10.0\textbackslash{}VC\textbackslash{}crt\textbackslash{}src\textbackslash{}intel}

Интересен случай с функцией \TT{alloca()}
\footnote{В MSVC, реализацию функции можно посмотреть в файлах \TT{alloca16.asm} и \TT{chkstk.asm} в \\
\TT{\AllocaSrcPath{}}}. 
Эта функция работает как \TT{malloc()}, но выделяет память прямо в стеке.
Память освобождать через \TT{free()} не нужно, так как эпилог функции~(\myref{sec:prologepilog})
вернет \ESP в изначальное состояние и выделенная память просто \emph{выкидывается}.
Интересна реализация функции \TT{alloca()}.
Эта функция, если упрощенно, просто сдвигает \ESP вглубь стека на столько байт, сколько вам нужно и возвращает \ESP в качестве указателя на выделенный блок.

Попробуем:

\lstinputlisting[style=customc]{patterns/02_stack/04_alloca/2_1.c}

Функция \TT{\_snprintf()} работает так же, как и \printf, только вместо выдачи результата в \gls{stdout} (т.е. на терминал или в консоль),
записывает его в буфер \TT{buf}. Функция \puts выдает содержимое буфера \TT{buf} в \gls{stdout}. Конечно, можно было бы
заменить оба этих вызова на один \printf, но здесь нужно проиллюстрировать использование небольшого буфера.

\myparagraph{MSVC}

Компилируем (MSVC 2010):

\lstinputlisting[caption=MSVC 2010,style=customasmx86]{patterns/02_stack/04_alloca/2_2_msvc.asm}

\myindex{Compiler intrinsic}
Единственный параметр в \TT{alloca()} передается через \EAX, а не как обычно через стек
\footnote{Это потому, что alloca()~--- это не сколько функция, сколько т.н. \emph{compiler intrinsic} (\myref{sec:compiler_intrinsic})
Одна из причин, почему здесь нужна именно функция, а не несколько инструкций прямо в коде в том, что в реализации 
функции alloca() от \ac{MSVC}
есть также код, читающий из только что выделенной памяти, чтобы \ac{OS} подключила физическую память к этому региону \ac{VM}.
После вызова \TT{alloca()} \ESP указывает на блок в 600 байт, который мы можем использовать под \TT{buf}.}.

\myparagraph{GCC + \IntelSyntax}

А GCC 4.4.1 обходится без вызова других функций:

\lstinputlisting[caption=GCC 4.7.3,style=customasmx86]{patterns/02_stack/04_alloca/2_1_gcc_intel_O3_RU.asm}

\myparagraph{GCC + \ATTSyntax}

Посмотрим на тот же код, только в синтаксисе AT\&T:

\lstinputlisting[caption=GCC 4.7.3,style=customasmx86]{patterns/02_stack/04_alloca/2_1_gcc_ATT_O3.s}

\myindex{\ATTSyntax}
Всё то же самое, что и в прошлом листинге.

Кстати, \INS{movl \$3, 20(\%esp)}~--- это аналог \INS{mov DWORD PTR [esp+20], 3} в синтаксисе Intel.
Адресация памяти в виде \emph{регистр+смещение} записывается в синтаксисе AT\&T как \TT{смещение(\%{регистр})}.

}
\PTBR{\mysection{\Stack}
\label{sec:stack}
\myindex{\Stack}

A pilha é uma das estruturas mais fundamentais na ciência da computação.
\footnote{\href{http://go.yurichev.com/17119}{wikipedia.org/wiki/Call\_stack}}.
\ac{AKA} \ac{LIFO}.

Tecnicamente, é só um bloco de memória junto com os registradores \ESP ou \RSP em x86 e x64, ou o \ac{SP} no ARM, como um ponteiro para aquele bloco.

\myindex{ARM!\Instructions!PUSH}
\myindex{ARM!\Instructions!POP}
\myindex{x86!\Instructions!PUSH}
\myindex{x86!\Instructions!POP}
As instruções mais frequente para o acesso da pilha são \PUSH e \POP (em ambos x86 e x64).
\PUSH subtrai de \ESP/\RSP/\ac{SP} 4 no modo 32-bits (ou 8 no modo 64-bits) e então escreve o conteúdo desse operando único para o endereço de memória apontado por \ESP/\RSP/\ac{SP}.

\POP é a operação reversa: recupera a informação da localização de memória que é apontada por \ac{SP}, 
carrega a mesma no operando da instrução (geralmente um registrador) e então adiciona 4 (ou 8) para o ponteiro da pilha.

Depois da alocação da pilha, o ponteiro aponta para o fundo da pilha.
\PUSH decrementa o ponteiro da pilha e \POP incrementa. O fundo da pilha está na verdade no começo do bloco de memória alocado para ela.
Pode parecer estranho, mas é a maneira como é feita.

ARM: \PTBRph{}

\subsection{Por que a pilha ``cresce'' para trás?}
\label{stack_grow_backwards}

Intuitivamente, nós podemos pensar que a pilha cresce para frente, em direção a endereços mais altos, como qualquer outra estrutura de informação.

O motivo da pilha crescer para trás é provavelmente histórico. Quando os computadores era grandes e ocupavam um cômodo todo, era mais fácil dividir a memória em duas partes, uma para a ‘heap’ e outra para a pilha.
Logicamente, era desconhecido o quão grande a heap e a pilha seriam durante a execução do programa, então essa solução era a mais simples possível.

\input{patterns/02_stack/stack_and_heap}

No \RitchieThompsonUNIX nós podemos ler:

\begin{framed}
\begin{quotation}
A parte relacionada ao usuário é dividida em três segmentos lógicos. O segmento de texto do programa começa na localização 0 no espaço virtual de endereçamento.
Durante a execução, esse segmento é protegido para não ser reescrito e uma única cópia dele é compartilhado entre
todos os processos executando o mesmo programa.
Começando no limite de 8Kbytes acima do segmento de texto do programa no espaço de endereçamento virtual começa um segmento de informação gravável,
não compartilhável e de um tamanho que pode ser extendido por uma chamada do sistema.
Começando no endereço mais alto no espaço de endereçamento virtual está a pilha, que automaticamente cresce para trás conforme o ponteiro da pilha do hardware se altera.
\end{quotation}
\end{framed}

Isso pode ser análogo a como um estudante escreve notas de duas matérias diferentes em um caderno só:
as notas para a primeira matéria são escritas como de costume e as notas para a segunda são escritas do final do caderno,
virando o mesmo. As anotações de uma matéria podem encontrar as da outra no meio, no caso de haver falta de espaço.

\subsection{Para que a pilha é usada?}

% subsections
\subsubsection{Salvar o endereço de retorno de uma função}

\myparagraph{x86}

\myindex{x86!\Instructions!CALL}
Quando você chama outra função utilizando a instrução CALL, o endereço do ponto exato onde a 
instrução \CALL se encontra é salvo na pilha e então um jump incondicional para o endereço no operando de \CALL é executado.

\myindex{x86!\Instructions!PUSH}
\myindex{x86!\Instructions!JMP}
A instrução \CALL é equivalente a usar o par de instruções \TT{PUSH endereço\_depois\_chamada / JMP}.

\myindex{x86!\Instructions!RET}
\myindex{x86!\Instructions!POP}
\RET pega um valor da pilha e usa um jump para ele --- isso é equivalente a usar \INS{POP tmp / JMP tmp}.

\myindex{\Stack!\MLStackOverflow}
\myindex{\Recursion}
Estourar uma stack é fácil. Só execute alguma recursão externa:

\begin{lstlisting}[style=customc]
void f()
{
	f();
};
\end{lstlisting}

O compilador MSVC 2008 informa o problema:

\begin{lstlisting}
c:\tmp6>cl ss.cpp /Fass.asm
Microsoft (R) 32-bit C/C++ Optimizing Compiler Version 15.00.21022.08 for 80x86
Copyright (C) Microsoft Corporation.  All rights reserved.

ss.cpp
c:\tmp6\ss.cpp(4) : warning C4717: 'f' : recursive on all control paths, function will cause runtime stack overflow
\end{lstlisting}

\dots mas gera o código de qualquer maneira:

\lstinputlisting[style=customasmx86]{patterns/02_stack/1.asm}

\dots também, se ativarmos a otimização do compilador (opção \TT{/Ox}) 
o código otimizado não vai estourar a pilha e funcionará \emph{corretamente} \footnote{ironia aqui}:

\lstinputlisting[style=customasmx86]{patterns/02_stack/2.asm}

\PTBRph{}


\input{patterns/02_stack/02_args_passing_PTBR}
\input{patterns/02_stack/03_local_vars_PTBR}
\EN{\input{patterns/02_stack/04_alloca/main_EN}}
\FR{\input{patterns/02_stack/04_alloca/main_FR}}
\RU{\input{patterns/02_stack/04_alloca/main_RU}}
\PTBR{\input{patterns/02_stack/04_alloca/main_PTBR}}
\IT{\input{patterns/02_stack/04_alloca/main_IT}}
\DE{\input{patterns/02_stack/04_alloca/main_DE}}
\PL{\input{patterns/02_stack/04_alloca/main_PL}}
\JA{\input{patterns/02_stack/04_alloca/main_JA}}

\input{patterns/02_stack/05_SEH}
\ifdefined\ENGLISH
\subsubsection{Buffer overflow protection}

More about it here~(\myref{subsec:bufferoverflow}).
\fi

\ifdefined\RUSSIAN
\subsubsection{Защита от переполнений буфера}

Здесь больше об этом~(\myref{subsec:bufferoverflow}).
\fi

\ifdefined\BRAZILIAN
\subsubsection{Proteção contra estouro de buffer}

Mais sobre aqui~(\myref{subsec:bufferoverflow}).
\fi

\ifdefined\ITALIAN
\subsubsection{Protezione contro buffer overflow}

Maggiori informazioni qui~(\myref{subsec:bufferoverflow}).
\fi

\ifdefined\FRENCH
\subsubsection{Protection contre les débordements de tampon}

Lire à ce propos~(\myref{subsec:bufferoverflow}).
\fi


\ifdefined\POLISH
\subsubsection{Metody zabiezpieczenia przed przepełnieniem stosu}

Więcej o tym tutaj~(\myref{subsec:bufferoverflow}).
\fi

\ifdefined\JAPANESE
\subsubsection{バッファオーバーフロー保護}

詳細はこちら~(\myref{subsec:bufferoverflow})
\fi


\subsubsection{\PTBRph{}}

Talvez, o motivo para armazenar variáveis locais e registros SEH na pilha é que eles são desvinculados automaticamente depois do fim da função,
usando somente uma instrução para corrigir o ponteiro da pilha (geralmente é \ADD). Argumentos de funções, como podemos dizer, são
também desalocados automaticamente com o fim da função.
Como contraste, tudo armazenado na memória heap tem de ser desalocado explicitamente.

% sections
\input{patterns/02_stack/07_layout_PTBR}
\EN{\input{patterns/02_stack/08_noise/main_EN}}
\FR{\input{patterns/02_stack/08_noise/main_FR}}
\RU{\input{patterns/02_stack/08_noise/main_RU}}
\IT{\input{patterns/02_stack/08_noise/main_IT}}
\DE{\input{patterns/02_stack/08_noise/main_DE}}
\PL{\input{patterns/02_stack/08_noise/main_PL}}
\JA{\input{patterns/02_stack/08_noise/main_JA}}

\input{patterns/02_stack/exercises}

}
\IT{\subsubsection{x86: la funzione alloca() }
\label{alloca}
\myindex{\CStandardLibrary!alloca()}

\newcommand{\AllocaSrcPath}{C:\textbackslash{}Program Files (x86)\textbackslash{}Microsoft Visual Studio 10.0\textbackslash{}VC\textbackslash{}crt\textbackslash{}src\textbackslash{}intel}

Vale la pena esaminare la funzione \TT{alloca()}
\footnote{In MSVC, l'implementazione della funzione si trova in \TT{alloca16.asm} e \TT{chkstk.asm} in \\
\TT{\AllocaSrcPath{}}}.
Questa funzione opera come \TT{malloc()}, ma alloca memoria direttamente nello stack.
% page break added to prevent "\vref on page boundary" error. it may be dropped in future.
Il pezzo di memoria allocato non necessita di essere liberato tramite una chiamata alla funzione \TT{free()} function call, \\
poichè l'epilogo della funzione (\myref{sec:prologepilog}) ripristina \ESP al suo valore iniziale e la memoria allocata viene semplicemente \emph{abbandonata}.
Vale anche la pena notare come è implementata la funzione \TT{alloca()}.
In termini semplici, questa funzione sposta \ESP verso il basso, verso la base dello stack, per il numero di byte necessari e setta \ESP
per puntare al blocco \emph{allocato}.

Proviamo:

\lstinputlisting[style=customc]{patterns/02_stack/04_alloca/2_1.c}

La funzione \TT{\_snprintf()} opera come \printf, ma invece di inviare il risultato a \gls{stdout} (es. al terminale o console),
lo scrive nel buffer \TT{buf}. La funzione \puts copia il contenuto di \TT{buf} in \gls{stdout}.
Ovviamente queste due chiamate potrebbero essere rimpiazzate da una sola chiamata a \printf, ma questo è solo un esempio per illustrare
l'uso di un piccolo buffer.

\myparagraph{MSVC}

Compiliamo (MSVC 2010):

\lstinputlisting[caption=MSVC 2010,style=customasmx86]{patterns/02_stack/04_alloca/2_2_msvc.asm}

\myindex{Compiler intrinsic}
L'unico argomento di \TT{alloca()} viene passato tramite il registro \EAX (anzichè inserirlo nello stack)
\footnote{Questo perchè alloca() è più una "compiler intrinsic" (\myref{sec:compiler_intrinsic}) che una funzione normale.
Una delle ragioni per cui abbiamo bisogno di una funzione separata, invece di utilizzare semplicemente un paio di istruzioni nel codice, è che
l'implementazione di alloca() usata da \ac{MSVC} include anche del codice che legge dalla memoria appena allocata, per far si che l'\ac{OS} effettui il mapping
della memoria fisica in questa regione della \ac{VM}.
Dopo la chiamata a \TT{alloca()}, \ESP punta al blocco di 600 byte, ed è possibile utilizzarlo come memoria per l'array \TT{buf}.}.

\myparagraph{GCC + \IntelSyntax}

GCC 4.4.1 fa lo stesso senza chiamare funzioni esterne:

\lstinputlisting[caption=GCC 4.7.3,style=customasmx86]{patterns/02_stack/04_alloca/2_1_gcc_intel_O3_EN.asm}

\myparagraph{GCC + \ATTSyntax}

Esaminiamo lo stesso codice, ma in sintassi AT\&T:

\lstinputlisting[caption=GCC 4.7.3,style=customasmx86]{patterns/02_stack/04_alloca/2_1_gcc_ATT_O3.s}

\myindex{\ATTSyntax}
Il codice è uguale a quello del listato precedente.

A proposito, \INS{movl \$3, 20(\%esp)} corrisponde a \INS{mov DWORD PTR [esp+20], 3} in sintassi Intel.
In sintassi AT\&T, il formato registro+offset per indirizzare memoria appare come \TT{offset(\%{register})}.
}
\DE{\subsubsection{x86: alloca() Funktion}
\label{alloca}
\myindex{\CStandardLibrary!alloca()}

\newcommand{\AllocaSrcPath}{C:\textbackslash{}Program Files (x86)\textbackslash{}Microsoft Visual Studio 10.0\textbackslash{}VC\textbackslash{}crt\textbackslash{}src\textbackslash{}intel}

Es macht Sinn einen Blick auf die \TT{alloca()} Funktion zu werfen
\footnote{In MSVC, kann die Funktions Implementierung in \TT{alloca16.asm} und \TT{chkstk.asm} in \\
\TT{\AllocaSrcPath{}}} gefunden werden.
Diese Funktion arbeitet wie \TT{malloc()}, nur das sie Speicher direkt auf dem Stack bereit stellt.

Der allozierte Speicher Chunk muss nicht wieder mit \TT{free()} freigegeben werden, weil
der Funktions Epilog (\myref{sec:prologepilog}) das \ESP Register wieder in seinen ursprünglichen 
Zustand versetzt und der allozierte Speicher wird einfach \emph{verworfen}. 
Es macht Sinn sich anzuschauen wie \TT{alloca()} implementiert ist.
Mit einfachen Begriffen erklärt, diese Funktion verschiebt \ESP in Richtung des Stack ende mit der 
Anzahl der Bytes die alloziert werden müssen und setzt \ESP als einen Zeiger auf den \emph{allozierten} block.

Beispiel:

\lstinputlisting[style=customc]{patterns/02_stack/04_alloca/2_1.c}


Die \TT{\_snprintf()} Funktion arbeitetet genau wie \printf, nur statt die Ergebnisse nach \gls{stdout} aus zu geben ( bsp. auf dem Terminal oder Konsole), schreibt sie in den \TT{buf} buffer. Die Funktion \puts kopiert den Inhalt aus \TT{buf} nach \gls{stdout}. Sicher könnte man die beiden Funktions Aufrufe könnten durch einen \printf Aufruf ersetzt werden, aber wir sollten einen genaueren Blick auf die Benutzung kleiner Buffer anschauen.

\myparagraph{MSVC}

Compilierung mit MSVC 2010: 

\lstinputlisting[caption=MSVC 2010,style=customasmx86]{patterns/02_stack/04_alloca/2_2_msvc.asm}

\myindex{Compiler intrinsisch}
Das einzige \\TT{alloca()} Argument wird über \EAX übergeben (anstatt es erst auf den Stack zu pushen)
\footnote{Das liegt daran, das alloca() Verhalten Compiler intrinsisch bestimmt (\myref{sec:compiler_intrinsic}) im Gegensatz zu einer normalen Funktion. Einer der Gründe dafür das man braucht eine separate Funktion braucht, statt ein paar Code Instruktionen im Code,  ist weil die \ac{MSCV} alloca() Implementierung ebenfalls Code hat welcher aus dem gerade allozierten Speicher gelesen wird. Damit in Folge das \ac{Betriebssystem} physikalischen Speicher in dieser \ac{VM} Region zu allozieren. Nach dem \TT{alloca()} Aufruf, zeigt \ESP auf den Block von 600 Bytes der nun als Speicher für das \TT{buf} Array dienen kann.}.

\myparagraph{GCC + \IntelSyntax}

GCC 4.4.1 macht das selbe, aber ohne externe Funktions aufrufe.

\lstinputlisting[caption=GCC 4.7.3,style=customasmx86]{patterns/02_stack/04_alloca/2_1_gcc_intel_O3_EN.asm}

\myparagraph{GCC + \ATTSyntax}

Nun der gleiche Code, aber in AT\&T Syntax:

\lstinputlisting[caption=GCC 4.7.3,style=customasmx86]{patterns/02_stack/04_alloca/2_1_gcc_ATT_O3.s}

\myindex{\ATTSyntax}
Der Code ist der gleiche wie im vorherigen listig.

Übrigens, \INS{movl \$3, 20(\%esp)} in AT\&T Syntax wird zu \
\INS{mov DWORD PTR [esp+20], 3} in Intel-syntax.
In der AT\&T Syntax, sehen Register+Offset Formatierungen einer Adresse so aus:
\TT{offset(\%{register})}.
}
\PL{\subsubsection{x86: Funkcja alloca()}
\label{alloca}
\myindex{\CStandardLibrary!alloca()}

\newcommand{\AllocaSrcPath}{C:\textbackslash{}Program Files (x86)\textbackslash{}Microsoft Visual Studio 10.0\textbackslash{}VC\textbackslash{}crt\textbackslash{}src\textbackslash{}intel}

Przypadek z funkcją \TT{alloca()} jest całkiem ciekawy
\footnote{W MSVC, implementację funkcji można podejrzeć w plikach \TT{alloca16.asm} i \TT{chkstk.asm} w \\
\TT{\AllocaSrcPath{}}}. 
Ta funkcja działa jak \TT{malloc()}, ale przydziela pamięć od razu na stosie.
Nie potrzebne jest zwalnianie pamięci \TT{free()}, dlatego że epilog funkcji~(\myref{sec:prologepilog})
przywróci \ESP do stanu początkowego i przeznaczona pamięć \emph{zostaje wyrzucona}.
Ciekawa jest również realizacja tej funkcji.
Ona, w skrócie, po prostu przesuwa \ESP wgłąb stosu i zwraca \ESP jako wskaźnik na przydzielony obszar.

Spróbujmy:

\lstinputlisting[style=customc]{patterns/02_stack/04_alloca/2_1.c}

Funkcja \TT{\_snprintf()} działa tak samo, jak i \printf, tylko zamiast wyprowadzenia wyniku na wyjście standardowe \gls{stdout} (czyli do terminalu),
ona go zapisuje do buforu \TT{buf}. Funkcja \puts, z kolei, wyrzuca zawartość buforu \TT{buf} na \gls{stdout}. Oczywiście można by było tutaj zamienić
tę parę instrukcji na \printf, ale tutaj chcielibyśmy zobaczyć wykorzystanie niewielkiego buforu.

\myparagraph{MSVC}

Skompilujmy (MSVC 2010):

\lstinputlisting[caption=MSVC 2010,style=customasmx86]{patterns/02_stack/04_alloca/2_2_msvc.asm}

\myindex{Compiler intrinsic}
Jedyny parametr \TT{alloca()} jest przekazywany przez \EAX, a nie jak zwykle, przez stos
\footnote{To dlatego, że alloca()~--- to nie tyle co funkcja, a raczej \emph{compiler intrinsic} (\myref{sec:compiler_intrinsic})
Jedną z przyczyn, flaczego tu potrzeba funkcji, a nie kilku instrukcji w samym kodzie, polega na tym, że w realizacji
funkcji alloca() w \ac{MSVC}
zawarty również kod, czytający z dopiero co przydzielonej pamięci po to, żeby \ac{OS} zaalokowała pamięć fizyczną dla tego obszaru \ac{VM}.
Po wywołaniu \TT{alloca()} \ESP wskazuje na blok o długości 600 bajtów, z którego możemy korzystać na potrzeby naszego \TT{buf}.}.

\myparagraph{GCC + \IntelSyntax}

GCC 4.4.1 nie wywołuje innych funkcji:

\lstinputlisting[caption=GCC 4.7.3,style=customasmx86]{patterns/02_stack/04_alloca/2_1_gcc_intel_O3_RU.asm}

\myparagraph{GCC + \ATTSyntax}

Spójrzmy na ten sam kod w syntaksie AT\&T:

\lstinputlisting[caption=GCC 4.7.3,style=customasmx86]{patterns/02_stack/04_alloca/2_1_gcc_ATT_O3.s}

\myindex{\ATTSyntax}
Wygląda to tak samo jak i poprzedni listing.

A propos, \INS{movl \$3, 20(\%esp)}~--- jest analogiem do \INS{mov DWORD PTR [esp+20], 3} w syntaksie Intel.
Adresowanie pamięci typu \emph{rejestr+przesunięcie} jest zapisywane w syntaksie AT\&T jako \TT{przesunięcie(\%{rejestr})}.


}
\JA{\subsubsection{x86: alloca()関数}
\label{alloca}
\myindex{\CStandardLibrary!alloca()}

\newcommand{\AllocaSrcPath}{C:\textbackslash{}Program Files (x86)\textbackslash{}Microsoft Visual Studio 10.0\textbackslash{}VC\textbackslash{}crt\textbackslash{}src\textbackslash{}intel}

\TT{alloca()}関数に注目することは重要です
\footnote{MSVCでは、関数の実装は\TT{\AllocaSrcPath{}}の\TT{alloca16.asm} と \TT{chkstk.asm}にあります}
この関数は\TT{malloc()}のように動作しますが、スタックに直接メモリを割り当てます。 
% page break added to prevent "\vref on page boundary" error. it may be dropped in future.
関数のエピローグ(\myref{sec:prologepilog})は \ESP を初期状態に戻し、割り当てられたメモリは単に\emph{破棄}されるため、
割り当てられたメモリチャンクは\TT{free()}関数呼び出しで解放する必要はありません。 \TT{alloca()}がどのように実装されているかは注目に値する。
簡単に言えば、この関数は必要なバイト数だけスタック底部に向かって \ESP を下にシフトさせ、\emph{割り当てられた}ブロックへのポインタとして \ESP を設定します。

やってみましょう。

\lstinputlisting[style=customc]{patterns/02_stack/04_alloca/2_1.c}

\TT{\_snprintf()}関数は \printf と同じように動作しますが、結果を\gls{stdout}(ターミナルやコンソールなど)
にダンプする代わりに、\TT{buf}バッファに書き込みます。 \puts 関数は\TT{buf}の内容を\gls{stdout}にコピーします。
もちろん、これらの2つの関数呼び出しは1つの \printf 呼び出しで置き換えることができますが、小さなバッファの使用法を説明する必要があります。

\myparagraph{MSVC}

コンパイルしてみましょう(MSVC 2010で)

\lstinputlisting[caption=MSVC 2010,style=customasmx86]{patterns/02_stack/04_alloca/2_2_msvc.asm}

\myindex{Compiler intrinsic}
\TT{alloca()}の唯一の引数は \EAX 経由で(スタックにプッシュするのではなく)渡されます。
\footnote{alloca()はコンパイラ組み込み関数((\myref{sec:compiler_intrinsic}))ではなく、通常の関数です。 
\ac{MSVC}のalloca()の実装には、割り当てられたメモリから読み込むコードが含まれているため、\ac{OS}が物理メモリをVM領域にマップするために、
コード内の命令が数個ではなく別々の関数を必要とする理由の1つです。 \TT{alloca()}呼び出しの後、ESPは600バイトのブロックを指し、\TT{buf}配列のメモリとして使用できます。}

\myparagraph{GCC + \IntelSyntax}

GCC 4.4.1は、外部関数を呼び出すことなく同じことを行います

\lstinputlisting[caption=GCC 4.7.3,style=customasmx86]{patterns/02_stack/04_alloca/2_1_gcc_intel_O3_EN.asm}

\myparagraph{GCC + \ATTSyntax}

同じコードをAT\&T構文で見てみましょう

\lstinputlisting[caption=GCC 4.7.3,style=customasmx86]{patterns/02_stack/04_alloca/2_1_gcc_ATT_O3.s}

\myindex{\ATTSyntax}
コードは前のリストと同じです。

ちなみに、\INS{movl \$3, 20(\%esp)}は、
Intel構文の\INS{mov DWORD PTR [esp+20], 3}に対応しています。 
AT\&Tの構文では、アドレス指定メモリのレジスタ+オフセット形式は
\TT{offset(\%{register})}のように見えます。
}

\input{patterns/02_stack/05_SEH}
\ifdefined\ENGLISH
\subsubsection{Buffer overflow protection}

More about it here~(\myref{subsec:bufferoverflow}).
\fi

\ifdefined\RUSSIAN
\subsubsection{Защита от переполнений буфера}

Здесь больше об этом~(\myref{subsec:bufferoverflow}).
\fi

\ifdefined\BRAZILIAN
\subsubsection{Proteção contra estouro de buffer}

Mais sobre aqui~(\myref{subsec:bufferoverflow}).
\fi

\ifdefined\ITALIAN
\subsubsection{Protezione contro buffer overflow}

Maggiori informazioni qui~(\myref{subsec:bufferoverflow}).
\fi

\ifdefined\FRENCH
\subsubsection{Protection contre les débordements de tampon}

Lire à ce propos~(\myref{subsec:bufferoverflow}).
\fi


\ifdefined\POLISH
\subsubsection{Metody zabiezpieczenia przed przepełnieniem stosu}

Więcej o tym tutaj~(\myref{subsec:bufferoverflow}).
\fi

\ifdefined\JAPANESE
\subsubsection{バッファオーバーフロー保護}

詳細はこちら~(\myref{subsec:bufferoverflow})
\fi


\subsubsection{Automatyczne zwalnianie danych na stosie}
Możliwym powodem przechowywania zmiennych lokalnych i rekordów SEH na stosie jest to, że kiedy funkcja zakończy działanie są one automatycznie zwalniane ze stosu używając tylko jednej instrukcji w celu przywrócenia poprzedniego stanu stosu (często jest to instrukcja ADD). Argumenty funkcji także są automatycznie zwalniane z pamięci pod koniec funkcji. Natomiast wszystko co jest przechowywane na stercie(\emph{heap}) trzeba zwalniać jawnie.

% sections
\input{patterns/02_stack/07_layout_PL}
\EN{\subsubsection{ARM}

\myparagraph{\NonOptimizingKeilVI (\ARMMode)}

\lstinputlisting[label=Keil_number_sign,style=customasmARM]{patterns/09_loops/simple/ARM/Keil_ARM_O0.asm}

Iteration counter $i$ is to be stored in the \Reg{4} register.
The \INS{MOV R4, \#2} instruction just initializes $i$.
The \INS{MOV R0, R4} and \INS{BL printing\_function} instructions
compose the body of the loop, the first instruction preparing the argument for 
\ttf function and the second calling the function.
\myindex{ARM!\Instructions!ADD}
The \INS{ADD R4, R4, \#1} instruction just adds 1 to the $i$ variable at each iteration.
\myindex{ARM!\Instructions!CMP}
\myindex{ARM!\Instructions!BLT}
\INS{CMP R4, \#0xA} compares $i$ with \TT{0xA} (10). 
The next instruction \INS{BLT} (\emph{Branch Less Than}) jumps if $i$ is less than 10.
Otherwise, 0 is to be written into \Reg{0} (since our function returns 0)
and function execution finishes.

\myparagraph{\OptimizingKeilVI (\ThumbMode)}

\lstinputlisting[style=customasmARM]{patterns/09_loops/simple/ARM/Keil_thumb_O3.asm}

Practically the same.

\myparagraph{\OptimizingXcodeIV (\ThumbTwoMode)}
\label{ARM_unrolled_loops}

\lstinputlisting[style=customasmARM]{patterns/09_loops/simple/ARM/xcode_thumb_O3.asm}

In fact, this was in my \ttf function:

\begin{lstlisting}[style=customc]
void printing_function(int i)
{
    printf ("%d\n", i);
};
\end{lstlisting}

\myindex{Unrolled loop}
\myindex{Inline code}
So, LLVM not just \emph{unrolled} the loop, 
but also \emph{inlined} my 
very simple function \ttf,
and inserted its body 8 times instead of calling it. 

This is possible when the function is so simple (like mine) and when it is not called too much (like here).

\myparagraph{ARM64: \Optimizing GCC 4.9.1}

\lstinputlisting[caption=\Optimizing GCC 4.9.1,style=customasmARM]{patterns/09_loops/simple/ARM/ARM64_GCC491_O3_EN.s}

\myparagraph{ARM64: \NonOptimizing GCC 4.9.1}

\lstinputlisting[caption=\NonOptimizing GCC 4.9.1 -fno-inline,style=customasmARM]{patterns/09_loops/simple/ARM/ARM64_GCC491_O0_EN.s}
}
\FR{\subsection{Méthodes de protection contre les débordements de tampon}
\label{subsec:BO_protection}

Il existe quelques méthodes pour protéger contre ce fléau, indépendamment de la négligence
des programmeurs \CCpp.
MSVC possède des options comme\footnote{méthode de protection contre les débordements
de tampons côté compilateur:\href{http://go.yurichev.com/17133}{wikipedia.org/wiki/Buffer\_overflow\_protection}}:

\begin{lstlisting}
 /RTCs Stack Frame runtime checking
 /GZ Enable stack checks (/RTCs)
\end{lstlisting}

\myindex{x86!\Instructions!RET}
\myindex{Function prologue}
\myindex{Security cookie}

Une des méthodes est d'écrire une valeur aléatoire entre les variables locales sur
la pile dans le prologue de la fonction et de la vérifier dans l'épilogue, avant de
sortir de la fonction.
Si la valeur n'est pas la même, ne pas exécuter la dernière instruction \RET, mais
stopper (ou bloquer).
Le processus va s'arrêter, mais c'est mieux qu'une attaque distante sur votre ordinateur.
    
\newcommand{\CANARYURL}{\href{http://go.yurichev.com/17134}{wikipedia.org/wiki/Domestic\_canary\#Miner.27s\_canary}}

\myindex{Canary}

Cette valeur aléatoire est parfois appelé un \q{canari}, c'est lié au canari\footnote{\CANARYURL}
que les mineurs utilisaient dans le passé afin de détecter rapidement les gaz toxiques.

Les canaris sont très sensibles aux gaz, ils deviennent très agités en cas de danger,
et même meurent.

Si nous compilons notre exemple de tableau très simple~(\myref{arrays_simple}) dans
\ac{MSVC} avec les options RTC1 et RTCs, nous voyons un appel à \TT{@\_RTC\_CheckStackVars@8}
une fonction à la fin de la fonction qui vérifie si le \q{canari} est correct.

Voyons comment GCC gère ceci.
Prenons un exemple \TT{alloca()}~(\myref{alloca}):

\lstinputlisting[style=customc]{patterns/02_stack/04_alloca/2_1.c}

Par défaut, sans option supplémentaire, GCC 4.7.3 insère un test de  \q{canari} dans
le code:

\lstinputlisting[caption=GCC 4.7.3,style=customasmx86]{patterns/13_arrays/3_BO_protection/gcc_canary_FR.asm}

\myindex{x86!\Registers!GS}
La valeur aléatoire se trouve en \TT{gs:20}.
Elle est écrite sur la pile et à la fin de la fonction, la valeur sur la pile est
comparée avec le \q{canari} correct dans \TT{gs:20}.
Si les valeurs ne sont pas égales, la fonction \TT{\_\_stack\_chk\_fail} est appelée
et nous voyons dans la console quelque chose comme ça (Ubuntu 13.04 x86):

\begin{lstlisting}
*** buffer overflow detected ***: ./2_1 terminated
======= Backtrace: =========
/lib/i386-linux-gnu/libc.so.6(__fortify_fail+0x63)[0xb7699bc3]
/lib/i386-linux-gnu/libc.so.6(+0x10593a)[0xb769893a]
/lib/i386-linux-gnu/libc.so.6(+0x105008)[0xb7698008]
/lib/i386-linux-gnu/libc.so.6(_IO_default_xsputn+0x8c)[0xb7606e5c]
/lib/i386-linux-gnu/libc.so.6(_IO_vfprintf+0x165)[0xb75d7a45]
/lib/i386-linux-gnu/libc.so.6(__vsprintf_chk+0xc9)[0xb76980d9]
/lib/i386-linux-gnu/libc.so.6(__sprintf_chk+0x2f)[0xb7697fef]
./2_1[0x8048404]
/lib/i386-linux-gnu/libc.so.6(__libc_start_main+0xf5)[0xb75ac935]
======= Memory map: ========
08048000-08049000 r-xp 00000000 08:01 2097586    /home/dennis/2_1
08049000-0804a000 r--p 00000000 08:01 2097586    /home/dennis/2_1
0804a000-0804b000 rw-p 00001000 08:01 2097586    /home/dennis/2_1
094d1000-094f2000 rw-p 00000000 00:00 0          [heap]
b7560000-b757b000 r-xp 00000000 08:01 1048602    /lib/i386-linux-gnu/libgcc_s.so.1
b757b000-b757c000 r--p 0001a000 08:01 1048602    /lib/i386-linux-gnu/libgcc_s.so.1
b757c000-b757d000 rw-p 0001b000 08:01 1048602    /lib/i386-linux-gnu/libgcc_s.so.1
b7592000-b7593000 rw-p 00000000 00:00 0
b7593000-b7740000 r-xp 00000000 08:01 1050781    /lib/i386-linux-gnu/libc-2.17.so
b7740000-b7742000 r--p 001ad000 08:01 1050781    /lib/i386-linux-gnu/libc-2.17.so
b7742000-b7743000 rw-p 001af000 08:01 1050781    /lib/i386-linux-gnu/libc-2.17.so
b7743000-b7746000 rw-p 00000000 00:00 0
b775a000-b775d000 rw-p 00000000 00:00 0
b775d000-b775e000 r-xp 00000000 00:00 0          [vdso]
b775e000-b777e000 r-xp 00000000 08:01 1050794    /lib/i386-linux-gnu/ld-2.17.so
b777e000-b777f000 r--p 0001f000 08:01 1050794    /lib/i386-linux-gnu/ld-2.17.so
b777f000-b7780000 rw-p 00020000 08:01 1050794    /lib/i386-linux-gnu/ld-2.17.so
bff35000-bff56000 rw-p 00000000 00:00 0          [stack]
Aborted (core dumped)
\end{lstlisting}

\myindex{MS-DOS}
gs est ainsi appelé registre de segment. Ces registres étaient beaucoup utilisés
du temps de MS-DOS et des extensions de DOS.
Aujourd'hui, sa fonction est différente.
\myindex{TLS}
\myindex{Windows!TIB}

Dit brièvement, le registre \TT{gs} dans Linux pointe toujours sur le
\ac{TLS}~(\myref{TLS})---des informations spécifiques au thread sont stockées là.
À propos, en win32 le registre \TT{fs} joue le même rôle, pointant sur \ac{TIB}
\footnote{\href{http://go.yurichev.com/17104}{wikipedia.org/wiki/Win32\_Thread\_Information\_Block}}.

Il y a plus d'information dans le code source du noyau Linux (au moins dans la version 3.11),
dans\\
\emph{arch/x86/include/asm/stackprotector.h} cette variable est décrite dans les commentaires.

\subsubsection{ARM: \OptimizingKeilVI (\ARMMode)}
\myindex{\CLanguageElements!switch}

\lstinputlisting[style=customasmARM]{patterns/08_switch/1_few/few_ARM_ARM_O3.asm}

A nouveau, en investiguant ce code, nous ne pouvons pas dire si il y avait un switch()
dans le code source d'origine ou juste un ensemble de déclarations if().

\myindex{ARM!\Instructions!ADRcc}

En tout cas, nous voyons ici des instructions conditionnelles (comme \ADREQ (\emph{Equal}))
qui ne sont exécutées que si $R0=0$, et qui chargent ensuite l'adresse de la chaîne
\emph{<<zero\textbackslash{}n>>} dans \Reg{0}.
\myindex{ARM!\Instructions!BEQ}
L'instruction suivante \ac{BEQ} redirige le flux d'exécution en \TT{loc\_170}, si $R0=0$.

Le lecteur attentif peut se demander si \ac{BEQ} s'exécute correctement puisque \ADREQ
a déjà mis une autre valeur dans le registre \Reg{0}.

Oui, elle s'exécutera correctement, car \ac{BEQ} vérifie les flags mis par l'instruction
\CMP et \ADREQ ne modifie aucun flag.

Les instructions restantes nous sont déjà familières.
Il y a seulement un appel à \printf, à la fin, et nous avons déjà examiné cette
astuce ici~(\myref{ARM_B_to_printf}).
A la fin, il y a trois chemins vers \printf{}.

\myindex{ARM!\Instructions!ADRcc}
\myindex{ARM!\Instructions!CMP}
La dernière instruction, \TT{CMP R0, \#2}, est nécessaire pour vérifier si $a=2$.

Si ce n'est pas vrai, alors \ADRNE charge un pointeur sur la chaîne \emph{<<something unknown \textbackslash{}n>>}
dans \Reg{0}, puisque $a$ a déjà été comparée pour savoir s'elle est égale
à 0 ou 1, et nous sommes sûrs que la variable $a$ n'est pas égale à l'un de
ces nombres, à ce point.
Et si $R0=2$, un pointeur sur la chaîne \emph{<<two\textbackslash{}n>>} sera chargé
par \ADREQ dans \Reg{0}.

\subsubsection{ARM: \OptimizingKeilVI (\ThumbMode)}

\lstinputlisting[style=customasmARM]{patterns/08_switch/1_few/few_ARM_thumb_O3.asm}

% FIXME а каким можно? к каким нельзя? \myref{} ->

Comme il y déjà été dit, il n'est pas possible d'ajouter un prédicat conditionnel
à la plupart des instructions en mode Thumb, donc ce dernier est quelque peu similaire
au code \ac{CISC}-style x86, facilement compréhensible.

\subsubsection{ARM64: GCC (Linaro) 4.9 \NonOptimizing}

\lstinputlisting[style=customasmARM]{patterns/08_switch/1_few/ARM64_GCC_O0_FR.lst}

Le type de la valeur d'entrée est \Tint, par conséquent le registre \RegW{0} est
utilisé pour garder la valeur au lieu du registre complet \RegX{0}.

Les pointeurs de chaîne sont passés à \puts en utilisant la paire d'instructions
\INS{ADRP}/\INS{ADD} comme expliqué dans l'exemple \q{\HelloWorldSectionName}:~\myref{pointers_ADRP_and_ADD}.

\subsubsection{ARM64: GCC (Linaro) 4.9 \Optimizing}

\lstinputlisting[style=customasmARM]{patterns/08_switch/1_few/ARM64_GCC_O3_FR.lst}

Ce morceau de code est mieux optimisé.
L'instruction \TT{CBZ} (\emph{Compare and Branch on Zero} comparer et sauter si zéro)
effectue un saut si \RegW{0} vaut zéro.
Il y a alors un saut direct à \puts au lieu de l'appeler, comme cela a été expliqué
avant:~\myref{JMP_instead_of_RET}.


}
\RU{\mysection{Функция toupper()}
\myindex{\CStandardLibrary!toupper()}

Еще одна очень востребованная функция конвертирует символ из строчного в заглавный, если нужно:

\lstinputlisting[style=customc]{\CURPATH/toupper.c}

Выражение \TT{'a'+'A'} оставлено в исходном коде для удобства чтения, 
конечно, оно соптимизируется

\footnote{Впрочем, если быть дотошным, вполне могут до сих пор существовать компиляторы,
которые не оптимизируют подобное и оставляют в коде.}.

\ac{ASCII}-код символа \q{a} это 97 (или 0x61), и 65 (или 0x41) для символа \q{A}.

Разница (или расстояние) между ними в \ac{ASCII}-таблица это 32 (или 0x20).

Для лучшего понимания, читатель может посмотреть на стандартную 7-битную таблицу \ac{ASCII}:

\begin{figure}[H]
\centering
\includegraphics[width=0.7\textwidth]{ascii.png}
\caption{7-битная таблица \ac{ASCII} в Emacs}
\end{figure}

\subsection{x64}

\subsubsection{Две операции сравнения}

\NonOptimizing MSVC прямолинеен: код проверят, находится ли входной символ в интервале [97..122]
(или в интервале [`a'..`z'] ) и вычитает 32 в таком случае.

Имеется также небольшой артефакт компилятора:

\lstinputlisting[caption=\NonOptimizing MSVC 2013 (x64),numbers=left,style=customasmx86]{\CURPATH/MSVC_2013_x64_RU.asm}

Важно отметить что (на строке 3) входной байт загружается в 64-битный слот локального стека.

Все остальные биты ([8..63]) не трогаются, т.е. содержат случайный шум (вы можете увидеть его в отладчике).
% TODO add debugger example

Все инструкции работают только с байтами, так что всё нормально.

Последняя инструкция \TT{MOVZX} на строке 15 берет байт из локального стека и расширяет его 
до 32-битного \Tint, дополняя нулями.

\NonOptimizing GCC делает почти то же самое:

\lstinputlisting[caption=\NonOptimizing GCC 4.9 (x64),style=customasmx86]{\CURPATH/GCC_49_x64_O0.s}

\subsubsection{Одна операция сравнения}
\label{toupper_one_comparison}

\Optimizing MSVC работает лучше, он генерирует только одну операцию сравнения:

\lstinputlisting[caption=\Optimizing MSVC 2013 (x64),style=customasmx86]{\CURPATH/MSVC_2013_Ox_x64.asm}

Уже было описано, как можно заменить две операции сравнения на одну: \myref{one_comparison_instead_of_two}.

Мы бы переписал это на \CCpp так:

\begin{lstlisting}[style=customc]
int tmp=c-97;

if (tmp>25)
        return c;
else
        return c-32;
\end{lstlisting}

Переменная \emph{tmp} должна быть знаковая.

При помощи этого, имеем две операции вычитания в случае конверсии плюс одну операцию сравнения.

В то время как оригинальный алгоритм использует две операции сравнения плюс одну операцию вычитания.

\Optimizing GCC 
даже лучше, он избавился от переходов (а это хорошо: \myref{branch_predictors}) используя инструкцию CMOVcc:

\lstinputlisting[caption=\Optimizing GCC 4.9 (x64),numbers=left,style=customasmx86,label=toupper_GCC_O3]{\CURPATH/GCC_49_x64_O3.s}

На строке 3 код готовит уже сконвертированное значение заранее, как если бы конверсия всегда происходила.

На строке 5 это значение в EAX заменяется нетронутым входным значением, если конверсия не нужна.
И тогда это значение (конечно, неверное), просто выбрасывается.

Вычитание с упреждением это цена, которую компилятор платит за отсутствие условных переходов.

\subsection{ARM}

\Optimizing Keil для режима ARM также генерирует только одну операцию сравнения:

\lstinputlisting[caption=\OptimizingKeilVI (\ARMMode),style=customasmARM]{\CURPATH/Keil_ARM_O3.s}

\myindex{ARM!\Instructions!SUBcc}
\myindex{ARM!\Instructions!ANDcc}

\INS{SUBLS} и \INS{ANDLS} исполняются только если значение \Reg{1} меньше чем 0x19 (или равно).
Они и делают конверсию.

\Optimizing Keil для режима Thumb также генерирует только одну операцию сравнения:

\lstinputlisting[caption=\OptimizingKeilVI (\ThumbMode),style=customasmARM]{\CURPATH/Keil_thumb_O3.s}

\myindex{ARM!\Instructions!LSLS}
\myindex{ARM!\Instructions!LSLR}

Последние две инструкции \INS{LSLS} и \INS{LSRS} работают как \INS{AND reg, 0xFF}:
это аналог \CCpp-выражения $(i<<24)>>24$.

Очевидно, Keil для режима Thumb решил, что две 2-байтных инструкции это короче чем код, загружающий
константу 0xFF плюс инструкция AND.

\subsubsection{GCC для ARM64}

\lstinputlisting[caption=\NonOptimizing GCC 4.9 (ARM64),style=customasmARM]{\CURPATH/GCC_49_ARM64_O0.s}

\lstinputlisting[caption=\Optimizing GCC 4.9 (ARM64),style=customasmARM]{\CURPATH/GCC_49_ARM64_O3.s}

\subsection{Используя битовые операции}
\label{toupper_bit}

Учитывая тот факт, что 5-й бит (считая с 0-его) всегда присутствует после проверки, вычитание его это просто
сброс этого единственного бита, но точно такого же эффекта можно достичть при помощи обычного применения операции
``И'' (\myref{AND_OR_as_SUB_ADD}).

И даже проще, с исключающим ИЛИ:

\lstinputlisting[style=customc]{\CURPATH/toupper2.c}

Код близок к тому, что сгенерировал оптимизирующий GCC для предыдущего примера (\myref{toupper_GCC_O3}):

\lstinputlisting[caption=\Optimizing GCC 5.4 (x86),style=customasmx86]{\CURPATH/toupper2_GCC540_x86_O3.s}

\dots но используется \INS{XOR} вместо \INS{SUB}.

Переворачивание 5-го бита это просто перемещение \textit{курсора} в таблице \ac{ASCII} вверх/вниз на 2 ряда.

Некоторые люди говорят, что буквы нижнего/верхнего регистра были расставлены в \ac{ASCII}-таблице таким манером намеренно,
потому что:

\begin{framed}
\begin{quotation}
Very old keyboards used to do Shift just by toggling the 32 or 16 bit, depending on the key; this is why the relationship between small and capital letters in ASCII is so regular, and the relationship between numbers and symbols, and some pairs of symbols, is sort of regular if you squint at it.
\end{quotation}
\end{framed}

( Eric S. Raymond, \url{http://www.catb.org/esr/faqs/things-every-hacker-once-knew/} )

Следовательно, мы можем написать такой фрагмент кода, который просто меняет регистр букв:

\lstinputlisting[style=customc]{\CURPATH/flip_EN.c}

\subsection{Итог}

Все эти оптимизации компиляторов очень популярны в наше время и практикующий
reverse engineer обычно часто видит такие варианты кода.
}
\IT{\subsection{Rumore nello stack}
\label{noise_in_stack}

\epigraph{When one says that something seems random, what one usually
means in practice is that one cannot see any regularities in it.}{Stephen Wolfram, A New Kind of Science.}

In questo libro si fa spesso riferimento a \q{rumore} o \q{spazzatura} (garbage) nello stack o in memoria.
Da dove arrivano?
Si tratta di ciò che resta dopo l'esecuzione di altre funzioni.
Un piccolo esempio:

\lstinputlisting[style=customc]{patterns/02_stack/08_noise/st.c}

Compilando si ottiene:

\lstinputlisting[caption=\NonOptimizing MSVC 2010,style=customasmx86]{patterns/02_stack/08_noise/st.asm}

Il compilatore si lamenterà un pochino\dots

\begin{lstlisting}
c:\Polygon\c>cl st.c /Fast.asm /MD
Microsoft (R) 32-bit C/C++ Optimizing Compiler Version 16.00.40219.01 for 80x86
Copyright (C) Microsoft Corporation.  All rights reserved.

st.c
c:\polygon\c\st.c(11) : warning C4700: uninitialized local variable 'c' used
c:\polygon\c\st.c(11) : warning C4700: uninitialized local variable 'b' used
c:\polygon\c\st.c(11) : warning C4700: uninitialized local variable 'a' used
Microsoft (R) Incremental Linker Version 10.00.40219.01
Copyright (C) Microsoft Corporation.  All rights reserved.

/out:st.exe
st.obj
\end{lstlisting}

Ma quando avvieremo il programma \dots

\begin{lstlisting}
c:\Polygon\c>st
1, 2, 3
\end{lstlisting}

Oh, che cosa strana! Non abbiamo impostato il valore di alcuna variabile in \TT{f2()}.
Si tratta di valori \q{fantasma}, che si trovano ancora nello stack.

\clearpage
Carichiamo l'esempio in \olly:

\begin{figure}[H]
\centering
\myincludegraphics{patterns/02_stack/08_noise/olly1.png}
\caption{\olly: \TT{f1()}}
\label{fig:stack_noise_olly1}
\end{figure}

Quando \TT{f1()} assegna le variabili $a$, $b$ e $c$, i loro valori sono memorizzati all'indirizzo \TT{0x1FF860} e seguenti.

\clearpage
E quando viene eseguita \TT{f2()}:

\begin{figure}[H]
\centering
\myincludegraphics{patterns/02_stack/08_noise/olly2.png}
\caption{\olly: \TT{f2()}}
\label{fig:stack_noise_olly2}
\end{figure}

... $a$, $b$ e $c$ di \TT{f2()} si trovano agli stessi indirizzi!
Nessuno ha ancora sovrascritto quei valori, quindi in quel punto sono ancora intatti.
Quindi, affinchè questa strana situazione si verifichi, più funzioni devono essere chiamate una dopo l'altra e
\ac{SP} deve essere uguale ad ogni ingresso nella funzione (ovvero le funzioni devono avere lo stesso numero di argomenti).
A quel punto le variabili locali si troveranno nelle stesse posizioni nello stack.
Per riassumere, tutti i valori nello stack (e nelle celle di memoria in generale) hanno valori lasciati lì dall'esecuzione di funzioni precedenti.
Non sono letteralmente casuali, piuttosto hanno valori non predicibili.
C'è un'altra opzione?
Sarebbe possibile ripulire porzioni dello stack prima di ogni esecuzione di una funzione, ma sarebbe un lavoro extra probabilmente inutile.

\subsubsection{MSVC 2013}

L'esempio è stato compilato con MSVC 2010.
Un lettore di questo libro ha provato a compilare l'esempio con MSVC 2013, lo ha eseguito, ed ha ottenuto i 3 numeri in ordine inverso:%

\begin{lstlisting}
c:\Polygon\c>st
3, 2, 1
\end{lstlisting}

Perchè?
Ho compilato anche io l'esempio in MSVC 2013 ed ho visto questo:


\begin{lstlisting}[caption=MSVC 2013,style=customasmx86]
_a$ = -12	; size = 4
_b$ = -8	; size = 4
_c$ = -4	; size = 4
_f2	PROC

...

_f2	ENDP

_c$ = -12	; size = 4
_b$ = -8	; size = 4
_a$ = -4	; size = 4
_f1	PROC

...

_f1	ENDP
\end{lstlisting}

Contrariamente a MSVC 2010, MSVC 2013 ha allocato le variabili a/b/c nella funzione \TT{f2()} in ordine inverso.%
E ciò è del tutto corretto, perchè lo standard \CCpp non ha una regola che definisce in quale ordine le variabili locali devono essere allocate nello stack.
La ragione per cui si presenta questa differenza è che MSVC 2010 lo fa in un certo modo, mentre MSVC 2013 ha probabilmente subito modifiche all'interno del compilatore, e si comporta quindi in modo leggermente diverso.
}
\DE{\subsubsection{Struct als Menge von Werten}
Um zu veranschaulichen, dass ein struct nur eine Menge von nebeneinanderliegenden Variablen ist, überarbeiten wir unser
Beispiel, indem wir auf die Definition des \emph{tm} structs schauen:\lstref{struct_tm}.

\lstinputlisting[style=customc]{patterns/15_structs/3_tm_linux/as_array/GCC_tm2.c}

\myindex{\CStandardLibrary!localtime\_r()}
Der Pointer auf das Feld \TT{tm\_sec} wird nach \TT{localtime\_r} übergeben, d.h. an das erste Element des structs.

Der Compiler warnt uns:

\begin{lstlisting}[caption=GCC 4.7.3]
GCC_tm2.c: In function 'main':
GCC_tm2.c:11:5: warning: passing argument 2 of 'localtime_r' from incompatible pointer type [enabled by default]
In file included from GCC_tm2.c:2:0:
/usr/include/time.h:59:12: note: expected 'struct tm *' but argument is of type 'int *'
\end{lstlisting}

Trotzdem erzeugt er folgenden Code:

\lstinputlisting[caption=GCC 4.7.3,style=customasmx86]{patterns/15_structs/3_tm_linux/as_array/GCC_tm2.asm}
Dieser Code ist zum vorherigen identisch und es ist unmöglich zu sagen, ob es sich im originalen Quellcode um ein struct
oder nur um eine Menge von Variablen handelt.

Es funktioniert also, ist aber in der Praxis nicht empfehlenswert. 

Nicht optimierende Compiler legen normalerweise Variablen auf dem lokalen Stack in der Reihenfolge an, in der sie in der
Funktion deklariert wurden.

Ein Garantie dafür gibt es freilich nicht.

Andere Compiler könnten an dieser Stelle übrigens davor warnen, dass die Variablen \TT{tm\_year}, \TT{tm\_mon}, \TT{tm\_mday},
\TT{tm\_hour}, \TT{tm\_min} - nicht aber \TT{tm\_sec} - ohne Initialisierung verwendet werden.

Der Compiler weiß nicht, dass diese durch die Funktion \TT{localtime\_r()} befüllt werden.

Wir haben dieses Beispiel ausgewählt, da alle Felder im struct vom Typ \Tint sind.

Es würde nicht funktionieren, wenn die Felder 16 Bit (\TT{WORD}) groß wären, wie im Beispiel des \TT{SYSTEMTIME}
structs---\TT{GetSystemTime()} würde sie falsch befüllen (da die lokalen Variablen auf 32-Bit-Grenzen angeordnet sind).
Mehr dazu im folgenden Abschnitt: \q{\StructurePackingSectionName} (\myref{structure_packing}).

Ein struct ist also nichts als eine Menge von an einer Stelle gespeicherten Variablen.
Man kan sagen, dass das struct ein Befehl an den Compiler ist, diese Variablen an einer Stelle zu halten.
In ganz frühen Versionen von C (vor 1972) gab es übrigens gar keine structs \RitchieDevC.

Dieses Beispiel wird nicht im Debugger gezeigt, da es dem gerade gezeigten entspricht.

\subsubsection{Struct als Array aus 32-Bit-Worten}

\lstinputlisting[style=customc]{patterns/15_structs/3_tm_linux/as_array/GCC_tm3.c}
Wir können einen Pointer auf ein struct in ein Array aus \Tint{}s casten und es funktioniert.
Wir lassen dieses Beispiel zur Systemzeit 23:51:45 26-July-2014 laufen.

\begin{lstlisting}[label=GCC_tm3_output]
0x0000002D (45)
0x00000033 (51)
0x00000017 (23)
0x0000001A (26)
0x00000006 (6)
0x00000072 (114)
0x00000006 (6)
0x000000CE (206)
0x00000001 (1)
\end{lstlisting}
Die Variablen sind hier in der gleichen Reihenfolge, in der die in der Definition des structs aufgezählt
werden:\myref{struct_tm}.

Hier ist der erzeugte Code:

\lstinputlisting[caption=\Optimizing GCC
4.8.1,style=customasmx86]{patterns/15_structs/3_tm_linux/as_array/GCC_tm3_DE.lst}
Tatsächlich: der Platz auf dem lokalen Stack wird zuerst wie in struct und dann wie ein Array behandelt.

Es ist sogar möglich, die Felder des structs über diesen Pointer zu verändern.

Und wiederum ist es zweifellos ein seltsamer Weg die Dinge umzusetzen; er ist für produktiven Code definitiv nicht
empfehlenswert.

\mysubparagraph{\Exercise}
Versuchen Sie als Übung die Monatsnummer zu verändern (um 1 zu erhöhen), indem Sie das struct wie ein Array behandeln.

\subsubsection{Struct als Bytearray}
Wir können sogar noch weiter gehen. Casten wir den Pointer zu einem Bytearray und ziehen einen Dump:

\lstinputlisting[style=customc]{patterns/15_structs/3_tm_linux/as_array/GCC_tm4.c}

\begin{lstlisting}
0x2D 0x00 0x00 0x00 
0x33 0x00 0x00 0x00 
0x17 0x00 0x00 0x00 
0x1A 0x00 0x00 0x00 
0x06 0x00 0x00 0x00 
0x72 0x00 0x00 0x00 
0x06 0x00 0x00 0x00 
0xCE 0x00 0x00 0x00 
0x01 0x00 0x00 0x00 
\end{lstlisting}
Wir haben dieses Beispiel auch zur Systemzeit 23:51:45 26-July-2014 ausgeführt
\footnote{Datum und Uhrzeit sind zu Demonstrationszwecken identisch. Die Bytewerte sind modifiziert.}.
Die Werte sind genau dieselben wie im vorherigen Dump(\myref{GCC_tm3_output}) und natürlich steht das LSB vorne, da es
sich um eine Little-Endian-Architektur handelt(\myref{sec:endianness}). 

\lstinputlisting[caption=\Optimizing GCC
4.8.1,style=customasmx86]{patterns/15_structs/3_tm_linux/as_array/GCC_tm4_DE.lst}
}
\PL{\mysection{Pusta funkcja}
\label{empty_func}

Najprostszą istniejącą funkcją jest funkcja, która nic nie robi:

\lstinputlisting[caption= Kod w \CCpp,style=customc]{patterns/00_empty/1.c}

Kompilujemy!

\subsection{x86}

Dla x86 i MSVC i GCC generuje się ten sam kod:

\lstinputlisting[caption=\Optimizing GCC/MSVC (\assemblyOutput),style=customasmx86]{patterns/00_empty/1.s}

\myindex{x86!\Instructions!RET}
Tu jest tylko jedna instrukcja \RET, która zwraca zarządzanie do funkcji wywołującej \glslink{caller}{funkcję wywołującą}.

\subsection{ARM}

\lstinputlisting[caption=\OptimizingKeilVI (\ARMMode) ASM Output,style=customasmARM]{patterns/00_empty/1_Keil_ARM_O3.s}

Adres powrotu (\ac{RA}) w ARM zapisuję się nie na stosie lokalnym, a w rejestrze \ac{LR}.
Dlatego instrukcja \INS{BX LR} wykonuje skok pod ten adres, czyli zwraca zarządzanie do funkcji wywołującej.

\subsection{MIPS}

Są dwa sposoby deklaracji nazwy rejestru w MIPS. Według numeracji (od \$0 do \$31) lub nadając pseudo-nazwę (\$V0, \$A0, itd.).

Output asemblera w GCC pokazuję rejestry według numerów:

\lstinputlisting[caption=\Optimizing GCC 4.4.5 (\assemblyOutput),style=customasmMIPS]{patterns/00_empty/MIPS.s}

\dots a \IDA --- nadając pseudonazwę:

\lstinputlisting[caption=\Optimizing GCC 4.4.5 (IDA),style=customasmMIPS]{patterns/00_empty/MIPS_IDA.lst}

\myindex{MIPS!\Instructions!J}

Pierwsza instrukcja jest instrukcją skoku (J lub JR),
która zwraca zarządzanie \glslink{caller}{funkcję wywołującą}, skacząc pod adres w rejestrze \$31 (lub \$RA).

Jest to analogiczne do \ac{LR} w ARM.

Druga instrukcja to \ac{NOP}, która nic nie robi.
Na razie możemy ją zignorować.

\subsubsection{Jeszcze małe co nieco o konwecji nazewnictwa w MIPS}
Nazwy rejestrów i instrukcji w MIPS tradycyjnie są zapisywane małymi literami,
lecz my będziemy je zapisywać dużymi literami, dlatego że nazwy instrukcji i rejestrów innych 
\ac{ISA} w tej książce są zapisywane dużymi.

\subsection{Puste funkcje w praktyce}

Mimo że puste funkcje są bezużyteczne, są one dość często spotykane w niskopoziomowym kodzie.

Po pierwsze, popularne są funkcje wyprowadzające informację do logów, na przykład:

\lstinputlisting[caption=Kod w \CCpp,style=customc]{patterns/00_empty/dbg_print_EN.c}

Po drugie, popularnym sposobem na ochronę oprogramowania jest kompilacja kilku wersji: pierwsza dla legalnych konsumentów, druga - demonstracyjna. Demonstracyjna wersja może nie zawierać jakiejś ważnej funkcjonalności, naprzykład:

\lstinputlisting[caption=Kod w \CCpp,style=customc]{patterns/00_empty/demo_EN.c}

Funkcja \TT{save\_file()} może być wywołana, kiedy użytkownik klika na menu \TT{File->Save}.
Demo-wersja może zawierać wyłączony przycisk menu, ale nawet jeśli kraker go włączy, to będzie wywoływana pusta funkcja w której nie ma użytecznego kodu.
IDA oznacza takie funkcje np. w ten sposób \TT{nullsub\_00}, \TT{nullsub\_01}, itd.


}
\JA{\subsection{スタックのノイズ}
\label{noise_in_stack}

\epigraph{ある人が何かがランダムに見えると言うとき、実際には、その中に何らかの規則性を見ることができないということです}{Stephen Wolfram, A New Kind of Science.}

多くの場合、この本では\q{ノイズ}や\q{ガベージ}の値がスタックやメモリに記述されています。
彼らはどこから来たのか?
これらは、他の関数の実行後にそこに残っているものです。 
短い例:

\lstinputlisting[style=customc]{patterns/02_stack/08_noise/st.c}

コンパイルすると \dots

\lstinputlisting[caption=\NonOptimizing MSVC 2010,style=customasmx86]{patterns/02_stack/08_noise/st.asm}

コンパイラは少し不満そうです\dots

\begin{lstlisting}
c:\Polygon\c>cl st.c /Fast.asm /MD
Microsoft (R) 32-bit C/C++ Optimizing Compiler Version 16.00.40219.01 for 80x86
Copyright (C) Microsoft Corporation.  All rights reserved.

st.c
c:\polygon\c\st.c(11) : warning C4700: uninitialized local variable 'c' used
c:\polygon\c\st.c(11) : warning C4700: uninitialized local variable 'b' used
c:\polygon\c\st.c(11) : warning C4700: uninitialized local variable 'a' used
Microsoft (R) Incremental Linker Version 10.00.40219.01
Copyright (C) Microsoft Corporation.  All rights reserved.

/out:st.exe
st.obj
\end{lstlisting}

しかし、コンパイルされたプログラムを実行すると \dots

\begin{lstlisting}
c:\Polygon\c>st
1, 2, 3
\end{lstlisting}

ああ、なんて奇妙なんでしょう! 我々は\TT{f2()}に変数を設定しませんでした。
これらは\q{ゴースト}値であり、まだスタックに入っています。

\clearpage
サンプルを \olly にロードしましょう。

\begin{figure}[H]
\centering
\myincludegraphics{patterns/02_stack/08_noise/olly1.png}
\caption{\olly: \TT{f1()}}
\label{fig:stack_noise_olly1}
\end{figure}

\TT{f1()}が変数$a$、$b$、$c$を代入すると、その値はアドレス\TT{0x1FF860}に格納されます。

\clearpage
そして\TT{f2()}が実行されるとき:

\begin{figure}[H]
\centering
\myincludegraphics{patterns/02_stack/08_noise/olly2.png}
\caption{\olly: \TT{f2()}}
\label{fig:stack_noise_olly2}
\end{figure}

... \TT{f2()}の$a$、$b$、$c$は同じアドレスにあります!
誰もまだ値を上書きしていないので、その時点でまだ変更はありません。
したがって、この奇妙な状況が発生するためには、いくつかの関数を次々と呼び出さなければならず、
\ac{SP}は各関数エントリで同じでなければならない(すなわち、それらは同じ数の引数を有する)。 
次に、ローカル変数はスタック内の同じ位置に配置されます。 
要約すると、スタック(およびメモリセル)内のすべての値は、以前の関数実行から残った値を持ちます。 
彼らは厳密な意味でランダムではなく、むしろ予測不可能な値を持っています。 
別のオプションがありますか? 
各関数の実行前にスタックの一部をクリアすることはおそらく可能ですが、余計な(そして不要な)作業です。

\subsubsection{MSVC 2013}

この例はMSVC 2010によってコンパイルされました。
しかし、この本の読者は、このサンプルをMSVC 2013でコンパイルして実行し、3つの数字がすべて逆の結果になるでしょう。

\begin{lstlisting}
c:\Polygon\c>st
3, 2, 1
\end{lstlisting}

どうして?
私もMSVC 2013でこの例をコンパイルし、見てみました。

\begin{lstlisting}[caption=MSVC 2013,style=customasmx86]
_a$ = -12	; size = 4
_b$ = -8	; size = 4
_c$ = -4	; size = 4
_f2	PROC

...

_f2	ENDP

_c$ = -12	; size = 4
_b$ = -8	; size = 4
_a$ = -4	; size = 4
_f1	PROC

...

_f1	ENDP
\end{lstlisting}

MSVC 2010とは異なり、MSVC 2013は関数\TT{f2()}のa/b/c変数を逆順に割り当てました。
これは完全に正しい動作です。 \CCpp 標準にはルールがありません。ローカル変数をローカルスタックに割り当てる必要があれば、どういう順番でもよいのです。 
理由の違いは、MSVC 2010にはその方法があり、MSVC 2013はおそらくコンパイラの心臓部で何かが変わったと考えられるからです。
}

\input{patterns/02_stack/exercises}


