\subsubsection{関数のリターンアドレスを保存する}

\myparagraph{x86}

\myindex{x86!\Instructions!CALL}
\CALL 命令で別の関数を呼び出すと、 \CALL 命令の直後のポイントのアドレスがスタックに保存され、 
\CALL オペランドのアドレスへの無条件ジャンプが実行されます。

\myindex{x86!\Instructions!PUSH}
\myindex{x86!\Instructions!JMP}
\CALL 命令は、PUSHの\INS{PUSH address\_after\_call / JMP operand}命令対に相当する。

\RET はスタックから値を取り出し、ジャンプします。これは\TT{POP tmp / JMP tmp}命令の対に相当します。

\myindex{\Stack!\MLStackOverflow}
\myindex{\Recursion}
スタックのオーバーフローは簡単です。 永遠の再帰を実行するだけです:


\begin{lstlisting}[style=customc]
void f()
{
	f();
};
\end{lstlisting}

MSVC 2008が問題をレポートします:

\begin{lstlisting}
c:\tmp6>cl ss.cpp /Fass.asm
Microsoft (R) 32-bit C/C++ Optimizing Compiler Version 15.00.21022.08 for 80x86
Copyright (C) Microsoft Corporation.  All rights reserved.

ss.cpp
c:\tmp6\ss.cpp(4) : warning C4717: 'f' : recursive on all control paths, function will cause runtime stack overflow
\end{lstlisting}

\dots しかし、正しいコードを生成します。

\lstinputlisting[style=customasmx86]{patterns/02_stack/1.asm}

\dots また、コンパイラ最適化(\TT{\Ox}オプション)を有効にすると、最適化されたコードはスタックをオーバーフローせず、
代わりに\emph{正しく}\footnote{ここの皮肉}動作します。

\lstinputlisting[style=customasmx86]{patterns/02_stack/2.asm}

GCC 4.4.1はどちらの場合も問題の警告を出さずに同様のコードを生成します。

\myparagraph{ARM}

\myindex{ARM!\Registers!Link Register}
また、ARMプログラムはスタックを使用してリターンアドレスを保存しますが、別の方法でスタックを使用します。 
\q{\HelloWorldSectionName}~(\myref{sec:hw_ARM})で述べたように、\ac{RA}は\ac{LR}(\gls{link register})に保存されます。 
ただし、別の関数を呼び出してもう一度\ac{LR}レジスタを使用する必要がある場合は、その値を保存する必要があります。 
\myindex{Function prologue}
通常、関数プロローグに保存されます。

\myindex{ARM!\Instructions!PUSH}
\myindex{ARM!\Instructions!POP}
多くの場合、\INS{PUSH {R4-R7,LR}}のような命令が、エピローグで\INS{POP {R4-R7,PC}}とともに見られます。
したがって、関数で使用されるレジスタ値は、\ac{LR}を含めてスタックに保存されます。

\myindex{ARM!Leaf function}
それにもかかわらず、ある関数が他の関数を呼び出すことがなければ、\ac{RISC}の用語ではそれを
\emph{\gls{leaf function}}\footnote{\href{http://go.yurichev.com/17064}{infocenter.arm.com/help/index.jsp?topic=/com.arm.doc.faqs/ka13785.html}}と呼びます。
その結果、リーフ関数は\ac{LR}レジスタを保存しません(\ac{LR}レジスタを変更しないため)。 
このような関数が小さく、少数のレジスタを使用する場合は、スタックをまったく使用しないことがあります。 
したがって、スタックを使用せずにリーフ関数を呼び出すことができます。
\footnote{いくつかの時間前、PDP-11とVAXでは、CALL命令(他の関数を呼び出す)は高価でした。 
実行時間の50%までが費やされる可能性があるため、小さな機能を多数持つことは\gls{anti-pattern} \InSqBrackets{\TAOUP Chapter 4, Part II}}
これは、外部RAMがスタックに使用されないため、古いx86マシンよりも高速になる可能性があります。これは、スタックのメモリがまだ割り当てられていない状況 または利用できません。

リーフ関数のいくつかの例:
\myref{ARM_leaf_example1}, \myref{ARM_leaf_example2}, 
\myref{ARM_leaf_example3}, \myref{ARM_leaf_example4}, \myref{ARM_leaf_example5},
\myref{ARM_leaf_example6}, \myref{ARM_leaf_example7}, \myref{ARM_leaf_example10}.
