\subsubsection{ARM}

\myindex{Linux}

\TT{O\_CREAT}ビットはLinuxカーネル3.8.0ではチェックは異なります。

\begin{lstlisting}[caption=linux kernel 3.8.0,style=customc]
struct file *do_filp_open(int dfd, struct filename *pathname,
		const struct open_flags *op)
{
...
	filp = path_openat(dfd, pathname, &nd, op, flags | LOOKUP_RCU);
...
}

static struct file *path_openat(int dfd, struct filename *pathname,
		struct nameidata *nd, const struct open_flags *op, int flags)
{
...
	error = do_last(nd, &path, file, op, &opened, pathname);
...
}


static int do_last(struct nameidata *nd, struct path *path,
		   struct file *file, const struct open_flags *op,
		   int *opened, struct filename *name)
{
...
	if (!(open_flag & O_CREAT)) {
    ...
		error = lookup_fast(nd, path, &inode);
    ...
	} else {
    ...
		error = complete_walk(nd);
        }
...
}
\end{lstlisting}

ARMモード用にコンパイルされたカーネルが \IDA でどのように見えるかは次のとおりです。

\lstinputlisting[caption=do\_last() from vmlinux (IDA),style=customasmARM]{patterns/14_bitfields/1_check/IDA.lst}

\myindex{ARM!\Instructions!TST}
\myindex{x86!\Instructions!TEST}
\TT{TST}は、x86の \TEST 命令に似ています。 
\TT{lookup\_fast()}はあるケースでは実行され、もう一つのケースでは\TT{complete\_walk()}が実行されるという事実によって、
このコードフラグメントを視覚的に\q{発見}することができます。 
\TT{do\_last()}関数のソースコードに相当します。 
\TT{O\_CREAT}マクロはここでも\TT{0x40}と同じです。
