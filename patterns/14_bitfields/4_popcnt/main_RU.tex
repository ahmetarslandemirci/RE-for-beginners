\subsection{Подсчет выставленных бит}

Вот этот несложный пример иллюстрирует функцию, считающую количество бит-единиц во входном значении.

Эта операция также называется \q{population count}\footnote{современные x86-процессоры (поддерживающие SSE4) даже имеют инструкцию POPCNT для этого}.

\lstinputlisting[style=customc]{patterns/14_bitfields/4_popcnt/shifts.c}

В этом цикле счетчик итераций $i$ считает от 0 до 31, а $1 \ll i$ будет от 1 до \TT{0x80000000}. 
Описывая это словами, можно сказать 
\emph{сдвинуть единицу на $n$ бит влево}.
Т.е. в некотором смысле, выражение $1 \ll i$ последовательно выдает все возможные позиции бит в 32-битном числе. 
Освободившийся бит справа всегда обнуляется.

Вот таблица всех возможных значений $1 \ll i$ для $i=0 \ldots 31$:

%\small
\label{2n_numbers_table}
\begin{center}
\begin{tabular}{ | l | l | l | l | }
\hline
\HeaderColor Выражение & 
\HeaderColor Степень двойки & 
\HeaderColor Десятичная форма & 
\HeaderColor Шестнадцатеричная \\
\hline
$1 \ll 0$ & $2^{0}$ & 1 & 1 \\
\hline
$1 \ll 1$ & $2^{1}$ & 2 & 2 \\
\hline
$1 \ll 2$ & $2^{2}$ & 4 & 4 \\
\hline
$1 \ll 3$ & $2^{3}$ & 8 & 8 \\
\hline
$1 \ll 4$ & $2^{4}$ & 16 & 0x10 \\
\hline
$1 \ll 5$ & $2^{5}$ & 32 & 0x20 \\
\hline
$1 \ll 6$ & $2^{6}$ & 64 & 0x40 \\
\hline
$1 \ll 7$ & $2^{7}$ & 128 & 0x80 \\
\hline
$1 \ll 8$ & $2^{8}$ & 256 & 0x100 \\
\hline
$1 \ll 9$ & $2^{9}$ & 512 & 0x200 \\
\hline
$1 \ll 10$ & $2^{10}$ & 1024 & 0x400 \\
\hline
$1 \ll 11$ & $2^{11}$ & 2048 & 0x800 \\
\hline
$1 \ll 12$ & $2^{12}$ & 4096 & 0x1000 \\
\hline
$1 \ll 13$ & $2^{13}$ & 8192 & 0x2000 \\
\hline
$1 \ll 14$ & $2^{14}$ & 16384 & 0x4000 \\
\hline
$1 \ll 15$ & $2^{15}$ & 32768 & 0x8000 \\
\hline
$1 \ll 16$ & $2^{16}$ & 65536 & 0x10000 \\
\hline
$1 \ll 17$ & $2^{17}$ & 131072 & 0x20000 \\
\hline
$1 \ll 18$ & $2^{18}$ & 262144 & 0x40000 \\
\hline
$1 \ll 19$ & $2^{19}$ & 524288 & 0x80000 \\
\hline
$1 \ll 20$ & $2^{20}$ & 1048576 & 0x100000 \\
\hline
$1 \ll 21$ & $2^{21}$ & 2097152 & 0x200000 \\
\hline
$1 \ll 22$ & $2^{22}$ & 4194304 & 0x400000 \\
\hline
$1 \ll 23$ & $2^{23}$ & 8388608 & 0x800000 \\
\hline
$1 \ll 24$ & $2^{24}$ & 16777216 & 0x1000000 \\
\hline
$1 \ll 25$ & $2^{25}$ & 33554432 & 0x2000000 \\
\hline
$1 \ll 26$ & $2^{26}$ & 67108864 & 0x4000000 \\
\hline
$1 \ll 27$ & $2^{27}$ & 134217728 & 0x8000000 \\
\hline
$1 \ll 28$ & $2^{28}$ & 268435456 & 0x10000000 \\
\hline
$1 \ll 29$ & $2^{29}$ & 536870912 & 0x20000000 \\
\hline
$1 \ll 30$ & $2^{30}$ & 1073741824 & 0x40000000 \\
\hline
$1 \ll 31$ & $2^{31}$ & 2147483648 & 0x80000000 \\
\hline
\end{tabular}
\end{center}
%\normalsize

Это числа-константы (битовые маски), которые крайне часто попадаются в практике reverse engineer-а, 
и их нужно уметь распознавать.

Числа в десятичном виде, до 65536 и числа в шестнадцатеричном виде легко запомнить и так.
А числа в десятичном виде после 65536, пожалуй, заучивать не нужно.

Эти константы очень часто используются для определения отдельных бит как флагов.

Например, это из файла \TT{ssl\_private.h} из исходников Apache 2.4.6:

\begin{lstlisting}[style=customc]
/**
 * Define the SSL options
 */
#define SSL_OPT_NONE           (0)
#define SSL_OPT_RELSET         (1<<0)
#define SSL_OPT_STDENVVARS     (1<<1)
#define SSL_OPT_EXPORTCERTDATA (1<<3)
#define SSL_OPT_FAKEBASICAUTH  (1<<4)
#define SSL_OPT_STRICTREQUIRE  (1<<5)
#define SSL_OPT_OPTRENEGOTIATE (1<<6)
#define SSL_OPT_LEGACYDNFORMAT (1<<7)
\end{lstlisting}

Вернемся назад к нашему примеру.

Макрос \TT{IS\_SET} проверяет наличие этого бита в $a$.

\myindex{x86!\Instructions!AND}
Макрос \TT{IS\_SET} на самом деле это операция логического И (\emph{AND}) 
и она возвращает 0 если бита там нет, 
либо эту же битовую маску, если бит там есть. 
В \CCpp, конструкция \TT{if()} срабатывает, если выражение внутри её не ноль, пусть хоть 123456, 
поэтому все будет работать.

% subsections

\subsubsection{x86}

\myparagraph{\NonOptimizing MSVC}

Это дает в итоге (MSVC 2010):

\lstinputlisting[caption=MSVC 2010,style=customasmx86]{patterns/08_switch/1_few/few_msvc.asm}

Наша функция с оператором switch(), с небольшим количеством вариантов, 
это практически аналог подобной конструкции:

\lstinputlisting[label=switch_few_ifelse,style=customc]{patterns/08_switch/1_few/few_analogue.c}

\myindex{\CLanguageElements!switch}
\myindex{\CLanguageElements!if}
Когда вариантов немного и мы видим подобный код, невозможно сказать с уверенностью, был ли
в оригинальном исходном коде switch(), либо просто набор операторов if().

\myindex{\SyntacticSugar}
То есть, switch() это синтаксический сахар для большого количества вложенных проверок 
при помощи if().

В самом выходном коде ничего особо нового, 
за исключением того, что компилятор зачем-то 
перекладывает входящую переменную ($a$) во временную в локальном стеке \TT{v64}\footnote{Локальные переменные в стеке с префиксом \TT{tv}~--- 
так MSVC называет внутренние переменные для своих нужд}.

Если скомпилировать это при помощи GCC 4.4.1, то будет почти то же самое, даже с максимальной оптимизацией (ключ \Othree).

\myparagraph{\Optimizing MSVC}

% TODO separate various kinds of \TT
% idea: enclose command lines in a specific environment, like \cmdline{} 
% assembly instructions in \asm{} (now both \TT and \q{} are used),
% variables in,  like \var{}
% messages (string constants) in something else, like \strconst
% to separate them all. Now they all use \TT, which is not best
% \INS{} for all instructions including operands? --DY

Попробуем включить оптимизацию кодегенератора MSVC (\Ox): \TT{cl 1.c /Fa1.asm /Ox}

\label{JMP_instead_of_RET}
\lstinputlisting[caption=MSVC,style=customasmx86]{patterns/08_switch/1_few/few_msvc_Ox.asm}

Вот здесь уже всё немного по-другому, причем не без грязных трюков.

\myindex{x86!\Instructions!JZ}
\myindex{x86!\Instructions!JE}
\myindex{x86!\Instructions!SUB}
Первое: \TT{а} помещается в \EAX и от него отнимается 0. Звучит абсурдно, но нужно это для того, чтобы проверить, 
0 ли в \EAX был до этого? Если да, то выставится флаг \ZF (что означает, что результат вычитания 0 от числа 
стал 0) и первый условный переход \JE (\emph{Jump if Equal} или его синоним \JZ~--- \emph{Jump if Zero}) 
сработает на метку \TT{\$LN4@f}, где выводится сообщение \TT{'zero'}.
Если первый переход не сработал, от значения отнимается по единице, 
и если на какой-то стадии в результате образуется 0, то сработает соответствующий переход.

И в конце концов, если ни один из условных переходов не сработал, управление передается \printf
со строковым аргументом \TT{'something unknown'}.

\label{jump_to_last_printf}
\myindex{\Stack}
Второе: мы видим две, мягко говоря, необычные вещи: указатель на сообщение помещается в переменную $a$, 
и затем \printf вызывается не через \CALL, а через \JMP. Объяснение этому простое. 
Вызывающая функция заталкивает в стек некоторое значение и через \CALL вызывает нашу функцию. 
\CALL в свою очередь заталкивает в стек адрес возврата (\ac{RA}) и делает безусловный переход на адрес нашей функции. 
Наша функция в самом начале (да и в любом её месте, потому что в теле функции нет ни одной инструкции, 
которая меняет что-то в стеке или в \ESP) имеет следующую разметку стека:

\begin{itemize}
\item\ESP --- хранится \ac{RA}
\item\TT{ESP+4} --- хранится значение $a$ 
\end{itemize}

С другой стороны, чтобы вызвать \printf, нам нужна почти такая же разметка стека, 
только в первом аргументе нужен указатель на строку. Что, собственно, этот код и делает.

Он заменяет свой первый аргумент на адрес строки, и затем передает управление \printf, как если бы вызвали не 
нашу функцию \ttf, а сразу \printf. 
\printf выводит некую строку на \gls{stdout}, затем исполняет инструкцию \RET, 
которая из стека достает \ac{RA} и управление передается в ту функцию, 
которая вызывала \ttf, минуя при этом конец функции \ttf.

\myindex{\CStandardLibrary!longjmp()}
\newcommand{\URLSJ}{\href{http://go.yurichev.com/17121}{wikipedia}}
% TODO \myref{}
Всё это возможно, потому что \printf вызывается в \ttf в самом конце. 
Всё это чем-то даже похоже на \TT{longjmp()}\footnote{\URLSJ}.
И всё это, разумеется, сделано для экономии времени исполнения.

Похожая ситуация с компилятором для ARM описана в секции \q{\PrintfSeveralArgumentsSectionName}~(\myref{ARM_B_to_printf}).

\clearpage
\myparagraph{MSVC + \olly}
\myindex{\olly}

2 пары 32-битных слов обведены в стеке красным.
Каждая пара --- это числа двойной точности в формате IEEE 754, переданные из \main.

Видно, как первая \FLD загружает значение 1,2 из стека и помещает в регистр \ST{0}:

\begin{figure}[H]
\centering
\myincludegraphics{patterns/12_FPU/1_simple/olly1.png}
\caption{\olly: первая \FLD исполнилась}
\label{fig:FPU_simple_olly_1}
\end{figure}

Из-за неизбежных ошибок конвертирования числа из 64-битного IEEE 754 в 80-битное (внутреннее в FPU),
мы видим здесь 1,1999\ldots, что очень близко к 1,2.

Прямо сейчас \EIP указывает на следующую инструкцию (\FDIV), загружающую константу двойной точности 
из памяти.

Для удобства, \olly показывает её значение: 3,14.

\clearpage
Трассируем дальше. 
\FDIV исполнилась, теперь \ST{0} содержит 0,382\ldots
(\gls{quotient}):

\begin{figure}[H]
\centering
\myincludegraphics{patterns/12_FPU/1_simple/olly2.png}
\caption{\olly: \FDIV исполнилась}
\label{fig:FPU_simple_olly_2}
\end{figure}

\clearpage
Третий шаг: вторая \FLD 
исполнилась, загрузив в \ST{0} 3,4 (мы видим приближенное число 3,39999\ldots): 

\begin{figure}[H]
\centering
\myincludegraphics{patterns/12_FPU/1_simple/olly3.png}
\caption{\olly: вторая \FLD исполнилась}
\label{fig:FPU_simple_olly_3}
\end{figure}

В это время \gls{quotient} \emph{провалилось} 
в \ST{1}.
\EIP указывает на следующую инструкцию: \FMUL. 
Она загружает константу 4,1 из памяти, так что \olly тоже показывает её здесь.

\clearpage
Затем: \FMUL исполнилась, теперь в \ST{0} произведение:

\begin{figure}[H]
\centering
\myincludegraphics{patterns/12_FPU/1_simple/olly4.png}
\caption{\olly: \FMUL исполнилась}
\label{fig:FPU_simple_olly_4}
\end{figure}

\clearpage
Затем: \FADDP исполнилась, теперь в \ST{0} сумма, а \ST{1} очистился:

\begin{figure}[H]
\centering
\myincludegraphics{patterns/12_FPU/1_simple/olly5.png}
\caption{\olly: \FADDP исполнилась}
\label{fig:FPU_simple_olly_5}
\end{figure}

Сумма остается в \ST{0} потому что функция возвращает результат своей работы через \ST{0}.

Позже \main возьмет это значение оттуда.

Мы также видим кое-что необычное: значение 13,93\ldots теперь находится в \ST{7}.

Почему?

\label{FPU_is_rather_circular_buffer}
Мы читали в этой книге, что регистры в \ac{FPU} представляют собой стек: \myref{FPU_is_stack}. 
Но это упрощение.
Представьте, если бы \emph{в железе} было бы так, как описано. Тогда при каждом заталкивании (или выталкивании) в стек,
все остальные 7 значений нужно было бы передвигать (или копировать) в соседние регистры, 
а это слишком затратно.

Так что в реальности у
\ac{FPU} есть просто 8 регистров и указатель (называемый \GTT{TOP}), содержащий номер регистра,
который в текущий момент является \q{вершиной стека}.

При заталкивании значения в стек регистр \GTT{TOP} меняется, и указывает на свободный регистр. 
Затем значение записывается в этот регистр.

При выталкивании значения из стека процедура обратная. Однако освобожденный регистр не обнуляется
(наверное, можно было бы сделать, чтобы обнулялся, но это лишняя работа и работало бы медленнее).
Так что это мы здесь и видим. 
Можно сказать, что \FADDP сохранила сумму, а затем вытолкнула один элемент.

Но в реальности, эта инструкция сохранила сумму и затем передвинула регистр \GTT{TOP}.

Было бы ещё точнее сказать, что регистры \ac{FPU} представляют собой кольцевой буфер.




\subsubsection{x64}
\label{subsec:popcnt}

Немного изменим пример, расширив его до 64-х бит:

\lstinputlisting[label=popcnt_x64_example,style=customc]{patterns/14_bitfields/4_popcnt/shifts64.c}

\myparagraph{\NonOptimizing GCC 4.8.2}

Пока всё просто.

\lstinputlisting[caption=\NonOptimizing GCC 4.8.2,style=customasmx86]{patterns/14_bitfields/4_popcnt/shifts64_GCC_O0_RU.s}

\myparagraph{\Optimizing GCC 4.8.2}

\lstinputlisting[caption=\Optimizing GCC 4.8.2,numbers=left,label=shifts64_GCC_O3,style=customasmx86]{patterns/14_bitfields/4_popcnt/shifts64_GCC_O3_RU.s}

Код более лаконичный, но содержит одну необычную вещь.
Во всех примерах, что мы пока видели, инкремент значения переменной \q{rt} происходит после сравнения 
определенного бита с единицей, но здесь \q{rt} увеличивается на 1 до этого (строка 6), записывая новое значение
в регистр \EDX.

Затем, если последний бит был 1, инструкция \CMOVNE\footnote{Conditional MOVe if Not Equal (\MOV если не равно)}
(которая синонимична \CMOVNZ\footnote{Conditional MOVe if Not Zero (\MOV если не ноль)}) \emph{фиксирует} 
новое значение \q{rt}
копируя значение из \EDX (\q{предполагаемое значение rt}) 
в \EAX (\q{текущее rt} которое будет возвращено в конце функции).
Следовательно, инкремент происходит на каждом шаге цикла, т.е. 64 раза, вне всякой связи с входным
значением.

Преимущество этого кода в том, что он содержит только один условный переход (в конце цикла) вместо
двух (пропускающий инкремент \q{rt} и ещё одного в конце цикла).

И это может работать быстрее на современных CPU с предсказателем переходов: \myref{branch_predictors}.

\label{FATRET}
\myindex{x86!\Instructions!FATRET}
Последняя инструкция это \INS{REP RET} (опкод \TT{F3 C3}) 
которая также называется \INS{FATRET} в MSVC.
Это оптимизированная версия \RET, рекомендуемая AMD для вставки в конце функции, если \RET идет
сразу после условного перехода: 
\InSqBrackets{\AMDOptimization p.15}
\footnote{Больше об этом: \url{http://go.yurichev.com/17328}}.

\myparagraph{\Optimizing MSVC 2010}

\lstinputlisting[caption=\Optimizing MSVC 2010,style=customasmx86]{patterns/14_bitfields/4_popcnt/MSVC_2010_x64_Ox_RU.asm}

\myindex{x86!\Instructions!ROL}
Здесь используется инструкция \ROL вместо 
\SHL, которая на самом деле \q{rotate left} (прокручивать влево) 
вместо \q{shift left} (сдвиг влево),
но здесь, в этом примере, она работает так же как и  \TT{SHL}.

Об этих \q{прокручивающих} инструкциях больше читайте здесь: \myref{ROL_ROR}.

\Reg{8} здесь считает от 64 до 0. 
Это как бы инвертированная переменная $i$.

Вот таблица некоторых регистров в процессе исполнения:

\begin{center}
\begin{tabular}{ | l | l | }
\hline
\HeaderColor RDX & \HeaderColor R8 \\
\hline
0x0000000000000001 & 64 \\
\hline
0x0000000000000002 & 63 \\
\hline
0x0000000000000004 & 62 \\
\hline
0x0000000000000008 & 61 \\
\hline
... & ... \\
\hline
0x4000000000000000 & 2 \\
\hline
0x8000000000000000 & 1 \\
\hline
\end{tabular}
\end{center}

\myindex{x86!\Instructions!FATRET}
В конце видим инструкцию \INS{FATRET}, которая была описана здесь: \myref{FATRET}.

\myparagraph{\Optimizing MSVC 2012}

\lstinputlisting[caption=\Optimizing MSVC 2012,style=customasmx86]{patterns/14_bitfields/4_popcnt/MSVC_2012_x64_Ox_RU.asm}

\myindex{\CompilerAnomaly}
\label{MSVC2012_anomaly}
\Optimizing MSVC 2012 делает почти то же самое что и оптимизирующий MSVC 2010, но почему-то он генерирует 2 идентичных тела цикла и счетчик цикла теперь 32
вместо 64.
Честно говоря, нельзя сказать, почему. Какой-то трюк с оптимизацией? Может быть, телу цикла лучше быть
немного длиннее?

Так или иначе, такой код здесь уместен, чтобы показать, что результат компилятора
иногда может быть очень странный и нелогичный, но прекрасно работающий, конечно же.


\subsubsection{ARM + \OptimizingXcodeIV (\ARMMode)}

\lstinputlisting[caption=\OptimizingXcodeIV (\ARMMode),label=ARM_leaf_example4,style=customasmARM]{patterns/14_bitfields/4_popcnt/ARM_Xcode_O3_RU.lst}

\myindex{ARM!\Instructions!TST}
\TST это то же что и \TEST в x86.

\myindex{ARM!Optional operators!LSL}
\myindex{ARM!Optional operators!LSR}
\myindex{ARM!Optional operators!ASR}
\myindex{ARM!Optional operators!ROR}
\myindex{ARM!Optional operators!RRX}
\myindex{ARM!\Instructions!MOV}
\myindex{ARM!\Instructions!TST}
\myindex{ARM!\Instructions!CMP}
\myindex{ARM!\Instructions!ADD}
\myindex{ARM!\Instructions!SUB}
\myindex{ARM!\Instructions!RSB}
Как уже было указано~(\myref{shifts_in_ARM_mode}),
в режиме ARM нет отдельной инструкции для сдвигов.

Однако, модификаторами 
LSL (\emph{Logical Shift Left}), 
LSR (\emph{Logical Shift Right}), 
ASR (\emph{Arithmetic Shift Right}), 
ROR (\emph{Rotate Right}) и
RRX (\emph{Rotate Right with Extend}) можно дополнять некоторые инструкции, такие как \MOV, \TST,
\CMP, \ADD, \SUB, \RSB\footnote{\DataProcessingInstructionsFootNote}.

Эти модификаторы указывают, как сдвигать второй операнд, и на сколько.

\myindex{ARM!\Instructions!TST}
\myindex{ARM!Optional operators!LSL}
Таким образом, инструкция  \TT{\q{TST R1, R2,LSL R3}} здесь работает как $R1 \land (R2 \ll R3)$.

\subsubsection{ARM + \OptimizingXcodeIV (\ThumbTwoMode)}

\myindex{ARM!\Instructions!LSL.W}
\myindex{ARM!\Instructions!LSL}
Почти такое же, только здесь применяется пара инструкций \INS{LSL.W}/\TST вместо одной \TST,
ведь в режиме Thumb нельзя добавлять модификатор \LSL прямо в \TST.

\begin{lstlisting}[label=ARM_leaf_example5,style=customasmARM]
                MOV             R1, R0
                MOVS            R0, #0
                MOV.W           R9, #1
                MOVS            R3, #0
loc_2F7A
                LSL.W           R2, R9, R3
                TST             R2, R1
                ADD.W           R3, R3, #1
                IT NE
                ADDNE           R0, #1
                CMP             R3, #32
                BNE             loc_2F7A
                BX              LR
\end{lstlisting}

\subsubsection{ARM64 + \Optimizing GCC 4.9}

Возьмем 64-битный пример, который уже был здесь использован: \myref{popcnt_x64_example}.

\lstinputlisting[caption=\Optimizing GCC (Linaro) 4.8,style=customasmARM]{patterns/14_bitfields/4_popcnt/ARM64_GCC_O3_RU.s}
Результат очень похож на тот, что GCC сгенерировал для x64: \myref{shifts64_GCC_O3}.

\myindex{ARM!\Instructions!CSEL}
Инструкция \CSEL это \q{Conditional SELect} (выбор при условии). 
Она просто выбирает одну из переменных, в зависимости от флагов выставленных \TST и копирует значение в регистр \RegW{2}, содержащий переменную \q{rt}.

\subsubsection{ARM64 + \NonOptimizing GCC 4.9}

И снова будем использовать 64-битный пример, который мы использовали ранее: \myref{popcnt_x64_example}.
Код более многословный, как обычно.

\lstinputlisting[caption=\NonOptimizing GCC (Linaro) 4.8,style=customasmARM]{patterns/14_bitfields/4_popcnt/ARM64_GCC_O0_RU.s}


\subsubsection{MIPS}

\lstinputlisting[caption=\Optimizing GCC 4.4.5 (IDA),style=customasmMIPS]{patterns/08_switch/1_few/MIPS_O3_IDA_RU.lst}

\myindex{MIPS!\Instructions!JR}

Функция всегда заканчивается вызовом \puts, так что здесь мы видим переход на \puts (\INS{JR}: \q{Jump Register})
вместо перехода с сохранением \ac{RA} (\q{jump and link}).

Мы говорили об этом ранее: \myref{JMP_instead_of_RET}.

\myindex{MIPS!Load delay slot}
Мы также часто видим NOP-инструкции после \INS{LW}.
Это \q{load delay slot}: ещё один \emph{delay slot} в MIPS.
\myindex{MIPS!\Instructions!LW}
Инструкция после \INS{LW} может исполняться в тот момент, когда \INS{LW} загружает значение из памяти.

Впрочем, следующая инструкция не должна использовать результат \INS{LW}.

Современные MIPS-процессоры ждут, если следующая инструкция использует результат \INS{LW}, так что всё это уже
устарело, но GCC всё еще добавляет NOP-ы для более старых процессоров.

Вообще, это можно игнорировать.



