\mysection{Una funzione vuota}
\label{empty_func}

La funzione più semplice è sicuramente quella che non fa niente:

\lstinputlisting[caption=\EN{\CCpp Code},style=customc]{patterns/00_empty/1.c}

Compiliamola!

\subsection{x86}

Questo è quello che i compilatori GCC e MSVC producono per una piattaforma x86:

\lstinputlisting[caption=\Optimizing GCC/MSVC (\assemblyOutput),style=customasmx86]{patterns/00_empty/1.s}

\myindex{x86!\Instructions!RET}
C'è solo un'istruzione: \RET, la quale ritorna l'esecuzione al \gls{caller}.

\subsection{ARM}

\lstinputlisting[caption=\OptimizingKeilVI (\ARMMode) \assemblyOutput,style=customasmARM]{patterns/00_empty/1_Keil_ARM_O3.s}

L-indirizzo di ritorno non viene salvato nello stack locale nella \ac{ISA} ARM, ma invece nel registro link,
quindi l-istruzione \INS{BX LR} causa l-esecuzione di un salto (jump) a quell'indirizzo---effettivamente ritornando l'esecuzione
al \gls{caller}.

\subsection{MIPS}

Esistono due convenzioni utilizzate nel mondo MIPS quando vengono chiamati i registri:
per numero (da \$0 a \$31) o per pseudonimo (\$V0, \$A0, etc.).

L'outputo in assembly di GCC qua sotto elenca i registri per numero:

\lstinputlisting[caption=\Optimizing GCC 4.4.5 (\assemblyOutput),style=customasmMIPS]{patterns/00_empty/MIPS.s}

\dots mentre \IDA utilizza gli pseudonimi:

\lstinputlisting[caption=\Optimizing GCC 4.4.5 (IDA),style=customasmMIPS]{patterns/00_empty/MIPS_IDA.lst}

\myindex{MIPS!\Instructions!J}
La prima istruzione è un salto (J or JR) che ritorna il flusso di esecuzione al \gls{caller},
saltando all'indirizzo contenuto nel registro \$31 (o \$RA).

Questo è il registro analogo a \ac{LR} in ARM.

La seconda istruzione è \ac{NOP}, che non fa niente.
Per il momento possiamo ignorarla.

\subsubsection{Una nota sulle istruzioni MIPS ed i nomi dei registri}

I nomi dei registri e delle istruzioni nel mondo MIPS sono tradizionalmente scritti in minuscolo.
Tuttavia, per uniformità, questo libro manterrà l'utilizzo del maiuscolo,
che è la convenzione utilizzata in tutti gli altri \ac{ISA} mostrati in questo libro.

\subsection{Le funzioni vuote in pratica}

Anche se le funzioni vuote sembrano inutili, sono abbastanza utilizzate nel codice a basso livello.

Prima di tutto, sono abbastanza popolari nelle funzioni per debug, come questa:

\lstinputlisting[caption=\CCpp Code,style=customc]{patterns/00_empty/dbg_print_EN.c}

In una build non di debug (come in una ``release''), \TT{\_DEBUG} non è definito,
quindi la funzione \TT{dbg\_print()}, nonostante venga ancora chiamata durante l'esecuzione,
sarà vuota.

Similmente, un metodo popolare per la protezione del software è di creare una build per gli acquirenti regolari, ed una build di demo.
In una build di demo possono mancare alcune funzionalità importanti, come in questo esempio:

\lstinputlisting[caption=\CCpp Code,style=customc]{patterns/00_empty/demo_EN.c}

La funzione \TT{save\_file()} può essere chiamata quando l'utente fa click sul menu \TT{File->Salva}.
La versione demo può essere distribuita con questa voce di menu disattivata, ma anche se un cracker la riattiva,
verrà chiamata semplicemente una funzione vuota che non contiene del codice utile.

IDA segnala queste funzioni con dei nomi come \TT{nullsub\_00}, \TT{nullsub\_01}, etc.
