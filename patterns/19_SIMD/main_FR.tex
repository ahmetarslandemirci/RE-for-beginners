\mysection{SIMD}

\label{SIMD_x86}
\ac{SIMD} est un acronyme: \emph{Single Instruction, Multiple Data} (simple instruction,
multiple données).

Comme son nom le laisse entendre, cela traite des données multiples avec une seule
instruction.

Comme le \ac{FPU}, ce sous-système du \ac{CPU} ressemble à un processeur séparé à
l'intérieur du x86.

\myindex{x86!MMX}

SIMD a commencé avec le MMX en x86. 8 nouveaux registres apparurent: MM0-MM7.

Chaque registre MMX contient 2 valeurs 32-bit, 4 valeurs 16-bit ou 8 octets.
Par exemple, il est possible d'ajouter 8 valeurs 8-bit (octets) simultanément en
ajoutant deux valeurs dans des registres MMX.

Un exemple simple est un éditeur graphique qui représente une image comme un tableau
à deux dimensions.
Lorsque l'utilisateur change la luminosité de l'image, l'éditeur doit ajouter ou
soustraire un coefficient à/de chaque valeur du pixel.
Dans un soucis de concision, si l'on dit que l'image est en niveau de gris et que
chaque pixel est défini par un octet de 8-bit, alors il est possible de changer la
luminosité de 8 pixels simultanément.

À propos, c'est la raison pour laquelle les instructions de \emph{saturation} sont
présentes en SIMD.

Lorsque l'utilisateur change la luminosité dans l'éditeur graphique, les dépassements
% TODO: meilleure traduction underflow overflow?
au dessus ou en dessous ne sont pas souhaitables, donc il y a des instructions d'addition
en SIMD qui n'additionnent pas si la valeur maximum est atteinte, etc.

Lorsque le MMX est apparu, ces registres étaient situés dans les registres du FPU.
Il était seulement possible d'utiliser soit le FPU ou soit le MMX. On peut penser
qu'Intel économisait des transistors, mais en fait, la raison d'une telle symbiose
était plus simple~---les anciens \ac{OS}es qui n'étaient pas au courant de ces registres
supplémentaires et ne les sauvaient pas lors du changement de contexte, mais sauvaient
les registres FPU.
Ainsi, CPU avec MMX + ancien \ac{OS} + processus utilisant les capacités MMX fonctionnait
toujours.

\myindex{x86!SSE}
\myindex{x86!SSE2}
SSE---est une extension des registres SIMD à 128 bits, maintenant séparé du FPU.

\myindex{x86!AVX}
AVX---une autre extension, à 256 bits.

Parlons maintenant de l'usage pratique.

Bien sûr, il s'agit de routines de copie en mémoire (\TT{memcpy}), de comparaison
de mémoire (\TT{memcmp}) et ainsi de suite.

\myindex{DES}

Un autre exemple: l'algorithme de chiffrement DES prend un bloc de 64-bit et une
clef de 56-bit, chiffre le bloc et produit un résultat de 64-bit.
L'algorithme DES peut être considéré comme un grand circuit électronique, avec des
fils et des portes AND/OR/NOT.

\label{bitslicedes}
\newcommand{\URLBS}{\url{http://go.yurichev.com/17329}}

Le bitslice DES\footnote{\URLBS}~---est l'idée de traiter des groupes de blocs et
de clés simultanément.
Disons, une variable de type \emph{unsigned int} en x86 peut contenir jusqu'à 32-bit,
donc il est possible d'y stocker des résultats intermédiaires pour 32 paires de blocs-clé
simultanément, en utilisant 64+56 variables de type \emph{unsigned int}.

\myindex{\oracle}
Il existe un utilitaire pour brute-forcer les mots de passe/hashes d'\oracle (certains
basés sur DES) en utilisant un algorithme bitslice DES légèrement modifié pour SSE2
et AVX---maintenant il est possible de chiffrer 128 ou 256 paires de blocs-clé simultanément.

\url{http://go.yurichev.com/17313}

% sections
\subsection{Vectorisation}

\newcommand{\URLVEC}{\href{http://go.yurichev.com/17080}{Wikipédia: vectorisation}}

La vectorisation\footnote{\URLVEC}, c'est lorsque, par exemple, vous avez une boucle
qui prend une paire de tableaux en entrée et produit un tableau.
Le corps de la boucle prend les valeurs dans les tableaux en entrée, fait quelque
chose et met le résultat dans le tableau de sortie.
%It is important that there is only a single operation applied to each element.
%Il est important qu'il n'y ait qu'une seule opération appliquée à chaque élément.
La vectorisation est le fait de traiter plusieurs éléments simultanément.

La vectorisation n'est pas une nouvelle technologie: l'auteur de ce livre l'a vu
au moins sur la série du super-calculateur Cray Y-MP de 1988 lorsqu'il jouait avec
sa version \q{lite} le Cray Y-MP EL\footnote{À distance. Il est installé dans le
musée des super-calculateurs: \url{http://go.yurichev.com/17081}}.

% FIXME! add assembly listing!
Par exemple:

\begin{lstlisting}[style=customc]
for (i = 0; i < 1024; i++)
{
    C[i] = A[i]*B[i];
}
\end{lstlisting}

Ce morceau de code prend des éléments de A et de B, les multiplie et sauve le résultat
dans C.

\myindex{x86!\Instructions!PLMULLD}
\myindex{x86!\Instructions!PLMULHW}
\newcommand{\PMULLD}{\emph{PMULLD} (\emph{Multiply Packed Signed Dword Integers and Store Low Result})}
\newcommand{\PMULHW}{\TT{PMULHW} (\emph{Multiply Packed Signed Integers and Store High Result})}

Si chaque élément du tableau que nous avons est un \Tint 32-bit, alors il est possible
de charger 4 éléments de A dans un registre XMM 128-bit, 4 de B dans un autre registre
XMM, et en exécutant \PMULLD{} et \PMULHW{}, il est possible d'obtenir 4 \glslink{product}{produits}
64-bit en une fois.

Ainsi, le nombre d'exécution du corps de la boucle est $1024/4$ au lieu de 1024,
ce qui est 4 fois moins et, bien sûr, est plus rapide.

\newcommand{\URLINTELVEC}{\href{http://go.yurichev.com/17082}{Extrait: Vectorisation automatique efficace}}

\subsubsection{Exemple d'addition}

\myindex{Intel C++}

Certains compilateurs peuvent effectuer la vectorisation automatiquement dans des
cas simples, e.g., Intel C++\footnote{Sur la vectorisation automatique d'Intel C++:
\URLINTELVEC}.

Voici une fonction minuscule:

\begin{lstlisting}[style=customc]
int f (int sz, int *ar1, int *ar2, int *ar3)
{
	for (int i=0; i<sz; i++)
		ar3[i]=ar1[i]+ar2[i];

	return 0;
};
\end{lstlisting}

\myparagraph{Intel C++}

Compilons la avec Intel C++ 11.1.051 win32:

\begin{verbatim}
icl intel.cpp /QaxSSE2 /Faintel.asm /Ox
\end{verbatim}

Nous obtenons (dans \IDA):

\lstinputlisting[style=customasmx86]{patterns/19_SIMD/18_1_FR.asm}

Les instructions relatives à SSE2 sont:
\myindex{x86!\Instructions!MOVDQA}
\myindex{x86!\Instructions!MOVDQU}
\myindex{x86!\Instructions!PADDD}
\begin{itemize}
\item
\MOVDQU (\emph{Move Unaligned Double Quadword} déplacer double quadruple mot non
alignés)---charge juste 16 octets depuis la mémoire dans un registre XMM.

\item
\PADDD (\emph{Add Packed Integers} ajouter entier packé)---ajoute 4 paires de nombres
32-bit et laisse le résultat dans le premier opérande.
À propos, aucune exception n'est levée en cas de débordement et aucun flag n'est mis,
seuls les 32-bit bas du résultat sont stockés.
Si un des opérandes de \PADDD est l'adresse d'une valeur en mémoire, alors l'adresse
doit être alignée sur une limite de 16 octets.
Si elle n'est pas alignée, une exception est levée\footnote{En savoir plus sur l'alignement
des données: \URLWPDA}.

\item
\MOVDQA (\emph{Move Aligned Double Quadword}) est la même chose que \MOVDQU, mais nécessite
que l'adresse de la valeur en mémoire soit alignée sur une limite de 16 octets. Si
elle n'est pas alignée, une exception est levée.
\MOVDQA fonctionne plus vite que \MOVDQU, mais nécessite la condition qui vient d'être
écrite.

\end{itemize}

Donc, ces instructions SSE2 sont exécutées seulement dans le cas où il y a plus
de 4 paires à traiter et que le pointeur \TT{ar3} est aligné sur une limite de 16
octets.

Ainsi, si \TT{ar2} est également aligné sur une limite de 16 octets, ce morceau de
code sera exécuté:

\begin{lstlisting}[style=customasmx86]
movdqu  xmm0, xmmword ptr [ebx+edi*4] ; ar1+i*4
paddd   xmm0, xmmword ptr [esi+edi*4] ; ar2+i*4
movdqa  xmmword ptr [eax+edi*4], xmm0 ; ar3+i*4
\end{lstlisting}

Autrement, la valeur de \TT{ar2} est chargée dans \XMM{0} avec \MOVDQU, qui ne nécessite
pas que le pointeur soit aligné, mais peut s'exécuter plus lentement.

\lstinputlisting[style=customasmx86]{patterns/19_SIMD/18_1_excerpt_FR.asm}

Dans tous les autres cas, le code non-SSE2 sera exécuté.

\myparagraph{GCC}

\newcommand{\URLGCCVEC}{\url{http://go.yurichev.com/17083}}

GCC peut aussi vectoriser dans des cas simples\footnote{Plus sur le support de la
vectorisation dans GCC: \URLGCCVEC}, si l'option \Othree est utilisée et le support
de SSE2 activé: \TT{-msse2}.

Ce que nous obtenons (GCC 4.4.1):

\lstinputlisting[style=customasmx86]{patterns/19_SIMD/18_2_gcc_O3.asm}

Presque le même, toutefois, pas aussi méticuleux qu'Intel C++.

\subsubsection{Exemple de copie de mémoire}
\label{vec_memcpy}

Revoyons le simple exemple memcpy()
(\myref{loop_memcpy}):

\lstinputlisting[style=customc]{memcpy.c}

Et ce que les optimisations de GCC 4.9.1 font:

\lstinputlisting[caption=GCC 4.9.1 x64 \Optimizing,style=customasmx86]{patterns/19_SIMD/memcpy_GCC49_x64_O3_FR.s}

\subsection{Implémentation SIMD de \strlen}
\label{SIMD_strlen}

\newcommand{\URLMSDNSSE}{\href{http://go.yurichev.com/17262}{MSDN: particularités MMX, SSE, et SSE2}}

Il faut noter que les instructions \ac{SIMD} peuvent être insérées en code \CCpp via
des macros\footnote{\URLMSDNSSE} spécifiques.
Pour MSVC, certaines d'entre elles se trouvent dans le fichier \TT{intrin.h}.

\newcommand{\URLSTRLEN}{http://go.yurichev.com/17330}

\myindex{\CStandardLibrary!strlen()}

Il est possible d'implémenter la fonction \strlen\footnote{strlen()~---fonction de
la bibliothèque C standard pour calculer la longueur d'une chaîne} en utilisant des
instructions SIMD qui fonctionne 2-2.5 fois plus vite que l'implémentation habituelle.
Cette fonction charge 16 caractères dans un registre XMM et teste chacun d'eux avec
zéro.
\footnote{L'exemple est basé sur le code source de: \url{\URLSTRLEN}.}.

\lstinputlisting[style=customc]{patterns/19_SIMD/18_3.c}

Compilons la avec MSVC 2010 avec l'option \Ox:

\lstinputlisting[caption=MSVC 2010 \Optimizing,style=customasmx86]{patterns/19_SIMD/18_4_msvc_Ox_FR.asm}

Comment est-ce que ça fonctionne?
Premièrement, nous devons comprendre la but de la fonction.
Elle calcule la longueur de la chaîne C, mais nous pouvons utiliser différents termes:
sa tâche est de chercher l'octet zéro, et de calculer sa position relativement au
début de la chaîne.

Premièrement, nous testons si le pointeur \TT{str} est aligné sur une limite de
16 octets.
Si non, nous appelons l'implémentation générique de \strlen.

Puis, nous chargeons les 16 octets suivants dans le registre \XMM{1} en utilisant
\MOVDQA.

Un lecteur observateur pourrait demander, pourquoi \MOVDQU ne pourrait pas être utilisé
ici, puisqu'il peut charger des données depuis la mémoire quelque soit l'alignement
du pointeur?

Oui, cela pourrait être fait comme ça: si le pointeur est aligné, charger les données
en utilisant \MOVDQA, si non~---utiliser \MOVDQU moins rapide.

Mais ici nous pourrions rencontrer une autre restriction:

\myindex{Page (mémoire)}
\newcommand{\URLPAGE}{\href{http://go.yurichev.com/17136}{Wikipédia}}

Dans la série d'\ac{OS} \gls{Windows NT} (mais pas seulement), la mémoire est allouée
par pages de 4 KiB (4096 octets).
Chaque processus win32 a 4GiB de disponible, mais en fait, seulement une partie
de l'espace d'adressage est connecté à de la mémoire réelle.
Si le processus accède a un bloc mémoire absent, une exception est levée.
C'est comme cela que fonctionnent les \ac{VM}\footnote{\URLPAGE}.

Donc, une fonction qui charge 16 octets à la fois peut dépasser la limite d'un bloc
de mémoire allouée.
Disons que l'\ac{OS} a alloué 8192 (0x2000) octets à l'adresse 0x008c0000.
Ainsi, le bloc comprend les octets démarrant à l'adresse 0x008c0000 jusqu'à 0x008c1fff
inclus.

Après ce bloc, c'est à dire à partir de l'adresse 0x008c2000 il n'y a rien du tout,
e.g. l'\ac{OS} n'y a pas alloué de mémoire.
Toutes tentatives d'accéder à la mémoire à partir de cette adresse va déclencher
une exception.

Et maintenant, considérons l'exemple dans lequel le programme possède une chaîne
contenant 5 caractères presque à la fin d'un bloc, et ce n'est pas un crime.

\begin{center}
  \begin{tabular}{ | l | l | }
    \hline
        0x008c1ff8 & 'h' \\
        0x008c1ff9 & 'e' \\
        0x008c1ffa & 'l' \\
        0x008c1ffb & 'l' \\
        0x008c1ffc & 'o' \\
        0x008c1ffd & '\textbackslash{}x00' \\
        0x008c1ffe & random noise \\
        0x008c1fff & random noise \\
    \hline
  \end{tabular}
\end{center}

Donc, dans des conditions normales, le programme appelle \strlen, en lui passant un
pointeur sur la chaîne \TT{'hello'} se trouvant en mémoire à l'adresse 0x008c1ff8.
\strlen lit un octet à la fois jusqu'à 0x008c1ffd, où se trouve un octet à zéro,
et puis s'arrête.

Maintenant, si nous implémentons notre propre \strlen lisant 16 octets à la fois,
à partir de n'importe quelle adresse, alignée ou pas, \MOVDQU pourrait essayer de
charger 16 octets à la fois depuis l'adresse 0x008c1ff8 jusqu'à 0x008c2008, et ainsi
déclencherait une exception.
Cette situation peut être évitée, bien sûr.

Nous allons donc ne travailler qu'avec des adresses alignées sur une limite de 16
octets, ce qui en combinaison avec la connaissance que les pages de l'\ac{OS} sont
en général alignées sur une limite de 16 octets nous donne quelques garanties que
notre fonction ne va pas lire de la mémoire non allouée.

Retournons à notre fonction.

\myindex{x86!\Instructions!PXOR}
\verb|_mm_setzero_si128()|---est une macro générant \TT{pxor xmm0, xmm0}~---elle
efface juste le registre \XMM{0}.

\verb|_mm_load_si128()|---est une macro pour \MOVDQA, elle charge 16 octets depuis l'adresse dans le registre \XMM{1}.

\myindex{x86!\Instructions!PCMPEQB}
\verb|_mm_cmpeq_epi8()|---est une macro pour \PCMPEQB, une instruction qui compare deux registres XMM par octet.

Et si l'un des octets est égal à celui dans l'autre registre, il y aura \TT{0xff}
à ce point dans le résultat ou 0 sinon.

Par exemple:

\begin{verbatim}
XMM1: 0x11223344556677880000000000000000
XMM0: 0x11ab3444007877881111111111111111
\end{verbatim}

Après l'exécution de \TT{pcmpeqb xmm1, xmm0}, le registre \XMM{1} contient:

\begin{verbatim}
XMM1: 0xff0000ff0000ffff0000000000000000
\end{verbatim}

Dans notre cas, cette instruction compare chacun des 16 octets avec un bloc de 16
octets à zéro, qui ont été mis dans le registre \XMM{0} par \TT{pxor xmm0, xmm0}.

\myindex{x86!\Instructions!PMOVMSKB}

La macro suivante est \TT{\_mm\_movemask\_epi8()}~---qui est l'instruction \TT{PMOVMSKB}.

Elle est très utile avec \PCMPEQB.

\TT{pmovmskb eax, xmm1}

Cette instruction met d'abord le premier bit d'\EAX à 1 si le bit le plus significatif
du premier octet dans \XMM{1} est 1.
En d'autres mots, si le premier octet du registre \XMM{1} est \TT{0xff}, alors le
premier bit de \EAX sera 1 aussi.

Si le second octet du registre \XMM{1} est \TT{0xff}, alors le second bit de \EAX
sera mis à 1 aussi.
En d'autres mots, cette instruction répond à la question \q{quels octets de \XMM{1}
ont le bit le plus significatif à 1 ou sont plus grand que \TT{0x7f}?} et renvoie
16 bits dans le registre \EAX.
Les autres bits du registre \EAX sont mis à zéro.

À propos, ne pas oublier cette bizarrerie dans notre algorithme.
Il pourrait y avoir 16 octets dans l'entrée, comme:

\input{patterns/19_SIMD/strlen_hello_and_garbage}

Il s'agit de la chaîne \TT{'hello'}, terminée par un zéro, et du bruit aléatoire
dans la mémoire.

Si nous chargeons ces 16 octets dans \XMM{1} et les comparons avec ceux à zéro dans
\XMM{0}, nous obtenons quelque chose comme \footnote{Un ordre de \ac{MSB} à \ac{LSB}
est utilisé ici.}:

\begin{verbatim}
XMM1: 0x0000ff00000000000000ff0000000000
\end{verbatim}

Cela signifie que cette instruction a trouvé deux octets à zéro, et ce n'est pas
surprenant.

\TT{PMOVMSKB} dans notre cas va mettre \EAX à\\
\emph{0b0010000000100000}.

Bien sûr, notre fonction doit seulement prendre le premier octet à zéro et ignorer
le reste.

\myindex{x86!\Instructions!BSF}
\label{instruction_BSF}
L'instruction suivante est \TT{BSF} (\emph{Bit Scan Forward}).

Cette instruction trouve le premier bit mis à 1 et met sa position dans le premier
opérande.

\begin{verbatim}
EAX=0b0010000000100000
\end{verbatim}

Après l'exécution de \TT{bsf eax, eax}, \EAX contient 5, signifiant que 1 a été trouvé
au bit d'index 5 (en commençant à zéro).

MSVC a une macro pour cette instruction: \TT{\_BitScanForward}.

Maintenant c'est simple. Si un octet à zéro a été trouvé, sa position est ajoutée
à ce que nous avions déjà compté et nous pouvons renvoyer le résultat.

Presque tout.

À propos, il faut aussi noter que le compilateur MSVC a généré deux corps de boucle
côte-à-côte, afin d'optimiser.

Et aussi, SSE 4.2 (apparu dans les Intel Core i7) offre encore plus d'instructions
avec lesquelles ces manipulations de chaîne sont encore plus facile: \url{http://go.yurichev.com/17331}


