\subsection{Implémentation SIMD de \strlen}
\label{SIMD_strlen}

\newcommand{\URLMSDNSSE}{\href{http://go.yurichev.com/17262}{MSDN: particularités MMX, SSE, et SSE2}}

Il faut noter que les instructions \ac{SIMD} peuvent être insérées en code \CCpp via
des macros\footnote{\URLMSDNSSE} spécifiques.
Pour MSVC, certaines d'entre elles se trouvent dans le fichier \TT{intrin.h}.

\newcommand{\URLSTRLEN}{http://go.yurichev.com/17330}

\myindex{\CStandardLibrary!strlen()}

Il est possible d'implémenter la fonction \strlen\footnote{strlen()~---fonction de
la bibliothèque C standard pour calculer la longueur d'une chaîne} en utilisant des
instructions SIMD qui fonctionne 2-2.5 fois plus vite que l'implémentation habituelle.
Cette fonction charge 16 caractères dans un registre XMM et teste chacun d'eux avec
zéro.
\footnote{L'exemple est basé sur le code source de: \url{\URLSTRLEN}.}.

\lstinputlisting[style=customc]{patterns/19_SIMD/18_3.c}

Compilons la avec MSVC 2010 avec l'option \Ox:

\lstinputlisting[caption=MSVC 2010 \Optimizing,style=customasmx86]{patterns/19_SIMD/18_4_msvc_Ox_FR.asm}

Comment est-ce que ça fonctionne?
Premièrement, nous devons comprendre la but de la fonction.
Elle calcule la longueur de la chaîne C, mais nous pouvons utiliser différents termes:
sa tâche est de chercher l'octet zéro, et de calculer sa position relativement au
début de la chaîne.

Premièrement, nous testons si le pointeur \TT{str} est aligné sur une limite de
16 octets.
Si non, nous appelons l'implémentation générique de \strlen.

Puis, nous chargeons les 16 octets suivants dans le registre \XMM{1} en utilisant
\MOVDQA.

Un lecteur observateur pourrait demander, pourquoi \MOVDQU ne pourrait pas être utilisé
ici, puisqu'il peut charger des données depuis la mémoire quelque soit l'alignement
du pointeur?

Oui, cela pourrait être fait comme ça: si le pointeur est aligné, charger les données
en utilisant \MOVDQA, si non~---utiliser \MOVDQU moins rapide.

Mais ici nous pourrions rencontrer une autre restriction:

\myindex{Page (mémoire)}
\newcommand{\URLPAGE}{\href{http://go.yurichev.com/17136}{Wikipédia}}

Dans la série d'\ac{OS} \gls{Windows NT} (mais pas seulement), la mémoire est allouée
par pages de 4 KiB (4096 octets).
Chaque processus win32 a 4GiB de disponible, mais en fait, seulement une partie
de l'espace d'adressage est connecté à de la mémoire réelle.
Si le processus accède a un bloc mémoire absent, une exception est levée.
C'est comme cela que fonctionnent les \ac{VM}\footnote{\URLPAGE}.

Donc, une fonction qui charge 16 octets à la fois peut dépasser la limite d'un bloc
de mémoire allouée.
Disons que l'\ac{OS} a alloué 8192 (0x2000) octets à l'adresse 0x008c0000.
Ainsi, le bloc comprend les octets démarrant à l'adresse 0x008c0000 jusqu'à 0x008c1fff
inclus.

Après ce bloc, c'est à dire à partir de l'adresse 0x008c2000 il n'y a rien du tout,
e.g. l'\ac{OS} n'y a pas alloué de mémoire.
Toutes tentatives d'accéder à la mémoire à partir de cette adresse va déclencher
une exception.

Et maintenant, considérons l'exemple dans lequel le programme possède une chaîne
contenant 5 caractères presque à la fin d'un bloc, et ce n'est pas un crime.

\begin{center}
  \begin{tabular}{ | l | l | }
    \hline
        0x008c1ff8 & 'h' \\
        0x008c1ff9 & 'e' \\
        0x008c1ffa & 'l' \\
        0x008c1ffb & 'l' \\
        0x008c1ffc & 'o' \\
        0x008c1ffd & '\textbackslash{}x00' \\
        0x008c1ffe & random noise \\
        0x008c1fff & random noise \\
    \hline
  \end{tabular}
\end{center}

Donc, dans des conditions normales, le programme appelle \strlen, en lui passant un
pointeur sur la chaîne \TT{'hello'} se trouvant en mémoire à l'adresse 0x008c1ff8.
\strlen lit un octet à la fois jusqu'à 0x008c1ffd, où se trouve un octet à zéro,
et puis s'arrête.

Maintenant, si nous implémentons notre propre \strlen lisant 16 octets à la fois,
à partir de n'importe quelle adresse, alignée ou pas, \MOVDQU pourrait essayer de
charger 16 octets à la fois depuis l'adresse 0x008c1ff8 jusqu'à 0x008c2008, et ainsi
déclencherait une exception.
Cette situation peut être évitée, bien sûr.

Nous allons donc ne travailler qu'avec des adresses alignées sur une limite de 16
octets, ce qui en combinaison avec la connaissance que les pages de l'\ac{OS} sont
en général alignées sur une limite de 16 octets nous donne quelques garanties que
notre fonction ne va pas lire de la mémoire non allouée.

Retournons à notre fonction.

\myindex{x86!\Instructions!PXOR}
\verb|_mm_setzero_si128()|---est une macro générant \TT{pxor xmm0, xmm0}~---elle
efface juste le registre \XMM{0}.

\verb|_mm_load_si128()|---est une macro pour \MOVDQA, elle charge 16 octets depuis l'adresse dans le registre \XMM{1}.

\myindex{x86!\Instructions!PCMPEQB}
\verb|_mm_cmpeq_epi8()|---est une macro pour \PCMPEQB, une instruction qui compare deux registres XMM par octet.

Et si l'un des octets est égal à celui dans l'autre registre, il y aura \TT{0xff}
à ce point dans le résultat ou 0 sinon.

Par exemple:

\begin{verbatim}
XMM1: 0x11223344556677880000000000000000
XMM0: 0x11ab3444007877881111111111111111
\end{verbatim}

Après l'exécution de \TT{pcmpeqb xmm1, xmm0}, le registre \XMM{1} contient:

\begin{verbatim}
XMM1: 0xff0000ff0000ffff0000000000000000
\end{verbatim}

Dans notre cas, cette instruction compare chacun des 16 octets avec un bloc de 16
octets à zéro, qui ont été mis dans le registre \XMM{0} par \TT{pxor xmm0, xmm0}.

\myindex{x86!\Instructions!PMOVMSKB}

La macro suivante est \TT{\_mm\_movemask\_epi8()}~---qui est l'instruction \TT{PMOVMSKB}.

Elle est très utile avec \PCMPEQB.

\TT{pmovmskb eax, xmm1}

Cette instruction met d'abord le premier bit d'\EAX à 1 si le bit le plus significatif
du premier octet dans \XMM{1} est 1.
En d'autres mots, si le premier octet du registre \XMM{1} est \TT{0xff}, alors le
premier bit de \EAX sera 1 aussi.

Si le second octet du registre \XMM{1} est \TT{0xff}, alors le second bit de \EAX
sera mis à 1 aussi.
En d'autres mots, cette instruction répond à la question \q{quels octets de \XMM{1}
ont le bit le plus significatif à 1 ou sont plus grand que \TT{0x7f}?} et renvoie
16 bits dans le registre \EAX.
Les autres bits du registre \EAX sont mis à zéro.

À propos, ne pas oublier cette bizarrerie dans notre algorithme.
Il pourrait y avoir 16 octets dans l'entrée, comme:

\input{patterns/19_SIMD/strlen_hello_and_garbage}

Il s'agit de la chaîne \TT{'hello'}, terminée par un zéro, et du bruit aléatoire
dans la mémoire.

Si nous chargeons ces 16 octets dans \XMM{1} et les comparons avec ceux à zéro dans
\XMM{0}, nous obtenons quelque chose comme \footnote{Un ordre de \ac{MSB} à \ac{LSB}
est utilisé ici.}:

\begin{verbatim}
XMM1: 0x0000ff00000000000000ff0000000000
\end{verbatim}

Cela signifie que cette instruction a trouvé deux octets à zéro, et ce n'est pas
surprenant.

\TT{PMOVMSKB} dans notre cas va mettre \EAX à\\
\emph{0b0010000000100000}.

Bien sûr, notre fonction doit seulement prendre le premier octet à zéro et ignorer
le reste.

\myindex{x86!\Instructions!BSF}
\label{instruction_BSF}
L'instruction suivante est \TT{BSF} (\emph{Bit Scan Forward}).

Cette instruction trouve le premier bit mis à 1 et met sa position dans le premier
opérande.

\begin{verbatim}
EAX=0b0010000000100000
\end{verbatim}

Après l'exécution de \TT{bsf eax, eax}, \EAX contient 5, signifiant que 1 a été trouvé
au bit d'index 5 (en commençant à zéro).

MSVC a une macro pour cette instruction: \TT{\_BitScanForward}.

Maintenant c'est simple. Si un octet à zéro a été trouvé, sa position est ajoutée
à ce que nous avions déjà compté et nous pouvons renvoyer le résultat.

Presque tout.

À propos, il faut aussi noter que le compilateur MSVC a généré deux corps de boucle
côte-à-côte, afin d'optimiser.

Et aussi, SSE 4.2 (apparu dans les Intel Core i7) offre encore plus d'instructions
avec lesquelles ces manipulations de chaîne sont encore plus facile: \url{http://go.yurichev.com/17331}
