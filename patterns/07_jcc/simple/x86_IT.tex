\subsubsection{x86}

\myparagraph{x86 + MSVC}

La funzione \TT{f\_signed()} appare così:

\lstinputlisting[caption=\NonOptimizing MSVC 2010,style=customasmx86]{patterns/07_jcc/simple/signed_MSVC.asm}

\myindex{x86!\Instructions!JLE}

La prima istruzione, \JLE, sta per \emph{Jump if Less or Equal} (\emph{salta se è minore o uguale}). 
In altre parole, se il secondo operando è
maggiore o uguale al primo, il flusso di controllo sarà pasato all'indirizzo o alla label specificata nell'istruzione. 
Se questa condizione non è soddisfatta, poiché il secondo operando è più piccolo del primo, il flusso non viene alterato e la prima \printf sarà eseguita.

\myindex{x86!\Instructions!JNE}
Il secondo controllo è \JNE: \emph{Jump if Not Equal}.
Il flusso non cambia se i due operandi sono uguali.

\myindex{x86!\Instructions!JGE}
Il terzo controllo è \JGE: \emph{Jump if Greater or Equal}---salta se il primo operando è maggiore del secondo, o se sono uguali.

Quindi, se tutti i tre salti condizionali vengono innescati, nessuna delle chiamate a \printf sarà eseguita.
Ciò è chiaramente impossibile, almeno senza un intervento speciale.

Diamo ora un'occhiata alla funzione \TT{f\_unsigned()}.
La funzione \TT{f\_unsigned()} è uguale a \TT{f\_signed()}, con l'eccezione che le istruzioni \JBE e \JAE
sono utilizzate al posto di \JLE e \JGE:

\lstinputlisting[caption=GCC,style=customasmx86]{patterns/07_jcc/simple/unsigned_MSVC.asm}

\myindex{x86!\Instructions!JBE}
\myindex{x86!\Instructions!JAE}

Come già detto, le istruzioni di salto (branch instructions) sono diverse:
\JBE---\emph{Jump if Below or Equal} e \JAE---\emph{Jump if Above or Equal}.
Queste istruzioni (\INS{JA}/\JAE/\JB/\JBE) differiscono da \JG/\JGE/\JL/\JLE in quanto operano con numeri senza segno (unsigned).

\myindex{x86!\Instructions!JA}
\myindex{x86!\Instructions!JB}
\myindex{x86!\Instructions!JG}
\myindex{x86!\Instructions!JL}
\myindex{Signed numbers}

Vedi anche la sezione sulle rappresentazioni di numeri con segno (signed) ~(\myref{sec:signednumbers}).

Questo è il motivo per cui se vediamo usare \JG/\JL al posto di \INS{JA}/\JB, o viceversa, 
possiamo essere quasi certi che le variabili sono rispettivamente di tipo signed o unsigned.

Di seguito è riportata anche la funzione \main, dove non c'è niente di nuovo:

\lstinputlisting[caption=\main,style=customasmx86]{patterns/07_jcc/simple/main_MSVC.asm}

\clearpage
\myparagraph{x86 + MSVC + \olly}
\myindex{\olly}
\myindex{x86!\Registers!\Flags}

Possiamo vedere come vengono settati i flag facendo girare l'esempio in \olly.
Iniziamo con \TT{f\_unsigned()}, che funziona con numeri di tipo unsigned.

L'istruzione \CMP è eseguita tre volte e con gli stessi argomenti, pertanto i flag saranno ogni volta gli stessi.

Risultato del primo confronto:

\begin{figure}[H]
\centering
\myincludegraphics{patterns/07_jcc/simple/olly_unsigned1.png}
\caption{\olly: \TT{f\_unsigned()}: primo salto condizionale}
\label{fig:jcc_olly_unsigned_1}
\end{figure}

I flag sono: C=1, P=1, A=1, Z=0, S=1, T=0, D=0, O=0.
In \olly sono riportati per brevità con la sola iniziale.


\olly suggerisce che il jump (\JBE) sarà innescato.
Infatti, consultando i manuali Intel (\myref{x86_manuals}), vediamo che \JBE è innescato se CF=1 o ZF=1.
La condizione è vera, e quindi il salto viene effettuato.

\clearpage
Jump condizionale successivo:

\begin{figure}[H]
\centering
\myincludegraphics{patterns/07_jcc/simple/olly_unsigned2.png}
\caption{\olly: \TT{f\_unsigned()}: secondo conditional jump}
\label{fig:jcc_olly_unsigned_2}
\end{figure}

\olly dice che il \JNZ verrà seguito.
Infatti, \JNZ è innescato se ZF=0 (zero flag).

\clearpage
Il terzo conditional jump, \JNB:

\begin{figure}[H]
\centering
\myincludegraphics{patterns/07_jcc/simple/olly_unsigned3.png}
\caption{\olly: \TT{f\_unsigned()}: terzo conditional jump}
\label{fig:jcc_olly_unsigned_3}
\end{figure}

Nei manuali Intel (\myref{x86_manuals}) vediamo che \JNB è innescato se CF=0 (carry flag).
Nel nostro caso questa condizione non è vera, e quindi la terza \printf sarà eseguita.

\clearpage
Rivediamo ora la funzione \TT{f\_signed()}, che opera con valori signed, in \olly.
I flag sono settati allo stesso modo: C=1, P=1, A=1, Z=0, S=1, T=0, D=0, O=0.
Il primo salto condizionale \JLE sarà seguito:

\begin{figure}[H]
\centering
\myincludegraphics{patterns/07_jcc/simple/olly_signed1.png}
\caption{\olly: \TT{f\_signed()}: primo conditional jump}
\label{fig:jcc_olly_signed_1}
\end{figure}

Nei manuali Intel (\myref{x86_manuals}) vediamo che questa istruzione viene innescata se ZF=1 o SF$\neq$OF.
SF$\neq$OF nel nostro caso, quindi il salto viene effettuato.

\clearpage
Il secondo jump condizionale \JNZ viene innescato, se ZF=0 (zero flag):

\begin{figure}[H]
\centering
\myincludegraphics{patterns/07_jcc/simple/olly_signed2.png}
\caption{\olly: \TT{f\_signed()}: secondo conditional jump}
\label{fig:jcc_olly_signed_2}
\end{figure}

\clearpage
Il terzo jump \JGE non sarà innescato in quanto lo sarebbe solo se SF=OF, condizione non vera nel nostro caso:

\begin{figure}[H]
\centering
\myincludegraphics{patterns/07_jcc/simple/olly_signed3.png}
\caption{\olly: \TT{f\_signed()}: terzo conditional jump}
\label{fig:jcc_olly_signed_3}
\end{figure}


\clearpage
\myparagraph{x86 + MSVC + Hiew}
\myindex{Hiew}

Possiamo provare a patchare l'eseguibile (applicare una patch) in maniera tale che la funzione \TT{f\_unsigned()} stampi sempre \q{a==b}, 
a prescindere dai valori in input.
In Hiew appare così:

\begin{figure}[H]
\centering
\myincludegraphics{patterns/07_jcc/simple/hiew_unsigned1.png}
\caption{Hiew: \TT{f\_unsigned()} function}
\label{fig:jcc_hiew_1}
\end{figure}

Essenzialmente, per ottenere il risultato desiderato, dobbiamo:
\begin{itemize}
\item forzare il primo jump in modo che sia sempre seguito;
\item forzare il secondo jump a non essere mai seguito;
\item forzare il terzo jump ad essere sempre seguito.
\end{itemize}

Possiamo così diriggere il flusso di esecuzione in modo tale da farlo sempre passare attraverso la seconda \printf, dando in output \q{a==b}.

Devono essere corrette (patchate) tre istruzioni (o byte):

\begin{itemize}
\item Il primo jump diventa \JMP, ma il \gls{jump offset} resta invariato.

\item 
Il secondo jump potrebbe essere innescato in alcune occasioni, ma in ogni caso salterebbe alla prossima istruzione, poiché settiamo il \gls{jump offset} a 0.

In queste istruzioni il \gls{jump offset} viene sommato all'indirizzo della prossima istruzione.
Quindi se l'offset è 0, il jump trasferirà il controllo all'istruzione successiva,

\item 
Possiamo sostituire il terzo jump con \JMP allo stesso modo del primo, in modo che sia sempre innescato.

\end{itemize}

\clearpage
Ecco il codice modificato:

\begin{figure}[H]
\centering
\myincludegraphics{patterns/07_jcc/simple/hiew_unsigned2.png}
\caption{Hiew: funzione \TT{f\_unsigned()} modificata}
\label{fig:jcc_hiew_2}
\end{figure}

Se ci dimentichiamo di cambiare uno di questi jump, potrebbero essere eseguite diverse chiamate a \printf, ma noi vogliamo eseguirne solo una.

\myparagraph{\NonOptimizing GCC}

\myindex{puts() instead of printf()}
\NonOptimizing GCC 4.4.1 
procude pressoché lo stesso codice, ma usa \puts~(\myref{puts}) invece di \printf.

\myparagraph{\Optimizing GCC}

Un lettore attento potrebbe domandare: perchè eseguire \CMP più volte se i flag hanno gli stessi valori dopo ogni esecuzione?

Forse MSVC con ottimizzazioni non è in grado di applicare questa ottimizzazione, al contrario di GCC 4.8.1:

\lstinputlisting[caption=GCC 4.8.1 f\_signed(),style=customasmx86]{patterns/07_jcc/simple/GCC_O3_signed.asm}

% should be here instead of 'switch' section?
Notiamo anche l'uso di \TT{JMP puts} al posto di \TT{CALL puts / RETN}.
Questo trucco sarà spiegato più avanti: \myref{JMP_instead_of_RET}.

Questo tipo di codice x86 è piuttosto raro.
MSVC 2012 apparentemente non è in grado di generarne di simile.
Dall'altro lato, i programmatori assembly sanno perfettamente che le istruzioni \TT{Jcc} possono essere disposte in fila.

Se vedete codice con una disposizione simile, è molto probabile che sia stato scritto a mano.

La funzione \TT{f\_unsigned()} non è esteticamente corta allo stesso modo:

\lstinputlisting[caption=GCC 4.8.1 f\_unsigned(),style=customasmx86]{patterns/07_jcc/simple/GCC_O3_unsigned_EN.asm}

Ciò nonostante, ci sono due istruzioni \TT{CMP} invece di tre.
Gli algoritmi di ottimizzazione di GCC 4.8.1 probabilmente non sono ancora perfetti. 
