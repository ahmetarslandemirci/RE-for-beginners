\subsubsection{MIPS}

1つの特徴的なMIPS機能は、フラグが存在しないことです。 
明らかに、データ依存性の分析を簡素化するために行われました。

\myindex{x86!\Instructions!SETcc}
\myindex{MIPS!\Instructions!SLT}
\myindex{MIPS!\Instructions!SLTU}

x86には\INS{SETcc}に似た命令があります。\INS{SLT}(\q{Set on Less Than}:符号付きバージョン)と\INS{SLTU}(符号なしバージョン)です。 
これらの命令は、条件が真であれば宛先レジスタの値を1に設定し、そうでない場合は0に設定します。

\myindex{MIPS!\Instructions!BEQ}
\myindex{MIPS!\Instructions!BNE}

宛先レジスタは、\INS{BEQ} (\q{Branch on Equal}) または \INS{BNE} (\q{Branch on Not Equal})を使用してチェックされ、
ジャンプが発生することがあります。 
したがって、この命令ペアは比較および分岐のためにMIPSで使用されなければなりません。
最初に関数の符号付きバージョンから始めましょう。

\lstinputlisting[caption=\NonOptimizing GCC 4.4.5 (IDA),style=customasmMIPS]{patterns/07_jcc/simple/O0_MIPS_signed_IDA_JA.lst}

\INS{SLT REG0, REG0, REG1}は、IDAによって短縮形式
\INS{SLT REG0, REG1}に縮小されます。
\myindex{MIPS!\Pseudoinstructions!BEQZ}

実際には\INS{BEQ REG, \$ZERO, LABEL}の
\INS{BEQZ}擬似命令もあります(\q{Branch if Equal to Zero})。

\myindex{MIPS!\Instructions!SLTU}

符号なしバージョンはまったく同じですが、\INS{SLT}の代わりに\INS{SLTU}(符号なしバージョン、したがって\q{U}という名前)が使用されます。

\lstinputlisting[caption=\NonOptimizing GCC 4.4.5 (IDA),style=customasmMIPS]{patterns/07_jcc/simple/O0_MIPS_unsigned_IDA.lst}
