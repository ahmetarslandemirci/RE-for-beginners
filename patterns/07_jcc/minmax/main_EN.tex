\subsection{Getting minimal and maximal values}

\subsubsection{32-bit}

\lstinputlisting[style=customc]{patterns/07_jcc/minmax/minmax.c}

\lstinputlisting[caption=\NonOptimizing MSVC 2013,style=customasmx86]{patterns/07_jcc/minmax/minmax_MSVC_2013_EN.asm}

\myindex{x86!\Instructions!Jcc}

These two functions differ only in the conditional jump instruction: 
\INS{JGE} (\q{Jump if Greater or Equal}) is used in the first one
and \INS{JLE} (\q{Jump if Less or Equal}) in the second.

\myindex{\CompilerAnomaly}
\label{MSVC_double_JMP_anomaly}

There is one unneeded \JMP instruction in each function, which MSVC presumably left by mistake.

\myparagraph{Branchless}

ARM for Thumb mode reminds us of x86 code:

\lstinputlisting[caption=\OptimizingKeilVI (\ThumbMode),style=customasmARM]{patterns/07_jcc/minmax/minmax_Keil_Thumb_O3_EN.s}

\myindex{ARM!\Instructions!Bcc}

The functions differ in the branching instruction: \INS{BGT} and \INS{BLT}.
It's possible to use conditional suffixes in ARM mode, so the code is shorter.

\myindex{ARM!\Instructions!MOVcc}
\INS{MOVcc} is to be executed only if the condition is met:

\lstinputlisting[caption=\OptimizingKeilVI (\ARMMode),style=customasmARM]{patterns/07_jcc/minmax/minmax_Keil_ARM_O3_EN.s}

\myindex{x86!\Instructions!CMOVcc}
\Optimizing GCC 4.8.1 and optimizing MSVC 2013 can use \INS{CMOVcc} instruction, which is analogous to \INS{MOVcc} in ARM:

\lstinputlisting[caption=\Optimizing MSVC 2013,style=customasmx86]{patterns/07_jcc/minmax/minmax_GCC481_O3_EN.s}

\subsubsection{64-bit}

\lstinputlisting[style=customc]{patterns/07_jcc/minmax/minmax64.c}

There is some unneeded value shuffling, but the code is comprehensible:

\lstinputlisting[caption=\NonOptimizing GCC 4.9.1 ARM64,style=customasmARM]{patterns/07_jcc/minmax/minmax64_GCC_49_ARM64_O0.s}

\myparagraph{Branchless}

No need to load function arguments from the stack, as they are already in the registers:

\lstinputlisting[caption=\Optimizing GCC 4.9.1 x64,style=customasmx86]{patterns/07_jcc/minmax/minmax64_GCC_49_x64_O3_EN.s}

MSVC 2013 does almost the same.

\myindex{ARM!\Instructions!CSEL}

ARM64 has the \INS{CSEL} instruction, which works just as \INS{MOVcc} in ARM or \INS{CMOVcc} in x86, just the name is different:
\q{Conditional SELect}.

\lstinputlisting[caption=\Optimizing GCC 4.9.1 ARM64,style=customasmARM]{patterns/07_jcc/minmax/minmax64_GCC_49_ARM64_O3_EN.s}

\subsubsection{MIPS}

Unfortunately, GCC 4.4.5 for MIPS is not that good:

\lstinputlisting[caption=\Optimizing GCC 4.4.5 (IDA),style=customasmMIPS]{patterns/07_jcc/minmax/minmax_MIPS_O3_IDA_EN.lst}

Do not forget about the \emph{branch delay slots}: the first \INS{MOVE} is executed \emph{before} \INS{BEQZ}, 
the second \INS{MOVE} is executed only if the branch hasn't been taken.

