\subsection{Funktion zum Kopieren von Speicherblöcken}
\label{loop_memcpy}

Echte Funktionen zum Kopieren von Speicherblöcken kopieren in jedem Schritt 4
oder 8 Byte und verwenden \ac{SIMD}, Vektorisierung, etc.
Aber um einen Eindruck zu erhalten betrachte dieses einfachstmögliche Beispiel.

\lstinputlisting[style=customc]{memcpy.c}

\subsubsection{Grundlegende Implementierung}

\lstinputlisting[caption=GCC 4.9 x64 optimized for size (-Os),style=customasmx86]{patterns/09_loops/memcpy/memcpy_GCC49_x64_Os_DE.s}

\lstinputlisting[caption=GCC 4.9 ARM64 optimized for size (-Os),style=customasmARM]{patterns/09_loops/memcpy/memcpy_GCC49_ARM64_Os_DE.s}

\lstinputlisting[caption=\OptimizingKeilVI (\ThumbMode),style=customasmARM]{patterns/09_loops/memcpy/memcpy_Keil_Thumb_O3_DE.s}

\subsubsection{ARM in ARM mode}

Keil in ARM mode macht Gebrauch von konditionalen Suffixen:

\lstinputlisting[caption=\OptimizingKeilVI (\ARMMode),style=customasmARM]{patterns/09_loops/memcpy/memcpy_Keil_ARM_O3_DE.s}

Deshalb gibt es hier nur einen Verzweigungsbefehl anstatt deren zwei.

\subsubsection{MIPS}

\lstinputlisting[caption=GCC 4.4.5 optimized for size (-Os) (IDA),style=customasmMIPS]{patterns/09_loops/memcpy/memcpy_MIPS_Os_IDA_DE.lst}

\myindex{MIPS!\Instructions!LBU}
\myindex{MIPS!\Instructions!SB}

Hier tauchen zwei neue Befehle auf: \INS{LBU} (\q{Load Byte Unsigned}) und
\INS{SB} (\q{Store Byte}).

Wie in ARM haben alle MIPS Register eine Breite von 32 bit, es gibt keine
byte-breiten Teile wie bei x86.

Wenn wir also mit einzelnen Bytes arbeiten, müssen wir stets ein komplettes
32-bit-Register hierfür verwenden.

\INS{LBU} lädt ein Byte und löscht alle anderen Bits (\q{Unsigned}).
\myindex{MIPS!\Instructions!LB}

Der Befehl \INS{LB} (\q{Load Byte}) dagegen erweitert das geladene Byte zu einem
vorzeichenbehafteten 32-bit-Wert.

\INS{SB} schreibt ein Byte der niederwertigsten 8 Bit des Registers in den
Speicher.


\subsubsection{Vektorisierung}

\Optimizing GCC kann viel mehr; siehe dieses Beispiel: \myref{vec_memcpy}.
