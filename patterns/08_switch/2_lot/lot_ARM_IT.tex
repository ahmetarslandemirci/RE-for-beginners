\subsubsection{ARM: \OptimizingKeilVI (\ARMMode)}
\label{sec:SwitchARMLot}

\lstinputlisting[caption=\OptimizingKeilVI (\ARMMode),style=customasmARM]{patterns/08_switch/2_lot/lot_ARM_ARM_O3.asm}

Il codice fa uso della modalità ARM in cui tutte le istruzioni hanno dimensione fissa di 4 byte.
Ricordiamoci che il massimo valore previsto per $a$ è 4 e ogni valore maggiore causerà la stampa della stringa 
\emph{<<something unknown\textbackslash{}n>>}.

\myindex{ARM!\Instructions!CMP}
\myindex{ARM!\Instructions!ADDCC}
La prima istruzione \TT{CMP R0, \#5} confronta il valore di $a$ con 5.

\footnote{ADD---addition}
La successiva \TT{ADDCC PC, PC, R0,LSL\#2} viene eseguita solo se $R0 < 5$ (\emph{CC=Carry clear / Less than}). 
Di conseguenza, se \TT{ADDCC} non viene innescata (è il caso $R0 \geq 5$), si verificherà un jump a \emph{default\_case}.

Se invece $R0 < 5$ e \TT{ADDCC} viene innescata, succede quanto segue:

Il valore in \Reg{0} viene moltiplicato per 4.
Infatti \TT{LSL\#2} nel suffisso dell'istruzione sta per \q{shift left by 2 bits} (shift a sinistra di 2 bit).
Come vedremo più avanti ~(\myref{division_by_shifting}) nella sezione \q{\ShiftsSectionName}, uno shift a sinistra
di 2 bit equivale a moltiplicare per 4.

In seguito viene aggiunto $R0*4$ all'attuale valore in \ac{PC}, saltando quindi ad una delle istruzioni \TT{B} (\emph{Branch}) poste sotto.

Al momento dell'esecuzione di \TT{ADDCC}, il valore in \ac{PC} si trova 8 byte più avanti (\TT{0x180})
rispetto all'indirizzo a cui si trova l'istruzione \TT{ADDCC} (\TT{0x178}), 
o, in altre parole, 2 istruzioni più avanti.

\myindex{ARM!Pipeline}

La pipeline nei processori ARM funziona così: nel momento in cui \TT{ADDCC} viene eseguita,
il processore sta iniziando a processare l'istruzione successiva, e questo è il motivo per cui \ac{PC} punta a quella.
Bisogna ricordarsi di ciò e tenerne conto.

Se $a=0$, viene aggiunta al valore in \ac{PC},
e l'attuale valore del \ac{PC} sarà scritto in \ac{PC} (che è 8 byte più avanti)
e si verificherà un salto alla label \emph{loc\_180},
che si trova 8 byte più avanti rispetto al punto in cui si trova l'istruzione \TT{ADDCC}.

Se $a=1$, allora $PC+8+a*4 = PC+8+1*4 = PC+12 = 0x184$ sarà scritto in \ac{PC}, ovvero l'indirizzo della label \emph{loc\_184}.

Ogni volta che si aggiunge 1 ad $a$, il risultante \ac{PC} è incrementato di 4.

4 è la lunghezza delle istruzioni in modalità ARM, comprese le istruzioni \TT{B} di cui ne abbiamo 5.

Ognuna di queste istruzioni \TT{B} passa il controllo più avanti, a quello che era previsto nello \emph{switch()}.
Li' avviene il caricamento del puntatore alla stringa corrispondente al caso, etc.

\subsubsection{ARM: \OptimizingKeilVI (\ThumbMode)}

\lstinputlisting[caption=\OptimizingKeilVI (\ThumbMode),style=customasmARM]{patterns/08_switch/2_lot/lot_ARM_thumb_O3.asm}

\myindex{ARM!\ThumbMode}
\myindex{ARM!\ThumbTwoMode}

Non possiamo essere certi che tutte le istruzioni in modalità Thumb e Thumb-2 siano della stessa lunghezza. Si può dire
che in queste modalità le istruzioni hanno lunghezza variabile, proprio come in x86.

\myindex{jumptable}

E' stata aggiunta una speciale tabella che contiene informazioni su quanti casi sono previsti (escluso il default-case),
ed un offset per ciascuno di essi, con una label a cui deve essere passato il controllo per il caso corrispondente.


\myindex{ARM!Mode switching}
\myindex{ARM!\Instructions!BX}

E' anche presente una funzione speciale per gestire la tabella e passare il controllo, \\
chiamata \emph{\_\_ARM\_common\_switch8\_thumb}. 
Inizia con l'istruzione \TT{BX PC}, la cui funzione è quella di mettere il processore in ARM-mode.
Subito dopo c'è la funzione per il processamento della tabella.

E' troppo avanzata per essere analizzata adesso, e per il momento la saltiamo.
% TODO explain it...

\myindex{ARM!\Registers!Link Register}

E' interessante notare ch ela funzione usa il registro \ac{LR} come puntatore alla tabella.

Infatti, dopo la chiamata a questa funzione, \ac{LR} contiene l'indirizzo subito dopo l'istruzione\\
\TT{BL \_\_ARM\_common\_switch8\_thumb}, dove inizia appunto la tabella.

Vale anche la pena notare che il codice è generato come una funzione separata, così che possa essere riutilizzata, e il compilatore
debba generare lo stesso codice per ogni istruzione switch().

\IDA ha correttamente capito che si tratta di una funzione di servizio e di una tabella, ed ha aggiunto i commenti alle label come\\
\TT{jumptable 000000FA case 0}.

