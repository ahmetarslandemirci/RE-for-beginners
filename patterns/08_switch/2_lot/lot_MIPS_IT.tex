\subsubsection{MIPS}

\lstinputlisting[caption=\Optimizing GCC 4.4.5 (IDA),style=customasmMIPS]{patterns/08_switch/2_lot/MIPS_O3_IDA_EN.lst}

\myindex{MIPS!\Instructions!SLTIU}

La nuova istruzione che incontriamo è \INS{SLTIU} (\q{Set on Less Than Immediate Unsigned}).
\myindex{MIPS!\Instructions!SLTU}

E' uguale a \INS{SLTU} (\q{Set on Less Than Unsigned}), e la \q{I} sta per \q{immediate}, 
e prevede cioè che un valore sia specificato nell'istruzione stessa.

\myindex{MIPS!\Instructions!BNEZ}
\INS{BNEZ} is \q{Branch if Not Equal to Zero}.

Il codice è molto simile a quello di altre \ac{ISA}.
\myindex{MIPS!\Instructions!SLL}
\INS{SLL} (\q{Shift Word Left Logical}) moltiplica per 4.

MIPS è una CPU a 32-bit, e tutti gli indirizzi contenuti nella \emph{jumptable} sono quindi a 32-bit.

