\subsection{Molti casi}

Se uno statement \TT{switch()} contiene molti casi, per il compilatore non è molto conveniente enettere codice troppo lungo con un sacco di 
istruzioni \JE/\JNE.

\lstinputlisting[label=switch_lot_c,style=customc]{patterns/08_switch/2_lot/lot.c}

\subsubsection{x86}

\myparagraph{\NonOptimizing MSVC}

Con MSVC 2010 otteniamo:

\lstinputlisting[caption=MSVC 2010,style=customasmx86]{patterns/08_switch/2_lot/lot_msvc_EN.asm}

\myindex{jumptable}

Vediamo una serie di chiamate a \printf con vari argomenti. Hanno tutte non solo indirizzi nella memoria del processo, ma anche etichette
simboliche assegnate dal compilatore. Queste label sono anche menzionate nalle tabella interna \TT{\$LN11@f}.

All'inizio della funzione, se $a$ è maggiore di 4, il controllo del flusso è passato alla label 
\TT{\$LN1@f}, dove viene chiamata \printf con argomento \TT{'something unknown'}.

Se invece il valore di $a$ è minore o uguale a 4, viene moltiplicato per 4 e sommato all'indirizzo della tabella \TT{\$LN11@f}. 
In questo modo vengono costruiti gli indirizzi della tabell, facendo puntare esattamente all'elemento giusto per ogni caso,

Poniamo ad esempio che $a$ sia uguale a 2. $2*4 = 8$ (tutti gli elementi della tabella sono indirizzi in un processo a 32-bit, perciò tutti gli elementi sono larghi 4 byte).
L'indirizzo della tabella \TT{\$LN11@f} + 8 corrisponde all'elemento della tabella in cui è memorizzata la label \TT{\$LN4@f}.
L'istruzione \JMP recupera quindi l'indirizzo di \TT{\$LN4@f} dalla tabella e salta.

Questa tabella è talvolta detta \emph{jumptable} o \emph{branch table}\footnote{L'intero metodo una volta era noto come 
\emph{computed GOTO} nelle prime versioni di Fortran:
\href{http://go.yurichev.com/17122}{wikipedia}.
Non è molto rilevante oggigiorno, ma che termine!}.

Successivamente la corrispondente \printf viene chiamata con argomento \TT{'two'}.\\
Letteralmente, l'istruzione \TT{jmp DWORD PTR \$LN11@f[ecx*4]} corrisponde a 
\emph{salta alla DWORD che è memorizzata all'indirizzo} \TT{\$LN11@f + ecx * 4}.

\TT{npad} (\myref{sec:npad}) è una macro del linguaggio assembly che allinea la prossima label in modo tale che sia memorizzata 
ad un indirizzo allineato a 4 byte (or a 16 byte).

Ciò è molto utile in termini di performance poiché così il processore è in grado di recuperare valori a 32-bit dalla memoria attraverso il memory bus, 
la cache, etc., in maniera più efficiente.y if it is aligned.

\clearpage
\mysubparagraph{\olly}
\myindex{\olly}

Esaminiamo questo esempio con \olly.
Il valore di input della funzione (2) viene caricato \EAX: 

\begin{figure}[H]
\centering
\myincludegraphics{patterns/08_switch/2_lot/olly1.png}
\caption{\olly: il valore di input è caricato in \EAX}
\label{fig:switch_lot_olly1}
\end{figure}

\clearpage
Il valore viene controllato, è maggiore di 4?
Se no, il \q{default} jump non viene innescato:
\begin{figure}[H]
\centering
\myincludegraphics{patterns/08_switch/2_lot/olly2.png}
\caption{\olly: 2 non è maggiore di 4: il salto non viene fatto}
\label{fig:switch_lot_olly2}
\end{figure}

\clearpage
Qui vediamo un jumptable:

\begin{figure}[H]
\centering
\myincludegraphics{patterns/08_switch/2_lot/olly3.png}
\caption{\olly: calcolo dell'indirizzo di destinazione mediante jumptable}
\label{fig:switch_lot_olly3}
\end{figure}

Qui abbiamo cliccato \q{Follow in Dump} $\rightarrow$ \q{Address constant}, così da vedere la \emph{jumptable} nella data window.
Sono 5 valori a 32-bit \footnote{Sono sottolineati da \olly poiché
sono anche FIXUPs: \myref{subsec:relocs}, torneremo su questo argomento più avanti}.
\ECX adesso è 2, quindi il terzo elemento (avente indice 2\footnote{Per l'indicizzazione, vedi anche: \ref{arrays_at_one}}) della tabella.
E' anche possibile cliccare su \q{Follow in Dump} $\rightarrow$ 
\q{Memory address} e \olly mostrerà l'elemento a cui punta l'istruzione \JMP. 
In questo caso è \TT{0x010B103A}.

\clearpage
Dopo il salto ci troviamo a \TT{0x010B103A}: il codice che stampa \q{two} sarà ora eseguito:

\begin{figure}[H]
\centering
\myincludegraphics{patterns/08_switch/2_lot/olly4.png}
\caption{\olly: ora ci troviamo alla label \emph{case:}}
\label{fig:switch_lot_olly4}
\end{figure}


\myparagraph{\NonOptimizing GCC}
\label{switch_lot_GCC}

Vediamo il codice generato da GCC 4.4.1:

\lstinputlisting[caption=GCC 4.4.1,style=customasmx86]{patterns/08_switch/2_lot/lot_gcc.asm}

\myindex{x86!\Registers!JMP}

E' pressoché identico, con una leggera variazione: l'argomento \TT{arg\_0} è moltiplicato per 4
effettuando uno shift a sinistra di 2 bit (quasi identico alla moltiplicazione per 4)~(\myref{SHR}).
Successivamente l'indirizzo della label è preso dall'array \TT{off\_804855C}, memorizzato in 
\EAX, e infine \TT{JMP EAX} effettua il salto.


\subsubsection{ARM: \OptimizingKeilVI (\ARMMode)}
\label{sec:SwitchARMLot}

\lstinputlisting[caption=\OptimizingKeilVI (\ARMMode),style=customasmARM]{patterns/08_switch/2_lot/lot_ARM_ARM_O3.asm}

Il codice fa uso della modalità ARM in cui tutte le istruzioni hanno dimensione fissa di 4 byte.
Ricordiamoci che il massimo valore previsto per $a$ è 4 e ogni valore maggiore causerà la stampa della stringa 
\emph{<<something unknown\textbackslash{}n>>}.

\myindex{ARM!\Instructions!CMP}
\myindex{ARM!\Instructions!ADDCC}
La prima istruzione \TT{CMP R0, \#5} confronta il valore di $a$ con 5.

\footnote{ADD---addition}
La successiva \TT{ADDCC PC, PC, R0,LSL\#2} viene eseguita solo se $R0 < 5$ (\emph{CC=Carry clear / Less than}). 
Di conseguenza, se \TT{ADDCC} non viene innescata (è il caso $R0 \geq 5$), si verificherà un jump a \emph{default\_case}.

Se invece $R0 < 5$ e \TT{ADDCC} viene innescata, succede quanto segue:

Il valore in \Reg{0} viene moltiplicato per 4.
Infatti \TT{LSL\#2} nel suffisso dell'istruzione sta per \q{shift left by 2 bits} (shift a sinistra di 2 bit).
Come vedremo più avanti ~(\myref{division_by_shifting}) nella sezione \q{\ShiftsSectionName}, uno shift a sinistra
di 2 bit equivale a moltiplicare per 4.

In seguito viene aggiunto $R0*4$ all'attuale valore in \ac{PC}, saltando quindi ad una delle istruzioni \TT{B} (\emph{Branch}) poste sotto.

Al momento dell'esecuzione di \TT{ADDCC}, il valore in \ac{PC} si trova 8 byte più avanti (\TT{0x180})
rispetto all'indirizzo a cui si trova l'istruzione \TT{ADDCC} (\TT{0x178}), 
o, in altre parole, 2 istruzioni più avanti.

\myindex{ARM!Pipeline}

La pipeline nei processori ARM funziona così: nel momento in cui \TT{ADDCC} viene eseguita,
il processore sta iniziando a processare l'istruzione successiva, e questo è il motivo per cui \ac{PC} punta a quella.
Bisogna ricordarsi di ciò e tenerne conto.

Se $a=0$, viene aggiunta al valore in \ac{PC},
e l'attuale valore del \ac{PC} sarà scritto in \ac{PC} (che è 8 byte più avanti)
e si verificherà un salto alla label \emph{loc\_180},
che si trova 8 byte più avanti rispetto al punto in cui si trova l'istruzione \TT{ADDCC}.

Se $a=1$, allora $PC+8+a*4 = PC+8+1*4 = PC+12 = 0x184$ sarà scritto in \ac{PC}, ovvero l'indirizzo della label \emph{loc\_184}.

Ogni volta che si aggiunge 1 ad $a$, il risultante \ac{PC} è incrementato di 4.

4 è la lunghezza delle istruzioni in modalità ARM, comprese le istruzioni \TT{B} di cui ne abbiamo 5.

Ognuna di queste istruzioni \TT{B} passa il controllo più avanti, a quello che era previsto nello \emph{switch()}.
Li' avviene il caricamento del puntatore alla stringa corrispondente al caso, etc.

\subsubsection{ARM: \OptimizingKeilVI (\ThumbMode)}

\lstinputlisting[caption=\OptimizingKeilVI (\ThumbMode),style=customasmARM]{patterns/08_switch/2_lot/lot_ARM_thumb_O3.asm}

\myindex{ARM!\ThumbMode}
\myindex{ARM!\ThumbTwoMode}

Non possiamo essere certi che tutte le istruzioni in modalità Thumb e Thumb-2 siano della stessa lunghezza. Si può dire
che in queste modalità le istruzioni hanno lunghezza variabile, proprio come in x86.

\myindex{jumptable}

E' stata aggiunta una speciale tabella che contiene informazioni su quanti casi sono previsti (escluso il default-case),
ed un offset per ciascuno di essi, con una label a cui deve essere passato il controllo per il caso corrispondente.


\myindex{ARM!Mode switching}
\myindex{ARM!\Instructions!BX}

E' anche presente una funzione speciale per gestire la tabella e passare il controllo, \\
chiamata \emph{\_\_ARM\_common\_switch8\_thumb}. 
Inizia con l'istruzione \TT{BX PC}, la cui funzione è quella di mettere il processore in ARM-mode.
Subito dopo c'è la funzione per il processamento della tabella.

E' troppo avanzata per essere analizzata adesso, e per il momento la saltiamo.
% TODO explain it...

\myindex{ARM!\Registers!Link Register}

E' interessante notare ch ela funzione usa il registro \ac{LR} come puntatore alla tabella.

Infatti, dopo la chiamata a questa funzione, \ac{LR} contiene l'indirizzo subito dopo l'istruzione\\
\TT{BL \_\_ARM\_common\_switch8\_thumb}, dove inizia appunto la tabella.

Vale anche la pena notare che il codice è generato come una funzione separata, così che possa essere riutilizzata, e il compilatore
debba generare lo stesso codice per ogni istruzione switch().

\IDA ha correttamente capito che si tratta di una funzione di servizio e di una tabella, ed ha aggiunto i commenti alle label come\\
\TT{jumptable 000000FA case 0}.


\subsubsection{MIPS}

\lstinputlisting[caption=\Optimizing GCC 4.4.5 (IDA),style=customasmMIPS]{patterns/08_switch/2_lot/MIPS_O3_IDA_EN.lst}

\myindex{MIPS!\Instructions!SLTIU}

La nuova istruzione che incontriamo è \INS{SLTIU} (\q{Set on Less Than Immediate Unsigned}).
\myindex{MIPS!\Instructions!SLTU}

E' uguale a \INS{SLTU} (\q{Set on Less Than Unsigned}), e la \q{I} sta per \q{immediate}, 
e prevede cioè che un valore sia specificato nell'istruzione stessa.

\myindex{MIPS!\Instructions!BNEZ}
\INS{BNEZ} is \q{Branch if Not Equal to Zero}.

Il codice è molto simile a quello di altre \ac{ISA}.
\myindex{MIPS!\Instructions!SLL}
\INS{SLL} (\q{Shift Word Left Logical}) moltiplica per 4.

MIPS è una CPU a 32-bit, e tutti gli indirizzi contenuti nella \emph{jumptable} sono quindi a 32-bit.



\subsubsection{\Conclusion{}}

Stuttura approssimativa di \emph{switch()}:

% TODO: ARM, MIPS skeleton
\lstinputlisting[caption=x86,style=customasmx86]{patterns/08_switch/2_lot/skel1_IT.lst}

Il salto agli indirizzi nella tabella di jump può anche essere implementato usando questa istruzione: \\
\TT{JMP jump\_table[REG*4]}.
oppure \TT{JMP jump\_table[REG*8]} in x64.

Una \emph{jumptable} è semplicemente un array di puntatori, come quello descritto più avanti: \myref{array_of_pointers_to_strings}.
