\subsection{И если много}

Если ветвлений слишком много, то генерировать слишком длинный код с многочисленными \JE/\JNE 
уже не так удобно.

\lstinputlisting[label=switch_lot_c,style=customc]{patterns/08_switch/2_lot/lot.c}

\subsubsection{x86}

\myparagraph{\NonOptimizing MSVC}

Рассмотрим пример, скомпилированный в (MSVC 2010):

\lstinputlisting[caption=MSVC 2010,style=customasmx86]{patterns/08_switch/2_lot/lot_msvc_RU.asm}

\myindex{jumptable}
Здесь происходит следующее: в теле функции есть набор вызовов \printf с разными аргументами. 
Все они имеют, конечно же, адреса, а также внутренние символические метки, которые присвоил им компилятор.
Также все эти метки указываются во внутренней таблице \TT{\$LN11@f}.

В начале функции, если $a$ больше 4, то сразу происходит переход на метку \TT{\$LN1@f}, 
где вызывается \printf с аргументом \TT{'something unknown'}.

А если $a$ меньше или равно 4, то это значение умножается на 4 и прибавляется адрес таблицы 
с переходами (\TT{\$LN11@f}). 
Таким образом, получается адрес внутри таблицы, где лежит нужный адрес внутри тела функции. 
Например, возьмем $a$ равным 2. $2*4 = 8$ (ведь все элементы таблицы~--- это адреса внутри 32-битного процесса, 
таким образом, каждый элемент занимает 4 байта). 8 прибавить к \TT{\$LN11@f}~--- это будет элемент таблицы,
где лежит \TT{\$LN4@f}. \JMP вытаскивает из таблицы адрес \TT{\$LN4@f} и делает безусловный переход туда.

Эта таблица иногда называется \emph{jumptable} или
\emph{branch table}\footnote{Сам метод раньше назывался 
\emph{computed GOTO} В ранних версиях Фортрана:
\href{http://go.yurichev.com/17122}{wikipedia}.
Не очень-то и полезно в наше время, но каков термин!}.

А там вызывается \printf с аргументом \TT{'two'}. 
Дословно, инструкция \TT{jmp DWORD PTR \$LN11@f[ecx*4]} 
означает \emph{перейти по DWORD, который лежит по адресу} \TT{\$LN11@f + ecx * 4}.

\TT{npad} (\myref{sec:npad}) это макрос ассемблера, выравнивающий начало таблицы, 
чтобы она располагалась по адресу кратному 4 (или 16).
Это нужно для того, чтобы процессор мог эффективнее загружать 32-битные 
значения из памяти через шину с памятью, кэш-память, итд.

\clearpage
\mysubparagraph{\olly}
\myindex{\olly}

Попробуем этот пример в \olly.
Входное значение функции (2) загружается в \EAX: 

\begin{figure}[H]
\centering
\myincludegraphics{patterns/08_switch/2_lot/olly1.png}
\caption{\olly: входное значение функции загружено в \EAX}
\label{fig:switch_lot_olly1}
\end{figure}

\clearpage
Входное значение проверяется, не больше ли оно чем 4? 
Нет, переход по умолчанию (\q{default}) не будет исполнен:

\begin{figure}[H]
\centering
\myincludegraphics{patterns/08_switch/2_lot/olly2.png}
\caption{\olly: 2 не больше чем 4: переход не сработает}
\label{fig:switch_lot_olly2}
\end{figure}

\clearpage
Здесь мы видим jumptable:

\begin{figure}[H]
\centering
\myincludegraphics{patterns/08_switch/2_lot/olly3.png}
\caption{\olly: вычисляем адрес для перехода используя jumptable}
\label{fig:switch_lot_olly3}
\end{figure}

Кстати, щелкнем по \q{Follow in Dump} $\rightarrow$ \q{Address constant}, так что теперь \emph{jumptable} видна в окне данных.

Это 5 32-битных значений\footnote{Они подчеркнуты в \olly, потому что это также и FIXUP-ы: \myref{subsec:relocs}, мы вернемся к ним позже}.
\ECX сейчас содержит 2, так что третий элемент (может индексироваться как 2\footnote{Об индексаци, см.также: \ref{arrays_at_one}}) таблицы будет использован.
Кстати, можно также щелкнуть \q{Follow in Dump} $\rightarrow$ \q{Memory address} и \olly покажет элемент, который сейчас адресуется в инструкции \JMP. 
Это \TT{0x010B103A}.

\clearpage
Переход сработал и мы теперь на \TT{0x010B103A}: сейчас будет исполнен код, выводящий строку \q{two}:

\begin{figure}[H]
\centering
\myincludegraphics{patterns/08_switch/2_lot/olly4.png}
\caption{\olly: теперь мы на соответствующей метке \emph{case:}}
\label{fig:switch_lot_olly4}
\end{figure}


\myparagraph{\NonOptimizing GCC}
\label{switch_lot_GCC}

Посмотрим, что сгенерирует GCC 4.4.1:

\lstinputlisting[caption=GCC 4.4.1,style=customasmx86]{patterns/08_switch/2_lot/lot_gcc.asm}

\myindex{x86!\Registers!JMP}
Практически то же самое, за исключением мелкого нюанса: аргумент из \TT{arg\_0} умножается на 4 
при помощи сдвига влево на 2 бита (это почти то же самое что и умножение на 4)~(\myref{SHR}).
Затем адрес метки внутри функции берется из массива \TT{off\_804855C} и адресуется при помощи 
вычисленного индекса.


\subsubsection{ARM: \OptimizingKeilVI (\ARMMode)}
\label{sec:SwitchARMLot}

\lstinputlisting[caption=\OptimizingKeilVI (\ARMMode),style=customasmARM]{patterns/08_switch/2_lot/lot_ARM_ARM_O3.asm}

В этом коде используется та особенность режима ARM, 
что все инструкции в этом режиме имеют фиксированную длину 4 байта.

Итак, не будем забывать, что максимальное значение для $a$ это 4: всё что выше, должно вызвать
вывод строки \emph{<<something unknown\textbackslash{}n>>}.

\myindex{ARM!\Instructions!CMP}
\myindex{ARM!\Instructions!ADDCC}
Самая первая инструкция \TT{CMP R0, \#5} сравнивает входное значение в $a$ c 5.

\footnote{ADD---складывание чисел}
Следующая инструкция \TT{ADDCC PC, PC, R0,LSL\#2} сработает только в случае если $R0 < 5$ (\emph{CC=Carry clear / Less than}). 
Следовательно, если \TT{ADDCC} не сработает (это случай с $R0 \geq 5$), выполнится переход на метку 
\emph{default\_case}.

Но если $R0 < 5$ и \TT{ADDCC} сработает, то произойдет следующее.

Значение в \Reg{0} умножается на 4.
Фактически, \TT{LSL\#2} в суффиксе инструкции означает \q{сдвиг влево на 2 бита}.

Но как будет видно позже~(\myref{division_by_shifting}) в секции \q{\ShiftsSectionName}, 
сдвиг влево на 2 бита, это эквивалентно его умножению на 4.

Затем полученное $R0*4$ прибавляется к текущему значению \ac{PC}, 
совершая, таким образом, переход на одну из расположенных ниже инструкций \TT{B} (\emph{Branch}).

На момент исполнения \TT{ADDCC},
содержимое \ac{PC} на 8 байт больше (\TT{0x180}), чем адрес по которому расположена сама инструкция \TT{ADDCC} (\TT{0x178}), 
либо, говоря иным языком, на 2 инструкции больше.

\myindex{ARM!Конвейер}
Это связано с работой конвейера процессора ARM:
пока исполняется инструкция \TT{ADDCC}, процессор уже начинает обрабатывать инструкцию после следующей, 
поэтому \ac{PC} указывает туда. Этот факт нужно запомнить.

Если $a=0$, тогда к \ac{PC} ничего не будет прибавлено и 
в \ac{PC} запишется актуальный на тот момент \ac{PC} (который больше на 8) 
и произойдет переход на метку \emph{loc\_180}. 
Это на 8 байт дальше места, где находится инструкция \TT{ADDCC}.

Если $a=1$, тогда в \ac{PC} запишется 
$PC+8+a*4 = PC+8+1*4 = PC+12 = 0x184$. Это адрес метки \emph{loc\_184}.

При каждой добавленной к $a$ единице итоговый \ac{PC} увеличивается на 4.

4 это длина инструкции в режиме ARM и одновременно с этим, 
длина каждой инструкции \TT{B}, их здесь следует 5 в ряд.

Каждая из этих пяти инструкций \TT{B} передает управление дальше, где собственно и происходит то, 
что запрограммировано в операторе \emph{switch()}.
Там происходит загрузка указателя на свою строку, итд.

\subsubsection{ARM: \OptimizingKeilVI (\ThumbMode)}

\lstinputlisting[caption=\OptimizingKeilVI (\ThumbMode),style=customasmARM]{patterns/08_switch/2_lot/lot_ARM_thumb_O3.asm}

\myindex{ARM!\ThumbMode}
\myindex{ARM!\ThumbTwoMode}
В режимах Thumb и Thumb-2 уже нельзя надеяться на то, что все инструкции имеют одну длину.

Можно даже сказать, что в этих режимах инструкции переменной длины, как в x86.

\myindex{jumptable}
Так что здесь добавляется специальная таблица, содержащая информацию о том, как много вариантов здесь,
не включая варианта по умолчанию, и смещения, для каждого варианта. Каждое смещение кодирует метку, куда нужно передать
управление в соответствующем случае.

\myindex{ARM!Переключение режимов}
\myindex{ARM!\Instructions!BX}
Для того чтобы работать с таблицей и совершить переход, вызывается служебная функция

\emph{\_\_ARM\_common\_switch8\_thumb}. 
Она начинается с инструкции \TT{BX PC}, чья функция~--- переключить процессор в ARM-режим.

Далее функция, работающая с таблицей. 
Она слишком сложная для рассмотрения в данном месте, так что пропустим это.

% TODO explain it...

\myindex{ARM!\Registers!Link Register}
Но можно отметить, что эта функция использует регистр \ac{LR} как указатель на таблицу.

Действительно, после вызова этой функции, в \ac{LR} был записан адрес после инструкции

\TT{BL \_\_ARM\_common\_switch8\_thumb}, а там как раз и начинается таблица.

Ещё можно отметить, что код для этого выделен в отдельную функцию для того, 
чтобы не нужно было каждый раз генерировать 
точно такой же фрагмент кода для каждого выражения switch().

\IDA распознала эту служебную функцию и таблицу автоматически дописала комментарии к меткам вроде \\
\TT{jumptable 000000FA case 0}.


\subsubsection{MIPS}

\lstinputlisting[caption=\Optimizing GCC 4.4.5 (IDA),style=customasmMIPS]{patterns/08_switch/2_lot/MIPS_O3_IDA_RU.lst}

\myindex{MIPS!\Instructions!SLTIU}
Новая для нас инструкция здесь это \INS{SLTIU} (\q{Set on Less Than Immediate Unsigned}~--- установить,
если меньше чем значение, беззнаковое сравнение).

\myindex{MIPS!\Instructions!SLTU}
На самом деле, это то же что и \INS{SLTU} (\q{Set on Less Than Unsigned}), но \q{I} означает \q{immediate},
т.е. число может быть задано в самой инструкции.

\myindex{MIPS!\Instructions!BNEZ}
\INS{BNEZ} это \q{Branch if Not Equal to Zero} (переход если не равно нулю).

Код очень похож на код для других \ac{ISA}.
\myindex{MIPS!\Instructions!SLL}
\INS{SLL} (\q{Shift Word Left Logical}~--- логический сдвиг влево) совершает умножение на 4.
MIPS всё-таки это 32-битный процессор, так что все адреса в таблице переходов (\emph{jumptable}) 32-битные.



\subsubsection{\Conclusion{}}

Примерный скелет оператора \emph{switch()}:

% TODO: ARM, MIPS skeleton
\lstinputlisting[caption=x86,style=customasmx86]{patterns/08_switch/2_lot/skel1_RU.lst}

Переход по адресу из таблицы переходов может быть также реализован такой инструкцией: \\
\TT{JMP jump\_table[REG*4]}. Или \TT{JMP jump\_table[REG*8]} в x64.

Таблица переходов (\emph{jumptable}) это просто массив указателей, как это будет вскоре описано: \myref{array_of_pointers_to_strings}.
