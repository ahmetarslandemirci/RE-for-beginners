\subsection{アドレッシングモード}
\myindex{ARM!Addressing modes}
\label{ARM_postindex_vs_preindex}
\myindex{\CLanguageElements!\PostIncrement}
\myindex{\CLanguageElements!\PostDecrement}
\myindex{\CLanguageElements!\PreIncrement}
\myindex{\CLanguageElements!\PreDecrement}

この命令はARM64でも使用できます。

\myindex{ARM!\Instructions!LDR}
\begin{lstlisting}[style=customasmARM]
ldr	x0, [x29,24]
\end{lstlisting}

これは、X29の値に24を加えて、このアドレスから値をロードすることを意味します。

24が括弧の内側にあることに注意してください。
数字が括弧の外側にある場合、意味は異なります。

\begin{lstlisting}[style=customasmARM]
ldr	w4, [x1],28
\end{lstlisting}

これは、X1のアドレスに値をロードしてから、X1に28を加算することを意味します。

\myindex{PDP-11}

ARMでは、ロードに使用されるアドレスに定数を追加したりそこから定数を減算したりできます。

そして、ロードの前後にそれをすることは可能です。

x86にはそのようなアドレッシングモードはありませんが、他のプロセッサには、PDP-11でさえ、存在します。

PDP-11の前置インクリメント、後置インクリメント、前置デクリメント、後置デクリメントの各モードは、(PDP-11上で開発された)そのようなC言語の構造体が
*ptr++, *++ptr, *ptr-{}-, *-{}-ptr として出現したことに対して\q{罪}がありました。

ところで、これはCの機能を暗記するのが難しいことの1つです。
このようになっています。

\small
% FIXME: add ARM assembly...
\begin{center}
\begin{tabular}{ | l | l | l | l | }
\hline
\headercolor{} C term & 
\headercolor{} ARM term & 
\headercolor{} C statement & 
\headercolor{} how it works \\
\hline
\PostIncrement & 
後置インデックスアドレッシング & 
\TT{*ptr++} & 
\TT{*ptr}の値を使い、 \\
& & & 次に \TT{ptr} ポインタを \\
& & & \gls{increment} \\
\hline
\PostDecrement & 
後置インデックスアドレッシング & 
\TT{*ptr-{}-} & 
\TT{*ptr}の値を使い、 \\
& & & 次に \TT{ptr} ポインタを \\
& & & \gls{decrement} \\
\hline
\PreIncrement & 
前置インデックスアドレッシング & 
\TT{*++ptr} & 
\TT{ptr}ポインタを\gls{increment}、 \\
& & & 次に \TT{*ptr}の値を \\
& & & 使用 \\
\hline
\PreDecrement & 
前置インデックスアドレッシング & 
\TT{*-{}-ptr} & 
\TT{ptr}の値を\gls{decrement}、 \\
& & & 次に \TT{*ptr}の値を \\
& & & 使用 \\
\hline
\end{tabular}
\end{center}
\normalsize

ARMのアセンブリ言語では、プレインデクシングに感嘆符が付けられています。
たとえば、\lstref{hw_ARM64_GCC} の2行目を参照してください。

デニス・リッチー(C言語の作成者の一人)は、このプロセッサの機能がPDP-7
\footnote{\url{http://yurichev.com/mirrors/C/c_dmr_postincrement.txt}} に存在していたため、
ケン・トンプソン(もう一人のC言語作成者)によって発明されたと考えていると述べました \RitchieDevC{}

したがって、C言語コンパイラは、それがターゲットプロセッサに存在する場合は使用できます。

配列処理にはとても便利です。
