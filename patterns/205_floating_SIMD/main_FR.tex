% FIXME1 divide this file into separate ones...
\mysection{Travailler avec des nombres à virgule flottante en utilisant SIMD}

\label{floating_SIMD}
\myindex{IEEE 754}
\myindex{SIMD}
\myindex{SSE}
\myindex{SSE2}

Bien sûr. le \ac{FPU} est resté dans les processeurs compatible x86 lorsque les extensions
\ac{SIMD} ont été ajoutées.

L'extension \ac{SIMD} (SSE2) offre un moyen facile de travailler avec des nombres
à virgule flottante.

Le format des nombres reste le même (IEEE 754).

\myindex{x86-64}
Donc, les compilateurs modernes (incluant ceux générant pour x86-64) utilisent les
instructions \ac{SIMD} au lieu de celles pour FPU.

On peut dire que c'est une bonne nouvelle, car il est plus facile de travailler avec
elles.

Nous allons ré-utiliser les exemples de la section FPU ici: \myref{sec:FPU}.

\subsection{Simple exemple}

\lstinputlisting[style=customc]{patterns/12_FPU/1_simple/simple.c}

\subsubsection{x64}

\lstinputlisting[caption=MSVC 2012 x64 \Optimizing,style=customasmx86]{patterns/205_floating_SIMD/simple_MSVC_2012_x64_Ox.asm}

Les valeurs en virgule flottante entrées sont passées dans les registres \XMM{0}-\XMM{3},
tout le reste---via la pile
\footnote{\href{http://go.yurichev.com/17263}{MSDN: Parameter Passing}}.

$a$ est passé dans \XMM{0}, $b$---via \XMM{1}.

Les registres XMM font 128-bit (comme nous le savons depuis la section à propos de
\ac{SIMD}: \myref{SIMD_x86}), mais les valeurs \Tdouble font 64-bit, donc seulement
la moitié basse du registre est utilisée.

\myindex{x86!\Instructions!DIVSD}
\TT{DIVSD} est une instruction SSE qui signifie \q{Divide Scalar Double-Precision
Floating-Point Values} (Diviser des nombres flottants double-précision), elle divise
une valeur de type \Tdouble par une autre, stockées dans la moitié basse des opérandes.

Les constantes sont encodées par le compilateur au format IEEE 754.

\myindex{x86!\Instructions!MULSD}
\myindex{x86!\Instructions!ADDSD}
\TT{MULSD} et \TT{ADDSD} fonctionnent de même, mais font la multiplication et l'addition.

Le résultat de l'exécution de la fonction de type \Tdouble est laissé dans le registre
\XMM{0}.\\
\\
C'est ainsi que travaille MSVC sans optimisation:

\lstinputlisting[caption=MSVC 2012 x64,style=customasmx86]{patterns/205_floating_SIMD/simple_MSVC_2012_x64.asm}

\myindex{Shadow space}
Légèrement redondant.
Les arguments en entrée sont sauvés dans le \q{shadow space} (\myref{shadow_space}),
mais seule leur moitié inférieure, i.e., seulement la valeur 64-bit de type \Tdouble.
GCC produit le même code.

\subsubsection{x86}

Compilons cet exemple pour x86. Bien qu'il compile pour x86, MSVC 2012 utilise des
instructions SSE2:

\lstinputlisting[caption=MSVC 2012 x86 \NonOptimizing,style=customasmx86]{patterns/205_floating_SIMD/simple_MSVC_2012_x86.asm}

\lstinputlisting[caption=MSVC 2012 x86 \Optimizing,style=customasmx86]{patterns/205_floating_SIMD/simple_MSVC_2012_x86_Ox.asm}

C'est presque le même code, toutefois, il y a quelques différences relatives aux
conventions d'appel:
1) les arguments ne sont pas passés dans des registres XMM, mais par la pile, comme
dans les exemples FPU (\myref{sec:FPU});
2) le résultat de la fonction est renvoyé dans \ST{0} --- afin de faire cela, il
est copié (à travers la variable locale \TT{tv}) depuis un des registres XMM dans
\ST{0}.

\clearpage
Essayons l'exemple optimisé dans \olly:

\begin{figure}[H]
\centering
\myincludegraphics{patterns/205_floating_SIMD/simple_olly1.png}
\caption{\olly: \TT{MOVSD} charge la valeur de $a$ dans \XMM{1}}
\label{fig:FPU_SIMD_simple_olly1}
\end{figure}

\clearpage
\begin{figure}[H]
\centering
\myincludegraphics{patterns/205_floating_SIMD/simple_olly2.png}
\caption{\olly: \TT{DIVSD} a calculé le \gls{quotient} 
et l'a stocké dans \XMM{1}}
\label{fig:FPU_SIMD_simple_olly2}
\end{figure}

\clearpage
\begin{figure}[H]
\centering
\myincludegraphics{patterns/205_floating_SIMD/simple_olly3.png}
\caption{\olly: \TT{MULSD} a calculé le \glslink{product}{produit} et l'a stocké
dans \XMM{0}}
\label{fig:FPU_SIMD_simple_olly3}
\end{figure}

\clearpage
\begin{figure}[H]
\centering
\myincludegraphics{patterns/205_floating_SIMD/simple_olly4.png}
\caption{\olly: \TT{ADDSD} ajoute la valeur dans \XMM{0} à celle dans \XMM{1}}
\label{fig:FPU_SIMD_simple_olly4}
\end{figure}

\clearpage
\begin{figure}[H]
\centering
\myincludegraphics{patterns/205_floating_SIMD/simple_olly5.png}
\caption{\olly: \FLD laisse le résultat de la fonction dans \ST{0}}
\label{fig:FPU_SIMD_simple_olly5}
\end{figure}

Nous voyons qu'\olly montre les registres XMM comme des paires de nombres \Tdouble,
mais seule la partie \emph{basse} est utilisée.

Apparemment, \olly les montre dans ce format car les instructions SSE2 (suffixées
avec \TT{-SD}) sont exécutées actuellement.

Mais bien sûr, il est possible de changer le format du registre et de voir le contenu
comme 4 nombres \Tfloat{} ou juste comme 16 octets.

\clearpage
\subsection{Passer des nombres à virgule flottante via les arguments}

\lstinputlisting[style=customc]{patterns/12_FPU/2_passing_floats/pow.c}

Ils sont passés dans la moitié basse des registres \XMM{0}-\XMM{3}.

\lstinputlisting[caption=MSVC 2012 x64 \Optimizing,style=customasmx86]{patterns/205_floating_SIMD/pow_MSVC_2012_x64_Ox.asm}

\myindex{x86!\Instructions!MOVSD}
\myindex{x86!\Instructions!MOVSDX}
Il n'y a pas d'instruction \TT{MOVSDX} dans les manuels Intel et AMD (\myref{x86_manuals}),
elle y est appelée \TT{MOVSD}.
Donc il y a deux instructions qui partagent le même nom en x86 (à propos de l'autre
lire: \myref{REP_MOVSx}).
Apparemment, les développeurs de Microsoft voulaient arrêter cette pagaille, donc
ils l'ont renommée \TT{MOVSDX}.
Elle charge simplement une valeur dans la moitié inférieure d'un registre XMM.

\TT{pow()} prends ses arguments de \XMM{0} et \XMM{1}, et renvoie le résultat dans
\XMM{0}.
Il est ensuite déplacé dans \RDX pour \printf.
Pourquoi?
Peut-être parce que \printf{}---est une fonction avec un nombre variable d'arguments?

\lstinputlisting[caption=GCC 4.4.6 x64 \Optimizing,style=customasmx86]{patterns/205_floating_SIMD/pow_GCC446_x64_O3_FR.s}

GCC génère une sortie plus claire.
La valeur pour \printf est passée dans \XMM{0}.
À propos, il y a un cas lorsque 1 est écrit dans \EAX pour \printf ---ceci implique
qu'un argument sera passé dans des registres vectoriels, comme le requiert le standard
\SysVABI.

\subsection{Exemple de comparaison}

\lstinputlisting[style=customc]{patterns/12_FPU/3_comparison/d_max.c}

\subsubsection{x64}

\lstinputlisting[caption=MSVC 2012 x64 \Optimizing,style=customasmx86]{patterns/205_floating_SIMD/d_max_MSVC_2012_x64_Ox.asm}

MSVC \Optimizing génère un code très facile à comprendre.

\myindex{x86!\Instructions!COMISD}
\TT{COMISD} is \q{Compare Scalar Ordered Double-Precision Floating-Point Values and
Set EFLAGS} (comparer des valeurs double précision en virgule flottante scalaire
ordrées et mettre les EFLAGS). Pratiquement, c'est ce qu'elle fait.\\
\\
MSVC \NonOptimizing génère plus de code redondant, mais il n'est toujours pas très
difficile à comprendre:

\lstinputlisting[caption=MSVC 2012 x64,style=customasmx86]{patterns/205_floating_SIMD/d_max_MSVC_2012_x64.asm}

\myindex{x86!\Instructions!MAXSD}
Toutefois, GCC 4.4.6 effectue plus d'optimisations et utilise l'instruction \TT{MAXSD}
(\q{Return Maximum Scalar Double-Precision Floating-Point Value}) qui choisit la
valeur maximum!

\lstinputlisting[caption=GCC 4.4.6 x64 \Optimizing,style=customasmx86]{patterns/205_floating_SIMD/d_max_GCC446_x64_O3.s}

\clearpage
\subsubsection{x86}

Compilons cet exemple dans MSVC 2012 avec l'optimisation activée:

\lstinputlisting[caption=MSVC 2012 x86 \Optimizing,style=customasmx86]{patterns/205_floating_SIMD/d_max_MSVC_2012_x86_Ox.asm}

Presque la même chose, mais les valeurs de $a$ et $b$ sont prises depuis la pile
et le résultat de la fonction est laissé dans \ST{0}.

Si nous chargeons cet exemple dans \olly, nous pouvons voir comment l'instruction
\TT{COMISD} compare les valeurs et met/efface les flags \CF et \PF:

\begin{figure}[H]
\centering
\myincludegraphics{patterns/205_floating_SIMD/d_max_olly.png}
\caption{\olly: \TT{COMISD} a changé les flags \CF et \PF}
\label{fig:FPU_SIMD_d_max_olly}
\end{figure}

\subsection{Calcul de l'epsilon de la machine: x64 et SIMD}
\label{machine_epsilon_x64_and_SIMD}

Revoyons l'exemple \q{calcul de l'epsilon de la machine} pour \Tdouble\ \lstref{machine_epsilon_double_c}.

Maintenant nous compilons pour x64:

\lstinputlisting[caption=MSVC 2012 x64 \Optimizing,style=customasmx86]{patterns/205_floating_SIMD/epsilon_double_MSVC_2012_x64_Ox.asm}

Il n'y a pas moyen d'ajouter 1 à une valeur dans un registre XMM 128-bit, donc il
doit être placé en mémoire.

Il y a toutefois l'instruction \INS{ADDSD} (\emph{Add Scalar Double-Precision Floating-Point
Values} ajouter des valeurs scalaires à virgule flottante double-précision), qui
peut ajouter une valeur dans la moitié 64-bit basse d'un registre XMM en ignorant
celle du haut, mais MSVC 2012 n'est probablement pas encore assez bon \footnote{À
titre d'exercice, vous pouvez retravailler ce code pour éliminer l'usage de la pile
locale}.

Néanmoins, la valeur est ensuite rechargée dans un registre XMM et la soustraction
est effectuée.
\INS{SUBSD} est \q{Subtract Scalar Double-Precision Floating-Point Values} (soustraire
des valeurs en virgule flottante double-précision), i.e., elle opère sur la partie
64-bit basse d'un registre XMM 128-bit.
Le résultat est renvoyé dans le registre XMM0.

\subsection{Méthodes de protection contre les débordements de tampon}
\label{subsec:BO_protection}

Il existe quelques méthodes pour protéger contre ce fléau, indépendamment de la négligence
des programmeurs \CCpp.
MSVC possède des options comme\footnote{méthode de protection contre les débordements
de tampons côté compilateur:\href{http://go.yurichev.com/17133}{wikipedia.org/wiki/Buffer\_overflow\_protection}}:

\begin{lstlisting}
 /RTCs Stack Frame runtime checking
 /GZ Enable stack checks (/RTCs)
\end{lstlisting}

\myindex{x86!\Instructions!RET}
\myindex{Function prologue}
\myindex{Security cookie}

Une des méthodes est d'écrire une valeur aléatoire entre les variables locales sur
la pile dans le prologue de la fonction et de la vérifier dans l'épilogue, avant de
sortir de la fonction.
Si la valeur n'est pas la même, ne pas exécuter la dernière instruction \RET, mais
stopper (ou bloquer).
Le processus va s'arrêter, mais c'est mieux qu'une attaque distante sur votre ordinateur.
    
\newcommand{\CANARYURL}{\href{http://go.yurichev.com/17134}{wikipedia.org/wiki/Domestic\_canary\#Miner.27s\_canary}}

\myindex{Canary}

Cette valeur aléatoire est parfois appelé un \q{canari}, c'est lié au canari\footnote{\CANARYURL}
que les mineurs utilisaient dans le passé afin de détecter rapidement les gaz toxiques.

Les canaris sont très sensibles aux gaz, ils deviennent très agités en cas de danger,
et même meurent.

Si nous compilons notre exemple de tableau très simple~(\myref{arrays_simple}) dans
\ac{MSVC} avec les options RTC1 et RTCs, nous voyons un appel à \TT{@\_RTC\_CheckStackVars@8}
une fonction à la fin de la fonction qui vérifie si le \q{canari} est correct.

Voyons comment GCC gère ceci.
Prenons un exemple \TT{alloca()}~(\myref{alloca}):

\lstinputlisting[style=customc]{patterns/02_stack/04_alloca/2_1.c}

Par défaut, sans option supplémentaire, GCC 4.7.3 insère un test de  \q{canari} dans
le code:

\lstinputlisting[caption=GCC 4.7.3,style=customasmx86]{patterns/13_arrays/3_BO_protection/gcc_canary_FR.asm}

\myindex{x86!\Registers!GS}
La valeur aléatoire se trouve en \TT{gs:20}.
Elle est écrite sur la pile et à la fin de la fonction, la valeur sur la pile est
comparée avec le \q{canari} correct dans \TT{gs:20}.
Si les valeurs ne sont pas égales, la fonction \TT{\_\_stack\_chk\_fail} est appelée
et nous voyons dans la console quelque chose comme ça (Ubuntu 13.04 x86):

\begin{lstlisting}
*** buffer overflow detected ***: ./2_1 terminated
======= Backtrace: =========
/lib/i386-linux-gnu/libc.so.6(__fortify_fail+0x63)[0xb7699bc3]
/lib/i386-linux-gnu/libc.so.6(+0x10593a)[0xb769893a]
/lib/i386-linux-gnu/libc.so.6(+0x105008)[0xb7698008]
/lib/i386-linux-gnu/libc.so.6(_IO_default_xsputn+0x8c)[0xb7606e5c]
/lib/i386-linux-gnu/libc.so.6(_IO_vfprintf+0x165)[0xb75d7a45]
/lib/i386-linux-gnu/libc.so.6(__vsprintf_chk+0xc9)[0xb76980d9]
/lib/i386-linux-gnu/libc.so.6(__sprintf_chk+0x2f)[0xb7697fef]
./2_1[0x8048404]
/lib/i386-linux-gnu/libc.so.6(__libc_start_main+0xf5)[0xb75ac935]
======= Memory map: ========
08048000-08049000 r-xp 00000000 08:01 2097586    /home/dennis/2_1
08049000-0804a000 r--p 00000000 08:01 2097586    /home/dennis/2_1
0804a000-0804b000 rw-p 00001000 08:01 2097586    /home/dennis/2_1
094d1000-094f2000 rw-p 00000000 00:00 0          [heap]
b7560000-b757b000 r-xp 00000000 08:01 1048602    /lib/i386-linux-gnu/libgcc_s.so.1
b757b000-b757c000 r--p 0001a000 08:01 1048602    /lib/i386-linux-gnu/libgcc_s.so.1
b757c000-b757d000 rw-p 0001b000 08:01 1048602    /lib/i386-linux-gnu/libgcc_s.so.1
b7592000-b7593000 rw-p 00000000 00:00 0
b7593000-b7740000 r-xp 00000000 08:01 1050781    /lib/i386-linux-gnu/libc-2.17.so
b7740000-b7742000 r--p 001ad000 08:01 1050781    /lib/i386-linux-gnu/libc-2.17.so
b7742000-b7743000 rw-p 001af000 08:01 1050781    /lib/i386-linux-gnu/libc-2.17.so
b7743000-b7746000 rw-p 00000000 00:00 0
b775a000-b775d000 rw-p 00000000 00:00 0
b775d000-b775e000 r-xp 00000000 00:00 0          [vdso]
b775e000-b777e000 r-xp 00000000 08:01 1050794    /lib/i386-linux-gnu/ld-2.17.so
b777e000-b777f000 r--p 0001f000 08:01 1050794    /lib/i386-linux-gnu/ld-2.17.so
b777f000-b7780000 rw-p 00020000 08:01 1050794    /lib/i386-linux-gnu/ld-2.17.so
bff35000-bff56000 rw-p 00000000 00:00 0          [stack]
Aborted (core dumped)
\end{lstlisting}

\myindex{MS-DOS}
gs est ainsi appelé registre de segment. Ces registres étaient beaucoup utilisés
du temps de MS-DOS et des extensions de DOS.
Aujourd'hui, sa fonction est différente.
\myindex{TLS}
\myindex{Windows!TIB}

Dit brièvement, le registre \TT{gs} dans Linux pointe toujours sur le
\ac{TLS}~(\myref{TLS})---des informations spécifiques au thread sont stockées là.
À propos, en win32 le registre \TT{fs} joue le même rôle, pointant sur \ac{TIB}
\footnote{\href{http://go.yurichev.com/17104}{wikipedia.org/wiki/Win32\_Thread\_Information\_Block}}.

Il y a plus d'information dans le code source du noyau Linux (au moins dans la version 3.11),
dans\\
\emph{arch/x86/include/asm/stackprotector.h} cette variable est décrite dans les commentaires.

\subsubsection{ARM: \OptimizingKeilVI (\ARMMode)}
\myindex{\CLanguageElements!switch}

\lstinputlisting[style=customasmARM]{patterns/08_switch/1_few/few_ARM_ARM_O3.asm}

A nouveau, en investiguant ce code, nous ne pouvons pas dire si il y avait un switch()
dans le code source d'origine ou juste un ensemble de déclarations if().

\myindex{ARM!\Instructions!ADRcc}

En tout cas, nous voyons ici des instructions conditionnelles (comme \ADREQ (\emph{Equal}))
qui ne sont exécutées que si $R0=0$, et qui chargent ensuite l'adresse de la chaîne
\emph{<<zero\textbackslash{}n>>} dans \Reg{0}.
\myindex{ARM!\Instructions!BEQ}
L'instruction suivante \ac{BEQ} redirige le flux d'exécution en \TT{loc\_170}, si $R0=0$.

Le lecteur attentif peut se demander si \ac{BEQ} s'exécute correctement puisque \ADREQ
a déjà mis une autre valeur dans le registre \Reg{0}.

Oui, elle s'exécutera correctement, car \ac{BEQ} vérifie les flags mis par l'instruction
\CMP et \ADREQ ne modifie aucun flag.

Les instructions restantes nous sont déjà familières.
Il y a seulement un appel à \printf, à la fin, et nous avons déjà examiné cette
astuce ici~(\myref{ARM_B_to_printf}).
A la fin, il y a trois chemins vers \printf{}.

\myindex{ARM!\Instructions!ADRcc}
\myindex{ARM!\Instructions!CMP}
La dernière instruction, \TT{CMP R0, \#2}, est nécessaire pour vérifier si $a=2$.

Si ce n'est pas vrai, alors \ADRNE charge un pointeur sur la chaîne \emph{<<something unknown \textbackslash{}n>>}
dans \Reg{0}, puisque $a$ a déjà été comparée pour savoir s'elle est égale
à 0 ou 1, et nous sommes sûrs que la variable $a$ n'est pas égale à l'un de
ces nombres, à ce point.
Et si $R0=2$, un pointeur sur la chaîne \emph{<<two\textbackslash{}n>>} sera chargé
par \ADREQ dans \Reg{0}.

\subsubsection{ARM: \OptimizingKeilVI (\ThumbMode)}

\lstinputlisting[style=customasmARM]{patterns/08_switch/1_few/few_ARM_thumb_O3.asm}

% FIXME а каким можно? к каким нельзя? \myref{} ->

Comme il y déjà été dit, il n'est pas possible d'ajouter un prédicat conditionnel
à la plupart des instructions en mode Thumb, donc ce dernier est quelque peu similaire
au code \ac{CISC}-style x86, facilement compréhensible.

\subsubsection{ARM64: GCC (Linaro) 4.9 \NonOptimizing}

\lstinputlisting[style=customasmARM]{patterns/08_switch/1_few/ARM64_GCC_O0_FR.lst}

Le type de la valeur d'entrée est \Tint, par conséquent le registre \RegW{0} est
utilisé pour garder la valeur au lieu du registre complet \RegX{0}.

Les pointeurs de chaîne sont passés à \puts en utilisant la paire d'instructions
\INS{ADRP}/\INS{ADD} comme expliqué dans l'exemple \q{\HelloWorldSectionName}:~\myref{pointers_ADRP_and_ADD}.

\subsubsection{ARM64: GCC (Linaro) 4.9 \Optimizing}

\lstinputlisting[style=customasmARM]{patterns/08_switch/1_few/ARM64_GCC_O3_FR.lst}

Ce morceau de code est mieux optimisé.
L'instruction \TT{CBZ} (\emph{Compare and Branch on Zero} comparer et sauter si zéro)
effectue un saut si \RegW{0} vaut zéro.
Il y a alors un saut direct à \puts au lieu de l'appeler, comme cela a été expliqué
avant:~\myref{JMP_instead_of_RET}.




\subsection{Résumé}

Seule la moitié basse des registres XMM est utilisée dans tous les exemples ici,
pour stocker un nombre au format IEEE 754.

Pratiquement, toutes les instructions préfixées par \TT{-SD} (\q{Scalar Double-Precision})---sont
des instructions travaillant avec des nombres à virgule flottante au format IEEE
754, stockés dans la moitié 64-bit basse d'un registre XMM.

Et c'est plus facile que dans le FPU, sans doute parce que les extensions SIMD ont
évolué dans un chemin moins chaotique que celles FPU dans le passé.
Le modèle de pile de registre n'est pas utilisé.

\myindex{x86!\Instructions!ADDSS}
\myindex{x86!\Instructions!MOVSS}
\myindex{x86!\Instructions!COMISS}
% TODO1: do this!
Si vous voulez, essayez de remplacer \Tdouble avec \Tfloat

% FIXME1 ... but their -SS versions
dans ces exemples, la même instruction sera utilisée, mais préfixée avec \TT{-SS}
(\q{Scalar Single-Precision} scalaire simple-précision), par exemple, \TT{MOVSS},
\TT{COMISS}, \TT{ADDSS}, etc.

\q{Scalaire} 
implique que le registre SIMD contienne seulement une valeur au lieu de plusieurs.

Les instructions travaillant avec plusieurs valeurs dans un registre simultanément
ont \q{Packed} dans leur nom.

Inutile de dire que les instructions SSE2 travaillent avec des nombres 64-bit au
format IEEE 754 (\Tdouble), alors que la représentation interne des nombres à virgule
flottante dans le FPU est sur 80-bit.

C'est pourquoi la FPU produit moins d'erreur d'arrondi et par conséquent, le FPU
peut donner des résultats de calcul plus précis.
