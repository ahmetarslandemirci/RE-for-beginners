% FIXME1 divide this file into separate ones...
\mysection{Arbeiten mit Fließkommazahlen und SIMD}

\label{floating_SIMD}
\myindex{IEEE 754}
\myindex{SIMD}
\myindex{SSE}
\myindex{SSE2}
Natürlich verblieb die \ac{FPU} in x86-kompatiblen Prozessoren als die \ac{SIMD} Erweiterungen hinzugefügt wurden.

Die \ac{SIMD} Erweiterungen (SSE2) bieten einen einfacheren Weg um mit Fließkommazahlen zu arbeiten.

Das Zahlenformat bleibt dabei das gleich (IEEE 754).

\myindex{x86-64}
Moderne Compiler (inklusive derer für x86-64) verwenden normalerweise \ac{SIMD} Befehle anstatt FPU-Befehle.

Wir werden hier die Beispiele aus dem Abschnitt über die FPU recyclen: \myref{sec:FPU}.

\subsection{Ein einfaches Beispiel}

\lstinputlisting[style=customc]{patterns/12_FPU/1_simple/simple.c}

\subsubsection{x64}

\lstinputlisting[caption=\Optimizing MSVC 2012 x64,style=customasmx86]{patterns/205_floating_SIMD/simple_MSVC_2012_x64_Ox.asm}
Die eingegebenen Fließkommawerte werden in die Register \XMM{0}-\XMM{3} übergeben und der Rest über den
Stackfootnote{\href{http://go.yurichev.com/17263}{MSDN: Parameterübergabe}}.

$a$ wird nach \XMM{0} übergeben und $b$ nach \XMM{1}.

Die XMM Register sind 128 Bit breit (wie wir bereits aus dem Abschnitt über \ac{SIMD} wissen: \myref{SIMD_x86}), aber
die \Tdouble Werte umfassen nur 64 Bit, sodass nur die niedere Hälfte der Register benötigt wird.

\myindex{x86!\Instructions!DIVSD}
\TT{DIVSD} ist ein SSE-Befehl, der für  
\q{Divide Scalar Double-Precision Floating-Point Values} steht;
er teilt einen \Tdouble Wert durch einen anderen, wobei beide in den niederen Hälften der Operanden gespeichert sind. 

Die Konstanten werden durch den Compiler im IEEE 754 Format kodiert.

\myindex{x86!\Instructions!MULSD}
\myindex{x86!\Instructions!ADDSD}
\TT{MULSD} und \TT{ADDSD} funktionieren genauso und führen Multiplikation und Addition durch.

Das Ergebnis der Funktionsausführung ist von Typ \Tdouble und wird im \XMM{0} Register abgelegt.\\\\
So arbeitet der nicht optimierende MSVC:

\lstinputlisting[caption=MSVC 2012 x64,style=customasmx86]{patterns/205_floating_SIMD/simple_MSVC_2012_x64.asm}

\myindex{Shadow space}
Ein wenig redundant. Die Eingabewerte bzw. deren niedere Registerhälften, d.h. die 64-Bit-Werte vom Typ \Tdouble, werden
im \q{shadow space}(\myref{shadow_space}) gespeichert.
GCC erzeugt identischen Code.

\subsubsection{x86}
Kompilieren wir das Beispiel für x86. Obwohl der Code für x86 erzeugt wird, verwendet MSVC 2012 SSE2 Befehle:

\lstinputlisting[caption=\NonOptimizing MSVC 2012 x86,style=customasmx86]{patterns/205_floating_SIMD/simple_MSVC_2012_x86.asm}

\lstinputlisting[caption=\Optimizing MSVC 2012 x86,style=customasmx86]{patterns/205_floating_SIMD/simple_MSVC_2012_x86_Ox.asm}
Es ist fast der gleiche Code, aber es gibt einige Unterschiede in Bezug auf die Aufrufkonventionen:
1) die Argumente werden nicht über die XMM Register, sondern über den Stack übergeben, wie in den FPU Beispielen
(\myref{sec:FPU});
2) das Ergebnis der Funktion wird über \ST{0} zurückgegeben---um dies zu erreichen wird es (durch die lokale Variable
\TT{tv}) aus einem der XMM Register nach \ST{0} kopiert.

\clearpage
Untersuchen wir das optimierte Beispiel mit \olly:

\begin{figure}[H]
\centering
\myincludegraphics{patterns/205_floating_SIMD/simple_olly1.png}
\caption{\olly: \TT{MOVSD} lädt den Wert von $a$ nach \XMM{1}}
\label{fig:FPU_SIMD_simple_olly1}
\end{figure}

\clearpage
\begin{figure}[H]
\centering
\myincludegraphics{patterns/205_floating_SIMD/simple_olly2.png}
\caption{\olly: \TT{DIVSD} hat den Quotienten berechnet und in \XMM{1} gespeichert} 
\label{fig:FPU_SIMD_simple_olly2}
\end{figure}

\clearpage
\begin{figure}[H]
\centering
\myincludegraphics{patterns/205_floating_SIMD/simple_olly3.png}
\caption{\olly: \TT{MULSD} hat das Produkt berechnet und in \XMM{0} gespeichert}
\label{fig:FPU_SIMD_simple_olly3}
\end{figure}

\clearpage
\begin{figure}[H]
\centering
\myincludegraphics{patterns/205_floating_SIMD/simple_olly4.png}
\caption{\olly: \TT{ADDSD} addiert den Wert in \XMM{0} zu \XMM{1}}
\label{fig:FPU_SIMD_simple_olly4}
\end{figure}

\clearpage
\begin{figure}[H]
\centering
\myincludegraphics{patterns/205_floating_SIMD/simple_olly5.png}
\caption{\olly: \FLD lässt das Funktionsergebnis in \ST{0}}
\label{fig:FPU_SIMD_simple_olly5}
\end{figure}
Wir sehen, dass \olly die XMM Register als Paare von \Tdouble Zahlen anzeigt, aber nur der niedere Teil davon verwendet
wird.

Offenbar zeigt \olly sie in diesem Format an, weil die SSE2 Befehle (die mit dem Suffix \TT{-SD}) jetzt ausgeführt
werden.

Natürlich ist es auch möglich das Registerformat zu ändern und sich die Inhalte als 4 \Tfloat-Zahlen oder nur als 16
Byte anzeigen zu lassen.

\clearpage
\subsection{Fließkommazahlen als Argumente übergeben}

\lstinputlisting[style=customc]{patterns/12_FPU/2_passing_floats/pow.c}
Sie werden über die niederen Hälften der Register \XMM{0}-\XMM{3} übergeben.

\lstinputlisting[caption=\Optimizing MSVC 2012 x64,style=customasmx86]{patterns/205_floating_SIMD/pow_MSVC_2012_x64_Ox.asm}

\myindex{x86!\Instructions!MOVSD}
\myindex{x86!\Instructions!MOVSDX}
Es gibt in Intel keinen \TT{MOVSDX} Befehl und AMD-Handbücher (\myref{x86_manuals}) bezeichnen ihn nur mit \TT{MOVSD}.
Es gibt insgesamt zwei Befehle, die sich in x86 denselben Namen teilen (für den anderen siehe: \myref{REP_MOVSx}).
Offenbar wollten die Microsoft Entwickler hier aufräumen und haben ihn deshalb in \TT{MOVSDX} umbenannt.
Er lädt lediglich einen Wert in die niedere Hälfte eines XMM Registers.

\TT{pow()} nimmt Argumente aus \XMM{0} und \XMM{1} und gibt das Ergebnis in \XMM{0} zurück.
Es wird dann für \printf nach \RDX verschoben.
Der Grund dafür ist möglicherweise, dass \printf eine Funktion mit einer variablen Anzahl an Argumenten ist.

\lstinputlisting[caption=\Optimizing GCC 4.4.6
x64,style=customasmx86]{patterns/205_floating_SIMD/pow_GCC446_x64_O3_DE.s}
GCC erzeugt klarer strukturierten Output.
Der Wert für \printf wird in \XMM{0} übergeben.
Hier ist übrigens ein Fall, in dem 1 nach \EAX für \printf geschrieben wird um anzuzeigen, dass ein Argument in den
Vektorregistern übergeben wird, genau wie es der Standard \SysVABI verlangt.

\subsection{Beispiel mit Vergleich}

\lstinputlisting[style=customc]{patterns/12_FPU/3_comparison/d_max.c}

\subsubsection{x64}

\lstinputlisting[caption=\Optimizing MSVC 2012 x64,style=customasmx86]{patterns/205_floating_SIMD/d_max_MSVC_2012_x64_Ox.asm}

\Optimizing MSVC erzeugt leicht verständlichen Code.

\myindex{x86!\Instructions!COMISD}
\TT{COMISD} bedeutet \q{Compare Scalar Ordered Double-Precision Floating-Point 
Values and Set EFLAGS}. Das beschreibt ziemlich genau, was der Befehl tatsächlich tut.\\
\\
\NonOptimizing MSVC erzeugt Code mit mehr Redundanzen, der aber immer noch gut verständlich ist:

\lstinputlisting[caption=MSVC 2012 x64,style=customasmx86]{patterns/205_floating_SIMD/d_max_MSVC_2012_x64.asm}

\myindex{x86!\Instructions!MAXSD}
GCC 4.4.6 hat mehr Optimierungen durchgeführt und den Befehl \TT{MAXSD} (\q{Return Maximum Scalar 
Double-Precision Floating-Point Value}) verwendet, der einfach den größten Wert auswählt!

\lstinputlisting[caption=\Optimizing GCC 4.4.6 x64,style=customasmx86]{patterns/205_floating_SIMD/d_max_GCC446_x64_O3.s}

\clearpage
\subsubsection{x86}
Kompilieren wir dieses Beispiel in MSVC 2012 mit aktivierter Optimierung:

\lstinputlisting[caption=\Optimizing MSVC 2012 x86,style=customasmx86]{patterns/205_floating_SIMD/d_max_MSVC_2012_x86_Ox.asm}
Fast identisch, nur dass die Werte $a$ und $b$ vom Stack geholt werden und das Funktionsergebnis in \ST{0} gelassen
wird.

Wenn wir dieses Beispiel in \olly laden, erkennen wir, wie der Befehl \TT{COMISD} Werte vergleicht und die \CF und \PF
Flags setzt bzw. löscht:

\begin{figure}[H]
\centering
\myincludegraphics{patterns/205_floating_SIMD/d_max_olly.png}
\caption{\olly: \TT{COMISD} hat die \CF und \PF Flags verändert}
\label{fig:FPU_SIMD_d_max_olly}
\end{figure}

\subsection{Berechnen der Maschinengenauigkeit: x64 und SIMD}
\label{machine_epsilon_x64_and_SIMD}
Betrachten wir erneut das Beispiel zur Berechnung der Maschinengenauigkeit für \Tdouble
\lstref{machine_epsilon_double_c}.

Wir kompilieren es jetzt für x64:

\lstinputlisting[caption=\Optimizing MSVC 2012 x64,style=customasmx86]{patterns/205_floating_SIMD/epsilon_double_MSVC_2012_x64_Ox.asm}
Es gibt keinen Weg 1 zum einem Wert in einem 128-Bit XMM Register zu addieren, also muss der Wert im Speicher abgelegt
werden.

Es gibt zwar den Befehl \INS{ADDSD} (\emph{Add Scalar Double-Precision Floating-Point Values}), der einen Wert zu niederen
64-Bit-Hälfte eines XMM Registers addieren kann und die höheren Bits ignoriert, aber MSVC 2012 scheint an dieser Stelle
nicht gut genug zu sein, um diese Möglichkeit zu erkennen\footnote{Als Übung können Sie versuchen, den Code so
umzugestalten, dass der lokale Stack nicht mehr verwendet wird}.

Nichtsdestotrotz wird der Wert dann wieder in ein XMM Register geladen und eine Subtraktion wird durchgeführt.
\INS{SUBSD} steht für \q{Subtract Scalar Double-Precision Floating-Point Values}, d.h. er arbeitet nur auf dem niederen
64-Bit-Teil des 128-Bit XMM Registers.
Das Ergebnis wird in das XMM0-Register zurückgegeben.

\subsubsection{Struct als Menge von Werten}
Um zu veranschaulichen, dass ein struct nur eine Menge von nebeneinanderliegenden Variablen ist, überarbeiten wir unser
Beispiel, indem wir auf die Definition des \emph{tm} structs schauen:\lstref{struct_tm}.

\lstinputlisting[style=customc]{patterns/15_structs/3_tm_linux/as_array/GCC_tm2.c}

\myindex{\CStandardLibrary!localtime\_r()}
Der Pointer auf das Feld \TT{tm\_sec} wird nach \TT{localtime\_r} übergeben, d.h. an das erste Element des structs.

Der Compiler warnt uns:

\begin{lstlisting}[caption=GCC 4.7.3]
GCC_tm2.c: In function 'main':
GCC_tm2.c:11:5: warning: passing argument 2 of 'localtime_r' from incompatible pointer type [enabled by default]
In file included from GCC_tm2.c:2:0:
/usr/include/time.h:59:12: note: expected 'struct tm *' but argument is of type 'int *'
\end{lstlisting}

Trotzdem erzeugt er folgenden Code:

\lstinputlisting[caption=GCC 4.7.3,style=customasmx86]{patterns/15_structs/3_tm_linux/as_array/GCC_tm2.asm}
Dieser Code ist zum vorherigen identisch und es ist unmöglich zu sagen, ob es sich im originalen Quellcode um ein struct
oder nur um eine Menge von Variablen handelt.

Es funktioniert also, ist aber in der Praxis nicht empfehlenswert. 

Nicht optimierende Compiler legen normalerweise Variablen auf dem lokalen Stack in der Reihenfolge an, in der sie in der
Funktion deklariert wurden.

Ein Garantie dafür gibt es freilich nicht.

Andere Compiler könnten an dieser Stelle übrigens davor warnen, dass die Variablen \TT{tm\_year}, \TT{tm\_mon}, \TT{tm\_mday},
\TT{tm\_hour}, \TT{tm\_min} - nicht aber \TT{tm\_sec} - ohne Initialisierung verwendet werden.

Der Compiler weiß nicht, dass diese durch die Funktion \TT{localtime\_r()} befüllt werden.

Wir haben dieses Beispiel ausgewählt, da alle Felder im struct vom Typ \Tint sind.

Es würde nicht funktionieren, wenn die Felder 16 Bit (\TT{WORD}) groß wären, wie im Beispiel des \TT{SYSTEMTIME}
structs---\TT{GetSystemTime()} würde sie falsch befüllen (da die lokalen Variablen auf 32-Bit-Grenzen angeordnet sind).
Mehr dazu im folgenden Abschnitt: \q{\StructurePackingSectionName} (\myref{structure_packing}).

Ein struct ist also nichts als eine Menge von an einer Stelle gespeicherten Variablen.
Man kan sagen, dass das struct ein Befehl an den Compiler ist, diese Variablen an einer Stelle zu halten.
In ganz frühen Versionen von C (vor 1972) gab es übrigens gar keine structs \RitchieDevC.

Dieses Beispiel wird nicht im Debugger gezeigt, da es dem gerade gezeigten entspricht.

\subsubsection{Struct als Array aus 32-Bit-Worten}

\lstinputlisting[style=customc]{patterns/15_structs/3_tm_linux/as_array/GCC_tm3.c}
Wir können einen Pointer auf ein struct in ein Array aus \Tint{}s casten und es funktioniert.
Wir lassen dieses Beispiel zur Systemzeit 23:51:45 26-July-2014 laufen.

\begin{lstlisting}[label=GCC_tm3_output]
0x0000002D (45)
0x00000033 (51)
0x00000017 (23)
0x0000001A (26)
0x00000006 (6)
0x00000072 (114)
0x00000006 (6)
0x000000CE (206)
0x00000001 (1)
\end{lstlisting}
Die Variablen sind hier in der gleichen Reihenfolge, in der die in der Definition des structs aufgezählt
werden:\myref{struct_tm}.

Hier ist der erzeugte Code:

\lstinputlisting[caption=\Optimizing GCC
4.8.1,style=customasmx86]{patterns/15_structs/3_tm_linux/as_array/GCC_tm3_DE.lst}
Tatsächlich: der Platz auf dem lokalen Stack wird zuerst wie in struct und dann wie ein Array behandelt.

Es ist sogar möglich, die Felder des structs über diesen Pointer zu verändern.

Und wiederum ist es zweifellos ein seltsamer Weg die Dinge umzusetzen; er ist für produktiven Code definitiv nicht
empfehlenswert.

\mysubparagraph{\Exercise}
Versuchen Sie als Übung die Monatsnummer zu verändern (um 1 zu erhöhen), indem Sie das struct wie ein Array behandeln.

\subsubsection{Struct als Bytearray}
Wir können sogar noch weiter gehen. Casten wir den Pointer zu einem Bytearray und ziehen einen Dump:

\lstinputlisting[style=customc]{patterns/15_structs/3_tm_linux/as_array/GCC_tm4.c}

\begin{lstlisting}
0x2D 0x00 0x00 0x00 
0x33 0x00 0x00 0x00 
0x17 0x00 0x00 0x00 
0x1A 0x00 0x00 0x00 
0x06 0x00 0x00 0x00 
0x72 0x00 0x00 0x00 
0x06 0x00 0x00 0x00 
0xCE 0x00 0x00 0x00 
0x01 0x00 0x00 0x00 
\end{lstlisting}
Wir haben dieses Beispiel auch zur Systemzeit 23:51:45 26-July-2014 ausgeführt
\footnote{Datum und Uhrzeit sind zu Demonstrationszwecken identisch. Die Bytewerte sind modifiziert.}.
Die Werte sind genau dieselben wie im vorherigen Dump(\myref{GCC_tm3_output}) und natürlich steht das LSB vorne, da es
sich um eine Little-Endian-Architektur handelt(\myref{sec:endianness}). 

\lstinputlisting[caption=\Optimizing GCC
4.8.1,style=customasmx86]{patterns/15_structs/3_tm_linux/as_array/GCC_tm4_DE.lst}


\subsection{Zusammenfassung}
In den Beispielen hier wird nur die niedere Hälfte der XMM Register verwendet, um eine Zahl im IEEE 754 Format zu
speichern.

Im Prinzip arbeiten alle Befehle, die um den Präfix \TT{-SD} (\q{Scalar Double-Precision}) ergänzt wurden, mit
Fließkommazahlen im IEEE 754 Format, die in der niederen 64-Bit-Hälfte eines XMM Registers gespeichert werden.

Dies ist einfacher als in der FPU, da die SIMD sich in weniger chaotischer Weise als die FPU entwickelt hat.
Das Stackregister Modell wird nicht verwendet.

\myindex{x86!\Instructions!ADDSS}
\myindex{x86!\Instructions!MOVSS}
\myindex{x86!\Instructions!COMISS}
% TODO1: do this!
Wenn man in diesen Beispielen \Tdouble durch \Tfloat ersetzen würde, würden die gleichen Befehle verwendet werden, aber
jeweils mit \TT{-SS}(\q{Scalar Single-Precision}) Präfix, z.B. \TT{MOVSS}, \TT{COMISS}, \TT{ADDSS}, etc.

\q{Scalar} bedeutet, dass die SIMD Register nur einen anstatt mehreren Werten enthalten.

Befehle, die mit mehreren Werte in einem Register gleichzeitig arbeiten, haben ein \q{Packed} in ihrem Namen.

Unnötig extra zu erwähnen, dass die SSE2 Befehle mit 64-Bit IEEE 743 Werten (\Tdouble) arbeiten, wohingegen die interne
Repräsentation von Fließkommazahlen in der FPU 80-Bit-Zahlen verwendet.

Die FPU wird deshalb manchmal weniger Rundungsfehler machen und als Konsequenz daraus möglicherweise präzisere
Berechnungsergebnisse liefern.
