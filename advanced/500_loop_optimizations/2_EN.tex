\subsection{Another loop optimization}

If you process all elements of some array which happens to be located in global memory, compiler can optimize it.
For example, let's calculate a sum of all elements of array of 128 \emph{int}'s:

\begin{lstlisting}[style=customc]
#include <stdio.h>

int a[128];

int sum_of_a()
{
	int rt=0;
	
	for (int i=0; i<128; i++)
		rt=rt+a[i];

	return rt;
};

int main()
{
	// initialize
	for (int i=0; i<128; i++)
		a[i]=i;
	
	// calculate the sum
	printf ("%d\n", sum_of_a());
};
\end{lstlisting}

Optimizing GCC 5.3.1 (x86) can produce this (\IDA):

\lstinputlisting[style=customasmx86]{advanced/500_loop_optimizations/tmp1.lst}

What the heck is \TT{\_\_libc\_start\_main@@GLIBC\_2\_0} at \TT{0x080484C5}?
This is a label just after end of \TT{a[]} array.
The function can be rewritten like this:

\begin{lstlisting}[style=customc]
int sum_of_a_v2()
{
	int *tmp=a;
	int rt=0;
	
	do
	{
		rt=rt+(*tmp);
		tmp++;
	}
	while (tmp<(a+128));

	return rt;
};
\end{lstlisting}

First version has \emph{i} counter, and the address of each element of array is to be calculated at each iteration.
The second version is more optimized: the pointer to each element of array is always ready and is sliding 4 bytes forward at each iteration.
How to check if the loop is ended?
Just compare the pointer with the address just behind array's end, which is, in our case, is happens to be address of imported \TT{\_\_libc\_start\_main()} function from Glibc 2.0.
Sometimes code like this is confusing, and this is very popular optimizing trick, so that's why I made this example.

My second version is very close to what GCC did, and when I compile it, the code is almost the same as in first version, but two first instructions are swapped:

\lstinputlisting[style=customasmx86]{advanced/500_loop_optimizations/tmp2.lst}

Needless to say, this optimization is possible if the compiler can calculate address of the end of array during compilation time.
This happens if the array is global and it's size is fixed.

However, if the address of array is unknown during compilation, but size is fixed, address of the label just behind array's end can be calculated at the beginning of the loop.
% FIXME <!-- \ref{} to example -->

