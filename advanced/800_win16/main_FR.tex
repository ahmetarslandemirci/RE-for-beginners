\mysection{Windows 16-bit}
\myindex{Windows!Windows 3.x}

Les programmes Windows 16-bit sont rares de nos jours, mais ils peuvent être utilisés
dans le cadre de rétrocomputing ou d'hacking de dongle (\myref{dongles}).

Il y a eu des versions 16-bit de Windows jusqu'à la 3.11.
95/98/ME
supportaient le code 16-bit, ainsi que les versions 32-bit de la série \gls{Windows NT}.
Les versions 64-bit de \gls{Windows NT} ne supportaient pas du tout le code exécutable
16-bit.

Le code ressemble a du code MS-DOS.

Les fichiers exécutables sont du type NE (appelé \q{new executable}).

Tous les exemples considérés ici ont été compilés avec le compilateur OpenWatcom 1.9,
en utilisant ces paramètres:\\

\begin{lstlisting}
wcl.exe -i=C:/WATCOM/h/win/ -s -os -bt=windows -bcl=windows example.c
\end{lstlisting}

\subsection{\Example \#1}

\begin{lstlisting}[style=customc]
#include <windows.h>

int PASCAL WinMain( HINSTANCE hInstance,
                    HINSTANCE hPrevInstance,
                    LPSTR lpCmdLine,
                    int nCmdShow )
{
	MessageBeep(MB_ICONEXCLAMATION);
	return 0;
};
\end{lstlisting}

\begin{lstlisting}[style=customasmx86]
WinMain         proc near
                push    bp
                mov     bp, sp
                mov     ax, 30h ; '0'   ; MB\_ICONEXCLAMATION constant
                push    ax
                call    MESSAGEBEEP
                xor     ax, ax          ; return 0
                pop     bp
                retn    0Ah
WinMain         endp
\end{lstlisting}

\RU{Пока всё просто}\EN{Seems to be easy, so far}\FR{Ça semble facile, jusqu'ici}.

\subsection{\Example{} \#2}
\label{win16_messagebox}

\begin{lstlisting}[style=customc]
#include <windows.h>

int PASCAL WinMain( HINSTANCE hInstance,
                    HINSTANCE hPrevInstance,
                    LPSTR lpCmdLine,
                    int nCmdShow )
{
	MessageBox (NULL, "hello, world", "caption", MB_YESNOCANCEL);
	return 0;
};
\end{lstlisting}

\begin{lstlisting}[style=customasmx86]
WinMain         proc near
                push    bp
                mov     bp, sp
                xor     ax, ax          ; NULL
                push    ax
                push    ds
                mov     ax, offset aHelloWorld ; 0x18. "hello, world"
                push    ax
                push    ds
                mov     ax, offset aCaption ; 0x10. "caption"
                push    ax
                mov     ax, 3           ; MB\_YESNOCANCEL
                push    ax
                call    MESSAGEBOX
                xor     ax, ax          ; return 0
                pop     bp
                retn    0Ah
WinMain         endp

dseg02:0010 aCaption        db 'caption',0
dseg02:0018 aHelloWorld     db 'hello, world',0
\end{lstlisting}

\myindex{x86!\Instructions!RET}
Quelques points importants ici: la convention d'appel \TT{PASCAL} impose de passer
le premier argument en premier (\TT{MB\_YESNOCANCEL}), et le dernier argument --- en dernier (NULL).
Cette convention demande aussi à l'\glslink{callee}{appelant} de restaurer le \glslink{stack pointer}{pointeur de pile}:
D'où l'instruction \TT{RETN} qui a \TT{0Ah} comme argument, ce qui implique que le
pointeur sera incrémenté de 10 octets lorsque l'on sortira de la fonction.
C'est comme stdcall (\myref{sec:stdcall}), mais les arguments sont passés dans l'ordre
\q{naturel}.

Les pointeurs sont passés par paire: d'abord le segment de données, puis le pointeur
dans le segment.
Il y a seulemnt un segment dans cet exemple, donc \TT{DS} pointe toujours sur le segment
de données de l'exécutable.

\subsection{\Example{} \#3}

\lstinputlisting[style=customc]{\CURPATH/ex3.c}

\lstinputlisting[style=customasmx86]{\CURPATH/ex3.lst}

\RU{Немного расширенная версия примера из предыдущей секции}
\EN{Somewhat extended example from the previous section}
\FR{Exemple un peu plus long de la section précédente}.

\subsection{\Example{} \#4}

\label{win16_32bit_values}

\lstinputlisting[style=customc]{\CURPATH/ex4.c}

\lstinputlisting[style=customasmx86]{\CURPATH/ex4.lst}

\myindex{MS-DOS}
Les valeurs 32-bit (le type de donnée \TT{long} implique 32 bits, tandis que \Tint
est 16-bit en code 16-bit (à la fois pour MS-DOS et Win16) sont passées par paires.
C'est tout comme lorsqu'une valeur 64-bit est utilisée dans un environnement 32-bit (\myref{sec:64bit_in_32_env}).

\TT{sub\_B2} 
voici une fonction de bibliothèques écrite par les développeurs du compilateurs qui
fait la \q{multiplication des long} (i.e., multiplie deux valeurs 32-bits).
D'autres fonctions de compilateur qui font la même chose sont listées ici: \myref{sec:MSVC_library_func}, \myref{sec:GCC_library_func}.

\myindex{x86!\Instructions!ADD}
\myindex{x86!\Instructions!ADC}
La paire d'instructions \TT{ADD}/\TT{ADC} est utilisée pour l'addition de valeurs
composées: \TT{ADD} peut mettre le flag \TT{CF} à 0/1, et \TT{ADC} l'utilise après.

La paire d'instructions \TT{SUB}/\TT{SBB} est utilisée pour la soustraction: \TT{SUB}
peut mettre la flag \TT{CF} à 0/1, et \TT{SBB} l'utilise après.

Les valeurs 32-bit sont renvoyées de la fonction dans la paire de registres \TT{DX:AX}.

Les constantes sont aussi passées par paires dans \TT{WinMain()} ici.

\myindex{x86!\Instructions!CWD}
La constante 123 typée \Tint{} est d'abord converti suivant le signe de la valeur
32-bit en utilisant l'instruction \TT{CWD}.

\subsection{\Example{} \#5}
\label{win16_near_far_pointers}

\lstinputlisting[style=customc]{\CURPATH/ex5.c}

\lstinputlisting[style=customasmx86]{\CURPATH/ex5.lst}

\myindex{Intel!8086!Modèle de mémoire}%%\EN{Memory model}}
Nous voyons ici une différence entre les pointeurs appelés \q{near} et \q{far}: un
autre effet bizarre de la mémoire segmentée en 16-bit 8086.

Vous pouvez en lire plus à ce sujet ici: \myref{8086_memory_mode}.

Les pointeurs \q{near} sont ceux qui pointent dans le segment de données courant.
C'est pourquoi la fonction \TT{string\_compare()} prend seulement deux pointeurs 16-bit,
et accède des données dans le segment sur lequel \TT{DS} pointe (L'instruction \TT{mov al, [bx]}
fonctionne en fait comme \TT{mov al, ds:[bx]}\EMDASH{}\TT{DS} est implicite ici).

Les pointeurs \q{far} sont ceux qui pointent sur des données dans un autre segment
de mémoire.
C'est pourquoi \TT{string\_compare\_far()} prend la paire de 16-bit comme un pointeur,
charge la partie haute dans le registre de segment \TT{ES} et accède aux données
à travers lui (\TT{mov al, es:[bx]}).
Les pointeurs \q{far} sont aussi utilisés dans mon exemple win16
\TT{MessageBox()}: \myref{win16_messagebox}.
En effet, le noyau de Windows n'est pas au courant du segment de données qui doit être
utilisé pour accéder aux chaînes de texte, donc il a besoin de l'information complète.
La raison de cette distinction est qu'un programme compact peut n'utiliser qu'un
segment de données de 64kb, donc il n'a pas besoin de passer la partie haute de l'adresse,
qui est toujours la même.
Un programme plus gros peut utiliser plusieurs segments de données de 64kb, donc il
doit spécifier le segment de données à chaque fois.

C'est la même histoire avec les segments de code.
Un programme compact peut avoir tout son code exécutable dans un seul segment de 64kb,
donc toutes les fonctions y seront appelées en utilisant l'instruction \TT{CALL NEAR},
et le contrôle du flux sera renvoyé en utilisant \TT{RETN}.
Mais si il y a plusieurs segments de code, alors l'adresse d'une fonction devra être
spécifiée par une paire, et sera appelée en utilisant l'instruction \TT{CALL FAR},
et le contrôle du flux renvoyé en utilisant \TT{RETF}.

Ceci est ce qui est mis dans le compilateur en spécifiant le \q{modèle de mémoire}.

Les compilateurs qui ciblent MS-DOS et Win16 ont des bibliothèques spécifiques pour
chaque modèle de mémoire: elles diffèrent par le type de pointeurs pour le code et
les données.


\subsection{\Example{} \#6}

\lstinputlisting[style=customc]{\CURPATH/ex6.c}

\lstinputlisting[style=customasmx86]{\CURPATH/ex6.lst}

\myindex{\CStandardLibrary!time()}
\myindex{\CStandardLibrary!localtime()}

Le temps UNIX est une valeur 32-bit, donc il est renvoyé dans la paire de registres
\TT{DX:AX} et est stocké dans deux variables locales 16-bit.
Puis, un pointeur sur la paire est passé à la fonction \TT{localtime()}.
La fonction \TT{localtime()} a une structure \TT{struct tm} allouée quelque part
dans les entrailles de la bibliothèque C, donc seul un pointeur est renvoyé.

À propos, ceci implique aussi que la fonction ne peut pas être appelée tant que le
résultat n'a pas été utilisé.

Pour les fonctions \TT{time()} et \TT{localtime()}, une convention d'appel Watcom
est utilisée ici:
les quatre premiers arguments sont passés dans les registres \TT{AX}, \TT{DX}, \TT{BX}
et \TT{CX}, et le reste des arguments par la pile.

Les fonctions utilisant cette convention sont aussi marquées par un souligné à la
fin de leur nom.

\TT{sprintf()} n'utilise pas la convention d'appel \TT{PASCAL}, ni la Watcom,\\ % varioref bug
donc les arguments sont passés de la manière \IT{cdecl} normale (\myref{cdecl}).

\subsubsection{Variables globales}

Ceci est le même exemple, mais cette fois les variables sont globales:

\lstinputlisting[style=customc]{\CURPATH/ex6_global.c}

\lstinputlisting[style=customasmx86]{\CURPATH/ex6_global.lst}

\TT{t} ne va pas être utilisée, mais le compilateur a généré le code qui stocke
la valeur.

Car il n'est pas sûr, peut-être que la valeur sera utilisée dans un autre module.


