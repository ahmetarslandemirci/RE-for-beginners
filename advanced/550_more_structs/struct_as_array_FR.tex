\subsection{Parfois une structure C peut être utilisée au lieu d'un tableau}

\subsubsection{Moyenne arithmétique}

\begin{lstlisting}[style=customc]
#include <stdio.h>

int mean(int *a, int len)
{
	int sum=0;
	for (int i=0; i<len; i++)
		sum=sum+a[i];
	return sum/len;
};

struct five_ints
{
	int a0;
	int a1;
	int a2;
	int a3;
	int a4;
};

int main()
{
	struct five_ints a;
	a.a0=123;
	a.a1=456;
	a.a2=789;
	a.a3=10;
	a.a4=100;
	printf ("%d\n", mean(&a, 5));
	// test: https://www.wolframalpha.com/input/?i=mean(123,456,789,10,100)
};
\end{lstlisting}

Ceci fonctionne: la fonction \IT{mean()} ne va jamais accéder après la fin de la
structure \IT{five\_ints}, car 5 est passé, signifiant que seuls 5 entiers vont être
accédés.

\subsubsection{Mettre une chaîne dans une structure}

\begin{lstlisting}[style=customc]
#include <stdio.h>

struct five_chars
{
	char a0;
	char a1;
	char a2;
	char a3;
	char a4;
} __attribute__ ((aligned (1),packed));

int main()
{
	struct five_chars a;
	a.a0='h';
	a.a1='i';
	a.a2='!';
	a.a3='\n';
	a.a4=0;
	printf (&a); // prints "hi!"
};
\end{lstlisting}

L'attribut \IT{((aligned (1),packed))} doit être utilisé, car sinon, chaque champ
de la structure sera aligné sur une limite de 4 ou 8 octets.

\subsubsection{Résumé}

Ceci est simplement un autre exemple de la façon dont les structures et les tableaux
sont stockés en mémoire.
Peut-être qu'aucun programmeur sain ne ferait quelque chose comme dans cet exemple,
excepté dans le cas d'astuces spécifiques.
Ou peut-être dans le cas d'obfuscation de code source?

