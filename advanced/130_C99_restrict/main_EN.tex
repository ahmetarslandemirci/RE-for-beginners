\mysection{C99 restrict}
\myindex{\CLanguageElements!C99!restrict}
\myindex{Fortran}

Here is a reason why Fortran programs, in some cases, work faster than \CCpp ones.

\begin{lstlisting}[style=customc]
void f1 (int* x, int* y, int* sum, int* product, int* sum_product, int* update_me, size_t s)
{
	for (int i=0; i<s; i++)
	{
		sum[i]=x[i]+y[i];
		product[i]=x[i]*y[i];
		update_me[i]=i*123; // some dummy value
		sum_product[i]=sum[i]+product[i];	
	};
};
\end{lstlisting}

That's very simple example with one specific
thing in it: 
the pointer to the \TT{update\_me} array could be
a pointer to the
\TT{sum} array, \TT{product} array, or even the 
\TT{sum\_product} array---nothing forbids that, right?

The compiler is fully aware of this, so it generates code with four stages in the loop body:
\begin{itemize}
\item calculate next \TT{sum[i]}
\item calculate next \TT{product[i]}
\item calculate next \TT{update\_me[i]}
\item calculate next \TT{sum\_product[i]}---on this stage, we need to load from memory
the already calculated \TT{sum[i]} and \TT{product[i]}
\end{itemize}

Is it possible to optimize the last stage?
Since we have already calculated \TT{sum[i]} and \TT{product[i]} 
it is not necessary to load them again from memory.

Yes, but compiler is not sure that nothing has been overwritten at the 3rd stage!
This is called \q{pointer aliasing}, 
a situation when the compiler cannot be sure that a memory to which a pointer is pointing hasn't been changed.

\emph{restrict} in the C99 standard \InSqBrackets{\CNineNineStd{} 6.7.3/1}
is a promise, given by programmer to the compiler that the function arguments 
marked by this keyword always points to different memory locations and never intersects.

To be more precise and describe this formally, \emph{restrict} shows that only this pointer is to be used
to access an object, and no other pointer will be used for it.

It can be even said the object will be accessed
only via one single pointer, if it is marked as \emph{restrict}.

Let's add this keyword to each pointer argument:

\begin{lstlisting}[style=customc]
void f2 (int* restrict x, int* restrict y, int* restrict sum, int* restrict product, int* restrict sum_product, 
	int* restrict update_me, size_t s)
{
	for (int i=0; i<s; i++)
	{
		sum[i]=x[i]+y[i];
		product[i]=x[i]*y[i];
		update_me[i]=i*123; // some dummy value
		sum_product[i]=sum[i]+product[i];	
	};
};
\end{lstlisting}

Let's see results:

\lstinputlisting[caption=GCC x64: f1(),style=customasmx86]{\CURPATH/f1_EN.asm}

\lstinputlisting[caption=GCC x64: f2(),style=customasmx86]{\CURPATH/f2_EN.asm}

The difference between the compiled \TT{f1()} and \TT{f2()} functions is as follows:
in \TT{f1()}, \TT{sum[i]} and \TT{product[i]} 
are reloaded in the middle of the loop,
and in \TT{f2()} 
there is no such thing,
the already calculated values are used, 
since we \q{promised} the compiler 
that no one and nothing will change the values in \TT{sum[i]} 
and \TT{product[i]} during the execution of the loop's body, 
so it is \q{sure} that there is no need to load the value from memory again.

Obviously, the second example works faster.

But what if the pointers in the function's arguments intersect somehow?

This is on the programmer's conscience, and the results will be incorrect.

Let's go back to Fortran. 

Compilers of this programming language treats all pointers as such, 
so when it was not possible to set \emph{restrict} in C, Fortran could generate faster code in these cases.

How practical is it? 

In the cases when the function works with several big blocks in memory.

There are a lot of such in linear algebra, for instance.

Supercomputers/\ac{HPC} are very busy with linear algebra, so probably that is why, traditionally, Fortran is still
used there [Eugene Loh, \emph{The Ideal HPC Programming Language}, (2010)].

But when the number of iterations is not very big,
certainly, the speed boost may not to be significant.
