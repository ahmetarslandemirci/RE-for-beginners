\subsection{Pin case}

\myindex{Pin}
It's interesting to note how some functions from Pin \ac{DBI} framework takes number of arguments:

\begin{lstlisting}[style=customc]
            INS_InsertPredicatedCall(
                ins, IPOINT_BEFORE, (AFUNPTR)RecordMemRead,
                IARG_INST_PTR,
                IARG_MEMORYOP_EA, memOp,
                IARG_END);
\end{lstlisting}
( \TT{pinatrace.cpp} )

And this is how \TT{INS\_InsertPredicatedCall()} function is declared:

\begin{lstlisting}[style=customc]
extern VOID INS_InsertPredicatedCall(INS ins, IPOINT ipoint, AFUNPTR funptr, ...);
\end{lstlisting}
( \TT{pin\_client.PH} )

Hence, constants with names starting with \TT{IARG\_} are some kinds of arguments to the function,
which are handled inside of \TT{INS\_InsertPredicatedCall()}.
You can pass as many arguments, as you need.
Some commands has additional argument(s), some are not.
Full list of arguments:
\url{https://software.intel.com/sites/landingpage/pintool/docs/58423/Pin/html/group__INST__ARGS.html}.
And it has to be a way to detect an end of arguments list, so the list must be terminated with \TT{IARG\_END} constant,
without which, the function will (try to) handle random noise in the local stack, treating it as additional arguments.

\index{Python}
Also, in [\RobPikePractice] we can find a nice example of C/C++ routines very similar to
\emph{pack/unpack}\footnote{\url{https://docs.python.org/3/library/struct.html}} in Python.

