\subsubsection{Windows x64}

\label{SEH_win64}

Vous imaginez bien qu'il n'est pas très performant de construire le contexte SEH dans le prologue 
de chaque fonction. S'y ajoute les nombreux changements de la valeur de \emph{previous try level} 
durant l'exécution de la fonction.

C'est pourquoi avec x64, la manière de faire a complèetement changé. Tous les pointeurs vers les 
blocs \TT{try}, les filtres et les gestionnaires sont désormais stockés dans une nouveau segment 
PE: \TT{.pdata} à partir duquel les gestionnaires d'exception de l'\ac{OS} récupèreront les 
informations.

Voici deux exemples tirés de la section précédente et compilés pour x64:

\lstinputlisting[caption=MSVC 2012,style=customasmx86]{OS/SEH/3/2_x64.asm}

\lstinputlisting[caption=MSVC 2012,style=customasmx86]{OS/SEH/3/3_x64.asm}

Pour plus d'informations sur le sujet, lisez \IgorSkochinsky.

Hormis les informations d'exception, \TT{.pdata} est aussi une section qui contient les adresses 
de début et de fin de toutes les fonctions. Elle revêt donc un intérêt particulier dans le cadre 
d'une analyse automatique d'un programme.

