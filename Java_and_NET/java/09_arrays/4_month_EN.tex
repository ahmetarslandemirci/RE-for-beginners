% TODO proof-reading
\subsubsection{Pre-initialized array of strings}
\label{Java_2D_array_month}

\begin{lstlisting}[style=customjava]
class Month
{
	public static String[] months = 
	{
		"January", 
		"February", 
		"March", 
		"April",
		"May",
		"June",
		"July",
		"August",
		"September",
		"October",
		"November",
		"December"
	};

	public String get_month (int i)
	{
		return months[i];
	};
} 
\end{lstlisting}

The \TT{get\_month()} function is simple:
Функция \TT{get\_month()} проста:

\begin{lstlisting}
  public java.lang.String get_month(int);
    flags: ACC_PUBLIC
    Code:
      stack=2, locals=2, args_size=2
         0: getstatic     #2         // Field months:[Ljava/lang/String;
         3: iload_1       
         4: aaload        
         5: areturn       
\end{lstlisting}

\TT{aaload} operates on an array of \emph{references}.

Java String are objects, so the \emph{a}-instructions are used to operate on them.

\TT{areturn} returns a \emph{reference} to a \TT{String} object.


How is the \TT{months[]} array initialized?


\begin{lstlisting}
  static {};
    flags: ACC_STATIC
    Code:
      stack=4, locals=0, args_size=0
         0: bipush        12
         2: anewarray     #3         // class java/lang/String
         5: dup           
         6: iconst_0      
         7: ldc           #4         // String January
         9: aastore       
        10: dup           
        11: iconst_1      
        12: ldc           #5         // String February
        14: aastore       
        15: dup           
        16: iconst_2      
        17: ldc           #6         // String March
        19: aastore       
        20: dup           
        21: iconst_3      
        22: ldc           #7         // String April
        24: aastore       
        25: dup           
        26: iconst_4      
        27: ldc           #8         // String May
        29: aastore       
        30: dup           
        31: iconst_5      
        32: ldc           #9         // String June
        34: aastore       
        35: dup           
        36: bipush        6
        38: ldc           #10        // String July
        40: aastore       
        41: dup           
        42: bipush        7
        44: ldc           #11        // String August
        46: aastore       
        47: dup           
        48: bipush        8
        50: ldc           #12        // String September
        52: aastore       
        53: dup           
        54: bipush        9
        56: ldc           #13        // String October
        58: aastore       
        59: dup           
        60: bipush        10
        62: ldc           #14        // String November
        64: aastore       
        65: dup           
        66: bipush        11
        68: ldc           #15        // String December
        70: aastore       
        71: putstatic     #2         // Field months:[Ljava/lang/String;
        74: return        
\end{lstlisting}

\TT{anewarray} creates a new array of \emph{references} (hence \emph{a} prefix).

The object's type is defined in the \TT{anewarray}'s operand, it is the \\
\q{java/lang/String} string.

The \TT{bipush 12} before \TT{anewarray} sets the array's size.

We see here a new instruction for us: \TT{dup}.


\myindex{Forth}
It's a standard instruction in stack computers (including the Forth programming language) 
which just duplicates the value at \ac{TOS}.

\myindex{x86!\Instructions!FDUP}
By the way, FPU 80x87 is also a stack computer and it has similar instruction -- \INS{FDUP}.


It is used here to duplicate a \emph{reference} to an array, because the \TT{aastore} instruction pops
the \emph{reference} to array from the stack, but subsequent \TT{aastore} will need it again.

The Java compiler concluded that it's better to generate a \TT{dup} instead of generating 
a \TT{getstatic} instruction before each array store operation (i.e., 11 times).


\TT{aastore} puts a \emph{reference} (to string) into the array at an index which is 
taken from \ac{TOS}.


Finally, \TT{putstatic} puts \emph{reference} to the newly created array into the second field 
of our object, i.e., \emph{months} field.

