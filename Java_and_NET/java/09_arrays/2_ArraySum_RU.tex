% TODO proof-reading
\subsubsection{Суммирование элементов массива}

Еще один пример:

\begin{lstlisting}[style=customjava]
public class ArraySum
{
	public static int f (int[] a)
	{
		int sum=0;
		for (int i=0; i<a.length; i++)
			sum=sum+a[i];
		return sum;
	}
}
\end{lstlisting}

\begin{lstlisting}
  public static int f(int[]);
    flags: ACC_PUBLIC, ACC_STATIC
    Code:
      stack=3, locals=3, args_size=1
         0: iconst_0      
         1: istore_1      
         2: iconst_0      
         3: istore_2      
         4: iload_2       
         5: aload_0       
         6: arraylength   
         7: if_icmpge     22
        10: iload_1       
        11: aload_0       
        12: iload_2       
        13: iaload        
        14: iadd          
        15: istore_1      
        16: iinc          2, 1
        19: goto          4
        22: iload_1       
        23: ireturn       
\end{lstlisting}

Нулевой слот в \ac{LVA} содержит указатель (\emph{reference}) на входной массив.

Первый слот \ac{LVA} содержит локальную переменную \emph{sum}.
