% TODO proof-reading
\subsection{Returning a value}

Probably the simplest Java function is the one which returns some value.

Oh, and we must keep in mind that there are no \q{free} functions in Java in common sense,
they are \q{methods}. 

Each method is related to some class, so it's not possible to define
a method outside of a class.

But we'll call them \q{functions} anyway, for simplicity.

\begin{lstlisting}[style=customjava]
public class ret
{
	public static int main(String[] args) 
	{
		return 0;
	}
}
\end{lstlisting}

Let's compile it:

\begin{lstlisting}
javac ret.java
\end{lstlisting}

\dots and decompile it using the standard Java utility:

\begin{lstlisting}
javap -c -verbose ret.class
\end{lstlisting}

And we get:

\begin{lstlisting}[caption=JDK 1.7 (excerpt)]
  public static int main(java.lang.String[]);
    flags: ACC_PUBLIC, ACC_STATIC
    Code:
      stack=1, locals=1, args_size=1
         0: iconst_0      
         1: ireturn       
\end{lstlisting}

The Java developers decided that 0 is one of the busiest constants in programming, 
so there is a separate short one-byte \GTT{iconst\_0} instruction which pushes 0

\footnote{Just like in MIPS, where a separate register for zero constant exists: \myref{MIPS_zero_register}.}.
There are also \GTT{iconst\_1} (which pushes 1), \GTT{iconst\_2}, etc., up to \GTT{iconst\_5}.

There is also \GTT{iconst\_m1} which pushes -1.

The stack is used in JVM for passing data to called functions and also for return values.
So \GTT{iconst\_0} pushes 0 into the stack.
\GTT{ireturn} returns an integer value (\emph{i} in name means \emph{integer}) from the \ac{TOS}.

Let's rewrite our example slightly, now we return 1234:

\begin{lstlisting}[style=customjava]
public class ret
{
	public static int main(String[] args)
	{
		return 1234;
	}
}
\end{lstlisting}

\dots we get:

\begin{lstlisting}[caption=JDK 1.7 (excerpt)]
  public static int main(java.lang.String[]);
    flags: ACC_PUBLIC, ACC_STATIC
    Code:
      stack=1, locals=1, args_size=1
         0: sipush        1234
         3: ireturn       
\end{lstlisting}

\GTT{sipush} (\emph{short integer}) pushes 1234 into the stack.
\emph{short} in name implies a 16-bit value is to be pushed. 
The number 1234 indeed fits well in a 16-bit value.

What about larger values?

\begin{lstlisting}[style=customjava]
public class ret
{
	public static int main(String[] args) 
	{
		return 12345678;
	}
}
\end{lstlisting}

\begin{lstlisting}[caption=Constant pool]
...
   #2 = Integer            12345678
...
\end{lstlisting}

\begin{lstlisting}
  public static int main(java.lang.String[]);
    flags: ACC_PUBLIC, ACC_STATIC
    Code:
      stack=1, locals=1, args_size=1
         0: ldc           #2                  // int 12345678
         2: ireturn       
\end{lstlisting}

It's not possible to encode a 32-bit number in a JVM instruction opcode, 
the developers didn't leave such possibility.

So the 32-bit number 12345678 is stored in so called \q{constant pool} which is, let's say,
the library of most used constants (including strings, objects, etc.).

This way of passing constants is not unique to JVM.

MIPS, ARM and other RISC CPUs also can't encode a 32-bit number
in a 32-bit opcode, so the RISC CPU code (including MIPS and ARM) has to construct the value 
in several steps, or to keep it in the data segment:
\myref{ARM_big_constants}, \myref{MIPS_big_constants}.

MIPS code also traditionally has a constant pool, named \q{literal pool}, the segments
are called \q{.lit4} (for 32-bit single precision floating point number constants) and \q{.lit8}
(for 64-bit double precision floating point number constants).

Let's try some other data types!

Boolean:

\begin{lstlisting}[style=customjava]
public class ret
{
	public static boolean main(String[] args) 
	{
		return true;
	}
}
\end{lstlisting}

\begin{lstlisting}
  public static boolean main(java.lang.String[]);
    flags: ACC_PUBLIC, ACC_STATIC
    Code:
      stack=1, locals=1, args_size=1
         0: iconst_1      
         1: ireturn       
\end{lstlisting}

This JVM bytecode is no different from one returning integer 1.

32-bit data slots in the stack are also used here for boolean values, like in \CCpp.

But one could not use returned boolean value as integer or vice versa --- type information is stored in the class file and checked at runtime.

It's the same story with a 16-bit \emph{short}:

\begin{lstlisting}[style=customjava]
public class ret
{
	public static short main(String[] args) 
	{
		return 1234;
	}
}
\end{lstlisting}

\begin{lstlisting}
  public static short main(java.lang.String[]);
    flags: ACC_PUBLIC, ACC_STATIC
    Code:
      stack=1, locals=1, args_size=1
         0: sipush        1234
         3: ireturn       
\end{lstlisting}

\dots and \emph{char}!

\begin{lstlisting}[style=customjava]
public class ret
{
	public static char main(String[] args) 
	{
		return 'A';
	}
}
\end{lstlisting}

\begin{lstlisting}
  public static char main(java.lang.String[]);
    flags: ACC_PUBLIC, ACC_STATIC
    Code:
      stack=1, locals=1, args_size=1
         0: bipush        65
         2: ireturn       
\end{lstlisting}

\INS{bipush} means \q{push byte}.
Needless to say that a \emph{char} in Java is 16-bit UTF-16 character, 
and it's equivalent to \emph{short}, but the ASCII code of the \q{A} character is 65, and it's possible
to use the instruction for pushing a byte in the stack.

Let's also try a \emph{byte}:

\begin{lstlisting}[style=customjava]
public class retc
{
	public static byte main(String[] args) 
	{
		return 123;
	}
}
\end{lstlisting}

\begin{lstlisting}
  public static byte main(java.lang.String[]);
    flags: ACC_PUBLIC, ACC_STATIC
    Code:
      stack=1, locals=1, args_size=1
         0: bipush        123
         2: ireturn       
\end{lstlisting}

One may ask, why bother with a 16-bit \emph{short} data type which internally works
as a 32-bit integer?

Why use a \emph{char} data type if it is the same as a \emph{short} data type?

The answer is simple: for data type control and source code readability.

A \emph{char} may essentially be the same as a \emph{short}, but we quickly grasp that it's a placeholder for
an UTF-16 character, and not for some other integer value.

When using \emph{short}, we show everyone that the variable's range is limited by 16 bits.

It's a very good idea to use the \emph{boolean} type where needed to, 
instead of the C-style \emph{int}.

There is also a 64-bit integer data type in Java:

\begin{lstlisting}[style=customjava]
public class ret3
{
	public static long main(String[] args)
	{
		return 1234567890123456789L;
	}
}
\end{lstlisting}

\begin{lstlisting}[caption=Constant pool]
...
   #2 = Long               1234567890123456789l
...
\end{lstlisting}

\begin{lstlisting}
  public static long main(java.lang.String[]);
    flags: ACC_PUBLIC, ACC_STATIC
    Code:
      stack=2, locals=1, args_size=1
         0: ldc2_w        #2                  // long 1234567890123456789l
         3: lreturn       
\end{lstlisting}

The 64-bit number is also stored in a constant pool, \INS{ldc2\_w} loads it and \INS{lreturn} 
(\emph{long return}) returns it.

The \INS{ldc2\_w} instruction is also used to load double precision floating point numbers 
(which also occupy 64 bits) from a constant pool:

\begin{lstlisting}[style=customjava]
public class ret
{
	public static double main(String[] args)
	{
		return 123.456d;
	}
}
\end{lstlisting}

\begin{lstlisting}[caption=Constant pool]
...
   #2 = Double             123.456d
...
\end{lstlisting}

\begin{lstlisting}
  public static double main(java.lang.String[]);
    flags: ACC_PUBLIC, ACC_STATIC
    Code:
      stack=2, locals=1, args_size=1
         0: ldc2_w        #2                  // double 123.456d
         3: dreturn       
\end{lstlisting}

\INS{dreturn} stands for \q{return double}.

And finally, a single precision floating point number:

\begin{lstlisting}[style=customjava]
public class ret
{
	public static float main(String[] args)
	{
		return 123.456f;
	}
}
\end{lstlisting}

\begin{lstlisting}[caption=Constant pool]
...
   #2 = Float              123.456f
...
\end{lstlisting}

\begin{lstlisting}
  public static float main(java.lang.String[]);
    flags: ACC_PUBLIC, ACC_STATIC
    Code:
      stack=1, locals=1, args_size=1
         0: ldc           #2                  // float 123.456f
         2: freturn       
\end{lstlisting}

The \INS{ldc} instruction used here is the same one as for loading 32-bit integer numbers
from a constant pool.

\INS{freturn} stands for \q{return float}.

Now what about function that return nothing?

\begin{lstlisting}[style=customjava]
public class ret
{
	public static void main(String[] args) 
	{
		return;
	}
}
\end{lstlisting}

\begin{lstlisting}
  public static void main(java.lang.String[]);
    flags: ACC_PUBLIC, ACC_STATIC
    Code:
      stack=0, locals=1, args_size=1
         0: return        
\end{lstlisting}

This means that the \INS{return} instruction is used to return control without returning 
an actual value.

Knowing all this, it's very easy to deduce the function's (or method's) returning type 
from the last instruction.

