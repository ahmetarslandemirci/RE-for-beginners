% TODO proof-reading
\subsection{Einfache Berechnungsfunktionen}

%Let's continue with a simple calculating functions.
%
%\begin{lstlisting}[style=customjava]
%public class calc
%{
%	public static int half(int a)
%	{
%		return a/2;
%	}
%}
%\end{lstlisting}
%
%Here's the output when the \INS{iconst\_2} instruction is used:
%
%
%\begin{lstlisting}
%  public static int half(int);
%    flags: ACC_PUBLIC, ACC_STATIC
%    Code:
%      stack=2, locals=1, args_size=1
%         0: iload_0       
%         1: iconst_2      
%         2: idiv          
%         3: ireturn       
%\end{lstlisting}
%         
%\INS{iload\_0} takes the zeroth function argument and pushes it to the stack.
%
%\INS{iconst\_2} pushes 2 in the stack.
%After the execution of these two instructions, this is how stack looks like:
%
%
%% FIXME: TikZ
%\begin{lstlisting}
%      +---+
%TOS ->| 2 |
%      +---+
%      | a |
%      +---+
%\end{lstlisting}
%
%\INS{idiv} just takes the two values at the \ac{TOS}, divides one by the other and leaves
%the result at \ac{TOS}:
%
%
%% FIXME: TikZ
%\begin{lstlisting}
%      +--------+
%TOS ->| result |
%      +--------+
%\end{lstlisting}
%
%\INS{ireturn} takes it and returns.
%
%Let's proceed with double precision floating point numbers:
%
%
%\begin{lstlisting}[style=customjava]
%public class calc
%{
%	public static double half_double(double a)
%	{
%		return a/2.0;
%	}
%}
%\end{lstlisting}
%
%\begin{lstlisting}[caption=Constant pool]
%...
%   #2 = Double             2.0d
%...
%\end{lstlisting}
%
%\begin{lstlisting}
%  public static double half_double(double);
%    flags: ACC_PUBLIC, ACC_STATIC
%    Code:
%      stack=4, locals=2, args_size=1
%         0: dload_0       
%         1: ldc2_w        #2                  // double 2.0d
%         4: ddiv          
%         5: dreturn       
%\end{lstlisting}
%
%It's the same, but the \INS{ldc2\_w} instruction is used to load the constant 
%2.0 from the constant pool.
%
%Also, the other three instructions have the \emph{d} prefix, 
%meaning they work with \emph{double} data type values.
%
%
%Let's now use a function with two arguments:
%
%
%\begin{lstlisting}[style=customjava]
%public class calc
%{
%	public static int sum(int a, int b)
%	{
%		return a+b;
%	}
%}
%\end{lstlisting}
%
%\begin{lstlisting}
%  public static int sum(int, int);
%    flags: ACC_PUBLIC, ACC_STATIC
%    Code:
%      stack=2, locals=2, args_size=2
%         0: iload_0       
%         1: iload_1       
%         2: iadd          
%         3: ireturn       
%\end{lstlisting}
%
%\INS{iload\_0} loads the first function argument (a), \INS{iload\_1}---second (b).
%
%Here is the stack after the execution of both instructions:
%
%
%\begin{lstlisting}
%      +---+
%TOS ->| b |
%      +---+
%      | a |
%      +---+
%\end{lstlisting}
%
%\INS{iadd} adds the two values and leaves the result at \ac{TOS}:
%
%
%\begin{lstlisting}
%      +--------+
%TOS ->| result |
%      +--------+
%\end{lstlisting}
%
%Let's extend this example to the \emph{long} data type:
%
%
%\begin{lstlisting}[style=customjava]
%	public static long lsum(long a, long b)
%	{
%		return a+b;
%	}
%\end{lstlisting}
%
%\dots we got:
%
%\begin{lstlisting}
%  public static long lsum(long, long);
%    flags: ACC_PUBLIC, ACC_STATIC
%    Code:
%      stack=4, locals=4, args_size=2
%         0: lload_0       
%         1: lload_2       
%         2: ladd          
%         3: lreturn       
%\end{lstlisting}
%
%The second \TT{lload} instruction takes the second argument from the 2nd slot.
%
%That's because a 64-bit \emph{long} value occupies exactly two 32-bit slots.
%
%
%Slightly more advanced example:
%
%
%\begin{lstlisting}[style=customjava]
%public class calc
%{
%	public static int mult_add(int a, int b, int c)
%	{
%		return a*b+c;
%	}
%}
%\end{lstlisting}
%
%\begin{lstlisting}
%  public static int mult_add(int, int, int);
%    flags: ACC_PUBLIC, ACC_STATIC
%    Code:
%      stack=2, locals=3, args_size=3
%         0: iload_0       
%         1: iload_1       
%         2: imul          
%         3: iload_2       
%         4: iadd          
%         5: ireturn       
%\end{lstlisting}
%
%The first step is multiplication. The product is left at the \ac{TOS}:
%
%
%\begin{lstlisting}
%      +---------+
%TOS ->| product |
%      +---------+
%\end{lstlisting}
%
%\TT{iload\_2} loads the third argument (c) in the stack:
%
%\begin{lstlisting}
%      +---------+
%TOS ->|    c    |
%      +---------+
%      | product |
%      +---------+
%\end{lstlisting}
%
%Now the \TT{iadd} instruction can add the two values.
%
