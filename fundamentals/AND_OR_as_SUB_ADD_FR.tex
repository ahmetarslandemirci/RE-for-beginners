\mysection{AND et OR comme soustraction et addition}
\label{AND_OR_as_SUB_ADD}

\subsection{Chaînes de texte de la ROM du ZX Spectrum}
\myindex{ZX Spectrum}

Ceux qui ont étudié une fois le contenu de la \ac{ROM} du ZX Spectrum, ont probablement
remarqués que le dernier caractère de chaque chaîne de texte est apparemment absent.

\begin{figure}[H]
\centering
\includegraphics[width=0.3\textwidth]{fundamentals/zx_spectrum_ROM.png}
\caption{Partie de la ROM du ZX Spectrum}
\end{figure}

Ils sont présents, en fait.

Voici un extrait de la ROM du ZX Spectrum 128K désassemblée:

\lstinputlisting{fundamentals/ZX_Spectrum_ROM.lst}
( \url{http://www.matthew-wilson.net/spectrum/rom/128_ROM0.html} )

Le dernier caractère a le bit le plus significatif mis, ce qui marque la fin de la
chaîne.
Vraisemblablement que ça a été fait pour économiser de l'espace.
Les vieux ordinateurs 8-bit avaient une mémoire très restreinte.

Les caractères de tous les messages sont toujours dans la table \ac{ASCII} 7-bit
standard, donc il est garanti que le 7ème bit n'est jamais utilisé pour les caractères.

Pour afficher une telle chaîne, nous devons tester le \ac{MSB} de chaque octet, et
s'il est mis, nous devons l'effacer, puis afficher le caractère et arrêter.
Voici un exemple en C:

\begin{lstlisting}[style=customc]
unsigned char hw[]=
{
	'H',
	'e',
	'l',
	'l',
	'o'|0x80
};

void print_string()
{
	for (int i=0; ;i++)
	{
		if (hw[i]&0x80) // check MSB
		{
			// clear MSB
			// (en d'autres mots, les effacer tous, mais laisser les 7 bits inférieurs intacts)
			printf ("%c", hw[i] & 0x7F);
			// stop
			break;
		};
		printf ("%c", hw[i]);
	};
};
\end{lstlisting}

Maintenant ce qui est intéressant, puisque le 7ème bit est le bit le plus significatif
(dans un octet), c'est que nous pouvons le tester, le mettre et le supprimer en utilisant
des opérations arithmétiques au lieu de logiques:

Je peux récrire mon exemple en C:

\begin{lstlisting}[style=customc]
unsigned char hw[]=
{
	'H',
	'e',
	'l',
	'l',
	'o'+0x80
};

void print()
{
	for (int i=0; ;i++)
	{
		// hw[] doit avoir le type 'unsigned char'
		if (hw[i] >= 0x80) // tester le MSB
		{
			printf ("%c", hw[i]-0x80); // clear MSB
			// stop
			break;
		};
		printf ("%c", hw[i]);
	};
};
\end{lstlisting}

Par défaut, \emph{char} est un type signé en \CCpp, donc pour le comparer avec une
variable comme 0x80 (qui est négative ($-128$) si elle est traitée comme signée),
nous devons traiter chaque caractère dans le texte du message comme non signé.

Maintenant si le 7ème bit est mis, le nombre est toujours supérieur ou égal à
0x80.
Si le 7ème est à zéro, le nombre est toujours plus petit que 0x80.

Et même plus que ça: si le 7ème bit est mis, il peut être effacé en soustrayant 0x80,
rien d'autre.
Si il n'est pas mis avant, toutefois, la soustraction va détruire d'autres bits.

De même, si le 7ème est à zéro, il est possible de le mettre en ajoutant 0x80.
Mais s'il est déjà mis, l'opération d'addition va détruire d'autres bits.

En fait, ceci est valide pour n'importe quel bit.
Si le 4ème bit est à zéro, vous pouvez le mettre juste en ajoutant 0x10: 0x100+0x10 = 0x110.
Si le 4ème bit est mis, vous pouvez l'effacer en soustrayant 0x10: 0x1234-0x10 = 0x1224.

Ça fonctionne, car il n'y a pas de retenue générée pendant l'addition/soustraction.
Elle le serait, toutefois, si le bit est déjà à 1 avant l'addition, ou à 0 avant
la soustraction.

De même, addition/soustraction peuvent être remplacées en utilisant une opération
OR/AND si deux conditions sont réunies:
1) vous voulez ajouter/soustraire un nombre de la forme $2^n$;
2) la valeur du bit d'indice $n$ dans la valeur source est 0/1.

Par exemple, l'addition de 0x20 est la même chose que OR-er la valeur avec 0x20 sous
la condition que ce bit est à zéro avant:
0x1204|0x20 = 0x1204+0x20 = 0x1224.

La soustraction de 0x20 est la même chose que AND-er la valeur avec ~0x20 (0x....FFDF),
mais si ce bit est mis avant:
0x1234\&(\~{}0x20) = 0x1234\&0xFFDF = 0x1234-0x20 = 0x1214.

À nouveau, ceci fonctionne parce qu'il n'y a pas de retenue générée lorsque vous ajoutez
le nombre $2^n$ et que ce bit n'est pas à 1 avant.

Cette propriété de l'algèbre booléenne est importante, elle vaut la peine d'être comprise
et gardée à l'esprit.

Un autre exemple dans ce livre: \myref{toupper_bit}.

