\subsection{Quelques ajouts à propos du complément à deux}

\epigraph{Exercice 2-1. Écrire un programme pour déterminer les intervalles des variables
\TT{char}, \TT{short}, \TT{int}, et \TT{long}, signées et non signées, en affichant
les valeurs appropriées depuis les headers standards et par calcul direct.}{\KRBook}

\subsubsection{Obtenir le nombre maximum de quelques \glslink{word}{mots}}

Le maximum d'un nombre non signé est simplement un nombre où tous les bits sont mis:
\emph{0xFF....FF} (ceci est -1 si le \glslink{word}{mot} est traité comme un entier
signé).
Donc, vous prenez un \glslink{word}{mot}, vous mettez tous les bits et vous obtenez
la valeur:

\begin{lstlisting}[style=customc]
#include <stdio.h>

int main()
{
	unsigned int val=~0; // changer à "unsigned char" pour obtenir la valeur maximale pour un octet 8-bit non-signé
	// 0-1 fonctionnera aussi, ou juste -1
	printf ("%u\n", val); // %u pour unsigned
};
\end{lstlisting}

C'est 4294967295 pour un entier 32-bit.

\subsubsection{Obtenir le nombre maximum de quelques \glslink{word}{mots} signés}

Le nombre signé minimum est encodé en \emph{0x80....00}, i.e., le bit le plus significatif
est mis, tandis que tous les autres sont à zéro.
Le nombre maximum signé est encodé de la même manière, mais tous les bits sont
inversés: \emph{0x7F....FF}.

Déplaçons un seul bit jusqu'à ce qu'il disparaisse:

\begin{lstlisting}[style=customc]
#include <stdio.h>

int main()
{
	signed int val=1; // changer à "signed char" pour trouver les valeurs pour un octet signé
	while (val!=0)
	{
		printf ("%d %d\n", val, ~val);
		val=val<<1;
	};
};
\end{lstlisting}

La sortie est:

\begin{lstlisting}
...

536870912 -536870913
1073741824 -1073741825
-2147483648 2147483647
\end{lstlisting}

Les deux dernier nombres sont respectivement le minimum et le maximum d'un entier
signé 32-bit \emph{int}.

