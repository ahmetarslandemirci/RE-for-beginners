\mysection{ARM}
\myindex{ARM}

\subsection{Terminologie}

ARM a été initialement développé comme un \ac{CPU} 32-bit, c'est pourquoi ici un
\emph{mot}, contrairement au x86, fait 32-bit.

\begin{description}
	\item[octet] 8-bit.
		La directive d'assemblage DB est utilisée pour définir des variables et des
		tableaux d'octets.
	\item[demi-mot] 16-bit. directive d'assemblage DCW \dittoclosing.
	\item[mot] 32-bit.  directive d'assemblage DCW \dittoclosing.
	\item[double mot] 64-bit.
	\item[quadruple mot] 128-bit.
\end{description}

\subsection{Versions}

\begin{itemize}
\item ARMv4: Le mode Thumb mode a été introduit.

\item ARMv6: Utilisé dans la 1ère génération d'iPhone, iPhone 3G (Samsung 32-bit RISC ARM 1176JZ(F)-S 
qui supporte Thumb-2)

\item ARMv7: Thumb-2 a été ajouté (2003).
Utilisé dans l'iPhone 3GS, iPhone 4, iPad 1ère génération (ARM Cortex-A8), iPad 2 (Cortex-A9),
iPad 3ème génération.

\item ARMv7s: De nouvelles instructions ont été ajoutées.
Utilisé dans l'iPhone 5, l'iPhone 5c, l'iPad 4ème génération. (Apple A6).

\item ARMv8: 64-bit CPU, \ac{AKA} ARM64 \ac{AKA} AArch64.
Utilisé dans l'iPhone 5S, l'iPad Air (Apple A7).
Il n'y a pas de mode Thumb en mode 64-bit, seulement ARM (instructions de 4 octets).
\end{itemize}

% sections
\subsection{ARM 32-bit (AArch32)}

\subsubsection{Registres d'usage général}

\begin{itemize}
\myindex{ARM!\Registers!R0}
	\item R0 --- le résultat d'une fonction est en général renvoyé dans R0
	\item R1...R12 --- \ac{GPR}s
	\item R13 --- \ac{AKA} SP (\glslink{stack pointer}{pointeur de pile})
\myindex{ARM!\Registers!Link Register}
	\item R14 --- \ac{AKA} LR (\gls{link register})
	\item R15 --- \ac{AKA} PC (program counter)
\end{itemize}

\myindex{ARM!\Registers!scratch registers}
\Reg{0}-\Reg{3} sont aussi appelés \q{registres scratch}: les arguments de la fonctions sont
d'habitude passés par eux, et leurs valeurs n'ont pas besoin d'être restaurées en sortant de la
fonction.

\subsubsection{Current Program Status Register (CPSR)}

\begin{center}
\begin{tabular}{ | l | l | }
\hline
\headercolor\ Bit &
\headercolor\ Description \\
\hline
0..4           & M --- processor mode \\
\hline
5              & T --- Thumb state \\
\hline
6              & F --- FIQ disable \\
\hline
7              & I --- IRQ disable \\
\hline
8              & A --- imprecise data abort disable \\
\hline
9              & E --- data endianness \\
\hline
10..15, 25, 26 & IT --- if-then state \\
\hline
16..19         & GE --- greater-than-or-equal-to \\
\hline
20..23         & DNM --- do not modify \\
\hline
24             & J --- Java state \\
\hline
27             & Q --- sticky overflow \\
\hline
28             & V --- overflow \\
\hline
29             & C --- carry/borrow/extend \\
\hline
\myindex{ARM!\Registers!Z}
30             & Z --- zero bit \\
\hline
31             & N --- negative/less than \\
\hline
\end{tabular}
\end{center}

% TODO
% \myindex{ARM!\Registers!APSR}
% \subsubsection{Application Program Status Register (APSR)}

% TODO
% \myindex{ARM!\Registers!FPSCR}
% \subsubsection{Floating-Point Status and Control Register (FPPSR)}
% http://infocenter.arm.com/help/index.jsp?topic=/com.arm.doc.ddi0344b/Chdfafia.html

\subsubsection{Registres VFP (virgule flottante) et registres NEON}
\label{ARM_VFP_registers}

% http://infocenter.arm.com/help/index.jsp?topic=/com.arm.doc.dht0002a/ch01s03s02.html

\myindex{ARM!D-\registers{}}
\myindex{ARM!S-\registers{}}
\begin{center}
\begin{tabular}{ | l | l | l | l | }
\hline
0..31\textsuperscript{bits} & 32..64 & 65..96 & 97..127 \\
\hline
\multicolumn{4}{ | c | }{Q0\textsuperscript{128 bits}} \\
\hline
\multicolumn{2}{ | c | }{D0\textsuperscript{64 bits}} & \multicolumn{2}{ c | }{D1} \\
\hline
S0\textsuperscript{32 bits} & S1 & S2 & S3 \\
\hline
\end{tabular}
\end{center}

Les registres-S sont 32-bit, utilisés pour le stockage de nombre en simple précision.
Les registres-D sont 64-bit, utilisés pour le stockage de nombre en double précision.

Les registres-D et -S partagent le même espace physique dans le CPU---il est possible d'accéder
un registre-D via les registres-S (mais c'est insensé).

De même, les registres \gls{NEON} sont des 128-bit et partagent le même espace physique dans le CPU
avec les autres registres en virgule flottante.

En VFP les registres-S sont présents: S0..S31.

En VFPv2 16 registres-D ont été ajoutés, qui occupent en fait le même espace que S0..S31.

En VFPv3 (\gls{NEON} ou \q{SIMD avancé}) il y a 16 registres-D de plus, D0..D31, mais les registres
D16..D31 ne partagent pas l'espace avec aucun autre registres-S.

En \gls{NEON} ou \q{SIMD avancé} 16 autres registres-Q 128-bit ont été ajoutés,
qui partagent le même espace que D0..D31.

\subsection{ARM 64-bit (AArch64)}

\subsubsection{Registres d'usage général}
\label{ARM64_GPRs}

Le nombre de registres a été doublé depuis AArch32.

\begin{itemize}
\myindex{ARM!\Registers!X0}
	\item X0 --- le résultat d'une fonction est en général renvoyé dans X0
	\item X0...X7 --- Les arguments de fonction sont passés ici
	\item X8
	\item X9...X15 --- sont des registres temporaires, la fonction appelée peut les utiliser sans en
	restaurer le contenu.
	\item X16
	\item X17
	\item X18
	\item X19...X29 --- la fonction appelée peut les utiliser mais doit restaurer leurs valeurs à sa
	sortie.
	\item X29 --- utilisé comme \ac{FP} (au moins dans GCC)
	\item X30 --- \q{Procedure Link Register} \ac{AKA} \ac{LR} (\gls{link register}).
	\item X31 --- ce registre contient toujours zéro \ac{AKA} XZR ou ZR \q{Zero Register}.
	Sa partie 32-bit est appelée WZR.
	\item \ac{SP}, n'est plus un registre d'usage général.
\end{itemize}

Voir aussi: \ARMPCS.

La partie 32-bit de chaque registre-X est aussi accessible par les registres-W (W0, W1, etc.).

\begin{center}
\begin{tabular}{ | l | l | }
\hline
\RU{Старшие 32 бита}\EN{High 32-bit part}\ES{Parte alta de 32 bits}\PTBRph{}\PLph{}\ITph{}\DE{Oberer 32-Bit-Teil}\THAph{}\NLph{}\FR{Partie 32 bits haute} & \RU{младшие 32 бита}\EN{low 32-bit part}\ES{parte baja de 32 bits}\PTBRph{}\PL{Starsze 32 bity}\ITph{}\DE{Unterer 32-Bit-Teil}\THAph{}\NLph{}\FR{Partie 32 bits basse} \\
\hline
\multicolumn{2}{ | c | }{X0} \\
\hline
\multicolumn{1}{ | c | }{} & \multicolumn{1}{ c | }{W0} \\
\hline
\end{tabular}
\end{center}


\subsection{Instructions}


Il il y a un suffixe \emph{-S}  pour certaines instructions en ARM, indiquant que
l'instruction met les flags en fonction du résultat.
Les instructions qui n'ont pas ce suffixe ne modifient pas les flags.
\myindex{ARM!\Instructions!ADD}
\myindex{ARM!\Instructions!ADDS}
\myindex{ARM!\Instructions!CMP}
Par exemple \TT{ADD} contrairement à \TT{ADDS}
ajoute deux nombres, mais les flags sont inchangés.
De telles instructions sont pratiques à utiliser entre \CMP où les flags sont mis et, e.g.
les sauts conditionnels, où les flags sont utilisés.
Elles sont aussi meilleures en termes d'analyse de dépendance de données (car moins
de registres sont modifiés pendant leurs exécution).

% ADD
% ADDAL
% ADDCC
% ADDS
% ADR
% ADREQ
% ADRGT
% ADRHI
% ADRNE
% ASRS
% B
% BCS
% BEQ
% BGE
% BIC
% BL
% BLE
% BLEQ
% BLGT
% BLHI
% BLS
% BLT
% BLX
% BNE
% BX
% CMP
% IDIV
% IT
% LDMCSFD
% LDMEA
% LDMED
% LDMFA
% LDMFD
% LDMGEFD
% LDR.W
% LDR
% LDRB.W
% LDRB
% LDRSB
% LSL.W
% LSL
% LSLS
% MLA
% MOV
% MOVT.W
% MOVT
% MOVW
% MULS
% MVNS
% ORR
% POP
% PUSH
% RSB
% SMMUL
% STMEA
% STMED
% STMFA
% STMFD
% STMIA
% STMIB
% STR
% SUB
% SUBEQ
% SXTB
% TEST
% TST
% VADD
% VDIV
% VLDR
% VMOV
% VMOVGT
% VMRS
% VMUL
%\myindex{ARM!Optional operators!ASR
%\myindex{ARM!Optional operators!LSL
%\myindex{ARM!Optional operators!LSR
%\myindex{ARM!Optional operators!ROR
%\myindex{ARM!Optional operators!RRX

% AArch64
% RET is BR X30 or BR LR but with additional hint to CPU

\subsubsection{Table des codes conditionnels}

% TODO rework this!
\small
\begin{center}
\begin{tabular}{ | l | l | l | }
\hline
\HeaderColor Code & 
\HeaderColor Description & 
\HeaderColor Flags \\
\hline
EQ & Égal & Z == 1 \\
\hline
NE & Non égal & Z == 0 \\
\hline
CS \ac{AKA} HS (Higher or Same) & Retenue mise/ Non-signé, Plus grand que, égal & C == 1 \\
\hline
CC \ac{AKA} LO (LOwer) & Retenue à zéro / non-signé, moins que & C == 0 \\
\hline
MI & moins, négatif / moins que & N == 1 \\
\hline
PL & plus, positif ou zéro / Plus grnad que, égal & N == 0 \\
\hline
VS & débordement & V == 1 \\
\hline
VC & Pas de débordement & V == 0 \\
\hline
HI & non signé supérieur / plus grand que & C == 1 \AndENRU \\
 & & Z == 0 \\
\hline
LS & non signé inférieur ou égal / inférieur ou égal & C == 0 \OrENRU \\
 & & Z == 1 \\
\hline
GE & signé supérieur ou égal / supérieur ou égal & N == V \\
\hline
LT & signé plus petit que / plus petit que & N != V \\
\hline
GT & signé plus grand que / plus grnad que & Z == 0 \AndENRU \\
 & & N == V \\
\hline
LE & signé inférieur ou égal / moins que, égal & Z == 1 \OrENRU \\
 & & N != V \\
\hline
None / AL & toujours & n'importe lequel \\
\hline
\end{tabular}
\end{center}
\normalsize

