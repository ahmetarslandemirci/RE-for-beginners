\newglossaryentry{tail call}
{
  name=tail call,
  description={It is when the compiler (or interpreter) transforms the recursion
  (\emph{tail recursion}) into an iteration for efficiency}
}

\newglossaryentry{endianness}
{
  name=endianness,
  description={Byte order}
}

\newglossaryentry{caller}
{
  name=caller,
  description={A function calling another}
}

\newglossaryentry{callee}
{
  name=callee,
  description={A function being called by another}
}

\newglossaryentry{debuggee}
{
  name=debuggee,
  description={A program being debugged}
}

\newglossaryentry{leaf function}
{
  name=leaf function,
  description={A function which does not call any other function}
}

\newglossaryentry{link register}
{
  name=link register,
  description=(RISC) {A register where the return address is usually stored.
  This makes it possible to call leaf functions without using the stack, i.e., faster}
}

\newglossaryentry{anti-pattern}
{
  name=anti-pattern,
  description={Generally considered as bad practice}
}

\newglossaryentry{stack pointer}
{
  name=stack pointer,
  description={A register pointing to a place in the stack}
}

\newglossaryentry{decrement}
{
  name=decrement,
  description={Decrease by 1}
}

\newglossaryentry{increment}
{
  name=increment,
  description={Increase by 1}
}

\newglossaryentry{loop unwinding}
{
  name=loop unwinding,
  description={It is when a compiler, instead of generating loop code for $n$ iterations, generates just $n$ copies of the
  loop body, in order to get rid of the instructions for loop maintenance}
}

\newglossaryentry{register allocator}
{
  name=register allocator,
  description=
  {The part of the compiler that assigns CPU registers to local variables}
}

\newglossaryentry{quotient}
{
  name=quotient,
  description={Division result}
}

\newglossaryentry{product}
{
  name=product,
  description={Multiplication result}
}

\newglossaryentry{NOP}
{
  name=NOP,
  description={\q{no operation}, idle instruction}
}

\newglossaryentry{POKE}
{
  name=POKE,
  description={BASIC language instruction for writing a byte at a specific address}
}

\newglossaryentry{keygenme}
{
  name=keygenme,
  description={A program which imitates software protection,
  for which one needs to make a key/license generator}
} % TODO clarify: A software which generate key/license value to bypass sotfware protection?

\newglossaryentry{dongle}
{
  name=dongle,
  description={Dongle is a small piece of hardware connected to LPT printer port (in past) or to USB}
}

\newglossaryentry{thunk function}
{
  name=thunk function,
  description={Tiny function with a single role: call another function}
}

\newglossaryentry{user mode}
{
  name=user mode,
  description={A restricted CPU mode in which it all application software code is executed. cf. \gls{kernel mode}}
}

\newglossaryentry{kernel mode}
{
  name=kernel mode,
  description={A restrictions-free CPU mode in which the OS kernel and drivers execute. cf. \gls{user mode}}
}

\newglossaryentry{Windows NT}
{
  name=Windows NT,
  description={Windows NT, 2000, XP, Vista, 7, 8, 10}
}

\newglossaryentry{atomic operation}
{
  name=atomic operation,
  description={
  \q{$\alpha{}\tau{}o\mu{}o\varsigma{}$}
  %\q{atomic}
  stands for \q{indivisible} in Greek, so an atomic operation is guaranteed not
  to be interrupted by other threads}
}

% to be proofreaded (begin)
\newglossaryentry{NaN}
{
  name=NaN,
  description={not a number: 
  	a special cases for floating point numbers, usually signaling about errors}
}

\newglossaryentry{basic block}
{
  name=basic block,
  description={
	a group of 
	instructions that do not have jump/branch instructions, and also don't have
	jumps inside the block from the outside.
	In \IDA it looks just like as a list of instructions without empty lines}
}

\newglossaryentry{NEON}
{
  name=NEON,
  description={\ac{AKA} \q{Advanced SIMD}---\ac{SIMD} from ARM}
}

\newglossaryentry{reverse engineering}
{
  name=reverse engineering,
  description={act of understanding how the thing works, sometimes in order to clone it}
}

\newglossaryentry{compiler intrinsic}
{
  name=compiler intrinsic,
  description={A function specific to a compiler which is not an usual library function.
	The compiler generates a specific machine code instead of a call to it.
	Often, it's a pseudofunction for a specific \ac{CPU} instruction. Read more:}
 (\myref{sec:compiler_intrinsic})
}

\newglossaryentry{heap}
{
  name=heap,
  description={usually, a big chunk of memory provided by the \ac{OS} so that applications can divide it by themselves as they wish.
  malloc()/free() work with the heap}
}

\newglossaryentry{name mangling}
{
  name=name mangling,
  description={used at least in \Cpp, where the compiler needs to encode the name of class, method and argument types in one string,
  which will become the internal name of the function. You can read more about it here: \myref{namemangling}}
}

\newglossaryentry{xoring}
{
  name=xoring,
  description={often used in the English language, which implying applying the \ac{XOR} operation}
}

\newglossaryentry{security cookie}
{
  name=security cookie,
  description={A random value, different at each execution. You can read more about it here:
  \myref{subsec:BO_protection}}
}

\newglossaryentry{tracer}
{
  name=tracer,
  description={My own simple debugging tool. You can read more about it here: \myref{tracer}}
}

\newglossaryentry{GiB}
{
  name=GiB,
  description={Gibibyte: $2^{30}$ or 1024 mebibytes or 1073741824 bytes}
}

\newglossaryentry{CP/M}
{
  name=CP/M,
  description={Control Program for Microcomputers: a very basic disk \ac{OS} used before MS-DOS}
}

\newglossaryentry{stack frame}
{
  name=stack frame,
  description={A part of the stack that contains information specific to the current function:
  local variables, function arguments, \ac{RA}, etc.}
}

\newglossaryentry{jump offset}
{
  name=jump offset,
  description={a part of the JMP or Jcc instruction's opcode, 
  to be added to the address
  of the next instruction, and this is how the new \ac{PC} is calculated. May be negative as well}
}

\newglossaryentry{integral type}
{
  name=integral data type,
  description={usual numbers, but not a real ones. may be used for passing variables of boolean data type and enumerations}
}

\newglossaryentry{real number}
{
  name={real number},
  description={numbers which may contain a dot. this is \Tfloat and \Tdouble in \CCpp}
}

\newglossaryentry{PDB}
{
  name=PDB,
  description={(Win32) Debugging information file, usually just function names, but sometimes also function
  arguments and local variables names}
}

\newglossaryentry{NTAPI}
{
  name=NTAPI,
  description={\ac{API} available only in the Windows NT line.  Largely not documented by Microsoft}
}

\newglossaryentry{stdout}
{
  name=stdout,
  description={standard output}
}

\newglossaryentry{word}
{
  name=word,
  description={data type fitting in \ac{GPR}.
  In the computers older than PCs, 
  the memory size was often measured in words rather than bytes.}
}

\newglossaryentry{arithmetic mean}
{
  name=arithmetic mean,
  description={a sum of all values divided by their count}
}
\newglossaryentry{padding}
{
  name=padding,
  description={
  \emph{Padding} in English language means to stuff a pillow with something
  to give it a desired (bigger) form.
  In computer science, padding means to add more bytes to a block so it will have desired size, like $2^n$ bytes.}
}

