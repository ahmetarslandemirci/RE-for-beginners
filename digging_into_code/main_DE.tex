\chapter{Finden von wichtigen / interessanten Stellen im Code}

Minimalismus ist kein beliebtes Feature moderner Software.

\myindex{\Cpp!STL}

Aber nicht weil die Programmierer so viel Code schreiben, sondern weil die Libaries
allgemein statisch zu ausf\"uhrbaren Dateien gelinkt werden. Wenn alle externen
Libraries in externe DLL Dateien verschoben werden w\"urden, w\"are die Welt ein
anderer Ort. (Ein weiterer Grund f\"ur C++ sind die \ac{STL} und andere Template-Libraries.)

\newcommand{\FOOTNOTEBOOST}{\footnote{\url{http://go.yurichev.com/17036}}}
\newcommand{\FOOTNOTELIBPNG}{\footnote{\url{http://go.yurichev.com/17037}}}

Deshalb ist es sehr wichtig den Ursprung einer Funktion zu bestimmen, wenn die
Funktion aus einer Standard-Library oder aus einer sehr bekannten Library stammt
(wie z.B Boost\FOOTNOTEBOOST, libpng\FOOTNOTELIBPNG), oder ob die Funktion sich
auf das bezieht was wir im Code versuchen zu finden.

Es ist ein wenig absurd s\"amtlichen Code in \CCpp neu zu schreiben, um das zu
finden was wir suchen.

Eine der Hauptaufgaben eines Reverse Enigneers ist es schnell Code zu finden den
er/sie sucht.

\myindex{\GrepUsage}

Der \IDA-Disassembler erlaubt es durch Textstrings, Byte-Sequenzen und Konstanten
zu suchen.  Es ist sogar m\"oglich den Code in .lst oder .asm Text Dateien zu
exportieren und diese mit \TT{grep}, \TT{awk}, etc. zu untersuchen.

Wenn man versucht zu verstehen wie ein bestimmter Code funktioniert, kann auch
eine einfache Open-Source-Library wie libpng als Beispiel dienen.
Wenn man also eine Konstante oder Textstrings findet die vertraut erscheinen, ist
es immer einen Versuch wert diese zu \emph{google}n .
Und wenn man ein Opensource Projekt findet in dem diese Funktion benutzt wird, 
reicht es meist aus diese Funktionen miteinander zu vergleichen.
Es k\"onnte helfen Teile des Problems zu l\"osen.

% When you try to understand what some code is doing, this easily could be some open-source library like libpng.
% So when you see some constants or text strings which look familiar, it is always worth to \emph{google} them.
% And if you find the opensource project where they are used, 
% then it's enough just to compare the functions.
% It may solve some part of the problem.

Zum Beispiel, wenn ein Programm XML Dateien benutzt, w\"are der erste Schritt zu ermitteln welche
XML-Library benutzt wird f\"ur die Verarbeitung, da die Standard (oder am weitesten verbreitete) libraries
normal benutzt werden anstatt selbst geschriebene librarys.

\myindex{SAP}
\myindex{Windows!PDB}

Zum Beispiel, der Autor dieser Zeilen wollte verstehen wie die Kompression/Dekompression von Netzwerkpaketen in SAP 6.0 funktioniert.
SAP ist ein gewaltiges St\"uck Software, aber detaillierte -\gls{PDB} Dateien mit Debug Informationen sind vorhanden, was sehr praktisch 
ist. Der Autor hat schließlich eine Ahnung gehabt, das eine Funktion genannt \emph{CsDecomprLZC} die Dekompression der Netzwerkpakete \"ubernahm.
Er hat nach dem Namen der Funktion auf google gesucht und ist schnell zum schluss gekommen das diese Funktion in 
MaxDB benutzt wurde (Das ist ein Open-Source SAP Projekt) \footnote{Mehr dar\"uber in der relevanten Sektion~(\myref{sec:SAPGbUI})}. 

\url{http://www.google.com/search?q=CsDecomprLZC}

Erstaunlich, das MaxDB und die SAP 6.0 Software den selben Code geteilt haben f\"ur die Kompression/Dekompression der Netzwerkpakete.

\mysection{Ausf\"uhrbare Dateien Identifizieren}

\subsection{Microsoft Visual C++}
\label{MSVC_versions}

MSVC Versionen und DLLs die Importiert werden k\"onnen:

%\small
\begin{center}
\begin{tabular}{ | l | l | l | l | l | }
\hline
\HeaderColor Marketing ver. & 
\HeaderColor Internal ver. & 
\HeaderColor CL.EXE ver. &
\HeaderColor DLLs imported &
\HeaderColor Release date \\
\hline
% 4.0, April 1995
% 97 & 5.0 & February 1997
6		&  6.0	& 12.00	& msvcrt.dll	& June 1998		\\
		&	&	& msvcp60.dll	&			\\
\hline
.NET (2002)	&  7.0	& 13.00	& msvcr70.dll	& February 13, 2002	\\
		&	&	& msvcp70.dll	&			\\
\hline
.NET 2003	&  7.1	& 13.10 & msvcr71.dll	& April 24, 2003	\\
		&	&	& msvcp71.dll	&			\\
\hline
2005		&  8.0	& 14.00 & msvcr80.dll	& November 7, 2005	\\
		&	&	& msvcp80.dll	&			\\
\hline
2008		&  9.0	& 15.00 & msvcr90.dll	& November 19, 2007	\\
		&	&	& msvcp90.dll	&			\\
\hline
2010		& 10.0	& 16.00 & msvcr100.dll	& April 12, 2010 	\\
		&	&	& msvcp100.dll	&			\\
\hline
2012		& 11.0	& 17.00 & msvcr110.dll	& September 12, 2012 	\\
		&	&	& msvcp110.dll	&			\\
\hline
2013		& 12.0	& 18.00 & msvcr120.dll	& October 17, 2013 	\\
		&	&	& msvcp120.dll	&			\\
\hline
\end{tabular}
\end{center}
%\normalsize

msvcp*.dll hat \Cpp{}-bezogene Funktionen, bedeutet wenn die library importiert wird,
ist das Programm das sie importiert wahrscheinlich ein \Cpp program.

\subsubsection{Name mangling} 

Die Namen fangen normal an mit dem \TT{?} Symbol.

Hier: \myref{namemangling} kann man mehr lesen \"uber MSVC's \gls{name mangling} . 

\subsection{GCC}
\myindex{GCC}

Neben *NIX Umgebungen, ist GCC auch in win32 Umgebungen pr\"asent, in der Form von Cygwin and MinGW. 

\subsubsection{Name mangling} 

Namen fangen hier normal mit dem \TT{\_Z} Symbolen an.

Man kann mehr lesen \"uber GCC's \gls{name mangling} hier: \myref{namemangling}.

\subsubsection{Cygwin}
\myindex{Cygwin}

cygwin1.dll wird oft importiert.

\subsubsection{MinGW}
\myindex{MinGW}

msvcrt.dll wird vielleicht importiert.

\subsection{Intel Fortran}
\myindex{Fortran}


libifcoremd.dll, libifportmd.dll and libiomp5md.dll (OpenMP Support) werden vielleicht importiert.

libifcoremd.dll hat eine menge an Funktionen die das \TT{for\_} Pr\"afix haben, was \emph{Fortran} bedeutet.

\subsection{Watcom, OpenWatcom}
\myindex{Watcom}
\myindex{OpenWatcom}

\subsubsection{Name mangling}

Namen fangen normal mit dem \TT{W} Symbol an. 

Zum Beispiel wird so eine Methode benannt \q{method} der Klasse \q{class} die keine Argumente hat und \Tvoid zur\"uck gibt: % <-- Finde was besseres!
% For example, that is how the method named \q{method} of the class \q{class} that does not have any arguments and returns
% \Tvoid is encoded:

\begin{lstlisting}
W?method$_class$n__v
\end{lstlisting}

\subsection{Borland}
\myindex{Borland Delphi}
\myindex{Borland C++Builder}

Hier ist ein Beispiel f\"ur Borland Delphi's und C++Builder's \gls{name mangling}:

\lstinputlisting{digging_into_code/identification/borland_mangling.txt}

Die Namen fangen immer mit dem \TT{@} Symbol an, dann haben wir den Namen
der Klassen Namen, Methoden Namen, und codiert die Typen der Argumente der Methode.

Diese Namen k\"onnen in den .exe Imports, .dll Exports, Debug Daten und etc existieren.

Borland Visual Component Libraries (VCL) 
werden in .bpl Dateien gehalten anstatt .dll's, zum Beispiel vcl50.dll, rtl60.dll.

Eine weitere DLL die vielleicht importiert wird: BORLNDMM.DLL

\subsubsection{Delphi}

Fast alle Delphi executables haben den \q{Boolean} Text String am Anfang des Code Segments, zusammen mit den Namen anderer Typen liegen.
% Almost all Delphi executables has the \q{Boolean} text string at the beginning of the code segment, along with other type names.

Dies ist ein sehr typischer Anfang f\"ur das \TT{CODE} Segment bei einem 
Delphi Programm, dieser Block kam direkt nach dem win32 PE Datei header:

\lstinputlisting{digging_into_code/identification/delphi.txt}

Die ersten 4 Btyes des Daten Segments (\TT{DATA}) k\"onnen \TT{00 00 00 00}, \TT{32 13 8B C0} oder \TT{FF FF FF FF} sein.

Diese Informationen k\"onnen n\"utzlich sein wenn man mit gepackten oder verschl\"usselten Delphi executables arbeiten muss. 

\subsection{Other known DLLs}

\myindex{OpenMP}
\begin{itemize}
\item vcomp*.dll---Microsoft's Implementierung von OpenMP. 
\end{itemize}

 

\mysection{Kommunikation mit der außen Welt (Funktion Level)} 
Oft ist es empfehlenswert die Funktionsargumente und die R\"uckgabewerte im
Debugger oder \ac{DBI} zu \"uberwachen. Zum Beispiel hat der Autor einmal
versucht die Bedeutung einer obskuren Funktion zu verstehen, die einen inkorrekten
Bubblesort-Algorithmus implementiert hatte\footnote{\url{https://yurichev.com/blog/weird_sort/}}
(Sie hat funktioniert, jedoch viel langsamer als normal). Die Eingaben und Ausgaben zur Laufzeit 
der Funktion zu \"uberwachen hilft sofort zu verstehen was die Funktion tut.

% TBT

% sections:
\input{digging_into_code/communication_win32_DE}
\mysection{Strings}
\label{sec:digging_strings}

\subsection{Text strings}

\subsubsection{\CCpp}

\label{C_strings}

Die normalen C-strings sind NULL-Terminiert (\ac{ASCIIZ}-strings).

Der Grund warum C Stringformatierung so ist wie sie ist (NULL-Terminiert) scheint ein Historischer zu sein.
In [Dennis M. Ritchie, \emph{The Evolution of the Unix Time-sharing System}, (1979)] kann man nach lesen:

\begin{framed}
\begin{quotation}
Ein kleiner Unterschied war das die I/O Einheit ein ``word'' war, nicht ein Byte, weil die PDP-7 eine word-adressierte
Maschine war. In der Praxis bedeutete das lediglich das alle Programme die mit Zeichen Streams arbeiteten, das NULL 
Zeichen ignorieren mussten, weil die NULL benutzt wurde um eine Datei bis zu einer Graden Zahl an Bytes auf zu f\"ullen.

\end{quotation}
\end{framed}

\myindex{Hiew}

In Hiew oder FAR Manager sehen diese Strings so aus:

\begin{lstlisting}[style=customc]
int main()
{
	printf ("Hello, world!\n");
};
\end{lstlisting}

\begin{figure}[H]
\centering
\includegraphics[width=0.6\textwidth]{digging_into_code/strings/C-string.png}
\caption{Hiew}
\end{figure}

% FIXME видно \n в конце, потом пробел

\subsubsection{Borland Delphi}
\myindex{Pascal}
\myindex{Borland Delphi}

Dem String in Passcal und Borland Delphi h\"angt eine 8 oder 32-Bit Zeichenkette an. 

Zum Beispiel:

\begin{lstlisting}[caption=Delphi,style=customasmx86]
CODE:00518AC8                 dd 19h
CODE:00518ACC aLoading___Plea db 'Loading... , please wait.',0

...

CODE:00518AFC                 dd 10h
CODE:00518B00 aPreparingRun__ db 'Preparing run...',0
\end{lstlisting}

\subsubsection{Unicode}

\myindex{Unicode}

Oft, ist das was Unicode genannt wird einfach eine Methode um Strings zu codieren, bei denen jedes Zeichen 2 Byte oder 
16 Bits verbraucht. Das ist ein h\a"ufiger Terminologischer Fehler. Unicode ist ein Standard bei dem eine Nummer 
zu einem der vielen Schreibsysteme der Welt zugeordnet wird, aber es beschreibt nicht die codierungs Methode. 

\myindex{UTF-8}
\myindex{UTF-16LE}

Die bekannteste Methode zu Codieren ist: UTF-8 ( ist weit verteilt im Internet und auf *NIX Systemen) und UTF-16LE ( wird bei Windows benutzt). 

\myparagraph{UTF-8}

\myindex{UTF-8}
UTF-8 ist eine der erfolgreichsten Methoden um Zeichen zu codieren.
Alle Latein Zeichen werden codiert so wie in ASCII, und alle Symbole nach der
ASCII Tabelle wurden codiert mit zus\"atzlichen Bytes. 0 wird codiert als davor,
also arbeiten alle Standard C String Funktionen mit UTF-8 Strings wie mit jedem anderen String auch.

Lasst uns anschauen wie die Symbole in verschiedenen anderen Sprachen nach UTF-8 Codiert werden und 
wie man sie als FAR aussehen lassen kann, durch das benutzen der codepage 437.

\footnote{Beispiel und \"Ubersetzung k\o"nnen von hier bezogen werden:  
\url{http://go.yurichev.com/17304}}:

\begin{figure}[H]
\centering
\includegraphics[width=0.6\textwidth]{digging_into_code/strings/multilang_sampler.png}
\end{figure}

% FIXME: cut it
\begin{figure}[H]
\centering
\myincludegraphics{digging_into_code/strings/multilang_sampler_UTF8.png}
\caption{FAR: UTF-8}
\end{figure}

Wie man hier sehen kann, der Englische String sieht genauso aus wie sein Gegenst\"uck in ASCII.

Die Ungarische Sprache benutzt Latein Symbole plus ein paar Symbole mit diacritic Markierungen.

Diese Symbole werden mit mehreren Bytes codiert, diese wurden rot unterstrichen.
Das gleiche gilt f\"ur die Isl\"andischen und Polnischen Sprachen.

Es gibt auch das \q{Euro} W\"ahrungs Symbol im Standard, das Symbol wurde mit 3 Bytes Codiert.

Der Rest der Schreibsysteme hat keinen Bezug zu Latein.

Zumindest in Russisch, Arabisch, Hebr\"aisch und Hindu k\"onnen wir wiederkehrende Bytes erkennen und das ist nicht mal \"uberraschend:
Alle Zeichen eines Schreibsystems werden normalerweise in der selben Unicode Tabelle angelegt, also f\"angt ihr code mit den 
immer gleichen nummern an. % <--- Wird anders \"ubersetzt.

Zu Anfang, noch vor dem \q{How much?} String sehen wir 3 Bytes, die tats\"achlich das \ac{BOM} darstellen.
Das \ac{BOM} definiert das Codierungssystem das benutzt werden soll.

\myparagraph{UTF-16LE}

\myindex{UTF-16LE}
\myindex{Windows!Win32}
Viele win32 Funktionen in Windows haben die Suffixe \TT{-A} und \TT{-W}. 
Der erste Typ Funktionen arbeitet mit normalen Strings, der andere Typ mit 
UTF-16LE Strings (\emph{wide}). 

Im zweiten Fall, wird jedes Symbol normal als 16-Bit Wert des Typs \emph{short} gespeichert.

Die Latein Symbole in UFT-16 Strings sehen in Hiew oder FAR aus als w\"aren sie mit Null Bytes verschachtelt:

\begin{lstlisting}[style=customc]
int wmain()
{
	wprintf (L"Hello, world!\n");
};
\end{lstlisting}

\begin{figure}[H]
\centering
\includegraphics[width=0.6\textwidth]{digging_into_code/strings/UTF16-string.png}
\caption{Hiew}
\end{figure}

Wir k\o"nnen das oft auch in gls{Windows NT} System Dateien sehen:

\begin{figure}[H]
\centering
\includegraphics[width=0.6\textwidth]{digging_into_code/strings/ntoskrnl_UTF16.png}
\caption{Hiew}
\end{figure}

\myindex{IDA}
Strings mit Zeichen die exakt 2 Bytes verbrauchen werden \q{Unicode} in \IDA genannt:

\begin{lstlisting}[style=customasmx86]
.data:0040E000 aHelloWorld:
.data:0040E000                 unicode 0, <Hello, world!>
.data:0040E000                 dw 0Ah, 0
\end{lstlisting}

Hier sieht man wie Russische Sprache in UTF-16LE Codiert wird:

\begin{figure}[H]
\centering
\includegraphics[width=0.6\textwidth]{digging_into_code/strings/russian_UTF16.png}
\caption{Hiew: UTF-16LE}
\end{figure}

Was man leicht sehen kann ist das die Symbole durchzogen sind von den Diamant Zeichen (das im ASCII code mit 4 codiert wird).
Tats\"achlich, findet man die Kyrillischen Symbole in der vierten Unicode Tabelle 
\footnote{\href{http://go.yurichev.com/17003}{wikipedia}}.
Deswegen, alle Kyrillischen Symbole in UTF-16LE findet man im Bereich \TT{0x400-0x4FF}.

Lass uns noch mal zu dem Beispiel gehen mit dem String der in verschiedenen Sprachen geschrieben ist.
Hier sieht man wie der String in UTF-16LE aussieht. 

% FIXME: cut it
\begin{figure}[H]
\centering
\myincludegraphics{digging_into_code/strings/multilang_sampler_UTF16.png}
\caption{FAR: UTF-16LE}
\end{figure}

Hier k\"onnen wir auch das \ac{BOM} am Anfang sehen. 
Alle Latein Zeichen enthalten Null Bytes.

Manche Zeichen mit unterschiedlichen Markierungen (Ungarisch und Isl\"andisch) wurden rot unterstrichen.

% subsection:
\subsubsection{Base64}
\myindex{Base64}

Die Base64 Codierung ist sehr weit verbreitet f\"ur f\"alle in denen man Bin\"ardaten als Textstring \"ubertragen will.

Im Grunde, codiert dieser Algorithmus 3 Bin\"ar Bytes in 4 druckbare Zeichen: 
Alle 26 Latein Zeichen (beides klein und groß Buchstaben), Ziffern, plus Zeichen (\q{+}) und slash Zeichen (\q{/}),
64 Zeichen insgesamt. 

Ein charakteristisches Feature von Base64 Strings ist das sie oft (aber nicht immer) mit 1 oder 2  \gls{padding}
Gleichheitszeichen (\q{=}) Enden, zum Beispiel: 

\begin{lstlisting}
AVjbbVSVfcUMu1xvjaMgjNtueRwBbxnyJw8dpGnLW8ZW8aKG3v4Y0icuQT+qEJAp9lAOuWs=
\end{lstlisting}

\begin{lstlisting}
WVjbbVSVfcUMu1xvjaMgjNtueRwBbxnyJw8dpGnLW8ZW8aKG3v4Y0icuQT+qEJAp9lAOuQ==
\end{lstlisting}

Das Gleichheitszeichen Symbol (q{=}) wird man niemals in der Mitte eines Base64-codierten
Strings sehen.

Jetzt ein Beispiel wie man per Hand Base64 codieren kann.
Lasst uns 0x00, 0x11 , 0x22 und 0x33 in Hexadezimalzahlen in einen Base64
String umwandeln: 

\lstinputlisting{digging_into_code/strings/base64_ex.sh}

Lasst uns alle 4 Bytes in Bin\"ar Form bringen und dann neu gruppieren in 6-Bit Gruppen:

\begin{lstlisting}
|  00  ||  11  ||  22  ||  33  ||      ||      |
00000000000100010010001000110011????????????????
| A  || B  || E  || i  || M  || w  || =  || =  |
\end{lstlisting}

Die ersten drei Bytes (0x00, 0x11, 0x22) k\"onnen in 4 Base64 Zeichen umgewandelt werden (``ABEi''),
aber nicht das letzte Byte (0x33), also wird das Byte codiert indem man zwei Buchstaben 
benutzt (``Mw'') und das \gls{padding} Symbol (``='') wird zweimal hinzugef\"ugt um die letzte
Gruppe auf 4 Zeichen zu erweitern. Das bedeutet das die L\"ange aller korrekten Base64 Strings
sich immer durch 4 Teilen l\"asst. 

\myindex{XML}
\myindex{PGP}
Base64 wird oft benutzt wenn es darum geht Bin\"ardaten in  XML Dateien zu speichern.
``Armored'' (z.B, in Text Form) PGP Cookie und Signaturen werden codiert mit Base64.

Manche Leute versuchen auch Base64 zu benutzen um Strings zu verschleiern. 
\url{http://blog.sec-consult.com/2016/01/deliberately-hidden-backdoor-account-in.html}
\footnote{\url{http://archive.is/nDCas}}.

\myindex{base64scanner}
Es gibt Werkzeuge zum scannen von beliebigen Bin\"ardateien nach Base64 Strings.
Ein solch ein Scanner ist base64scanner\footnote{\url{https://github.com/DennisYurichev/base64scanner}}.

\myindex{UseNet}
\myindex{FidoNet}
\myindex{Uuencoding}
\myindex{Phrack}
Ein weiteres Codierungssystem welches im UseNet und FidoNet sehr weit verbreitet
war, ist UUencoding. Binärdateien sind in Phrack Magazine immernoch mit UUencoding
codiert. Es hat eigentlich die gleichen Features, unterscheidet sich von Base64
jedoch insofern, dass der Dateiname auch im Header gespeichert wird.

\myindex{Tor}
\myindex{base32}
By the Way: Es gibt auch einen nahen Verwandten zu Base64: Base32., ein Alphabet das ~10 Zeichen und ~26 Latein Zeichen hat. 
Eine verbreitete Anwendung ist Onion Adressen zu codieren. 
\footnote{\url{https://trac.torproject.org/projects/tor/wiki/doc/HiddenServiceNames}},
z.B: \\
\url{http://3g2upl4pq6kufc4m.onion/}.
\ac{URL} kann keine mixed-case Latein Zeichen beinhalten, deshalb haben Tor Entwickler sich f\"ur Base32 entschieden.




\subsection{Strings in Bin\"ar finden}

\myindex{UNIX!strings}
Das Standard UNIX \emph{strings} Utility ist ein quick-n-dirty Weg um alle Strings in der 
Datei an zu schauen. Zum Beispiel, in der OpenSSH 7.2 sshd executable Datei gibt es einige Strings:

\lstinputlisting{digging_into_code/sshd_strings.txt}

Dort kann man Optionen, Fehler Meldungen, Datei Pfade, importierte dynamische Module, Funktionen und einige andere komische 
Strings (keys?) sehen. Es gibt auch nicht druckbare Zeichen---x86 Code enth\"alt chunks von druckbaren ASCII Zeichen, bis zu ca 8 Zeichen. % <-- bessere formulierung?

Sicher, OpenSSH ist ein open-source Programm.
Aber sich die lesbaren Strings eines unbekannten Programms an zuschauen ist meist der erste Schritt bei 
der Analyse. 
\myindex{UNIX!grep}

\emph{grep} kann genauso benutzt werden.

\myindex{Hiew}
\myindex{Sysinternals}
Hiew hat die gleichen F\"ahigkeiten (Alt-F6), genau wie der Sysinternals ProcessMonitor.

\subsection{Error/debug Narchichten}

Debugging Messages sind auch sehr n\"utzlich, wenn vorhanden.
Auf gewisse weise, melden die debug Narichten was gerade
im Programm vorgeht. Oft schreiben diese \printf-\"ahnlichen Funktionen, in
log-Dateien oder sie schreiben nirgends hin aber die calls zu den printf-\"ahnlichen Funktionen sind noch vorhanden, 
weil der build kein Debug build aber ein \emph{release} ist. % <-- nochmal \"uber formulierung nachdenken
\myindex{\oracle}

Wenn lokale oder globale Variablen in Debug messages geschrieben werden, kann das auch 
hilfreich sein da man so an die Variablen Namen kommt.
Zum Beispiel, eine solche Funktion in \oracle ist \TT{ksdwrt()}.

Textstrings mit Aussage sind auch Hilfreich.
Der \IDA disassembler zeigt welche Funktion und von welchem Punkt aus ein spezifischer String benutzt wird.
Manchmal passieren lustige Dinge dabei\footnote{\href{http://go.yurichev.com/17223}{blog.yurichev.com}}.

Fehlermeldungen helfen uns genauso.
In \oracle, werden Fehler von einer Gruppe von Funktionen gemeldet.
\"Uber das Thema kann man mehr hier erfahren: \href{http://go.yurichev.com/17224}{blog.yurichev.com}.

\myindex{Error messages}

Es ist M\"oglich heraus zu finden welche Funktionen Fehler melden und unter welchen Bedingungen.


\"Ubrigens, das ist f\"ur Kopierschutztsysteme oft der Grund kryptische Fehlermeldungen oder einfach nur 
Fehlernummer aus zu geben. Niemand ist gl\"ucklich dar\"uber wenn der Softwarecracker den Kopierschutz besser
versteht nur weil dieser durch eine Fehlermeldung ausgel\"ost wurde.

Ein Beispiel von verschl\"usselten Fehlermeldungen gibt es hier: \myref{examples_SCO}.

\subsection{Verd\"achtige magic strings}

Manche Magic Strings die in Hintert\"uren benutzt werden sehen schon ziemlich verd\"achtig aus.

Zum Beispiel, es gab eine Hintert\"ur im TP-Link WR740 Home Router\footnote{\url{http://sekurak.pl/tp-link-httptftp-backdoor/}}.
Die Hintert\"ur konnte aktiviert werden wenn man folgende URL aufrief:
\url{http://192.168.0.1/userRpmNatDebugRpm26525557/start_art.html}.\\

Tats\"achlich, kann man den Magic String \q{userRpmNatDebugRpm26525557} in der Firmware finden.

Der String war nicht googlebar bis die Information \"offentlich \"uber die Hintert\"ur \"offentlich verbreitet wurde.


Man w\"urde solche Informationen nat\"urlich auch nicht in irgendeinem \ac{RFC} finden.


Man w\"urde auch keinen Algorithmus finden der solch seltsame Byte Sequenzen benutzt.


Und es sieht auch nicht nach einer Fehler- order Debugnaricht aus.


Also es ist immer eine gute Idee so seltsamen Dinge genauer zu betrachten.

\myindex{base64}

Manchmal, sind solche Strings auch mit base64 codiert.

Es ist also immer eine gute Idee diese Stings zu Decodieren und sie visuell zu durchsuchen, ein Blick
kann schon gen\"ugen.

\myindex{Security through obscurity}
Pr\"aziser gesagt, diese Methode Hintert\"uren zu verstecken nennt man \q{security through obscurity}.

\input{digging_into_code/assert_DE}
\mysection{Konstanten}

Menschen, Programmierer eingeschlossen, neigen dazu Zahlen zu runden wie z.B 10, 100, 1000,
im realen Leben so wie in ihrem Code.

Der angehende Reverse Engineer kennt diese Werte und ihre hexadezimale Repr\"asentation sehr gut:
10=0xA, 100=0x64, 1000=0x3E8, 10000=0x2710.

Die Konstanten \TT{0xAAAAAAAA} (0b10101010101010101010101010101010) und \\
\TT{0x55555555} (0b01010101010101010101010101010101) sind auch sehr popul\"ar---
sie sind zusammengesetzt aus ver\"andernden Bits. % <-- Findest vielleicht noch ne bessere Bezeichnung

Dies hilft Signale voneinander zu unterscheiden bei denen alle Bits eingeschaltet (0b1111 \dots) oder ausgeschaltet (0b0000 \dots) werden .
Zum Beispiel wird die Konstante \TT{0x55AA} beim Boot Sektor, \ac{MBR},
und im \ac{ROM} von IBM-Kompatiblen Erweiterung Karten benutzt.

Manche Algorithmen, speziell die Kryptografischen benutzen eindeutige Konstanten, die mit der Hilfe von \IDA einfach im Code zu finden sind.

\myindex{MD5}
\newcommand{\URLMD}{http://go.yurichev.com/17111}

Zum Beispiel, der MD5\footnote{\href{\URLMD}{wikipedia}} Algorithmus initialisiert seine Internen Variablen wie folgt:


\begin{verbatim}
var int h0 := 0x67452301
var int h1 := 0xEFCDAB89
var int h2 := 0x98BADCFE
var int h3 := 0x10325476
\end{verbatim}

Wenn man diese vier Konstanten im Code hintereinander benutzt findet, dann ist die Wahrscheinlichkeit das diese Funktion 
sich auf MD5 bezieht.

\par Ein weiteres Beispiel sind die CRC16/CRC32 Algorithmen,
ihre Berechnungs Algorithmen benutzen oft vorberechnete Tabellen wie diese:

\begin{lstlisting}[caption=linux/lib/crc16.c,style=customc]
/** CRC table for the CRC-16. The poly is 0x8005 (x^16 + x^15 + x^2 + 1) */
u16 const crc16_table[256] = {
	0x0000, 0xC0C1, 0xC181, 0x0140, 0xC301, 0x03C0, 0x0280, 0xC241,
	0xC601, 0x06C0, 0x0780, 0xC741, 0x0500, 0xC5C1, 0xC481, 0x0440,
	0xCC01, 0x0CC0, 0x0D80, 0xCD41, 0x0F00, 0xCFC1, 0xCE81, 0x0E40,
	...
\end{lstlisting}

Man beachte auch die vorberechnete Tabelle f\"ur CRC32: \myref{sec:CRC32}.

In tabellenlosen CRC-Algorithmen werden bekannte Polynome benutzt, zum Beispiel, 0xEDB88320 f\"ur CRC32.

\subsection{Magic numbers}
\label{magic_numbers}

\newcommand{\FNURLMAGIC}{\footnote{\href{http://go.yurichev.com/17112}{wikipedia}}}

Viele Datei-Formate definieren einen Standard-Dateiheader in dem eine \emph{magic number(s)}\FNURLMAGIC{} benutzt wird, einzelne oder sogar mehrere. 

\myindex{MS-DOS}

Zum Beispiel, alle Win32 und MS-DOS executable starten mit zwei Zeichen \q{MZ}\footnote{\href{http://go.yurichev.com/17113}{wikipedia}}.


\myindex{MIDI}

Am Anfang einer MIDI Datei muss die \q{MThd} Signatur vorhanden sein.
Wenn wir ein Programm haben das auf MIDI Dateien zugreift um sonst was zu machen,
ist es sehr wahrscheinlich das das Programm die Datei validieren muss in dem es
mindestens die ersten 4 Bytes pr\"uft.

Das kann man wie folgt realisieren:
(\emph{buf} Zeigt auf den Anfang der geladenen Datei im Speicher) 

\begin{lstlisting}[style=customasmx86]
cmp [buf], 0x6468544D ; "MThd"
jnz _error_not_a_MIDI_file
\end{lstlisting}

\myindex{\CStandardLibrary!memcmp()}
\myindex{x86!\Instructions!CMPSB}

\dots oder durch das Aufrufen der Funktion f\"ur das vergleichen von Speicherbl\"ocken wie z.B \TT{memcmp()} oder 
beliebigen anderen Code bis hin zu einer \TT{CMPSB} (\myref{REPE_CMPSx}) Instruktion.

Wenn man so einen Punkt findet kann man bereits sagen das eine MIDI Datei geladen wird, % <-- \"Andern?
wir k\"onnen auch sehen wo der Puffer mit den Inhalten der MIDI Datei liegt und was/wie aus diesem
Puffer verwendet wird.

\subsubsection{Daten}

\myindex{UFS2}
\myindex{FreeBSD}
\myindex{HASP}

Oft findet man auch nur eine Zahl wie \TT{0x19870116}, was ganz klar nach einem Jahres Datum aussieht (Tag 16,  1 Monat (Januar),  Jahr 1987).
Das ist vielleicht das Geburtsdatum von jemandem (ein Programmierer. ihre/seine bekannte, Kind), oder ein anderes wichtiges Datum.
Das Datum kann auch in umgekehrter folge auftreten, wie z.B \TT{0x16011987}. 
Datums angaben im Amerikanischen-Stil sind auch weit verbreitet wie \TT{0x01161987}.

Ein ziemlich bekanntes Beispiel ist  \TT{0x19540119} (magic number wird in der UFS2 Superblock Struktur benutzt), das 
Geburtsdatum von Marschall Kirk McKusick ist, einem Prominenten FreeBSD Entwickler. 


\myindex{Stuxnet}
Stuxnet benutzt die Zahl ``19790509'' (nicht als 32-Bit Zahl, aber als String), was zu Spekulationen gef\"uhrt hat
weil die malware Verbindungen nach Israel aufzeigt.
\footnote{Das ist das Datum der Hinrichtung von Habib Elghanian, persischer Jude.}

Solche Zahlen sind auch sehr beliebt in Amateur Kryptografie, zum Beispiel, ein Ausschnitt aus den \emph{secret function} Interna aus dem HASP3 Dongle %  <-- Vielleicht bessere formulierung?
\footnote{\url{https://web.archive.org/web/20160311231616/http://www.woodmann.com/fravia/bayu3.htm}}:

\begin{lstlisting}[style=customc]
void xor_pwd(void) 
{ 
	int i; 
	
	pwd^=0x09071966;
	for(i=0;i<8;i++) 
	{ 
		al_buf[i]= pwd & 7; pwd = pwd >> 3; 
	} 
};

void emulate_func2(unsigned short seed)
{ 
	int i, j; 
	for(i=0;i<8;i++) 
	{ 
		ch[i] = 0; 
		
		for(j=0;j<8;j++)
		{ 
			seed *= 0x1989; 
			seed += 5; 
			ch[i] |= (tab[(seed>>9)&0x3f]) << (7-j); 
		}
	} 
}
\end{lstlisting}

\subsubsection{DHCP}

Das Trifft auf Netzwerk Protokolle ebenso zu. 
Zum Beispiel, die Pakete des DHCP Protokoll's beinhalten so genannte \emph{magic cookie}: \TT{0x63538263}.
Jeder Code der ein DHCP Pakete generiert, muss diese Konstante in das Pakete einbetten.
Wenn wir diesen Code finden, wissen wir auch wo es passiert und nicht nur was passiert.
Jedes Programm das DHCP Pakete empfangen kann muss verifizieren das der \emph{magic cookie} mit der Konstante 
\"ubereinstimmt. 

Zum Beispiel, lasst uns die dhcpcore.dll Datei aus Windows 7 x64 analysieren die nach der Konstante suchen.
Wir k\"onnen die Konstante zweimal finden:
Es sieht danach aus als w\"are die Konstante in zwei Funktionen benutzt mit dem selbst redenden Namen\\
\TT{DhcpExtractOptionsForValidation()} und \TT{DhcpExtractFullOptions()}:

\begin{lstlisting}[caption=dhcpcore.dll (Windows 7 x64),style=customasmx86]
.rdata:000007FF6483CBE8 dword_7FF6483CBE8 dd 63538263h          ; DATA XREF: DhcpExtractOptionsForValidation+79
.rdata:000007FF6483CBEC dword_7FF6483CBEC dd 63538263h          ; DATA XREF: DhcpExtractFullOptions+97
\end{lstlisting}

Und hier die (Speicher) Orte an denen auf die Konstante zugegriffen wird:

\begin{lstlisting}[caption=dhcpcore.dll (Windows 7 x64),style=customasmx86]
.text:000007FF6480875F  mov     eax, [rsi]
.text:000007FF64808761  cmp     eax, cs:dword_7FF6483CBE8
.text:000007FF64808767  jnz     loc_7FF64817179
\end{lstlisting}

Und:

\begin{lstlisting}[caption=dhcpcore.dll (Windows 7 x64),style=customasmx86]
.text:000007FF648082C7  mov     eax, [r12]
.text:000007FF648082CB  cmp     eax, cs:dword_7FF6483CBEC
.text:000007FF648082D1  jnz     loc_7FF648173AF
\end{lstlisting}

\subsection{Spezifische Konstanten}

Manchmal, gibt es spezifische Konstanten f\"ur gewissen Code % <-- Besser? 
Zum Beispiel, einmal hat der Autor sich in ein St\"uck Code gegraben wo die Nummer 12 verd\"achtig
oft vor kam. Arrays haben oft eine Gr\"oße von 12 oder ein vielfaches von 12 (24, etc). 
Wie sich raus stellte, hat der Code eine 12-Kanal Audiodatei an der Eingabe entgegen genommen und
sie verarbeitet.

Und umgekehrt: zum Beispiel, wenn ein Programm ein Textfeld verarbeitet das eine L\"ange von 120 Bytes hat,
dann gibt es auch eine Konstante 120 oder 119 irgendwo im Code.
Wenn UTF-16 Benutzt wird, dann $2 \cdot 120$. Wenn Code mit Netzwerkpaketen arbeitet die von fester Gr\"oße
sind, ist es eine gute Idee nach dieser Konstante im Code zu suchen.

Das trifft auch auf Amateur Kryptografie zu (Lizenz Schl\"ussel, etc). 
Bei einem verschl\"usselten Block von $n$ Bytes, will man versuchen die vorkommen dieser Nummer im Code zu suchen,
auch, wenn man ein St\"uck Code sieht der sich $n$ mal w\"ahrend einer Schleifen Ausf\"uhrung wiederholt, ist das vielleicht
eine ver-/Entschl\"usselung Routine.

\subsection{Nach Konstanten suchen}

Das ist einfach mit \IDA: Alt-B oder Alt-I.
\myindex{bin\"ar grep}
Und f\"ur das suchen von Konstanten in einem Haufen großer Dateien, oder f\"ur das suchen in nicht ausf\"uhrbaren Dateien,
gibt es ein kleines Utility genannt \emph{binary grep}\footnote{\BGREPURL}.

\input{digging_into_code/instructions_DE}
\mysection{Verd\"achtige Code muster}

\subsection{XOR Instruktionen}
\myindex{x86!\Instructions!XOR}

Instruktionen wie \TT{XOR op, op} (zum Beispiel, \TT{XOR EAX, EAX})
werden normal daf\"ur benutzt Register Werte auf Null zu setzen, wenn jedoch
einer der Operanden sich unterscheidet wird die \q{exclusive or} Operation 
ausgef\"uhrt.

Diese Operation wird allgemeinen selten benutzt beim programmieren, aber ist
weit verbreitet in der Kryptografie, besonders bei Amateuren der Kryptografie.
Sowas ist besonders Verd\"achtig wenn der zweite Operand eine große Zahl ist.

Das k\"onnte ein Hinweis sein das etwas ver-/entschl\"usselt wird oder Checksumme berechnet werden, etc.

Eine Ausnahme dieser Beobachtung ist der \q{canary} (\myref{subsec:BO_protection}). 
Die Generierung und das pr\"ufen des \q{canary} werden oft mit Hilfe der \XOR Instruktion gemacht. 

\myindex{AWK}

Dieses AWK Skript kann benutzt werden um \IDA{} listing (.lst) Dateien zu parsen:

\lstinputlisting{digging_into_code/awk.sh}

Es sollte auch noch erw\"ahnt werden das diese Art von Skript in der Lage ist inkorrekt disassemblierten Code zu erkennen
(\myref{sec:incorrectly_disasmed_code}).

\subsection{Hand geschriebener Assembler code}

\myindex{Function prologue}
\myindex{Function epilogue}
\myindex{x86!\Instructions!LOOP}
\myindex{x86!\Instructions!RCL}

Moderne Compiler benutzen keine \TT{LOOP} und \TT{RCL} Instruktionen.
Auf der anderen Seite sind diese Instruktionen sehr beliebt bei Programmieren die Code direkt in Assembler schreiben.
Wenn man diese Instruktionen sieht, kann man mit hoher Sicherheit sagen das dieses Code Fragment h\"andisch geschrieben wurde.,
Diese Instruktionen sind in der Instruktionsliste im Anhang mit (M) markiert: \myref{sec:x86_instructions}.

\par Die Funktions Prolog und Epilog sind allgemein nicht vorhanden bei handgeschriebenen Assembler Code.

\par Tats\"achlich gibt es kein bestimmtes System um Argumente an Funktionen zu \"ubergeben wenn der Code handgeschrieben wurde. 

\par Beispiel aus dem Windows 2003 Kernel (ntoskrnl.exe file):

\lstinputlisting[style=customasmx86]{digging_into_code/ntoskrnl.lst}

Tats\"achlich, wenn wir in den \ac{WRK} v1.2 source code schauen, kann dieser Code einfach in der Datei
\emph{WRK-v1.2\textbackslash{}base\textbackslash{}ntos\textbackslash{}ke\textbackslash{}i386\textbackslash{}cpu.asm} gefunden werden.

\mysection{Using magic numbers while tracing}

Oft ist unser Hauptziel zu verstehen wie ein Programm einen Wert behandelt der entweder \"uber eine Datei oder \"uber das Netzwerk erhalten wurde.
Das manuelle tracen eines Wertes ist meistens ein ziemlich arbeits-intensiver Task. Eine der einfachsten Techniken um Werte zu Tracen (auch wenn nicht 100\% verl\"asslich)
ist eigene \emph{magic number}'s zu benutzen. 

Das \"ahnelt ein wenig dem Vorgang beim R\"ontgen auf gewisser weise: ein radioaktives Kontrastmittel wird dem Patienten injeziert,
welches dann benutzt wird um die Gef\"asse des Patienten besser zu erkennen duch die R\"onthgenstahlung. Wie das blut bei 
gesunden Menschen in den Nieren gereinigt wird wenn das Kontrastmittel im Blut ist, man kann dann sehr einfach auf dem
Bild der Tomografie erkennen ob sich Nierensteine oder Tumore in den Nierenbefinden. 

Wir k\"onnen einfach eine 32-Bit Zahl nehmen z.B \TT{0xbadf00d}, oder ein Geburtsdatum wie \TT{0x11101979}
und diese 4-Byte Zahl wird an einem bestimmten Punkt in eine Datei geschrieben welche von dem Programm 
das wir untersuchen genutzt wird. 

\myindex{\GrepUsage}
\myindex{tracer}

Dann w\"ahrend das programm getraced wird mit \tracer im \emph{code coverage} modus, mit der Hilfe von \emph{grep}
oder durch einfaches durchsuchen der Textdatei (der trace Ergebnisse), k\"onnen wir ganz einfach sehen wo der 
Wert benutzt wurde und wie er benutzt wurde. 

Beispiel der \emph{grepable} \tracer Ergebnissen im \emph{cc} mode:

\begin{lstlisting}[style=customasmx86]
0x150bf66 (_kziaia+0x14), e=       1 [MOV EBX, [EBP+8]] [EBP+8]=0xf59c934 
0x150bf69 (_kziaia+0x17), e=       1 [MOV EDX, [69AEB08h]] [69AEB08h]=0 
0x150bf6f (_kziaia+0x1d), e=       1 [FS: MOV EAX, [2Ch]] 
0x150bf75 (_kziaia+0x23), e=       1 [MOV ECX, [EAX+EDX*4]] [EAX+EDX*4]=0xf1ac360 
0x150bf78 (_kziaia+0x26), e=       1 [MOV [EBP-4], ECX] ECX=0xf1ac360 
\end{lstlisting}
% TODO: good example!

Das gleiche verfahren kann man auch auf Netzwerkpakete anwenden.
F\"ur die \emph{magic number} ist es wichtig das diese einzigartig ist und nicht im Programm code vorkommt.

\newcommand{\DOSBOXURL}{\href{http://go.yurichev.com/17222}{blog.yurichev.com}}

\myindex{DosBox}
\myindex{MS-DOS}
Neben dem \tracer Befehl, gibt es noch den DosBox (MS-DOS emulator) im heavydebug Modus,
welcher in der Lage ist alle Informationen \"uber alle Register zust\"ande f\"ur jede ausgef\"uhrte Instruktion des Programmes in
eine einfache Textdatei\footnote{See also my blog post about this DosBox feature: \DOSBOXURL{}} zu schreiben, so kann
diese Technik f\"ur DOS Programme n\"utzlich sein. 


\input{digging_into_code/loops_DE}
\subsubsection{Struct als Menge von Werten}
Um zu veranschaulichen, dass ein struct nur eine Menge von nebeneinanderliegenden Variablen ist, überarbeiten wir unser
Beispiel, indem wir auf die Definition des \emph{tm} structs schauen:\lstref{struct_tm}.

\lstinputlisting[style=customc]{patterns/15_structs/3_tm_linux/as_array/GCC_tm2.c}

\myindex{\CStandardLibrary!localtime\_r()}
Der Pointer auf das Feld \TT{tm\_sec} wird nach \TT{localtime\_r} übergeben, d.h. an das erste Element des structs.

Der Compiler warnt uns:

\begin{lstlisting}[caption=GCC 4.7.3]
GCC_tm2.c: In function 'main':
GCC_tm2.c:11:5: warning: passing argument 2 of 'localtime_r' from incompatible pointer type [enabled by default]
In file included from GCC_tm2.c:2:0:
/usr/include/time.h:59:12: note: expected 'struct tm *' but argument is of type 'int *'
\end{lstlisting}

Trotzdem erzeugt er folgenden Code:

\lstinputlisting[caption=GCC 4.7.3,style=customasmx86]{patterns/15_structs/3_tm_linux/as_array/GCC_tm2.asm}
Dieser Code ist zum vorherigen identisch und es ist unmöglich zu sagen, ob es sich im originalen Quellcode um ein struct
oder nur um eine Menge von Variablen handelt.

Es funktioniert also, ist aber in der Praxis nicht empfehlenswert. 

Nicht optimierende Compiler legen normalerweise Variablen auf dem lokalen Stack in der Reihenfolge an, in der sie in der
Funktion deklariert wurden.

Ein Garantie dafür gibt es freilich nicht.

Andere Compiler könnten an dieser Stelle übrigens davor warnen, dass die Variablen \TT{tm\_year}, \TT{tm\_mon}, \TT{tm\_mday},
\TT{tm\_hour}, \TT{tm\_min} - nicht aber \TT{tm\_sec} - ohne Initialisierung verwendet werden.

Der Compiler weiß nicht, dass diese durch die Funktion \TT{localtime\_r()} befüllt werden.

Wir haben dieses Beispiel ausgewählt, da alle Felder im struct vom Typ \Tint sind.

Es würde nicht funktionieren, wenn die Felder 16 Bit (\TT{WORD}) groß wären, wie im Beispiel des \TT{SYSTEMTIME}
structs---\TT{GetSystemTime()} würde sie falsch befüllen (da die lokalen Variablen auf 32-Bit-Grenzen angeordnet sind).
Mehr dazu im folgenden Abschnitt: \q{\StructurePackingSectionName} (\myref{structure_packing}).

Ein struct ist also nichts als eine Menge von an einer Stelle gespeicherten Variablen.
Man kan sagen, dass das struct ein Befehl an den Compiler ist, diese Variablen an einer Stelle zu halten.
In ganz frühen Versionen von C (vor 1972) gab es übrigens gar keine structs \RitchieDevC.

Dieses Beispiel wird nicht im Debugger gezeigt, da es dem gerade gezeigten entspricht.

\subsubsection{Struct als Array aus 32-Bit-Worten}

\lstinputlisting[style=customc]{patterns/15_structs/3_tm_linux/as_array/GCC_tm3.c}
Wir können einen Pointer auf ein struct in ein Array aus \Tint{}s casten und es funktioniert.
Wir lassen dieses Beispiel zur Systemzeit 23:51:45 26-July-2014 laufen.

\begin{lstlisting}[label=GCC_tm3_output]
0x0000002D (45)
0x00000033 (51)
0x00000017 (23)
0x0000001A (26)
0x00000006 (6)
0x00000072 (114)
0x00000006 (6)
0x000000CE (206)
0x00000001 (1)
\end{lstlisting}
Die Variablen sind hier in der gleichen Reihenfolge, in der die in der Definition des structs aufgezählt
werden:\myref{struct_tm}.

Hier ist der erzeugte Code:

\lstinputlisting[caption=\Optimizing GCC
4.8.1,style=customasmx86]{patterns/15_structs/3_tm_linux/as_array/GCC_tm3_DE.lst}
Tatsächlich: der Platz auf dem lokalen Stack wird zuerst wie in struct und dann wie ein Array behandelt.

Es ist sogar möglich, die Felder des structs über diesen Pointer zu verändern.

Und wiederum ist es zweifellos ein seltsamer Weg die Dinge umzusetzen; er ist für produktiven Code definitiv nicht
empfehlenswert.

\mysubparagraph{\Exercise}
Versuchen Sie als Übung die Monatsnummer zu verändern (um 1 zu erhöhen), indem Sie das struct wie ein Array behandeln.

\subsubsection{Struct als Bytearray}
Wir können sogar noch weiter gehen. Casten wir den Pointer zu einem Bytearray und ziehen einen Dump:

\lstinputlisting[style=customc]{patterns/15_structs/3_tm_linux/as_array/GCC_tm4.c}

\begin{lstlisting}
0x2D 0x00 0x00 0x00 
0x33 0x00 0x00 0x00 
0x17 0x00 0x00 0x00 
0x1A 0x00 0x00 0x00 
0x06 0x00 0x00 0x00 
0x72 0x00 0x00 0x00 
0x06 0x00 0x00 0x00 
0xCE 0x00 0x00 0x00 
0x01 0x00 0x00 0x00 
\end{lstlisting}
Wir haben dieses Beispiel auch zur Systemzeit 23:51:45 26-July-2014 ausgeführt
\footnote{Datum und Uhrzeit sind zu Demonstrationszwecken identisch. Die Bytewerte sind modifiziert.}.
Die Werte sind genau dieselben wie im vorherigen Dump(\myref{GCC_tm3_output}) und natürlich steht das LSB vorne, da es
sich um eine Little-Endian-Architektur handelt(\myref{sec:endianness}). 

\lstinputlisting[caption=\Optimizing GCC
4.8.1,style=customasmx86]{patterns/15_structs/3_tm_linux/as_array/GCC_tm4_DE.lst}

\input{digging_into_code/snapshots_comparing_DE}
% TBT \input{digging_into_code/ISA_detect_DE}

\mysection{Andere Dinge}

\subsection{Die Idee}  

Ein Reverse Engineer sollte versuchen so oft wie m\"oglich in den Schuhen des
Programmierers zu laufen. Um ihren/seinen Standpunkt zu betrachten uns sich
selbst zu Fragen wie man einen Task in spezifischen F\"allen l\"osen w\"urde.

\subsection{Anordnung von Funktionen in Bin\"ar Code}  

S\"amtliche Funktionen die in einer einzelnen .c oder .cpp-Datei gefunden werden,
werden zu den entsprechenden Objekt Dateien (.o) kompiliert. Sp\"ater, f\"ugt
der Linker alle Objektdatein die er braucht zusammen, ohne die Reihenfolge oder
die Funktionen in Ihnen zu ver\"andern. Als eine Konsequenz, ergibt sich daraus
wenn man zwei oder mehr aufeinander folgende Funktionen sieht, bedeutet dass das
sie in der gleichen Source Code Datei platziert waren (Außer nat\"urlich man bewegt
sich an der Grenze zwischen zwei Dateien.). Das bedeutet das diese Funktionen etwas
gemeinsam haben, das sie aus dem gleichen \ac{API}-Level stammen oder aus der
gleichen Library, etc.

\subsection{kleine Funktionen} 

Sehr kleine oder leere Funktionen  (\myref{empty_func})
oder Funktionen die nur ``true'' (1) oder ``false'' (0) (\myref{ret_val_func}) sind weit verbreitet,
und fast jeder ordentlicher Compiler tendiert dazu nur solche Funktionen in den resultierenden ausf\"uhrbaren Code zu stecken,
sogar wenn es mehrere gleiche Funktionen im Source Code bereits gibt. 
Also, wann immer man solche kleinen Funktionen sieht die z.B nur aus \TT{mov eax, 1 / ret} bestehen und von mehreren 
Orten aus referenziert werden (und aufgerufen werden k\"onnen), und scheinbar keine Verbindung zu einander haben, dann 
ist das wahrscheinlich das Ergebnis einer Optimierung. 

\subsection{\Cpp}

\ac{RTTI}~(\myref{RTTI})-data ist vielleicht auch n\"utzlich f\"ur die \Cpp Klassen Identifikation.
