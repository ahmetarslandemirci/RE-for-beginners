\subsection{\TT{V\$TIMER} table in \oracle}
\myindex{\oracle}

\TT{V\$TIMER} 
is another \emph{fixed view} that reflects a rapidly changing value:

\begin{framed}
\begin{quotation}
V\$TIMER displays the elapsed time in hundredths of a second. Time is measured since the beginning of the epoch, 
which is operating system specific, and wraps around to 0 again whenever the value overflows four bytes 
(roughly 497 days).
\end{quotation}
\end{framed}(From \oracle documentation
\footnote{\url{http://go.yurichev.com/17088}})


It is interesting that the periods are different for Oracle for win32 and for Linux. 
Will we be able to find the function that generates this value?

As we can see, 
this information is finally taken from the \TT{X\$KSUTM} table.

\begin{lstlisting}
SQL> select * from V$FIXED_VIEW_DEFINITION where view_name='V$TIMER';

VIEW_NAME
------------------------------
VIEW_DEFINITION
------------------------------

V$TIMER
select  HSECS from GV$TIMER where inst_id = USERENV('Instance')

SQL> select * from V$FIXED_VIEW_DEFINITION where view_name='GV$TIMER';

VIEW_NAME
------------------------------
VIEW_DEFINITION
------------------------------

GV$TIMER
select inst_id,ksutmtim from x$ksutm
\end{lstlisting}

Now we are stuck in a small problem, there are no references to value generating function(s) 
in the tables \TT{kqftab}/\TT{kqftap}:

\begin{lstlisting}[caption=Result of \OracleTablesName]
kqftab_element.name: [X$KSUTM] ?: [ksutm] 0x1 0x4 0x4 0x0 0xffffc09b 0x3
kqftap_param.name=[ADDR] ?: 0x10917 0x0 0x0 0x0 0x4 0x0 0x0
kqftap_param.name=[INDX] ?: 0x20b02 0x0 0x0 0x0 0x4 0x0 0x0
kqftap_param.name=[INST_ID] ?: 0xb02 0x0 0x0 0x0 0x4 0x0 0x0
kqftap_param.name=[KSUTMTIM] ?: 0x1302 0x0 0x0 0x0 0x4 0x0 0x1e
kqftap_element.fn1=NULL
kqftap_element.fn2=NULL
\end{lstlisting}

When we try to find the string \TT{KSUTMTIM}, we see it in this function:

\begin{lstlisting}[style=customasmx86]
kqfd_DRN_ksutm_c proc near    ; DATA XREF: .rodata:0805B4E8

arg_0   = dword ptr  8
arg_8   = dword ptr  10h
arg_C   = dword ptr  14h

        push    ebp
        mov     ebp, esp
        push    [ebp+arg_C]
        push    offset ksugtm
        push    offset _2__STRING_1263_0 ; "KSUTMTIM"
        push    [ebp+arg_8]
        push    [ebp+arg_0]
        call    kqfd_cfui_drain
        add     esp, 14h
        mov     esp, ebp
        pop     ebp
        retn
kqfd_DRN_ksutm_c endp
\end{lstlisting}

The \TT{kqfd\_DRN\_ksutm\_c()} function is mentioned in the \\
\TT{kqfd\_tab\_registry\_0} table:

\begin{lstlisting}[style=customasmx86]
dd offset _2__STRING_62_0 ; "X\$KSUTM"
dd offset kqfd_OPN_ksutm_c
dd offset kqfd_tabl_fetch
dd 0
dd 0
dd offset kqfd_DRN_ksutm_c
\end{lstlisting}

There is a function \TT{ksugtm()} referenced here.
Let's see what's in it (Linux x86):

\begin{lstlisting}[caption=ksu.o,style=customasmx86]
ksugtm  proc near

var_1C  = byte ptr -1Ch
arg_4   = dword ptr  0Ch

        push    ebp
        mov     ebp, esp
        sub     esp, 1Ch
        lea     eax, [ebp+var_1C]
        push    eax
        call    slgcs
        pop     ecx
        mov     edx, [ebp+arg_4]
        mov     [edx], eax
        mov     eax, 4
        mov     esp, ebp
        pop     ebp
        retn
ksugtm  endp
\end{lstlisting}

The code in the win32 version is almost the same.

Is this the function we are looking for? Let's see:
\myindex{tracer}

\begin{lstlisting}
tracer -a:oracle.exe bpf=oracle.exe!_ksugtm,args:2,dump_args:0x4
\end{lstlisting}

Let's try again:

\begin{lstlisting}
SQL> select * from V$TIMER;

     HSECS
----------
  27294929

SQL> select * from V$TIMER;

     HSECS
----------
  27295006

SQL> select * from V$TIMER;

     HSECS
----------
  27295167
\end{lstlisting}

\begin{lstlisting}[caption=\tracer output]
TID=2428|(0) oracle.exe!_ksugtm (0x0, 0xd76c5f0) (called from oracle.exe!__VInfreq__qerfxFetch+0xfad (0x56bb6d5))
Argument 2/2
0D76C5F0: 38 C9                                           "8.              "
TID=2428|(0) oracle.exe!_ksugtm () -> 0x4 (0x4)
Argument 2/2 difference
00000000: D1 7C A0 01                                     ".|..            "
TID=2428|(0) oracle.exe!_ksugtm (0x0, 0xd76c5f0) (called from oracle.exe!__VInfreq__qerfxFetch+0xfad (0x56bb6d5))
Argument 2/2
0D76C5F0: 38 C9                                           "8.              "
TID=2428|(0) oracle.exe!_ksugtm () -> 0x4 (0x4)
Argument 2/2 difference
00000000: 1E 7D A0 01                                     ".}..            "
TID=2428|(0) oracle.exe!_ksugtm (0x0, 0xd76c5f0) (called from oracle.exe!__VInfreq__qerfxFetch+0xfad (0x56bb6d5))
Argument 2/2
0D76C5F0: 38 C9                                           "8.              "
TID=2428|(0) oracle.exe!_ksugtm () -> 0x4 (0x4)
Argument 2/2 difference
00000000: BF 7D A0 01                                     ".}..            "
\end{lstlisting}

Indeed---the value is the same we see in SQL*Plus and it is returned via the second argument.

Let's see what is in \TT{slgcs()} (Linux x86):

\begin{lstlisting}[style=customasmx86]
slgcs   proc near

var_4   = dword ptr -4
arg_0   = dword ptr  8

        push    ebp
        mov     ebp, esp
        push    esi
        mov     [ebp+var_4], ebx
        mov     eax, [ebp+arg_0]
        call    $+5
        pop     ebx
        nop                     ; PIC mode
        mov     ebx, offset _GLOBAL_OFFSET_TABLE_
        mov     dword ptr [eax], 0
        call    sltrgatime64    ; PIC mode
        push    0
        push    0Ah
        push    edx
        push    eax
        call    __udivdi3       ; PIC mode
        mov     ebx, [ebp+var_4]
        add     esp, 10h
        mov     esp, ebp
        pop     ebp
        retn
slgcs   endp
\end{lstlisting}

(it is just a call to \TT{sltrgatime64()}

and division of its result by 10 (\myref{sec:divisionbymult}))

And win32-version:

\begin{lstlisting}[style=customasmx86]
_slgcs  proc near     ; CODE XREF: \_dbgefgHtElResetCount+15
                      ; \_dbgerRunActions+1528
        db      66h
        nop
        push    ebp
        mov     ebp, esp
        mov     eax, [ebp+8]
        mov     dword ptr [eax], 0
        call    ds:__imp__GetTickCount@0 ; GetTickCount()
        mov     edx, eax
        mov     eax, 0CCCCCCCDh
        mul     edx
        shr     edx, 3
        mov     eax, edx
        mov     esp, ebp
        pop     ebp
        retn
_slgcs  endp
\end{lstlisting}

It is just the result of \TT{GetTickCount()}
\footnote{\href{http://go.yurichev.com/17248}{MSDN}}
divided by 10 (\myref{sec:divisionbymult}).

% TODO add calc

Voilà! That's why the win32 version and the Linux x86 version show different results, 
because they are generated by different \ac{OS} functions.

\emph{Drain} apparently implies \emph{connecting} a specific table column to a specific function.

We will add support of the table \TT{kqfd\_tab\_registry\_0} to \oracletables, 
now we can see how the table column's variables are \emph{connected} to a specific functions:

\begin{lstlisting}
[X$KSUTM] [kqfd_OPN_ksutm_c] [kqfd_tabl_fetch] [NULL] [NULL] [kqfd_DRN_ksutm_c]
[X$KSUSGIF] [kqfd_OPN_ksusg_c] [kqfd_tabl_fetch] [NULL] [NULL] [kqfd_DRN_ksusg_c]
\end{lstlisting}

\emph{OPN}, apparently stands for, \emph{open}, and \emph{DRN}, apparently, for \emph{drain}.

